\section{Verbale della riunione}

\subsection {Organizzazione dell'architettura del software}

\'E stata intavolato un confronto con il proponente per definire alcuni dettagli riguardanti l'architettura del software.

\subsubsection{Server}

Per quanto riguarda la componente server, sono stati definiti 3 livelli su cui si può articolare l'architettura:
\begin{enumerate}
	\item \textbf{layer di comunicazione}: gestisce la comunicazione tra il server e i client, e si può specializzare in due sezioni: la prima può essere gestita tramite Socket, e riguarderà l'invio dei dati per il monitor real-time (di visualizzazione del magazzino con i muletti che si muovono al suo interno); la seconda, gestibile tramite API di tipo REST, regola l'interazione dei due attori "Amministratore" e "Responsabile" con l'interfaccia grafica. Per approcciarsi a questa pratica, il proponente suggerisce di usufruire della libreria "Jersey" di Java, indicando come possibile fonte di studio una guida dedicata sul sito html.it;
	\item \textbf{layer di business}: il motore di calcolo del sistema;
	\item \textbf{layer di persistenza}: per la gestione della persistenza, nel caso del nostro applicativo per il momento è previsto il salvataggio in file \textit{.json}. \'E stato evidenziato come sia importante assicurare, a questo livello, la maggior indipendenza possibile dagli altri layer per consentire l'estensione ad altri tipi di persistenza.
\end{enumerate}

\subsubsection{Client}

I client si differenziano a seconda della tipologia, entrambi si compongono di una parte node e una angular:
\begin{itemize}
    \item muletti:
    \begin{itemize}
        \item node: comunicazione con il server;
        \item angular: interfaccia di visualizzazione propria mappa e comandi di guida;
    \end{itemize}
    \item utenti (admin e responsabile):
    \begin{itemize}
        \item node: comunicazione mappa real time;
        \item rest (da valutare): per le altre azioni (aggiunta/modifica/rimozione utenti/task ...).
        \item angular: interfaccia di login, visualizzazione e modifica dati e varie funzionalità in base ai privilegi.
    \end{itemize}
\end{itemize}

\subsection{Analisi dei design pattern da adottare}

Sono stati evidenziate alcune componenti del server che potrebbero essere modellate tramite il design pattern \textit{Singleton}:
\begin{enumerate}
	\item path finder: che incapsula l'algoritmo per la ricerca del percorso migliore;
	\item collision detection: per la gestione delle collisioni;
	\item warehouse map: la rappresentazione interna della mappa del magazzino.
\end{enumerate}

Per queste classi è necessario assicurare la presenza di un'unica istanza condivisa, motivo per cui il pattern Singleton può essere impiegato efficacemente.

Per quanto riguarda ConnectionAccepter (classe dedicata ad accettare le connessioni entranti) è stata pensata la possibilità di introdurre un pattern di tipo \textit{Factory}.

\subsection{Guida manuale}

\'E stata discussa l'implementazione della guida manuale per i muletti: in particolare è stato evidenziato come non debba esserci alcuna attesa di feedback da parte del server nel movimento azionato da guida manuale. Tuttavia è comunque necessario, per le unità guidate manualmente, la comunicazione con il server della propria posizione, necessarie per il calcolo (ed eventualmente, i ricalcoli) del percorso migliore per raggiungere la destinazione.
Una possibilità implementativa per assicurare queste caratteristiche è l'introduzione di un timer interno al client che regoli lo spostamento dell'unità sulla base degli input utente ricevuti.











