\section{Verbale della riunione}

\subsection{Feedback sul contenuto dei manuali utente e manutentore}
\label{manuali}
    I manuali svolgono la loro funzione, si riscontrano le seguenti possibili migliorie:
    \begin{itemize}
        \item tra i requisiti hardware, indicare:
        \begin{itemize}
            \item versione specifica o superiore;
            \item frequenza indicativa non vincolante o equivalente;
            \item RAM: quantità dedicata o condivisa (con il SO e gli altri applicativi);
        \end{itemize}
        \item tra quelli software:
        \begin{itemize}
            \item specificare gli ulteriori requisiti software, driver, libreria, sistema operativo (es versione di linux per docker…);

            \item specificare come installare (ad esempio docker, il tipo di installazione minima necessaria);

        \end{itemize}
        \item nella sezione dedicata al protocollo di comunicazione, fornire degli esempi di righe di codice che facciano capire come utilizzare, modificare ed aggiungere le funzionalità ed i comandi relativi alle API di comunicazione.
    \end{itemize}

\subsection{Discussione su test di sistema ed integrazione}
    Per i test di integrazione:
    \begin{itemize}
        \item si suggerisce di testare i layer a salire dalla componente più bassa:
        \begin{itemize}
            \item ciò che il layer di persistenza mette a disposizione agli altri layer;

            \item poi collisioni;

            \item mappa;

            \item client;

        \end{itemize}
        \item certifico il pezzo piccolo, poi certifico come i pezzi piccoli collaborano insieme.
    \end{itemize}


    Per i test di sistema:
    \begin{itemize}
        \item verificare in modo automatizzato o non tramite ambiente di test che le modifiche non rompano alcuni requisiti;

        \item anche dei semplici test delle componenti sfruttando \textit{netcat} per verificare le connessioni dei socket ed un primo scambio di messaggi può essere sufficiente.
    \end{itemize}

\subsection{Feedback su stato di avanzamento del prodotto esibendo demo}
    Il prodotto soddisfa le aspettative del proponente, viene suggerito di richiedere la conferma all'atto della modifica password.

\subsection{Discussione sul concetto di parcheggio}
\label{park}
    \begin{itemize}
        \item viene proposto di indicare una lista dei muletti parcheggiati/non attivi;
        \item questa potrebbe venire integrata in quella che indica i compiti presi in carico;
        \item predisporre uno o più punti di uscita dalla mappa, sui quali può passare un muletto alla volta e dopodiché viene considerato al di fuori della responsabilità del motore di calcolo.
    \end{itemize}











