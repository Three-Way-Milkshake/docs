Studio di Fattibilità & Si verifica lo \textsc{Studio di Fattibilità} redatto durante l'Avvio. & 2 & Verificatore
\tabularnewline 
Norme di Progetto & Vengono stabilite le norme di progetto\textsubscript{G} pianificando nel dettaglio i processi primari, i processi di sviluppo e i processi organizzativi. Il documento \textsc{Norme di Progetto} viene redatto. & 22 & Amministratore
\tabularnewline 
Piano di Progetto & Il Responsabile di Progetto redige il \textsc{Piano di Progetto} scandendo le fasi\textsubscript{G} e i periodi secondo cui si articolerà il lavoro. & 20 & Responsabile
\tabularnewline 
Analisi dei Requisiti & Uno studio approfondito del capitolato\textsubscript{G} ne individua i requisiti\textsubscript{G}: l'analisi si caratterizza da contatti frequenti con il proponente che fornirà supporto nella comprensione del problema. Viene completata la redazione dell'\textsc{Analisi dei Requisiti}. & 50 & Analista
\tabularnewline 
Piano di Qualifica & In questa attivita\textsubscript{G} si individuano i criteri che garantiscono la qualità del prodotto. Viene redatto il \textsc{Piano di Qualifica}. & 20 & Verificatore
\tabularnewline 
Glossario & Il \textsc{Glossario} conterrà i termini a cui si riterrà necessario dare definizione. & 3 & Responsabile
\tabularnewline 
Verifica dei documenti & Quest'attività si concentra nella settimana che precede la presentazione e ha l'obiettivo di verificare e certificare la qualità di tutti i documenti prodotti. & 23 & Verificatore
\tabularnewline 
Lettera di Presentazione & Avviene la stesura della lettera con cui il gruppo si candida alla Revisione dei Requisiti. & 1 & Responsabile
\tabularnewline 
\caption{Pianificazione preventiva - Analisi dei Requisiti - Periodo 1}