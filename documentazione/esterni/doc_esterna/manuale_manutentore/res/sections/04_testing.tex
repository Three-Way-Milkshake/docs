\section{Testing}
Questo capitolo ha lo scopo di indicare agli sviluppatori come controllare in che modo opera il codice e la sua sintassi. Vengono di seguito esposti gli strumenti utilizzati per effettuare i test di analisi statica e dinamica del nostro applicativo.

\subsection{Coveralls}
Per quanto riguarda l'attività di code coverage\textsubscript{G} è stato scelto di utilizzare \textit{Coveralls}. \textit{Coveralls} esegue revisioni automatiche al codice relative all'analisi statica attraverso \textit{GitHub Actions}.

\subsection{GitHub Actions}
Il servizio di Continuous Integration che è stato deciso di utilizzare è \textit{GitHub Actions}, fornito da \textit{GitHub}. \\\textit{GitHub Actions} permette di creare dei workflow, ovvero processi automatici, con l'obiettivo di automatizzare il ciclo di vita dello sviluppo del software.

% CI\textsubscript{A} e code coverage\textsubscript{G}, glossario



