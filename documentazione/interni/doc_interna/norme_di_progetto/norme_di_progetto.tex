\documentclass[a4paper]{article}

%Tutti gli usepackage vanno qui

\usepackage{geometry}
\usepackage[italian]{babel}
\usepackage[utf8]{inputenc}
\usepackage[T1]{fontenc}
\usepackage[normalem]{ulem}
\usepackage{tgschola}
%\usepackage{tgbonum}
\usepackage{tabularx}
\usepackage{longtable}
\usepackage{hyperref}
\usepackage{enumitem}
\usepackage[toc]{appendix}
\hypersetup{
	colorlinks=true,
	linkcolor=blue,
	filecolor=magenta,
	urlcolor=blue,
}
% Numerazione figure
\let\counterwithout\relax
\let\counterwithin\relax
\usepackage{chngcntr}

\counterwithin{table}{subsection}
\counterwithin{figure}{subsection}

\usepackage[bottom]{footmisc}
\usepackage{fancyhdr}
\setcounter{secnumdepth}{4}
\usepackage{amsmath, amssymb}
\usepackage{array}
\usepackage{graphicx}

\usepackage{ifthen}

%\usepackage{float}
\usepackage{layouts}
\usepackage{url}
\usepackage{comment}
\usepackage{float}
\usepackage{eurosym}

\usepackage{lastpage}
\usepackage{layouts}
\usepackage{float}
\usepackage{eurosym}

%Comandi di impaginazione uguale per tutti i documenti
\pagestyle{fancy}
\lhead{\includegraphics[scale=0.04]{../../../../latex/images/logoTWM.png}}
%Titolo del documento
\rhead{\doctitle{}}
%\rfoot{\thepage}
\cfoot{Pagina \thepage\ di \pageref{LastPage}}
\setlength{\headheight}{35pt}
\setcounter{tocdepth}{5}
\setcounter{secnumdepth}{5}
\renewcommand{\footrulewidth}{0.4pt}

% multirow per tabelle
\usepackage{multirow}

% Permette tabelle su più pagine
%\usepackage{longtable}


% colore di sfondo per le celle
\usepackage[table]{xcolor}

%COMANDI TABELLE
\newcommand{\rowcolorhead}{\rowcolor[HTML]{9b240a}} %intestazione
% check for missing commands
\newcommand{\headertitle}[1]{\textbf{\color{white}#1}} %titolo colonna
\definecolor{pari}{HTML}{FFDBCB}
\definecolor{dispari}{HTML}{F1F7FD}

% comandi glossario
\newcommand{\glo}{$_{G}$}
\newcommand{\glosp}{$_{G}$ }


%label custom
\makeatletter
\newcommand{\uclabel}[2]{%
	\protected@write \@auxout {}{\string \newlabel {#1}{{#2}{\thepage}{#2}{#1}{}} }%
	\hypertarget{#1}{#2}
}
\makeatother

%riportare pezzi di codice
\definecolor{codegray}{gray}{0.9}
\newcommand{\code}[1]{\colorbox{codegray}{\texttt{#1}}}



% Configurazione della pagina iniziale
\newcommand{\doctitle}{Verbale interno 15}
\newcommand{\docdate}{26 Febbraio 2021}
\newcommand{\rev}{1.0.0}
\newcommand{\stato}{Approvato}
\newcommand{\uso}{Interno}
\newcommand{\approv}{Tessari Andrea}
\newcommand{\red}{Crivellari Alberto}
\newcommand{\ver}{De Renzis Simone}
\newcommand{\dest}{Three Way Milkshake\\ Prof. Vardanega Tullio\\ Prof. Cardin Riccardo}
\newcommand{\describedoc}{Verbale del meeting del 2021-02-26 del gruppo Three Way Milkshake}
 % modifica questo file
\makeindex

\usepackage{hyperref}
\hypersetup{
    colorlinks=true,
    linkcolor=blue,
    urlcolor=blue,
    hyperfootnotes=false
}
\usepackage{verbatim}
\usepackage{multicol}



\newcommand{\group}{Three Way Milkshake }
\newcommand{\pparagraph}[1]{\paragraph{#1}\mbox{}\\}
\newcommand{\nop}[1]{#1}

\makeatletter
\renewcommand\subparagraph{%
    \@startsection {subparagraph}{5}{\z@ }{3.25ex \@plus 1ex
        \@minus .2ex}{-1em}{\normalfont \normalsize \bfseries }}%
\makeatother


\newcommand{\ssubparagraph}[1]{\subparagraph{#1}\mbox{}\\}
\newcommand{\hfoot}[1]{\footnote{\hyperref[ref]{#1}}}
\newcommand{\hfootiso}[1]{\hfoot{Standard ISO 12207 \S\ #1}}

\begin{document}
	\thispagestyle{empty}
\begin{titlepage}
	\begin{center}
		
		\includegraphics[scale = 0.17]{../../../../latex/images/logoTWM.png}\\[0.7cm]
		

		\noindent\rule{\textwidth}{1pt} \\[0.4cm]
		\Huge \textbf{\doctitle} \\[0.1cm]
		\ifthenelse{\equal{\docdate}{ }}{ }{ \huge \textbf{\docdate} \\[0.1cm] }
		
		\noindent\rule{\textwidth}{1pt}\\[0.7cm]
		
		\large \textbf{Three Way Milkshake - Progetto "PORTACS"} \\[0.4cm] 
                \texttt{threewaymilkshake@gmail.com} \\[0.4cm]
                
		
        
        
        \large

        \begin{tabular}{r|l}
                        \textbf{Versione} & \rev{} \\
                        \textbf{Stato} & \stato{} \\
                        \textbf{Uso} & \uso{} \\                         
                        \textbf{Approvazione} & \approv{} \\                      
                        \textbf{Redazione} & \red{} \\ 
                        \textbf{Verifica} &  \ver{} \\                         
                        \textbf{Destinatari} & \parbox[t]{5cm}{ \dest{} }
                \end{tabular} 
                \\[0.3cm]
                \large \textbf{Descrizione} \\ \describedoc{} 
               

	\end{center}
\end{titlepage}
	\pagebreak


	% Registro delle modifiche
	\section*{Registro delle modifiche}

\newcommand{\changelogTable}[1]{
	
	
	\renewcommand{\arraystretch}{1.5}
	\rowcolors{2}{pari}{dispari}
	\begin{longtable}{ 
			>{\centering}p{0.07\textwidth} 
			>{}p{0.21\textwidth}
			>{\centering}p{0.17\textwidth}
			>{\centering}p{0.13\textwidth} 
			>{\centering}p{0.17\textwidth} 
			>{\centering}p{0.13\textwidth} }
		\rowcolorhead
		\headertitle{Vers.} &
		\centering \headertitle{Descrizione} &	
		\headertitle{Redazione} &
		\headertitle{Data red.} & 
		\headertitle{Verifica} &
		\headertitle{Data ver.}
		\endfirsthead	
		\endhead
		
		#1
		
	\end{longtable}
	\vspace{-2em}
	
}


\newcommand{\approvingTable}[1]{ 
	
	
	\renewcommand{\arraystretch}{1.5}
	\rowcolors{2}{pari}{dispari}
	\begin{longtable}{ 
			>{\centering}p{0.07\textwidth} 
			>{\centering}p{0.415\textwidth}
			>{\centering}p{0.13\textwidth}
			>{\centering}p{0.322\textwidth}  }
		\rowcolorhead
		\headertitle{Vers.} &
		\centering \headertitle{Descrizione} &	
		\headertitle{Data appr.} &
		\headertitle{Approvazione}
		\endfirsthead	
		\endhead
		
		#1
		
	\end{longtable}
	\vspace{-2em}
	
}
	\approvingTable{
	1.0.0 & Approvazione del verbale & 2021-04-18 & Greggio Nicolò
}

\changelogTable{
	0.1.0 & Stesura e verifica del verbale & Crivellari Alberto & 2021-04-15 & De Renzis Simone & 2021-04-18
} % modifica questo file
	\pagebreak

	% indice
    {
        \hypersetup{linkcolor=black}
        \tableofcontents
        \pagebreak

        % indice delle figure
        \listoffigures
        \pagebreak

        % indice delle tabelle
        %\listoftables  non ci sono tabelle
        %\pagebreak
    }


	% contenuto del documento, ogni sezione in un file
	\section{Introduzione}




\subsection{Scopo del documento}
Lo scopo di questo documento è presentare tutte le informazioni necessarie al mantenimento e all'estensione del software PORTACS, mostrando nel dettaglio l'architettura del sistema e l'organizzazione del codice sorgente.\\
In questo documento saranno presentate le varie tecnologie usate, sia lato front end che back end, come anche le varie librerie e framework. Verrà inoltre mostrato il sistema di versionamento utilizzato e la Continuous Integration applicata.





\subsection{Scopo del prodotto}

Il capitolato\textsubscript{G} C5 propone un progetto\textsubscript{G} in cui viene richiesto lo sviluppo di un software per il monitoraggio in tempo reale di unità che si muovono in uno spazio definito. All'interno di questo spazio, creato dall’utente per riprodurre le caratteristiche di un ambiente reale, le unità dovranno essere in grado di circolare in autonomia, o sotto il controllo dell’utente, per raggiungere dei punti di interesse posti nella mappa.  La circolazione è sottoposta a vincoli di viabilità e ad ostacoli propri della topologia dell’ambiente, deve evitare le collisioni con le altre unità e prevedere la gestione di situazioni critiche nel traffico.




\subsection{Riferimenti}



\subsubsection{Normativi}

\begin{itemize}
	\item \textsc{Norme di progetto\textsubscript{G} v3.0.0 }: per qualsiasi convenzione sulla nomenclatura degli elementi presenti all’interno del documento;
	
	\item Regolamento progetto\textsubscript{G} didattico: \\ {\url{https://www.math.unipd.it/~tullio/IS-1/2020/Dispense/P1.pdf}};
	\item Model-View Patterns: \\ {\url{https://www.math.unipd.it/~rcardin/sweb/2020/L02.pdf}};
	\item SOLID Principles: \\ {\url{https://www.math.unipd.it/~rcardin/sweb/2020/L04.pdf}};
	\item Diagrammi delle classi: \\ {\url{https://www.math.unipd.it/~rcardin/swea/2021/Diagrammi delle Classi_4x4.pdf}};
	\item Diagrammi dei package: \\ {\url{https://www.math.unipd.it/~rcardin/swea/2021/Diagrammi dei Package_4x4.pdf}};
	\item Diagrammi di sequenza: \\ {\url{https://www.math.unipd.it/~rcardin/swea/2021/Diagrammi di Sequenza_4x4.pdf}};
	\item Design Pattern Creazionali: \\ {\url{https://www.math.unipd.it/~rcardin/swea/2021/Design Pattern Creazionali_4x4.pdf}};
	\item Design Pattern Strutturali: \\ {\url{https://www.math.unipd.it/~rcardin/swea/2021/Design Pattern Strutturali_4x4.pdf}};
	\item Design Pattern Comportamentali: \\ {\url{https://www.math.unipd.it/~rcardin/swea/2021/Design Pattern Comportamentali_4x4.pdf}}.
\end{itemize}



\subsubsection{Informativi}
\begin{itemize}
	\item \textsc{\href{https://github.com/Three-Way-Milkshake/docs/wiki/Glossario}{Glossario}}: per la definizione dei termini (pedice G) e degli acronimi (pedice A) evidenziati nel documento;
	\item Capitolato d'appalto C5-PORTACS: \\
{\url{https://www.math.unipd.it/~tullio/IS-1/2020/Progetto/C5.pdf}}
	\item Software Engineering - Iam Sommerville - $10^{th}$ Edition.
\end{itemize}
	\pagebreak

	\section{Processi primari}
\subsection{Fornitura}
\subsubsection{Scopo}
Lo scopo del processo di fornitura è di determinare

\subsection{Sviluppo}
\subsubsection{Scopo}
Lo scopo del processo di ...
    \pagebreak

    \section{Processi di Supporto}
\label{supporto}
\subsection{Documentazione}
    \subsubsection{Scopo}
        Lo scopo del processo di documentazione è regolamentare la creazione e la gestione dei documenti e fissare le modalità di stesura ed approvazione degli stessi.
    \subsubsection{Aspettative}
        Avere un approccio condiviso ed uniforme per la stesura e l'aggiornamento dei documenti all'interno del gruppo di lavoro è fondamentale per rendere la documentazione uno strumento costruttivo e di supporto, e non un mera formalità aggiuntiva.
        Inoltre fornire un aspetto uniforme attraverso tutti i documenti facilita qualunque lettore.
    \subsubsection{Descrizione}
        Il gruppo \group si doterà di due categorie di documentazione:
        \begin{itemize}
            \item \textbf{formale: }documenti interni o esterni che rispetteranno strettamente i vincoli descritti in seguito e che saranno interamente pubblici, realizzati in \LaTeX{} aderendo ad un template condiviso;
            \item \textbf{informale: }documenti interni che potranno svolgere diverse funzioni, tra cui:
            \begin{itemize}
                \item  raccolta appunti e ordini del giorno per riunioni;
                \item  raccolta argomenti delle discussioni delle riunioni, per tracciare l'evoluzione delle stesse e favorire la stesura dei verbali in seguito;
                \item  creazione di wiki per condivisione di materiale utile riguardo l'uso di tecnologie o strumenti a supporto di qualsiasi attività.
            \end{itemize}
            Questi documenti saranno realizzati sfruttando \href{https://www.atlassian.com/software/confluence}{Confluence} per garantire semplicità, accentramento e condivisione real-time.

        \end{itemize}
    \subsubsection{Ciclo di vita dei documenti}
    \label{ciclovitadoc}
        Ogni documento formale si redige ed incrementa tramite queste attività:
        \begin{itemize}
            \item \textbf{stesura: }la scrittura del documento in sé, riguarda sia la creazione di nuove parti che l'aggiornamento di queste. Uno o più redattori si occupano di ciò;
            \item \textbf{verifica: }eseguita da uno o più verificatori, necessariamente diversi dai redattori, consiste nel controllo della correttezza sintattica, semantica, grammaticale ed ortografica e della conformità del documento rispetto alle suo scopo. Nel caso in cui si rendano necessarie modifiche sostanziali, i verificatori notificheranno il responsabile che provvederà a riportare il documento in stesura e solleciterà i redattori affinché apportino le correzioni richieste;
            \item \textbf{approvazione: }quando i verificatori riporteranno la completa correttezza ed aderenza ai requisiti del documento, il responsabile provvederà all'approvazione finale ed al rilascio di una nuova versione dello stesso.
        \end{itemize}
        Adottando un approccio incrementale, queste attività possono ripetersi.
    \subsubsection{Struttura dei documenti formali}
        \pparagraph{Organizzazione in file e cartelle}
            Ogni documento, ha una sua cartella dedicata, all'interno della quale ci devono essere:
            \begin{itemize}
                \item \textbf{file principale del documento: }(nome\_documento.tex) che contiene l'impostazione della struttura, le dichiarazioni per importare i package \LaTeX{} aggiuntivi e le dichiarazioni di inclusione delle sezioni e del glossario;
                \item \textbf{cartella config: }contenente file di configurazione relativi al documento, che consentono l'impostazione del frontespizio e del registro delle modifiche;
                \item \textbf{cartella res: }per contenere le risorse del documento, a sua volta questa contiene:
                    \begin{itemize}
                        \item \textbf{cartella images: }per le eventuali immagini;
                        \item \textbf{cartella sections: }che contiene tutte le sezioni, ognuna in un file diverso, i quali seguono la convenzione di nomenclatura preceduta da un identificativo numerico progressivo ad indicare la posizione della stessa nel documento (e.g.: 01\_introduzione.tex, 02\_sezione1.tex)
                        %todo add link convenzione nomi
                    \end{itemize}
                \item \textbf{file glossario.txt: }dove definire tutti gli acronimi e le voci di glossario, che verranno poi processate \hyperref[glossario]{dall'automazione}.
            \end{itemize}
        \pparagraph{Frontespizio}
            Fornisce dele informazioni generali e di introduzione al documento, ed è così composto:
            \begin{itemize}
                \item logo esteso del gruppo;
                \item nome del documento;
                \item nome del gruppo e del progetto relativo al capitolato scelto;
                \item indirizzo email di riferimento del gruppo;
                \item versione del documento, secondo la \hyperref[versions]{convenzione};
                \item stato, in accordo con quanto descritto nel \hyperref[ciclovitadoc]{ciclo di vita dei documenti};
                \item elenco dei redattori;
                \item elenco dei verificatori;
                \item nome del responsabile che ha effettuato l'ultima approvazione
                \item destinatari del documento
                \item breve descrizione
            \end{itemize}
        \pparagraph{Registro delle modifiche}
            Raccoglie in forma tabellare tutte le modifiche e conseguentemente la storia del documento, così strutturate:
            \begin{itemize}
                \item versione del documento relativa alla modifica
                \item breve descrizione delle attività svolte
                \item data della modifica, secondo lo standard \href{https://www.iso.org/iso-8601-date-and-time-format.html}{ISO 8601}
                \item cognome e nome di chi ha apportato la modifica
                \item ruolo del modificatore rispetto al ciclo di vita del documento, può quindi essere: redattore, verificatore, responsabile.
            \end{itemize}
        \pparagraph{Indice}
            Riassume i contenuti del documento raccolti per sezioni ed intestazioni, indicando la pagina di inizio e collegando alla stessa
        \pparagraph{Elenchi delle figure e tabelle}
            Nel caso in cui un documento presenti una o più figure o tabelle, queste sezioni le indicheranno raccogliendole rispettivamente ed indicando la pagina di apparizione.
        \pparagraph{Sezione di introduzione}
            La prima sezione di contenuto di ogni documento formale è l'introduzione che si articola in:
            \begin{itemize}
                \item una sezione che descrive lo scopo del documento;
                \item due sezioni condivise fra tutti i documenti, non verbali, che illustrano rispettivamente scopo del prodotto del capitolato scelto e funzionamento di acronimi, voci di glossario e riferimenti ad altri documenti;
                \item una sezione che contiene tutti i riferimenti, di natura normativa o informativa.
            \end{itemize}
        \pparagraph{Ulteriori sezioni, appendici e glossari}
            Come indicato precedentemente ogni sezione di primo livello deve avere un suo file dedicato e si svilupperà in seguito all'introduzione. Al termine di tutte le sezioni vi può essere un'appendice mentre le ultime pagine elencano la lista degli acronimi e i termini di glossario utilizzati nel documento, con riferimenti alle pagine dove questi appaiono.

        \subsubsection{Verbali}
            I verbali possono essere interni, se relativi ad una riunione dei soli membri del gruppo, o esterni, se relativi ad un qualche tipo di incontro con persone esterne al gruppo. Come gli altri documenti formali, hanno un frontespizio ed un registro delle modifiche, poi si articolano in questo modo:
            \begin{itemize}
                \item informazioni generali
                    \subitem -- dettagli sull'incontro
                        \subsubitem -- luogo, data\footnote{Standard ISO 8601}, orari di inizio e fine e partecipanti
                    \subitem -- ordine del giorno;
                \item verbale della riunione
                    \subitem -- contiene tutte le sottosezioni necessarie a descrivere lo svolgimento dell'incontro;
                \item conclusioni
                    \subitem -- nel caso di verbale interno, una tabella di tracciamento delle decisioni;
                    \subitem -- nel caso di verbale esterno, un paragrafo riassuntivo.
            \end{itemize}
            Ogni verbale avrà come nome "verbale\_[tipo]\_[num]" dove tipo può essere interno o esterno, e num e il progressivo rispetto al tipo.
            \pparagraph{Tracciamento delle decisioni nei verbali interni}
                Ogni riga della tabella di tracciamento delle decisioni si compone di:
                \begin{itemize}
                    \item \textbf{codice: }indentificativo univoco della decisione, così formato: VI\_numRiunione.numDecisione (e.g.: VI\_2.3 indica la terza decisione presa durante la riunione interna 2);
                    \item \textbf{decisione: }breve riassunto che indichi in maniera chiara la decisione presa.
                \end{itemize}
        \subsubsection{Glossario e acronimi}
            Ogni documento ha una sua sezione dedicata al glossario e agli acronimi a fine documento, dove sono indicate anche le pagine di apparizione delle rispettive voci.
            \pparagraph{Definzione ed utilizzo delle voci}
                Gli acronimi ed i termini di glossario vanno definiti nel glossario.txt, istruzioni precise su come utilizzare correttamente questo documento si trovano nella \textsc{wiki how-to glossario}. Una volta che una voce è stata aggiunta, la si può usare liberamente in qualsiasi sezione tramite il suo nome, singolare o plurale nel caso di termine nel glossario, o abbreviazione nel caso di acronimo

            \pparagraph{Funzionamento}
                Ad ogni azione di push su un branch feature, un'automazione \href{https://docs.github.com/en/free-pro-team@latest/actions}{github action} si occupa di eseguire uno script bash, glossary\_builder.sh il quale:
                \begin{itemize}
                    \item controlla tutti i documenti che hanno un glossario.txt;
                    \item ne genera un glossario.tex;
                    \item sostituisce tutte le occorrenze delle voci di glossario e acronimi in ogni sezione con i rispettivi comandi \LaTeX{}
                    \item se ci sono stati cambiamenti, ricompila il glossario e ricompila il documento per ottenere un pdf aggiornato;
                    \item se ci sono stati dei documenti aggiornati, effettua commit e push.
                \end{itemize}
                Questo permette di avere sempre i file pdf dei documenti aggiornati e con le voci di glossario correttamente compilate e collegate, in maniera automatica e rapida. Si consiglia quindi ai membri di effettuare sempre una pull prima di apportare nuove modifiche ai documenti se si ha precedentemente eseguito una push, così da avere il documento aggioranto ed evitare problemi di conflitti e nel VCS.
        \subsubsection{Norme tipografiche}
            \pparagraph{Nomi di file e cartelle}
                Ogni file e cartella dovrà avere un nome:
                \begin{itemize}
                    \item che sia il più possibile conciso ed esplicativo;
                    \item composto di sole lettere minuscole, numeri, \_ e '-', ad eccetto del '.' per separare nome da estensione
                    \item che abbia alla fine un'estensione coerente col suo tipo
                    \item non contenere spazi o caratteri diversi da quelli indicati sopra, parole diverse si separano con '\_'
                \end{itemize}
            \pparagraph{Stile del testo}
                Ai seguenti stili si attribuisce una specifica funzione semantica:
                \begin{itemize}
                    \item \textbf{corsivo: }per denotare termini tecnici appartenenti ad una particolare tecnologia che si sta trattando;
                    \item \textbf{grassetto: }per evidenziare termini rilevanti o dei quali viene dato un significato esteso immediatamente in seguito, come questi in un elenco puntato;
                    \item \textbf{maiuscoletto: }...documenti...
                    \item \textbf{pedice: } ... G..A...
                \end{itemize}


\subsection{Versionamento} %TODO corretto averli qui o meglio in processi organizzativi?
    \subsubsection{Scopo}
    Lo scopo del processo di fornitura è di determinare
    \subsubsection{Aspettative}
    ...
    \subsubsection{Descrizione}
    ...
    \subsubsection{Attività}
    ...
    \paragraph{singole attività...}
    \subsubsection{Strumenti}
    ...

\subsection{Verifica}
\label{verifica}
    \subsubsection{Scopo}
    Lo scopo del processo di fornitura è di determinare
    \subsubsection{Aspettative}
    ...
    \subsubsection{Descrizione}
    ...
    \subsubsection{Attività}
    ...
    \paragraph{singole attività...}
    \subsubsection{Strumenti}
    ...

\subsection{Validazione}
    \subsubsection{Scopo}
    Lo scopo del processo di fornitura è di determinare
    \subsubsection{Aspettative}
    ...
    \subsubsection{Descrizione}
    ...
    \subsubsection{Attività}
    ...
    \paragraph{singole attività...}
    \subsubsection{Strumenti}
    ...
    \pagebreak

    \section{Processi Organizzativi}

% ruoli di progetto, gestione degli incontri, gestione degli strumenti di coordinamento e di versionamento, gestione dei rischi, gestione della formazione individuale

\subsection{Comunicazione}
\subsubsection{Scopo}
Lo scopo del processo di ...

\subsection{Riunioni}
\subsubsection{Scopo}
Lo scopo del processo di ...

\subsection{Ruoli di Progetto}
\subsubsection{Scopo}
Lo scopo del processo di ...

\subsection{Ambiente di Lavoro}
\subsubsection{Scopo}
Lo scopo del processo di ...
	\pagebreak
	\appendix
	\section{Standard di qualità}
\subsection{ISO/IEC 12207}
ISO/IEC 12207 è uno standard ISO per la gestione del ciclo di vita del software.\\
\begin{figure}[h!]
	\centering
	\includegraphics[scale=0.4]{res/images/ISO_12207.png}
	\caption{Processi del ciclo di vita del software, secondo lo standard ISO/IEC 25010:2011}
\end{figure}
Di tutti questi processi, elenchiamo quelli su cui ci siamo concentrati maggiormente.
\begin{itemize}
    \item Tra i \textbf{processi primari}:
    \begin{itemize}
        \item Sviluppo;
        \item Fornitura.
    \end{itemize}
    \item Tra i \textbf{processi di supporto}:
    \begin{itemize}
        \item Documentazione;
        \item Gestione della configurazione;
        \item Accertamento della qualità;
        \item Verifica;
        \item Validazione;
        \item Risoluzione dei problemi.
    \end{itemize}
    \item Tra i \textbf{processi organizzativi}:
    \begin{itemize}
        \item Gestione (dei processi, comunicazione e rischi);
        \item Infrastruttura.
    \end{itemize}
\end{itemize}
\subsection{ISO/IEC 25010:2011}
In questo standard troviamo la parte di qualità del software, sostituisce l'ISO/IEC 9126 dal 2011 in poi.\\
In particolare aggiunge il modello della qualità in uso del software.\\
\begin{enumerate}
	\item \textbf{Efficacia:} precisione e completezza con cui gli utenti raggiungono i risultati desiderati.
	\item \textbf{Efficienza:} risorse spese in relazione agli obiettivi raggiunti (e in relazione all'efficacia).
	\item \textbf{Soddisfazione:} soddisfazione dell'utente, relativo ai suoi bisogni soddisfatti dal software.\\Solitamente la soddisfazione dipende dal soddisfacimento di \textit{utilità}, \textit{fiducia nel software}, \textit{gradimento} e \textit{comfort}.
	\item \textbf{Libertà da rischi:} grado con cui il software mitiga i possibili rischi.\\In particolare:
		\begin{itemize}
			\item mitigazione rischi economici;
			\item mitigazione rischi di salute e sicurezza;
			\item mitigazione rischi ambientali.
		\end{itemize}
	\item \textbf{Context coverage:} grado con cui il software può essere usato con efficacia\textsubscript{G}, efficienza\textsubscript{G}, soddisfazione e libertà da rischi, in qualunque contesto.
\end{enumerate}
Consultare la sezione ISO/IEC 9126 per altre informazioni sulla qualità del software.
\subsection{ISO/IEC 9126}
ISO/IEC 9126 è uno standard internazionale per valutare la qualità del software.\\
Questo standard fornisce un modello di qualità e 3 tipologie di metriche, queste 4 sezioni vengono riportate di seguito.\\
\begin{figure}[h!]
	\centering
	\includegraphics[scale=0.15]{res/images/ISO_9126.png}
	\caption{Figura esplicativa del modello della qualità software esterna ed interna dello standard ISO/IEC 9126}
\end{figure}
\subsubsection{Metriche per la qualità interna}
Definisce metriche applicabili al codice sorgente non eseguibile. Idealmente, la qualità interna determina la qualità esterna.\\
Viene rilevata tramite \textbf{analisi statica}.
\subsubsection{Metriche per la qualità esterna}
Definisce metriche applicabili al software in esecuzione che ne misurano i comportamenti tramite test. Idealmente, la qualità esterna determina la qualità in uso.\\
Viene rilevata tramite \textbf{analisi dinamica}.
\subsubsection{Metriche per la qualità in uso}
Definisce metriche applicabili solo quando il prodotto è finito e utilizzato in condizioni reali.
\subsubsection{Modello della qualità del software}
\begin{enumerate}
	\item \textbf{Funzionalità:} il software deve fornire funzioni che soddisfino i bisogni emersi nell'\textsc{Analisi dei Requisiti}.\\In particolare il software deve avere le seguenti caratteristiche:
		\begin{itemize}
			\item Appropriatezza;
			\item Accuratezza;
			\item Interoperabilità;
			\item Sicurezza.
		\end{itemize}
	\item \textbf{Affidabilità:} il software deve mantenere un certo livello di prestazioni quando utilizzato in condizione specificate.\\In particolare il software deve avere le seguenti caratteristiche:
		\begin{itemize}
			\item Maturità;
			\item Robustezza;
			\item Recuperabilità.
		\end{itemize}
	\item \textbf{Efficienza:} il software deve eseguire le proprie funzioni con minimo tempo e consumo di risorse possibile.\\In particolare efficienza\textsubscript{G} nel tempo, con veloci tempi di risposta e nello spazio, con una appropriata quantità di risorse.
	\item \textbf{Usabilità:} il software deve essere comprensibile e poter essere studiato senza troppe difficoltà.\\In particolare il software deve avere le seguenti caratteristiche:
		\begin{itemize}
			\item Comprensibilità;
			\item Apprendibilità;
			\item Operabilità;
			\item Attrattiva.
		\end{itemize}
	\item \textbf{Manutenibilità:} il software deve potersi evolvere con modifiche, correzioni e adattamenti.\\In particolare il software deve avere le seguenti caratteristiche:
		\begin{itemize}
			\item Analizzabilità;
			\item Modificabilità;
			\item Stabilità;
			\item Testabilità.
		\end{itemize}
	\item \textbf{Portabilità:} il software deve poter essere trasferito da un ambiente hardware/software ad un altro seguendo le evoluzioni tecnologiche.\\In particolare il software deve avere le seguenti caratteristiche:
		\begin{itemize}
			\item Adattabilità;
			\item Installabilità;
			\item Conformità;
			\item Sostituibilità.
		\end{itemize}
\end{enumerate}


\end{document}
