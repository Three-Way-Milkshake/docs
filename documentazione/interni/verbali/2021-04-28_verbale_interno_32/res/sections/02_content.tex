\section{Verbale della riunione}
\subsection{Organizzazione e suddivisione slide RQ}

Sono stati individuati i temi da esporre alla presentazione e impostata la struttura della stessa. Per ogni blocco è stato individuato un incaricato alla realizzazione delle slide e alla loro esposizione.

\begin{itemize}
	\item \textbf{Introduzione} (Sofia)
	\begin{itemize}
		\item introduzione al dominio: una breve ricapitolazione del dominio del nostro prodotto;
		\item architettura ad alto livello del software: per dare una visione d'insieme;
		\item ruoli e funzionalità degli attori del sistema.
	\end{itemize}
	\item \textbf{Verifica} (Alberto)
	\begin{itemize}
		\item automazione dei test sul codice;
		\item automazione della formattazione dello stile del codice;
		\item automazione del calcolo del \textit{code coverage}.
	\end{itemize}
	\item \textbf{Requisiti soddisfatti} (Alberto)
	\begin{itemize}
		\item realizzazione di un istogramma che rappresenta i requisiti attualmente soddisfatti dal prodotto.
	\end{itemize}
	\item \textbf{Colloqui Product Baseline} (Simone)
	\begin{itemize}
		\item svolgimento del primo colloquio: argomenti trattati e criticità riscontrate;
		\item ragioni del fallimento;
		\item correzioni apportate: studio e applicazione attenta dei design pattern, adozione di un framework per la gestione delle dipendenze;
		\item la resipiscenza: la nuova esposizione ed il successo.
	\end{itemize}
	\item \textbf{Difficoltà incontrate e soluzioni} (Giada)
	\begin{itemize}
		\item impostazione di un'architettura soddisfacente;
		\item introduzione del framework Spring;
		\item librerie e strumenti di test;
		\item organizzazione e coordinazione per la suddivisione dei compiti di sviluppo;
		\item riduzione dei tempi di riunione per minimizzare la sincronizzazione del gruppo e massimizzare la parallelizzazione dei lavori.
	\end{itemize}
	\item \textbf{Documentazione} (Giada)
	\begin{itemize}
		\item stesura di: \textsc{Manuale Utente} e \textsc{Manuale Manutentore};
		\item incremento di: \textsc{Analisi dei Requisiti}, \textsc{Norme di Progetto}, \textsc{Piano di Qualifica} e \textsc{Piano di Progetto}.
	\end{itemize}
	\item \textbf{Rendicontazione di impegno e costi} (Andrea)
	\begin{itemize}
		\item consuntivi di periodo;
		\item preventivo a finire.
	\end{itemize}
	\item \textbf{Previsioni di completamento} (Andrea)
	\begin{itemize}
		\item finire test unità (raggiungere 75\% preferibile);
		\item realizzazione test di integrazione, test di sistema, test di accettazione;
		\item ottimizzazione delle performance;
		\item 14 maggio possibile data di consegna per la RA.
	\end{itemize}
	\item \textbf{Demo} (Nicolò)
	\begin{itemize}
		\item esposizione delle funzionalità base degli utenti;
		\item prova circolazione di 3 unità nella mappa.
	\end{itemize}

\subsection{Contattare il proponente per richiedere un incontro}

\`E stato deciso di contattare il proponente per fissare un incontro il prima possibile dopo la RQ al fine di sciogliere alcuni dubbi riguardanti il testing del prodotto e mostrare l'avanzamento tramite l'esposizione della demo.

\end{itemize}

