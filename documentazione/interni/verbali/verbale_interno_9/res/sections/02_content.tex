\section{Verbale della riunione}
\subsection{Refactoring changelog}
A seguito della decisione VI\_8.1, tutti i membri sono invitati ad eseguire al più presto il refactoring del registro delle modifiche dei documenti redatti seguendo le istruzioni della wiki "Refactoring del Registro delle Modifiche" su Confluence.

\subsection{Sistemazione del cruscotto}
Sono state evidenziate alcune correzioni da aggiungere al foglio di calcolo sul cruscotto:
\begin{itemize}
	\item bloccare range di formule e fogli (proteggi) dei grafici per evitare modifiche accidentali alla struttura;
	\item sistemare formato date secondo ISO 8601;
	\item completare la sincronizzazione i fogli "Sincronizzazione - Dati" e "Durata Riunioni", per quanto riguarda i valori di durata;
	\item su "Sincronizzazione - Dati" prevedere una sezione per il tracciamento dei tempi di verifica similmente a quanto accade per i tempi di realizzazione della task;
	\item aggiungere una metrica relativa allo scostamento tra data di completamento e verifica di un artefatto;
	\item collegare il consuntivo con dati effettivi derivanti dalle ore di ogni task.
\end{itemize}

\subsection{Organizzazione della verifica dei verbali}
Utilizzare il documento "Sequenza Stesura e Verifica verbali" nello spazio Verbali di Confluence per seguire l'ordine di stesura e verifica dei verbali. In particolare, terminata la scrittura, assegnare il verificatore secondo l'ordine stabilito.


\subsection{Rimozione del registro modifiche nei verbali}
\'E stato deciso di non rimuovere il registro modifiche nei verbali.

\subsection{Revisione del PdQ}
\begin{itemize}	
	\item rivedere la struttura del documento, riorganizzarne le sezioni per facilitare la leggibilità del documento (ad esempio, rimuovere paragrafi "Metriche" dove non servono);
	\item a fronte delle nuove metriche introdotte, definire con precisioni quali metriche siano da utilizzare, eliminare eventuali indici già presenti e stabilire quali metriche vengano tracciate nei vari periodi;
	\item modifica dello stile: rappresentare metriche in tabella, piuttosto che elenco puntato;
	\item aggiungere al cruscotto le metriche già valorizzate, con a seguire un paragrafo vuoto per metriche non valorizzate.
\end{itemize}

\subsection{Divisione tecnologie da studiare}
Per approcciarsi alla realizzazione del Proof Of Concept, il gruppo ha deciso di esplorare le tecnologie necessarie alla realizzazione del software: è stata operata la seguente suddivisione:
\begin{itemize}
	\item Angular (per interfaccia e collegamenti): Zuccolo Giada e Chiarello Sofia;
	\item NodeJs (per comunicazione tra unità e server): Tessari Andrea e Crivellari Alberto;
	\item Java (per il server centrale): Greggio Nicolò e De Renzis Simone;
	\item JSON: tutti.
\end{itemize}