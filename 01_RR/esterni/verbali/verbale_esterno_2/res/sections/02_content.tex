\section{Verbale della riunione}

\subsection{Sensori ed unità}
\begin{itemize}
    \item Nel contesto reale si rende naturalmente necessaria una parte di sensoristica
    \subitem -- per gli scopi di questo progetto si può tralasciare;
    \item Il sistema ha tutte le informazioni di cui ha bisogno
    \subitem -- il server centrale si occupa di tutto;
    \item Non è necessario simulare i sensori.
\end{itemize}

\subsection{Tecnologie e formazione}
Sono state confermate le tecnologie discusse in confronti precedenti con il proponente.\\Sono emerse diverse fonti da sfruttare.
    \subsubsection{Java}
        \begin{itemize}
            \item Corso html.it;
            \item Si possono approfondire alcune novità delle versioni > 11
                \subitem -- tuttavia versione 8 va bene.
        \end{itemize}

    \subsubsection{Node.js}
        \begin{itemize}
            \item Documentazione ufficiale;
            \item Corsi su html.it.
        \end{itemize}

    \subsubsection{Frontend}
        \begin{itemize}
            \item Angular o AngularJS
            \item Guide e documentazioni ufficiali su corrispondenti siti;
            \item TypeScript vs JavaScript
                \subitem -- dipende dal tempo che ci vogliamo dedicare.
        \end{itemize}

    \subsubsection{Design pattern}
        \begin{itemize}
            \item Nessun vincolo su design pattern;
            \item Potrebbe essere comodo il concetto di factory
                \subitem -- eventualmente bypassato da librerie;
                \subitem -- ma in generale come pattern più pulito è più pratico;
                \subitem -- adatto per sistemi composti da tanti piccoli componenti.
        \end{itemize}

    \subsubsection{Salvataggio dati}
        \begin{itemize}
            \item Non è per forza necessaria una base di dati;
            \item La configurazione iniziale può essere ricevuta da un file di testo;
            \item Consigliate strutture noSQL comunque;
            \item Semplici JSON possono essere sufficienti;
            \item Non è richiesto il tracciamento delle operazioni.
        \end{itemize}

    \subsubsection{Approccio al multithreading}
        \begin{itemize}
            \item L'introduzione di framework in questo ambito può portare ad una elevata complessità;
            \item Preferire come approccio l'uso di thread puri
                \subitem -- timer task, runnable...;
            \item Solo in caso di difficoltà valutare e discutere l'introduzione di un framework per il multithreading;
            \item Limitare l'uso in generale di librerie di terze parti.
        \end{itemize}

    \subsubsection{Simulazione muletti}
        \begin{itemize}
            \item Carta bianca, si può usare tutto ciò che può dare una mano:
            \item In questo contesto anche diverse librerie, anche se ne sfruttiamo solo  una piccola percentuale.
        \end{itemize}

    \subsubsection{Controllo diagrammi attività server}
        \begin{itemize}
            \item Bene in generale;
            \item Quando si introdurrà il concetto di tempo e timer task utilizzare apposita icona start con orologio.
        \end{itemize}

    \subsubsection{Possibili requisiti qualitativi}
        \begin{itemize}
            \item In genere 2 famiglie:
                \begin{itemize}
                    \item bontà software
                        \begin{itemize}
                            \item indicatori comuni sul sw;
                            \item non serve stare dentro certi limiti, basta misurarli;
                            \item editor moderni fanno quasi tutto da soli;
                        \end{itemize}
                    \item applicativi
                        \begin{itemize}
                            \item requisito sensato può essere: $\frac{\text{tempo risposta}}{\text{numero dispositivi}}$;
                            \item e.g.: entro 1 sec fino a 50, entro 2 fino a 100...;
                            \item sistema qualità basato su tempi di risposta e numero di fail, questi ultimi in ogni caso non devono mai bloccare l'applicativo (input received, no output producecd);
                        \end{itemize}
                \end{itemize}
            \item Decisioni che spettano al gruppo
                \begin{itemize}
                    \item i muletti vanno ad una certa velocità;
                    \item considerando tempi di reazione e velocità, conseguirà uno spazio di spostamento;
                    \item stabilire in quanto spazio massimo si vuole l'arresto;
                    \item ricavare tempi di risposta che si dovranno attendere dal sistema;
                    \item discorso analogo per il numero di fail.
                \end{itemize}
        \end{itemize}

    \subsubsection{Strategie di progettazione}
        \begin{itemize}
            \item Inizialmente si consiglia di procedere in parallelo:
                \subitem -- 3 progettazione;
                \subitem -- 3 studio tecnologie ed approccio codice;
            \item Questo da vantaggi;
            \item Valutando eventuali scostamenti, riequilibrare le partizioni.
        \end{itemize}








