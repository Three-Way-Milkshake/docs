\section{Verbale della riunione}

\subsection{Discussione su architettura}
Vengono discussi alcuni modelli di design patter da poter utilizzare. E` emerso quanto segue:
\begin{itemize}
\item Observer\\in quanto il segnale deve essere gestito a seconda della sua tipologia;
\item Layer: 
\\Il proponente consiglia di non usare più di 5 layer.\\Più precisamente vengono indicati dal proponente:
	\begin{itemize}
	\item DAO (Data Access Object);
	\item Business Logic;
	\item Strato servizi che riceve segnali HTTP;
	\item Layer di accesso al DB sono poco utilizzati;
	\end{itemize}
\item il proponente ha consigliato al gruppo di visualizzare il Singleton Pattern;
\item il proponente ha consigliato al gruppo di visualizzare il Factory Pattern;
\item no microservizi poichè si aggiunge tempo di latenza, quindi sono inutili ed è più difficile da creare. Il proponente consiglia un sistema più simile a un monolite.
\end{itemize}

\subsection{Docker}
Secondo il proponente il client va fatto containerizzando istanze di Node e Angular. 

\subsection{Sicurezza}
Discussione su come fare per la comunicazione sicura.
\begin{itemize}
	\item Si consiglia Java socket in HTTPS;
	\item procedura:\\
	lato java → collegarsi ad un'api rest in http → fatto in automatico → serve il certificato da chi viene invocato → scambio chiavi → canale crittografato;
	\item utilizzare Secure Socket Layer;
	\item il proponente consiglia:\\\href{https://docs.oracle.com/javase/10/security/sample-code-illustrating-secure-socket-connection-client-and-server.htm\#JSSEC-GUID-AA1C27A1-2CA8-4309-B281-D6199F60E666}{https://docs.oracle.com/javase/10/security/sample-code-illustrating-secure-socket-connection-client-and-server.htm\#JSSEC-GUID-AA1C27A1-2CA8-4309-B281-D6199F60E666};	
\end{itemize}

\subsection{Confronto sull'algoritmo per la rilevazione delle collisioni}
Mostrato al proponente la proposta di algoritmo prodotta per il rilevamento delle collisioni. Il proponente ha appoggiato la proposta e ha consigliato di fare autonomamente l'algoritmo senza cercare soluzioni accademiche ritenute troppo complicate.

\subsection{PoC}
E' stato mostrato al proponente il PoC e il suo funzionamento. 