\section{Verbale della riunione}

\subsection{Meccanismi di login}
    \begin{itemize}
        \item adottare login con user e password per amministratori e responsabili;
        \item muletti come entità diventano nuovi attori per le operazioni che si possono svolgere a bordo
            \subitem -- non serve login operatore;
            \subitem -- al momento della connessione scambio token con server per identificazione;
            \subitem utente a bordo è informazione in più
                \subsubitem * fa parte della sezione facoltativa "pedoni" solo quando scende dal mezzo;
            \subitem il muletto comunica con il server indipendentemente;
        \item non serve appoggiarsi a servizi esterni, può essere realizzato internamente dal gruppo, seguendo buone pratiche di sicurezza:
            \begin{itemize}
                \item comunicazioni con protocollo https
                    \subitem * certificato autogenerato "self signed" va bene;
                    \subitem * non serve acquistare/generare;
                \item salvare hash delle password
                    \subitem * potenzialmente anche salted hash;
                \item se si adottano sessioni queste devono avere scadenza.
            \end{itemize}
    \end{itemize}

\subsection{Interfaccia di guida}
    \begin{itemize}
        \item guida manuale rimane obbligatoria
            \subitem -- unico modo di testare capacità di adattamento del sistema;
        \item il controllo delle unità non deve per forza avvenire su dispositivi/schermi/interfacce dedicate ma può essere centralizzato in unico pannello tramite il quale l'amministratore può far intraprendere qualunque azione ad ogni unità
    \end{itemize}

\subsection{Riclassificazione dei Requisiti}
    Durante il confronto sono emersi i seguenti punti notabili riguardo le differenze tra requisiti di vincolo e funzionali:
    \begin{itemize}
        \item \textbf{vincolo: }
            \begin{itemize}
                \item tutto ciò che a sistema tratto come condizione iniziale:

                \item sistema in cui lavoro è rigido, non c'è possibilità di ridiscutere; questi punti fermi

                \item eg: tutto ciò che riguarda mappa e spazi;
            \end{itemize}
        \item \textbf{funzionale: }
            \begin{itemize}
                \item comportamento del sistema;
                \item ciò che il software fa, su cui posso quindi lavorare e migliorare.
            \end{itemize}
    \end{itemize}
    I requisiti di vincolo sono dunque stati rivisti e classificati come segue (\textbf{F: }funzionale, \textbf{V: }vincolo):
    \begin{enumerate}
        \item F;
        \item F;
        \item F;
        \item tutto quello che si muove all’interno del … è censita dal sistema
            \subitem (a) \; non esiste unità non riconosciuta e controllata dal sistema;
            \subitem (b) \; non c'è nulla che il sistema non conosca e non riceva dati;
        \item F;
        \item V;
        \item V;
        \item F, compito del nostro lavoro , derivato da 6 e 7, renderlo valido;
        \item F (sparisce se 10..12 vengono fatte) (10..12 sono funzionalità 2,3,4 della 9);
        \item F;
        \item F;
        \item F;
        \item F;
        \item F;
        \item F;
        \item F;
        \item F;
        \item F;
        \item F;
        \item F;
        \item F;
        \item F;
        \item F;
        \item F;
        \item F;
        \item F;
        \item F;
        \item F;
        \item F;
        \item F;
        \item F;
        \item F;
        \item F;
        \item F;
        \item F;

    \end{enumerate}