\documentclass[a4paper]{article}

%Tutti gli usepackage vanno qui

\usepackage{geometry}
\usepackage[italian]{babel}
\usepackage[utf8]{inputenc}
\usepackage[T1]{fontenc}
\usepackage[normalem]{ulem}
\usepackage{tgschola}
%\usepackage{tgbonum}
\usepackage{tabularx}
\usepackage{longtable}
\usepackage{hyperref}
\usepackage{enumitem}
\usepackage[toc]{appendix}
\hypersetup{
	colorlinks=true,
	linkcolor=blue,
	filecolor=magenta,
	urlcolor=blue,
}
% Numerazione figure
\let\counterwithout\relax
\let\counterwithin\relax
\usepackage{chngcntr}

\counterwithin{table}{subsection}
\counterwithin{figure}{subsection}

\usepackage[bottom]{footmisc}
\usepackage{fancyhdr}
\setcounter{secnumdepth}{4}
\usepackage{amsmath, amssymb}
\usepackage{array}
\usepackage{graphicx}

\usepackage{ifthen}

%\usepackage{float}
\usepackage{layouts}
\usepackage{url}
\usepackage{comment}
\usepackage{float}
\usepackage{eurosym}

\usepackage{lastpage}
\usepackage{layouts}
\usepackage{float}
\usepackage{eurosym}

%Comandi di impaginazione uguale per tutti i documenti
\pagestyle{fancy}
\lhead{\includegraphics[scale=0.04]{../../../../latex/images/logoTWM.png}}
%Titolo del documento
\rhead{\doctitle{}}
%\rfoot{\thepage}
\cfoot{Pagina \thepage\ di \pageref{LastPage}}
\setlength{\headheight}{35pt}
\setcounter{tocdepth}{5}
\setcounter{secnumdepth}{5}
\renewcommand{\footrulewidth}{0.4pt}

% multirow per tabelle
\usepackage{multirow}

% Permette tabelle su più pagine
%\usepackage{longtable}


% colore di sfondo per le celle
\usepackage[table]{xcolor}

%COMANDI TABELLE
\newcommand{\rowcolorhead}{\rowcolor[HTML]{9b240a}} %intestazione
% check for missing commands
\newcommand{\headertitle}[1]{\textbf{\color{white}#1}} %titolo colonna
\definecolor{pari}{HTML}{FFDBCB}
\definecolor{dispari}{HTML}{F1F7FD}

% comandi glossario
\newcommand{\glo}{$_{G}$}
\newcommand{\glosp}{$_{G}$ }


%label custom
\makeatletter
\newcommand{\uclabel}[2]{%
	\protected@write \@auxout {}{\string \newlabel {#1}{{#2}{\thepage}{#2}{#1}{}} }%
	\hypertarget{#1}{#2}
}
\makeatother

%riportare pezzi di codice
\definecolor{codegray}{gray}{0.9}
\newcommand{\code}[1]{\colorbox{codegray}{\texttt{#1}}}



% Configurazione della pagina iniziale
\newcommand{\doctitle}{Verbale interno 15}
\newcommand{\docdate}{26 Febbraio 2021}
\newcommand{\rev}{1.0.0}
\newcommand{\stato}{Approvato}
\newcommand{\uso}{Interno}
\newcommand{\approv}{Tessari Andrea}
\newcommand{\red}{Crivellari Alberto}
\newcommand{\ver}{De Renzis Simone}
\newcommand{\dest}{Three Way Milkshake\\ Prof. Vardanega Tullio\\ Prof. Cardin Riccardo}
\newcommand{\describedoc}{Verbale del meeting del 2021-02-26 del gruppo Three Way Milkshake}
 % modifica questo file
\makeindex

\usepackage{hyperref}
\hypersetup{
    colorlinks=true,
    linkcolor=blue,
    urlcolor=blue,
    hyperfootnotes=false
}
\usepackage{multicol}
\usepackage{pgfplots}
\usepackage{verbatim}
\usepackage{pgf-pie}
\usepackage{ragged2e}
\setlength{\columnseprule}{1pt}


\begin{document}
	\thispagestyle{empty}
\begin{titlepage}
	\begin{center}
		
		\includegraphics[scale = 0.17]{../../../../latex/images/logoTWM.png}\\[0.7cm]
		

		\noindent\rule{\textwidth}{1pt} \\[0.4cm]
		\Huge \textbf{\doctitle} \\[0.1cm]
		\ifthenelse{\equal{\docdate}{ }}{ }{ \huge \textbf{\docdate} \\[0.1cm] }
		
		\noindent\rule{\textwidth}{1pt}\\[0.7cm]
		
		\large \textbf{Three Way Milkshake - Progetto "PORTACS"} \\[0.4cm] 
                \texttt{threewaymilkshake@gmail.com} \\[0.4cm]
                
		
        
        
        \large

        \begin{tabular}{r|l}
                        \textbf{Versione} & \rev{} \\
                        \textbf{Stato} & \stato{} \\
                        \textbf{Uso} & \uso{} \\                         
                        \textbf{Approvazione} & \approv{} \\                      
                        \textbf{Redazione} & \red{} \\ 
                        \textbf{Verifica} &  \ver{} \\                         
                        \textbf{Destinatari} & \parbox[t]{5cm}{ \dest{} }
                \end{tabular} 
                \\[0.3cm]
                \large \textbf{Descrizione} \\ \describedoc{} 
               

	\end{center}
\end{titlepage}
	\pagebreak

	% Registro delle modifiche
	\section*{Registro delle modifiche}

\newcommand{\changelogTable}[1]{
	
	
	\renewcommand{\arraystretch}{1.5}
	\rowcolors{2}{pari}{dispari}
	\begin{longtable}{ 
			>{\centering}p{0.07\textwidth} 
			>{}p{0.21\textwidth}
			>{\centering}p{0.17\textwidth}
			>{\centering}p{0.13\textwidth} 
			>{\centering}p{0.17\textwidth} 
			>{\centering}p{0.13\textwidth} }
		\rowcolorhead
		\headertitle{Vers.} &
		\centering \headertitle{Descrizione} &	
		\headertitle{Redazione} &
		\headertitle{Data red.} & 
		\headertitle{Verifica} &
		\headertitle{Data ver.}
		\endfirsthead	
		\endhead
		
		#1
		
	\end{longtable}
	\vspace{-2em}
	
}


\newcommand{\approvingTable}[1]{ 
	
	
	\renewcommand{\arraystretch}{1.5}
	\rowcolors{2}{pari}{dispari}
	\begin{longtable}{ 
			>{\centering}p{0.07\textwidth} 
			>{\centering}p{0.415\textwidth}
			>{\centering}p{0.13\textwidth}
			>{\centering}p{0.322\textwidth}  }
		\rowcolorhead
		\headertitle{Vers.} &
		\centering \headertitle{Descrizione} &	
		\headertitle{Data appr.} &
		\headertitle{Approvazione}
		\endfirsthead	
		\endhead
		
		#1
		
	\end{longtable}
	\vspace{-2em}
	
}
	\approvingTable{
	1.0.0 & Approvazione del verbale & 2021-04-18 & Greggio Nicolò
}

\changelogTable{
	0.1.0 & Stesura e verifica del verbale & Crivellari Alberto & 2021-04-15 & De Renzis Simone & 2021-04-18
} % modifica questo file
	%\end{longtable}
	\pagebreak

	% indice
	{
        \hypersetup{linkcolor=black}
        \tableofcontents
        \pagebreak

        % indice delle figure
        \listoffigures
        \pagebreak

        % indice delle tabelle
        \listoftables
        \pagebreak
    }

	\newcommand{\contabilitaTable}[1]{

\begin{table}[H]
	\begin{center}
		\begin{tabular}{c
				!{\color[HTML]{9b240a}\vrule width 1pt}
				cccccc
				!{\color[HTML]{9b240a}\vrule width 1pt}	
				c}
			\rowcolorhead
			\headertitle{Nome} & \headertitle{R} & \headertitle{V} & \headertitle{An} & \headertitle{Am} & \headertitle{Pr} & \headertitle{Pt} & \headertitle{Tot} \\	
			#1
			\end{center}
			\end{table}	
}


\newcommand{\smallPreventivoTable}[1]{
	
	\begin{table}[H]
		\begin{center}
			\begin{tabular}{c
					!{\color[HTML]{9b240a}\vrule width 1pt}
					cccccc
					!{\color[HTML]{9b240a}\vrule width 1pt}	
					c}
				\rowcolorhead
				\headertitle{Ruolo} & \headertitle{Tempo (ore)} & \headertitle{Costo (euro)} \\
				#1
			\end{center}
		\end{table}	
	}


\newcommand{\planningTable}[1]{
	
	
	\renewcommand{\arraystretch}{1.5}
	\rowcolors{2}{pari}{dispari}
	\begin{longtable}{ 
			>{}p{0.25\textwidth} 
			>{}p{0.42\textwidth}
			>{\centering}p{0.05\textwidth}
			>{\centering}p{0.17\textwidth} }
		\rowcolorhead
		\headertitle{Attività} &
		\headertitle{Descrizione} &	
		\headertitle{Ore} &
		\headertitle{Ruolo} 
		\endfirsthead	
		\endhead
		#1		
	\end{longtable}
	
} % file con template tabelle con macro
	

	% contenuto del documento, ogni sezione in un file
	\section{Introduzione}




\subsection{Scopo del documento}
Lo scopo di questo documento è presentare tutte le informazioni necessarie al mantenimento e all'estensione del software PORTACS, mostrando nel dettaglio l'architettura del sistema e l'organizzazione del codice sorgente.\\
In questo documento saranno presentate le varie tecnologie usate, sia lato front end che back end, come anche le varie librerie e framework. Verrà inoltre mostrato il sistema di versionamento utilizzato e la Continuous Integration applicata.





\subsection{Scopo del prodotto}

Il capitolato\textsubscript{G} C5 propone un progetto\textsubscript{G} in cui viene richiesto lo sviluppo di un software per il monitoraggio in tempo reale di unità che si muovono in uno spazio definito. All'interno di questo spazio, creato dall’utente per riprodurre le caratteristiche di un ambiente reale, le unità dovranno essere in grado di circolare in autonomia, o sotto il controllo dell’utente, per raggiungere dei punti di interesse posti nella mappa.  La circolazione è sottoposta a vincoli di viabilità e ad ostacoli propri della topologia dell’ambiente, deve evitare le collisioni con le altre unità e prevedere la gestione di situazioni critiche nel traffico.




\subsection{Riferimenti}



\subsubsection{Normativi}

\begin{itemize}
	\item \textsc{Norme di progetto\textsubscript{G} v3.0.0 }: per qualsiasi convenzione sulla nomenclatura degli elementi presenti all’interno del documento;
	
	\item Regolamento progetto\textsubscript{G} didattico: \\ {\url{https://www.math.unipd.it/~tullio/IS-1/2020/Dispense/P1.pdf}};
	\item Model-View Patterns: \\ {\url{https://www.math.unipd.it/~rcardin/sweb/2020/L02.pdf}};
	\item SOLID Principles: \\ {\url{https://www.math.unipd.it/~rcardin/sweb/2020/L04.pdf}};
	\item Diagrammi delle classi: \\ {\url{https://www.math.unipd.it/~rcardin/swea/2021/Diagrammi delle Classi_4x4.pdf}};
	\item Diagrammi dei package: \\ {\url{https://www.math.unipd.it/~rcardin/swea/2021/Diagrammi dei Package_4x4.pdf}};
	\item Diagrammi di sequenza: \\ {\url{https://www.math.unipd.it/~rcardin/swea/2021/Diagrammi di Sequenza_4x4.pdf}};
	\item Design Pattern Creazionali: \\ {\url{https://www.math.unipd.it/~rcardin/swea/2021/Design Pattern Creazionali_4x4.pdf}};
	\item Design Pattern Strutturali: \\ {\url{https://www.math.unipd.it/~rcardin/swea/2021/Design Pattern Strutturali_4x4.pdf}};
	\item Design Pattern Comportamentali: \\ {\url{https://www.math.unipd.it/~rcardin/swea/2021/Design Pattern Comportamentali_4x4.pdf}}.
\end{itemize}



\subsubsection{Informativi}
\begin{itemize}
	\item \textsc{\href{https://github.com/Three-Way-Milkshake/docs/wiki/Glossario}{Glossario}}: per la definizione dei termini (pedice G) e degli acronimi (pedice A) evidenziati nel documento;
	\item Capitolato d'appalto C5-PORTACS: \\
{\url{https://www.math.unipd.it/~tullio/IS-1/2020/Progetto/C5.pdf}}
	\item Software Engineering - Iam Sommerville - $10^{th}$ Edition.
\end{itemize}
	\pagebreak

	\section{Tecnologie e librerie}

Nelle sezioni che seguono, vengono elencate le tecnologie e le librerie interessate dallo sviluppo del software. Per ognuna, viene presentata una breve descrizione e spiegato il suo impiego nel contesto del software. Dove necessario, viene fornito un collegamento per il download e l'installazione delle risorse necessarie per lo sviluppo e manutenzione del progetto\textsubscript{G}.





















	\pagebreak

	
	\section{Setup}
PORTACS viene distribuito tramite container Docker, per cui i dispositivi sui quali dovrà eseguire avranno minimi requisiti software. È possibile utilizzare i container anche per la fase\textsubscript{G} di sviluppo, altrimenti si possono scaricare ed utilizzare gli strumenti descritti in \ref{tecnologie} per l'esecuzione diretta in locale. PORTACS\textsubscript{A} si divide su tre immagini Docker:
\begin{enumerate}
    \item server;
    \item client muletto (forklift);
    \item client utente (user).
\end{enumerate}


\subsection{Requisiti di sistema}
Sotto elencati saranno descritti i requisiti minimi del sistema per un corretto funzionamento del software PORTACS\textsubscript{A}.


\subsubsection{Requisiti Hardware}
\begin{itemize}
	\item Client → unità:
\begin{itemize}
	\item CPU\textsubscript{A} dual-core o maggiore;
	\item memoria Ram >= 4GB;
	\item connessione con bassi tempi di risposta.
\end{itemize}
	\item Client → admin o responsabile:
\begin{itemize}
	\item CPU\textsubscript{A} dual-core o maggiore;
	\item memoria Ram >= 4GB;
	\item connessione con bassi tempi di risposta.
\end{itemize}
	\item Server:
\begin{itemize}
	\item CPU\textsubscript{A} quad-core o maggiore;
	\item memoria Ram >= 8GB;
	\item connessione con bassi tempi di risposta.
\end{itemize}
\end{itemize}

\subsubsection{Requisiti Software}
    \pparagraph{Esecuzione}
    \begin{itemize}
        \item Docker (v19.03.*);
        \item Google Chrome (v90).
    \end{itemize}

    \pparagraph{Sviluppo}
    Vedi \S\ \ref{tecnologie}

    \begin{comment}
    	\item Client:
    \begin{itemize}
    	\item Docker.
    	%\item Chrome Versione 90.0.4430.85 (Build ufficiale) (a 64 bit);
    	%\item Node.js;
    	%\item Angular;

    	%\item Windows 10.
    \end{itemize}
    	\item Server:
    \begin{itemize}
    	\item Docker.
    	%\item Java;
    	%\item gradle;

    	%\item Windows 10.
    \end{itemize}
    \end{comment}







\subsection{Installazione ed avvio degli applicativi}
    \subsubsection{Server}
    Nella macchina da utilizzare come server andrà scaricata l'immagine di portacs-server con la versione desiderata \href{https://hub.docker.com/r/threewaymilkshake}{docker hub di \group}.
    Per l'avvio, in ambiente Linux e MacOS:
    \begin{itemize}
        \item predisporre una cartella che contenga un file \texttt{config.txt} ed una cartella \texttt{resources} che manterrà la persistenza;
        \item posizionarsi su tale cartella;
        \item eseguire:
    \begin{verbatim}
    docker run -v $(pwd)/resources:/resources \
        --env-file config.txt threewaymilkshake/portacs-server
    \end{verbatim}
    \end{itemize}
    All'interno del file \texttt{config.txt} si possono specificare (come coppie key=value) delle configurazioni diverse da quelle di default per quanto riguardo percorso della persistenza e porta sulla quale il server dovrà esporre il Server Socket per il collegamento. Se non si desiderano modificare i valori di default il file e la relativa parte nel comando possono essere omessi.

    \subsubsection{Client}
    Come per il server, scaricare l'immagine dal docker hub di \group{} relativa al client voluto (forklift o user). Dopodiché sul dispositivo che dovrà fungere da client andranno impostata la configurazione in un file \texttt{config.txt}:
    \begin{itemize}
        \item per i client \textit{user} sarà sufficiente specificare l'indirizzo IP della macchina server così: \texttt{SERVER\_ADDR=ip};
        \item per i client \texttt{forklift}, oltre all'indirizzo come per gli utenti, bisognerà aggiungere altre 2 righe per la configurazione di ogni muletto:
        \begin{itemize}
            \item \texttt{ID=id del muletto};
            \item \texttt{TOKEN=token del muletto}.
        \end{itemize}
        Queste informazioni sono a disposizione degli admin.
    \end{itemize}

    Dopodiché, in entrambi i casi, sarà sufficiente eseguire:
    \begin{verbatim}
        docker run --env-file config.txt threewaymilshake/portacs-client-<type>
    \end{verbatim}
    con \texttt{<type>} tra \texttt{forklift} o \texttt{user}.



	\pagebreak
	

	\section{Testing}


\subsection{JUnit}



\subsection{Libreria test frontend}
	\newpage


	\section{Architettura del sistema}


Qui si potrebbe mettere uno schema simile a quello della slide iniziale per evidenziare l'architettura client-server



\subsection{Server}

Dire che è 3 layer architecture

Qui potrebbe esserci il diagramma minimale complessivo


\subsubsection{Diagramma delle classi}


\paragraph{Persistance layer}


\paragraph{Business layer}

\subparagraph{Mappa}

\subparagraph{Clients}

\subparagraph{Tasks}

\subparagraph{Collisioni}


\paragraph{Communication layer}



\subsection{Client}


\subsection{Comunicazione}

\subsubsection{Diagrammi di sequenza}

\subsubsection{Protocollo di comunicazione}



	\newpage


	\section{Estendere PORTACS}
In questa sezione verranno riportate tutte quelle informazioni utili ad una semplice e corretta estensione del prodotto PORTACS. Possono essere estensioni legate sia a nuovi algoritmi e framework, più efficaci o efficienti, che a nuove categorie di un sotto-sistemi di PORTACS.















	\newpage


\end{document}
