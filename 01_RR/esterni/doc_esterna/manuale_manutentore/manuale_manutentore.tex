\documentclass[a4paper]{article}

%Tutti gli usepackage vanno qui

\usepackage{geometry}
\usepackage[italian]{babel}
\usepackage[utf8]{inputenc}
\usepackage[T1]{fontenc}
\usepackage[normalem]{ulem}
\usepackage{tgschola}
%\usepackage{tgbonum}
\usepackage{tabularx}
\usepackage{longtable}
\usepackage{hyperref}
\usepackage{enumitem}
\usepackage[toc]{appendix}
\hypersetup{
	colorlinks=true,
	linkcolor=blue,
	filecolor=magenta,
	urlcolor=blue,
}
% Numerazione figure
\let\counterwithout\relax
\let\counterwithin\relax
\usepackage{chngcntr}

\counterwithin{table}{subsection}
\counterwithin{figure}{subsection}

\usepackage[bottom]{footmisc}
\usepackage{fancyhdr}
\setcounter{secnumdepth}{4}
\usepackage{amsmath, amssymb}
\usepackage{array}
\usepackage{graphicx}

\usepackage{ifthen}

%\usepackage{float}
\usepackage{layouts}
\usepackage{url}
\usepackage{comment}
\usepackage{float}
\usepackage{eurosym}

\usepackage{lastpage}
\usepackage{layouts}
\usepackage{float}
\usepackage{eurosym}

%Comandi di impaginazione uguale per tutti i documenti
\pagestyle{fancy}
\lhead{\includegraphics[scale=0.04]{../../../../latex/images/logoTWM.png}}
%Titolo del documento
\rhead{\doctitle{}}
%\rfoot{\thepage}
\cfoot{Pagina \thepage\ di \pageref{LastPage}}
\setlength{\headheight}{35pt}
\setcounter{tocdepth}{5}
\setcounter{secnumdepth}{5}
\renewcommand{\footrulewidth}{0.4pt}

% multirow per tabelle
\usepackage{multirow}

% Permette tabelle su più pagine
%\usepackage{longtable}


% colore di sfondo per le celle
\usepackage[table]{xcolor}

%COMANDI TABELLE
\newcommand{\rowcolorhead}{\rowcolor[HTML]{9b240a}} %intestazione
% check for missing commands
\newcommand{\headertitle}[1]{\textbf{\color{white}#1}} %titolo colonna
\definecolor{pari}{HTML}{FFDBCB}
\definecolor{dispari}{HTML}{F1F7FD}

% comandi glossario
\newcommand{\glo}{$_{G}$}
\newcommand{\glosp}{$_{G}$ }


%label custom
\makeatletter
\newcommand{\uclabel}[2]{%
	\protected@write \@auxout {}{\string \newlabel {#1}{{#2}{\thepage}{#2}{#1}{}} }%
	\hypertarget{#1}{#2}
}
\makeatother

%riportare pezzi di codice
\definecolor{codegray}{gray}{0.9}
\newcommand{\code}[1]{\colorbox{codegray}{\texttt{#1}}}



% Configurazione della pagina iniziale
\newcommand{\doctitle}{Verbale interno 15}
\newcommand{\docdate}{26 Febbraio 2021}
\newcommand{\rev}{1.0.0}
\newcommand{\stato}{Approvato}
\newcommand{\uso}{Interno}
\newcommand{\approv}{Tessari Andrea}
\newcommand{\red}{Crivellari Alberto}
\newcommand{\ver}{De Renzis Simone}
\newcommand{\dest}{Three Way Milkshake\\ Prof. Vardanega Tullio\\ Prof. Cardin Riccardo}
\newcommand{\describedoc}{Verbale del meeting del 2021-02-26 del gruppo Three Way Milkshake}
 % modifica questo file
\makeindex

\usepackage{hyperref}
\hypersetup{
    colorlinks=true,
    linkcolor=blue,
    urlcolor=blue,
    hyperfootnotes=false
}
\usepackage{multicol}
\usepackage{pgfplots}
\usepackage{verbatim}
\usepackage{pgf-pie}
\usepackage{ragged2e}
\usepackage[section]{placeins} %in teoria evita che elementi tipo tabelle rompano il flow e vadano in sezioni dopo rispetto a dove dichiarate (al momento semrba funzionare)
\usepackage{caption}


\setlength{\columnseprule}{1pt}
\newcommand{\group}{Three Way Milkshake}
\newcommand{\pparagraph}[1]{\paragraph{#1}\mbox{}\\}

\makeatletter
\renewcommand\subparagraph{%
    \@startsection {subparagraph}{5}{\z@ }{3.25ex \@plus 1ex
        \@minus .2ex}{-1em}{\normalfont \normalsize \bfseries }}%
\makeatother

\begin{document}
	\thispagestyle{empty}
\begin{titlepage}
	\begin{center}
		
		\includegraphics[scale = 0.17]{../../../../latex/images/logoTWM.png}\\[0.7cm]
		

		\noindent\rule{\textwidth}{1pt} \\[0.4cm]
		\Huge \textbf{\doctitle} \\[0.1cm]
		\ifthenelse{\equal{\docdate}{ }}{ }{ \huge \textbf{\docdate} \\[0.1cm] }
		
		\noindent\rule{\textwidth}{1pt}\\[0.7cm]
		
		\large \textbf{Three Way Milkshake - Progetto "PORTACS"} \\[0.4cm] 
                \texttt{threewaymilkshake@gmail.com} \\[0.4cm]
                
		
        
        
        \large

        \begin{tabular}{r|l}
                        \textbf{Versione} & \rev{} \\
                        \textbf{Stato} & \stato{} \\
                        \textbf{Uso} & \uso{} \\                         
                        \textbf{Approvazione} & \approv{} \\                      
                        \textbf{Redazione} & \red{} \\ 
                        \textbf{Verifica} &  \ver{} \\                         
                        \textbf{Destinatari} & \parbox[t]{5cm}{ \dest{} }
                \end{tabular} 
                \\[0.3cm]
                \large \textbf{Descrizione} \\ \describedoc{} 
               

	\end{center}
\end{titlepage}
	\pagebreak

	% Registro delle modifiche
	\section*{Registro delle modifiche}

\newcommand{\changelogTable}[1]{
	
	
	\renewcommand{\arraystretch}{1.5}
	\rowcolors{2}{pari}{dispari}
	\begin{longtable}{ 
			>{\centering}p{0.07\textwidth} 
			>{}p{0.21\textwidth}
			>{\centering}p{0.17\textwidth}
			>{\centering}p{0.13\textwidth} 
			>{\centering}p{0.17\textwidth} 
			>{\centering}p{0.13\textwidth} }
		\rowcolorhead
		\headertitle{Vers.} &
		\centering \headertitle{Descrizione} &	
		\headertitle{Redazione} &
		\headertitle{Data red.} & 
		\headertitle{Verifica} &
		\headertitle{Data ver.}
		\endfirsthead	
		\endhead
		
		#1
		
	\end{longtable}
	\vspace{-2em}
	
}


\newcommand{\approvingTable}[1]{ 
	
	
	\renewcommand{\arraystretch}{1.5}
	\rowcolors{2}{pari}{dispari}
	\begin{longtable}{ 
			>{\centering}p{0.07\textwidth} 
			>{\centering}p{0.415\textwidth}
			>{\centering}p{0.13\textwidth}
			>{\centering}p{0.322\textwidth}  }
		\rowcolorhead
		\headertitle{Vers.} &
		\centering \headertitle{Descrizione} &	
		\headertitle{Data appr.} &
		\headertitle{Approvazione}
		\endfirsthead	
		\endhead
		
		#1
		
	\end{longtable}
	\vspace{-2em}
	
}
	\approvingTable{
	1.0.0 & Approvazione del verbale & 2021-04-18 & Greggio Nicolò
}

\changelogTable{
	0.1.0 & Stesura e verifica del verbale & Crivellari Alberto & 2021-04-15 & De Renzis Simone & 2021-04-18
} % modifica questo file
	%\end{longtable}
	\pagebreak

	% indice
	{
        \hypersetup{linkcolor=black}
        \tableofcontents
        \pagebreak

        % indice delle figure
        \listoffigures
        \pagebreak

        % indice delle tabelle
        \listoftables
        \pagebreak
    }

	\newcommand{\contabilitaTable}[1]{

\begin{table}[H]
	\begin{center}
		\begin{tabular}{c
				!{\color[HTML]{9b240a}\vrule width 1pt}
				cccccc
				!{\color[HTML]{9b240a}\vrule width 1pt}	
				c}
			\rowcolorhead
			\headertitle{Nome} & \headertitle{R} & \headertitle{V} & \headertitle{An} & \headertitle{Am} & \headertitle{Pr} & \headertitle{Pt} & \headertitle{Tot} \\	
			#1
			\end{center}
			\end{table}	
}


\newcommand{\smallPreventivoTable}[1]{
	
	\begin{table}[H]
		\begin{center}
			\begin{tabular}{c
					!{\color[HTML]{9b240a}\vrule width 1pt}
					cccccc
					!{\color[HTML]{9b240a}\vrule width 1pt}	
					c}
				\rowcolorhead
				\headertitle{Ruolo} & \headertitle{Tempo (ore)} & \headertitle{Costo (euro)} \\
				#1
			\end{center}
		\end{table}	
	}


\newcommand{\planningTable}[1]{
	
	
	\renewcommand{\arraystretch}{1.5}
	\rowcolors{2}{pari}{dispari}
	\begin{longtable}{ 
			>{}p{0.25\textwidth} 
			>{}p{0.42\textwidth}
			>{\centering}p{0.05\textwidth}
			>{\centering}p{0.17\textwidth} }
		\rowcolorhead
		\headertitle{Attività} &
		\headertitle{Descrizione} &	
		\headertitle{Ore} &
		\headertitle{Ruolo} 
		\endfirsthead	
		\endhead
		#1		
	\end{longtable}
	
} % file con template tabelle con macro


	% contenuto del documento, ogni sezione in un file
	\section{Introduzione}




\subsection{Scopo del documento}
Lo scopo di questo documento è presentare tutte le informazioni necessarie al mantenimento e all'estensione del software PORTACS, mostrando nel dettaglio l'architettura del sistema e l'organizzazione del codice sorgente.\\
In questo documento saranno presentate le varie tecnologie usate, sia lato front end che back end, come anche le varie librerie e framework. Verrà inoltre mostrato il sistema di versionamento utilizzato e la Continuous Integration applicata.





\subsection{Scopo del prodotto}

Il capitolato\textsubscript{G} C5 propone un progetto\textsubscript{G} in cui viene richiesto lo sviluppo di un software per il monitoraggio in tempo reale di unità che si muovono in uno spazio definito. All'interno di questo spazio, creato dall’utente per riprodurre le caratteristiche di un ambiente reale, le unità dovranno essere in grado di circolare in autonomia, o sotto il controllo dell’utente, per raggiungere dei punti di interesse posti nella mappa.  La circolazione è sottoposta a vincoli di viabilità e ad ostacoli propri della topologia dell’ambiente, deve evitare le collisioni con le altre unità e prevedere la gestione di situazioni critiche nel traffico.




\subsection{Riferimenti}



\subsubsection{Normativi}

\begin{itemize}
	\item \textsc{Norme di progetto\textsubscript{G} v3.0.0 }: per qualsiasi convenzione sulla nomenclatura degli elementi presenti all’interno del documento;
	
	\item Regolamento progetto\textsubscript{G} didattico: \\ {\url{https://www.math.unipd.it/~tullio/IS-1/2020/Dispense/P1.pdf}};
	\item Model-View Patterns: \\ {\url{https://www.math.unipd.it/~rcardin/sweb/2020/L02.pdf}};
	\item SOLID Principles: \\ {\url{https://www.math.unipd.it/~rcardin/sweb/2020/L04.pdf}};
	\item Diagrammi delle classi: \\ {\url{https://www.math.unipd.it/~rcardin/swea/2021/Diagrammi delle Classi_4x4.pdf}};
	\item Diagrammi dei package: \\ {\url{https://www.math.unipd.it/~rcardin/swea/2021/Diagrammi dei Package_4x4.pdf}};
	\item Diagrammi di sequenza: \\ {\url{https://www.math.unipd.it/~rcardin/swea/2021/Diagrammi di Sequenza_4x4.pdf}};
	\item Design Pattern Creazionali: \\ {\url{https://www.math.unipd.it/~rcardin/swea/2021/Design Pattern Creazionali_4x4.pdf}};
	\item Design Pattern Strutturali: \\ {\url{https://www.math.unipd.it/~rcardin/swea/2021/Design Pattern Strutturali_4x4.pdf}};
	\item Design Pattern Comportamentali: \\ {\url{https://www.math.unipd.it/~rcardin/swea/2021/Design Pattern Comportamentali_4x4.pdf}}.
\end{itemize}



\subsubsection{Informativi}
\begin{itemize}
	\item \textsc{\href{https://github.com/Three-Way-Milkshake/docs/wiki/Glossario}{Glossario}}: per la definizione dei termini (pedice G) e degli acronimi (pedice A) evidenziati nel documento;
	\item Capitolato d'appalto C5-PORTACS: \\
{\url{https://www.math.unipd.it/~tullio/IS-1/2020/Progetto/C5.pdf}}
	\item Software Engineering - Iam Sommerville - $10^{th}$ Edition.
\end{itemize} %@Tex
	\pagebreak

	\section{Tecnologie e librerie}

Nelle sezioni che seguono, vengono elencate le tecnologie e le librerie interessate dallo sviluppo del software. Per ognuna, viene presentata una breve descrizione e spiegato il suo impiego nel contesto del software. Dove necessario, viene fornito un collegamento per il download e l'installazione delle risorse necessarie per lo sviluppo e manutenzione del progetto\textsubscript{G}.




















 %@Simone - intro

	\subsection{Server}

\subsubsection{Tecnologie}

\begin{itemize}
	\item \textbf{Java}
	\item \textbf{Json}
	\item \textbf{Docker}
	\item \textbf{Gradle}
\end{itemize}

\subsubsection{Librerie e Framework}

\begin{itemize}
	\item \textbf{Spring}
	\item \textbf{Gson}
	\item \textbf{Junit}
	\item \textbf{Mockito}

\end{itemize} %@Simone

	\subsection{Client}

\subsubsection{Tecnologie}

\begin{itemize}
	\item \textbf{Node.js}
	\item \textbf{HTML}
	\item \textbf{CSS}
	\item \textbf{Typescript}

\end{itemize}

\subsubsection{Librerie e Framework}

\begin{itemize}
	\item \textbf{Angular}
	\item \textbf{PrimeNG}
	\item \textbf{Libreria di test1}
	\item \textbf{Libreria di test2}
	
\end{itemize} %@TeamFE

	\subsection{Version Control System e Continuous Integration}

\subsubsection{Git e gitflow}
Git è un sistema di controllo per il versionamento veloce ed efficiente\textsubscript{G}. Gitflow è un workflow che aiuta lo sviluppo software dando delle linee guida sui branch che strutturano le repo\textsubscript{A} e le operazioni per l'implementazione di feature e rilascio di releases.\\
Maggiori informazioni:
\begin{itemize}
    \item \textbf{Git: }\url{https://git-scm.com/};
    \item \textbf{gitflow: }\url{https://www.atlassian.com/git/tutorials/comparing-workflows/gitflow-workflow}.
\end{itemize}

\subsubsection{GitHub}
GitHub è un provider di hosting internet per lo sviluppo di software e il controllo della versione utilizzando Git. Fornisce un intero ecosistema di strumenti (version control, issue tracking, project boards, continuous integration\textsubscript{G} e delivery...) e permette la creazione di account personali o di organizzazioni.
\begin{itemize}
    \item \textbf{Maggiori informazioni su GitHub:} \url{https://github.com/about};
    \item \textbf{\group{} su GitHub: }\url{https://github.com/Three-Way-Milkshake}.
\end{itemize}

\subsubsection{GitHub Actions}
È uno strumento integrato in ogni repo\textsubscript{A} di GitHub, permette di creare singole attività\textsubscript{G} combinabili al fine di realizzare complessi workflow personalizzati. 
Si possono creare nuove \textit{actions} ed eventualmente pubblicarle, o utilizzare il vasto catalogo di automazioni già realizzate dalla community.
\begin{itemize}
    \item \textbf{Documentazione: } \url{https://docs.github.com/en/actions}.
\end{itemize} %@Gregg - fatto

	\pagebreak


	\section{Setup}
PORTACS viene distribuito tramite container Docker, per cui i dispositivi sui quali dovrà eseguire avranno minimi requisiti software. È possibile utilizzare i container anche per la fase\textsubscript{G} di sviluppo, altrimenti si possono scaricare ed utilizzare gli strumenti descritti in \ref{tecnologie} per l'esecuzione diretta in locale. PORTACS\textsubscript{A} si divide su tre immagini Docker:
\begin{enumerate}
    \item server;
    \item client muletto (forklift);
    \item client utente (user).
\end{enumerate}


\subsection{Requisiti di sistema}
Sotto elencati saranno descritti i requisiti minimi del sistema per un corretto funzionamento del software PORTACS\textsubscript{A}.


\subsubsection{Requisiti Hardware}
\begin{itemize}
	\item Client → unità:
\begin{itemize}
	\item CPU\textsubscript{A} dual-core o maggiore;
	\item memoria Ram >= 4GB;
	\item connessione con bassi tempi di risposta.
\end{itemize}
	\item Client → admin o responsabile:
\begin{itemize}
	\item CPU\textsubscript{A} dual-core o maggiore;
	\item memoria Ram >= 4GB;
	\item connessione con bassi tempi di risposta.
\end{itemize}
	\item Server:
\begin{itemize}
	\item CPU\textsubscript{A} quad-core o maggiore;
	\item memoria Ram >= 8GB;
	\item connessione con bassi tempi di risposta.
\end{itemize}
\end{itemize}

\subsubsection{Requisiti Software}
    \pparagraph{Esecuzione}
    \begin{itemize}
        \item Docker (v19.03.*);
        \item Google Chrome (v90).
    \end{itemize}

    \pparagraph{Sviluppo}
    Vedi \S\ \ref{tecnologie}

    \begin{comment}
    	\item Client:
    \begin{itemize}
    	\item Docker.
    	%\item Chrome Versione 90.0.4430.85 (Build ufficiale) (a 64 bit);
    	%\item Node.js;
    	%\item Angular;

    	%\item Windows 10.
    \end{itemize}
    	\item Server:
    \begin{itemize}
    	\item Docker.
    	%\item Java;
    	%\item gradle;

    	%\item Windows 10.
    \end{itemize}
    \end{comment}







\subsection{Installazione ed avvio degli applicativi}
    \subsubsection{Server}
    Nella macchina da utilizzare come server andrà scaricata l'immagine di portacs-server con la versione desiderata \href{https://hub.docker.com/r/threewaymilkshake}{docker hub di \group}.
    Per l'avvio, in ambiente Linux e MacOS:
    \begin{itemize}
        \item predisporre una cartella che contenga un file \texttt{config.txt} ed una cartella \texttt{resources} che manterrà la persistenza;
        \item posizionarsi su tale cartella;
        \item eseguire:
    \begin{verbatim}
    docker run -v $(pwd)/resources:/resources \
        --env-file config.txt threewaymilkshake/portacs-server
    \end{verbatim}
    \end{itemize}
    All'interno del file \texttt{config.txt} si possono specificare (come coppie key=value) delle configurazioni diverse da quelle di default per quanto riguardo percorso della persistenza e porta sulla quale il server dovrà esporre il Server Socket per il collegamento. Se non si desiderano modificare i valori di default il file e la relativa parte nel comando possono essere omessi.

    \subsubsection{Client}
    Come per il server, scaricare l'immagine dal docker hub di \group{} relativa al client voluto (forklift o user). Dopodiché sul dispositivo che dovrà fungere da client andranno impostata la configurazione in un file \texttt{config.txt}:
    \begin{itemize}
        \item per i client \textit{user} sarà sufficiente specificare l'indirizzo IP della macchina server così: \texttt{SERVER\_ADDR=ip};
        \item per i client \texttt{forklift}, oltre all'indirizzo come per gli utenti, bisognerà aggiungere altre 2 righe per la configurazione di ogni muletto:
        \begin{itemize}
            \item \texttt{ID=id del muletto};
            \item \texttt{TOKEN=token del muletto}.
        \end{itemize}
        Queste informazioni sono a disposizione degli admin.
    \end{itemize}

    Dopodiché, in entrambi i casi, sarà sufficiente eseguire:
    \begin{verbatim}
        docker run --env-file config.txt threewaymilshake/portacs-client-<type>
    \end{verbatim}
    con \texttt{<type>} tra \texttt{forklift} o \texttt{user}.


 %TODO
	\pagebreak


	\section{Testing}


\subsection{JUnit}



\subsection{Libreria test frontend} %@Alberto

	\subsection{Testing lato server}
\subsubsection{Junit}
Un framework di unit testing per la programmazione java. Junit viene utilizzato per i test di unità relativi al codice in Java.
Per effettuare i test basterà eseguire il comando ./gradlew test.

\subsubsection{Spring}
Framework utilizzato per la risoluzione delle dipendenze e come supporto durante la scrittura degli unit test. %@Alberto (aiuto di teamBE)

	\subsection{Testing lato client}

\subsubsection{Jasmine}
Jasmine è un framework\textsubscript{G} di unit testing per il codice JavaScript.
Viene utilizzato per i test di unità relativi al codice in Angular CLI.
Si scrivono i test, in un formato leggibile ad occhio umano, e si eseguono attraverso dei file formato js, collegati a relativi file html e css.

\subsubsection{Karma}
Karma è un test runner, che permette di utilizzare Jasmine in maniera più efficace\textsubscript{G}.
Karma e Jasmin vengono utilizzati in quanto framework\textsubscript{G} di default per unit testing di Angular CLI.

\subsubsection{Mocha}
Mocha è un framework\textsubscript{G} di unit testing per il codice JavaScript.
Viene utilizzato per i test di unità relativi al codice in NodeJS. %@Alberto (aiuto di teamFE)

	\pagebreak


	\section{Architettura del sistema}


Qui si potrebbe mettere uno schema simile a quello della slide iniziale per evidenziare l'architettura client-server



\subsection{Server}

Dire che è 3 layer architecture

Qui potrebbe esserci il diagramma minimale complessivo


\subsubsection{Diagramma delle classi}


\paragraph{Persistance layer}


\paragraph{Business layer}

\subparagraph{Mappa}

\subparagraph{Clients}

\subparagraph{Tasks}

\subparagraph{Collisioni}


\paragraph{Communication layer}



\subsection{Client}


\subsection{Comunicazione}

\subsubsection{Diagrammi di sequenza}

\subsubsection{Protocollo di comunicazione}


 %@Simone - %Intro

	\clearpage
\subsection{Server}

L'architettura della componente server si articola in una 3-layer architecture, in cui si identificano i seguenti layer:
\begin{itemize}
	\item communication layer;
	\item business layer;
	\item persistence layer.
\end{itemize}

\begin{figure}[H]
	\centering
	\includegraphics[scale=0.22]{res/diagrams/server/server_complessivo_minimal.jpg}
	\caption{Visione complessiva dell'architettura del server}
\end{figure}

Le sezioni che seguono illustrano la struttura di ogni layer.

\clearpage
\subsubsection{Communication layer}
\label{communication-layer}

\begin{figure}[H]
	\centering
	\includegraphics[scale=0.55]{res/diagrams/server/server_communication.jpg}
	\caption{Visione di dettaglio del Communication Layer}
\end{figure}


Questo layer si interfaccia con i client esterni e ha lo scopo di gestire la comunicazione con questi. In particolare, la classe \texttt{ConnectionAccepter} si occupa di accettare le nuove connessioni entranti tramite \texttt{ServerSocket}: essa esegue su un thread dedicato in modo da non bloccare le altre operazioni all'arrivo di una nuova connessione.

Per ogni nuova connessione, crea un oggetto \texttt{Socket} che passa a \texttt{ConnectionHandler}. Quest'ultima è una componente che esegue su un altro thread dedicato: rimane in attesa fino al risveglio determinato da \texttt{ConnectionAccepter}: una volta attivato, procede a svuotare il buffer di \texttt{Socket} per creare oggetti di tipo \texttt{Connection}, istanziando per ognuno i buffer di input e output. Segue quindi il processo di autenticazione dei muletti o degli utenti, al termine del quale \texttt{ConnectionHandler} torna in attesa.






\clearpage
\subsubsection{Business layer}

Nel Business layer risiede il nucleo di elaborazione dei dati ricevuti dal layer superiore: i domini principali di cui si occupa sono:
\begin{itemize}
	\item gestione della mappa e path finding;
	\item autenticazione dei client;
    \item elaborazione delle richieste;
	\item gestione delle tasks e dei POI\textsubscript{A};
	\item rilevazione e risoluzione delle collisioni.
\end{itemize}
Per facilitare la consultazione, lo studio di questo layer si concentra separatamente sui package di cui si compone. Per una visione dall'alto, riferirsi al diagramma complessivo all'inizio della sezione 5.1.



\clearpage
\paragraph{Clients}
\subparagraph*{ }

\begin{figure}[H]
	\centering
	\includegraphics[scale=0.40]{res/diagrams/server/server_pack_clients.jpg}
	\caption{Visione di dettaglio del package Clients}
\end{figure}

La gerarchia dei \texttt{Client} prevede una prima suddivisione suddivisione tra \texttt{Forklift} e \texttt{User} (muletto e utente), gli \texttt{User} si specializzano ulteriormente in \texttt{Manager} (responsabile) e \texttt{Admin} (amministratore). Viene mantenuto a runtime lo stato di tutti i muletti e degli utenti registrati rispettivamente in \texttt{ForkliftsList} e \texttt{UsersList} così da ridurre le operazioni sulla persistenza.

I \texttt{Forklift} si caratterizzano dagli attributi:
\begin{itemize}
	\item \texttt{position}: rappresenta la posizione e orientamento attuali del muletto nella mappa;
	\item \texttt{tasks}: sequenza di task\textsubscript{G} da compiere;
	\item \texttt{pathToNextTask}: una lista di mosse atte a raggiungere il prossimo POI\textsubscript{A} (e quindi evadere la prossima task).
\end{itemize}

Notare che ogni \texttt{Client} possiede un attributo di tipo \texttt{Connection}, attraverso il quale viene regolata la comunicazione tramite Socket (per i dettagli si rimanda alle \S\ \ref{communication-layer} e \ref{communication-section}). Questo viene assegnato all'autenticazione e può cambiare un numero indefinito di volte, ogni qual volta che la connessione con il client verrà chiusa per qualsiasi ragione alla riconnessione questa verrà correttamente abbinata alla sua istanza.\\

La classe \texttt{Engine} è il cuore del motore di calcolo: essa esegue su un thread dedicato e tramite un timer scandisce l'esecuzione temporizzata dell'elaborazione. In particolare, interroga periodicamente \texttt{ForkliftsList} e \texttt{UsersList} con i seguenti obiettivi:
\begin{itemize}
	\item ricevere le nuove posizioni dai muletti;
	\item inviare le nuove informazioni agli utenti per la visualizzazione nel monitor real-time;
	\item rispondere ad eventuali altre richieste dell'iterazione precedente, avendo nel frattempo completato la loro elaborazione (calcolo percorso, aggiunta task\textsubscript{G}, modifica mappa).
\end{itemize}
Dopodichè la \texttt{ForkliftsList} viene utilizzata dal modulo di rilevazione e gestione delle collisioni.

In questo layer si concentra l'utilizzo del framework\textsubscript{G} Spring, utilizzato per gestire le dipendenze: alcune classi di utilizzo frequente e condiviso come \texttt{UsersList}, \texttt{ForkliftsList} e \texttt{TasksSequencesLists} (quest'ultima contenente tutte le liste di task\textsubscript{G} inserite dal responsabile) vengono istanziate tramite \textit{Dependency Injection} sfruttando il meccanismo dei \textit{Bean} di Spring.




\paragraph{Mappa}
\subparagraph*{ }

\begin{figure}[H]
	\centering
	\includegraphics[scale=0.60]{res/diagrams/server/server_pack_map.jpg}
	\caption{Visione di dettaglio del package Map}
\end{figure}


La classe \texttt{WarehouseMap} contiene la rappresentazione della planimetria\textsubscript{G} del magazzino: essa è rappresentata tramite una matrice di \texttt{CellType}, campo di tipo enumerazione che esprime le caratteristiche di ogni frazione spaziale. Alla mappa è associata una lista di POI\textsubscript{A}, e per ognuno la relativa locazione.
Si osserva l'applicazione di alcuni design pattern\textsubscript{G}:
\begin{itemize}
	\item \textbf{observer}: tramite \texttt{PropertyChangeSupport} e \texttt{PropertyChangeListener} di \texttt{java.beans} viene applicato il pattern \textit{observer}, definendo la \texttt{WarehouseMap} come Subject, e i \texttt{Client} come \textit{Observer}: essi verranno notificati ad ogni cambiamento della stessa in modo che possano comunicarlo tramite le connessioni, così da riflettere le modifiche ed aggiornare le interfacce grafiche che visualizzano la mappa;
	\item \textbf{strategy}: per l’algoritmo di path finding attualmente viene implementata una strategia di tipo \textit{breadth-first}, ma l’impostazione del pattern permette di aggiungere e variare dinamicamente eventuali altre implementazioni aggiunte in futuro. \texttt{WarehouseMap} assume il ruolo di \textit{context}, e i beneficiari sono i \texttt{Forklift}, i quali richiederanno il percorso ogni qualvolta si renderà necessario.
\end{itemize}





\paragraph{Collisioni}
\label{collision-details}
\subparagraph*{ }

\begin{figure}[H]
	\centering
	\includegraphics[scale=0.50]{res/diagrams/server/server_pack_collision.jpg}
	\caption{Visione di dettaglio del package Collision}
\end{figure}


Qui è contenuta la logica che gestisce le collisioni fra i muletti che circolano in guida autonoma all'interno del magazzino. L'elaborazione è scandita dal timer dell'\texttt{Engine}: ad ogni intervallo di tempo, vengono eseguite due operazioni sequenziali:
\begin{itemize}
	\item \textbf{rilevazione:} sulla base delle future mosse di ogni unità, vengono determinate le possibili collisioni;
	\item \textbf{risoluzione:} in caso vengano rilevate, vengono elaborate le mosse, da trasmettere alle unità coinvolte, che impediscano tali collisioni.
\end{itemize}
Le classi \texttt{CollisionDetector} e \texttt{CollisionSolver} incapsulano rispettivamente le due funzionalità elencate.

Viene applicato in questo contesto il design pattern\textsubscript{G} \texttt{Pipeline}\footnote{Variante del pattern \textit{Chain Of Responsibility}: \url{https://java-design-patterns.com/patterns/pipeline/}}, che permette di definire vari \texttt{Handler} da comporre come catena di operazioni. La pipeline può essere poi eseguita (se necessario, come in questo caso, ripetutamente) con un comando che attiva i vari step sequenzialmente. Ogni \texttt{Handler} specifica i tipi del proprio parametro di input e di output: l'output di un \texttt{Handler} sarà l'input dell'\texttt{Handler} successivo.
In questo caso l'input sarà una struttura rappresentante muletti attivi con le loro posizioni e prossime mosse, mentre l'output un'altra struttura che ad ogni punto con collisione associ i muletti coinvolti e le azioni da intraprendere così da poterle comunicare.



\subsubsection{Persistence layer}

\begin{figure}[H]
	\centering
	\includegraphics[scale=0.50]{res/diagrams/server/server_persistency.jpg}
	\caption{Visione di dettaglio del Persistence Layer}
\end{figure}

L'accesso a questo layer è regolato da 3 interfacce che gestiscono la persistenza delle tre tipologie di dati che vengono salvati:
\begin{itemize}
	\item i dati e le credenziali degli utenti;
	\item gli identificativi ed i token di autenticazione dei muletti;
	\item la rappresentazione della mappa.
\end{itemize}

Ogni interfaccia si rivolge alla relativa componente del layer superiore che conserva a runtime i dati impiegati nell'esecuzione. La presenza delle interfacce favorisce il disaccoppiamento tra i moduli e permette di estendere a tipi di persistenza alternativi. Attualmente è implementato il salvataggio dei dati su file di tipo .json, viene fatto uso della libreria standard java.io e GSON per gestire l'interazione con questo tipo di tecnologia.


\pagebreak
 %@Simone

	\subsection{Client}
\subsubsection{Diagramma di classe per l'unità}

\begin{figure}[H]
	\centering
	\includegraphics[scale=0.5]{res/images/UML_operatore.png}
	\caption{Diagramma UML delle classi per l'unità}
\end{figure}

Qui utilizziamo due tecnologie: Node per quanto riguarda il package \texttt{connection} e Angular per il resto. Queste due parti del front end comunicano attraverso il package esterno \texttt{Socket.io}, necessario gestire un flusso di dati attraverso i socket.\\
Il package \texttt{services}, come descritto anche nella documentazione di Angular, fa da intermediario tra il package \texttt{connection} e il package \texttt{component}, utilizzando degli Observer in ascolto di uno specifico socket e instradando l'informazione verso l'opportuno component.\\
Il package \texttt{component} permette di visualizzare sullo schermo le informazioni richieste grazie anche ai template di Angular. Ogni classe di questo package serve ad una specifica funzionalità:
\begin{itemize}
	\item \texttt{Map} → visualizza la mappa del magazzino con la posizione in real time dell'unità;
	\item \texttt{StartButton} → mostra un bottone che serve a far partire l'unità;
	\item \texttt{TaskList} → mostra la lista di task che l'operatore dovrà compiere;
	\item \texttt{Arrows} → visualizza le azioni che compie l'unità in real time;
	\item \texttt{ManualDrive} → permette di cambiare guida da manuale ad automatica e viceversa, facendo visualizzare anche i pulsanti da premere per far muovere l'unità manualmente in caso;
	\item \texttt{AdminNotification} → visulizza un pulsante che, se premuto, notifica all'admin un evento eccezionale;
	\item \texttt{ComeBack} → mostra un pulsante alla fine del turno dell'operatore che, se premuto, guida automaticamente il muletto verso la propria base.
\end{itemize}
Il package connection, attraverso le classi \texttt{Index} e \texttt{CommandsToJava}, instaura inoltre una comunicazione TCP Socket con java, permettendo di creare una connessione tra frontend e backend.\\

\subsubsection{Diagramma di classe per l'admin-manager}

\begin{figure}[H]
	\centering
	\includegraphics[scale=0.6]{res/images/UML_admin-manager.png}
	\caption{Diagramma UML delle classi per gli admin e i manager}
\end{figure}
Il diagramma di classe dell'admin-responsabile è molto simile a quello dell'unità: si avvale delle tecnologie Node, con il package \texttt{connection}, ed Angular, con tutti gli altri package.\\
Il contesto dei package è lo stesso dell'unità, di cui però si fa una differenziazione tra le funzionalità del'admin e quelle del manager grazie alla classe \texttt{Login} e al routing di \texttt{AppRouting} presente nel package \texttt{component}: permette all'utente di effettuare il login, "attivando" solamente le funzionalità riferite al tipo di utente loggato.\\
Il package \texttt{generic} contiene classi di funzionalità condivise tra manager e admin.\\
Le classi presenti in component permettono di attivare certe classi riferite al package (e quindi alle funzionalità di uno specifico tipo di utente):
\begin{itemize}
	\item admin:
	\begin{itemize}
		\item \texttt{ListUsers} → aggiunta o rimozione di manager;
		\item \texttt{EventAlert} → visualizzazione eventi eccezionali;
		\item \texttt{ManageMap} → modifica la planimetria e le caratteristiche della mappa;
	\end{itemize}
	\item manager:
	\begin{itemize}
		\item \texttt{TaskLists} → visualizzazione delle liste di task;
		\item \texttt{MangeListsTask} → aggiunta, modifica e rimozione di liste di task;
	\end{itemize}
	\item generic (sia per admin che per manager):
	\begin{itemize}
		\item \texttt{ViewMap} e \texttt{POIList} → visualizzazione in real time di tutte le unità nel magazzino.
	\end{itemize}
\end{itemize}



 %@TeamFE

	\clearpage
\subsection{Comunicazione}
\label{communication-section}

Le comunicazioni tra client e server avvengono tramite stringhe inviate sui TCPSocket che li connettono, secondo il protocollo descritto alla \S\ \ref{comm-protocol}. La creazione di un socket che viene mantenuto fino alla disconnessione permette di richiedere l'autenticazione di ogni client solo alla connessione, senza bisogno di scambi ulteriori di token o codici di sessioni, in quanto ogni client ha i suoi canali dedicati di input e output per cui si sa sempre a chi si scrive e da chi si legge. Le connessioni all'interno del server sono rappresentate con una classe dedicata, i client poi aggregano un attributo di tale classe: ciò permette di mantenere lo stato di questi anche in caso di disconnessione in quanto verrà distrutto solo l'oggetto connessione, e alla successiva autenticazione del client verrà correttamente abbinato lo stato interno che non avrà subito modifiche se non per l'associazione della nuova connessione.
 %@Gregg - Intro - fatto

	
\subsubsection{Diagrammi di sequenza}
Di seguito vengono riportati i diagrammi di sequenza.
\begin{figure}[H]
	\centering
	\includegraphics[scale=0.55]{res/diagrams/sequenza/connect unit.jpg}
	\caption{Visione complessiva della connessione server Java - unità muletto in NodeJS}
\end{figure}
Questo diagramma di sequenza rappresenta la prima connessione tra un'unità muletto in NodeJS e il server Java, attraverso il TCPsocket.
Nel diagramma viene raffigurato anche un socket.IO, che mette in comunicazione NodeJS con l'interfaccia grafica realizzata in Angular (non rappresentata nel diagramma).
\begin{figure}[H]
	\centering
	\includegraphics[scale=0.55]{res/diagrams/sequenza/connect unit.jpg}
	\caption{Esempio di comunicazione tra server e client, attraverso i socket TCP}
\end{figure}
Questo diagramma rappresenta un esempio di comunicazione tra server e client di tipo muletto. Il server legge i dati inviati dal muletto e invia una risposta, il muletto aggiorna l'interfaccia grafica in base alla risposta ricevuta dal server. %@Alberto

	\clearpage
\subsubsection{Protocollo di comunicazione}
\label{comm-protocol}

Ogni stringa può contenere uno o più comandi, separati da ‘;' e ogni comando può avere 0 o più parametri, separati da ',’. \\
\textbf{Esempio sequenza:} \texttt{POS,1,1,0;PATH,1}

\pparagraph{Modificare il protocollo}
    Per la modifica di parti esistenti o per l'aggiunta di nuove funzionalità riguardo la comunicazione è sufficiente andare a modificare le corrispondenti sezioni.
%    \ssubparagraph{Lato server}
    In base alla necessità:
    \begin{longtable}[h!]{
            %|m{3cm}|m{5.25cm}|m{5.25cm}|
            |>{\raggedright\arraybackslash}p{0.2\textwidth}|
            >{\raggedright\arraybackslash}p{0.4\textwidth}|
            >{\raggedright\arraybackslash}p{0.4\textwidth}|
        }
        %\endfirsthead
        \hline
        \rowcolorhead
        \headertitle{Contesto} & \headertitle{Lato Server} & \headertitle{Lato Client} \\
        \hline
        \endhead

        Connessione &
        \texttt{ConnectionAccepter.java: run} &
        \texttt{index.js: createConnectionServer}
        \\

        Autenticazione &
        \texttt{ConnectionHandler.java: execute}, \texttt{auth} di \texttt{Users/ForkliftsList} e conseguentemente \texttt{authenticate} di \texttt{Forklift/User}&
        \texttt{index.js: createConnectionServer} e quindi in \texttt{net.connect}
        \\

        Modificare una funzionalità generica utente esistente &
        \texttt{User.java: processCommuncation}, all'interno dello switch, identificare la funzionalità voluta (esiste un case per ogni possibile messaggio ricevibile) ed apportare le modifiche desiderate &
        Trovare all'interno dello switch in \texttt{index.js: createConnectionServer} il case desiderato e modificarlo
        \\

        Aggiungere una funzionalità generica utente &
        Aggiungere un case all'interno dello switch di cui sopra avendo cura di terminare con \texttt{break;} per evitare \textit{fallthrough} &
        Aggiungere un case all'interno dello switch in \texttt{index.js: createConnectionServer}
        \\

        Modificare una funzionalità amministratore / responsabile &
        \texttt{Admin/Manager.java: processCommuncation}, all'interno dello switch, identificare la funzionalità voluta (esiste un case per ogni possibile messaggio ricevibile) ed apportare le modifiche desiderate &
        Modificare il comportamento dell'opportuno case all'interno dello switch in \texttt{index.js: createConnectionServer}
        \\

        Aggiungere una funzionalità amministratore / responsabile &
        Aggiungere un case all'interno dello switch di cui sopra avendo cura di terminare con \texttt{break;} per evitare \textit{fallthrough} &
        Aggiungere un case all'interno dello switch in \texttt{index.js: createConnectionServer}
        \\

        Modificare una funzionalità muletto &
        \texttt{Forklift.java: processCommuncation}, all'interno dello switch, identificare la funzionalità voluta (esiste un case per ogni possibile messaggio ricevibile) ed apportare le modifiche desiderate &
        Modificare il comportamento dell'opportuno case all'interno dello switch in \texttt{index.js: createConnectionServer}
        \\

        Aggiungere una funzionalità muletto &
        Aggiungere un case all'interno dello switch di cui sopra avendo cura di terminare con \texttt{break;} per evitare \textit{fallthrough} &
        Aggiungere un case all'interno dello switch in \texttt{index.js: createConnectionServer}
        \\


        \hline
        \hiderowcolors
        \caption{Modificare il protocollo do comunicazione}\\
        \showrowcolors
    \end{longtable}

    \ssubparagraph{Esempio modifica lato server}
    Volendo modificare il comportamento del comando \texttt{RMU} (remove user) sarà sufficiente andare nel metodo \texttt{processCommunication} all'interno di \texttt{Admin.java}, identificare il case \texttt{"RMU"} e al suo interno fare il \textit{parsing} degli argomenti che ci si aspetta e processarli come voluto. Si consiglia di delegare queste operazioni ad altri metodi come visibile in gran parte del codice già esistente, per evitare di sovraccaricare \texttt{processCommuncation} di responsabilità. Poi come si può osservare all'interno di \texttt{removeUser} quando si deve inviare una risposta al client (la quale dovrà essere ben formata, vedi inizio \ref{comm-protocol}) sarà sufficiente chiamare \texttt{connection.writeToBuffer(...)} passandogli il messaggio da inviare.

    \ssubparagraph{Esempio modifica lato client}
    Volendo modificare il comportamento del comando \texttt{RMU} (remove user) sarà sufficiente andare nel metodo \texttt{createCommunicationServer} all'interno di \texttt{index.js}, identificare il case \texttt{"RMU"} e al suo interno modificarne il comportamento, conoscendo il protocollo di comunicazione e come queste informazioni vengono passate.

\pparagraph{Connessione: identificazione e login}
    Quando un client si connette deve essere identificato come tipo ed autenticato, perciò deve inviare separatamente ed in sequenza:
    \begin{enumerate}
        \item \textbf{TYPE: }\texttt{FORKLIFT} o \texttt{USER};
        \item \textbf{ID: } identificativo personale;
        \item \textbf{PWD/TOKEN: } password o token a seconda che sia rispettivamente un utente o un muletto.
    \end{enumerate}
    Quindi riceverà come risposta:
    \begin{itemize}

        \item nel caso di FORKLIFT: OK oppure FAIL,MSG

        \item nel caso di USER: OK,TYPE oppure FAIL,MSG

        \subitem -- dove TYPE indica il ruolo dell’utente (ADMIN o MANAGER)
    \end{itemize}
    dove MSG conterrà maggiori dettagli sulla causa.

    \subparagraph{Esempio connessione ed autenticazione muletto}
        Dato un muletto con id=f1 e token=abcdef:
        \begin{itemize}
            \item invia: \texttt{FORKLIFT\textbackslash nf1\textbackslash nabcdef};

            \item riceve: \texttt{OK} oppure \texttt{FAIL,messaggioErrore}
        \end{itemize}


        Funzionamento analogo per gli utenti con password al posto di token.

\clearpage
\pparagraph{Enumerazioni}
    In seguito si farà riferimento più volte ai diversi tipi enum presenti nella logica di business, per cui segue un riassunto:


    \begin{table}[h!]
        \centering
        \begin{tabular}{|c|c|c|c|c|}
            \hline
            \rowcolorhead
            \multicolumn{5}{|c|}{\headertitle{ENUM}}\\
            \hline
            \rowcolorhead
            \headertitle{↓Val \textbackslash{} Enum→} & \headertitle{PoiType} & \headertitle{Move}       & \headertitle{Orientation} & \headertitle{CellType} \\
            0          & LOAD    & GOSTRAIGHT & UP          & OBSTACLE \\
            1          & UNLOAD  & TURNAROUND & RIGHT       & NEUTRAL \\
            2          & EXIT    & TURNRIGHT  & DOWN        & UP \\
            3          & --      & TURNLEFT   & LEFT        & RIGHT \\
            4          & --      & STOP       & --          & DOWN \\
            5          & --      & --         & --          & LEFT \\
            6          & --      & --         & --          & POI\textsubscript{A} \\ [1ex]
            \hline
        \end{tabular}
        \caption{Riepilogo enumerazioni}
    \end{table}
\newcommand{\tabitem}{~~\llap{\textbullet}~~}

\clearpage
\pparagraph{Comandi clients → server}
Per i comandi di risposta, controllare i corrispondenti in \S\ \ref{commands-server-client}
    \begin{table}[h!]
        \centering
        \begin{tabular}{|c|p{8cm}|c|}
            \hline
            \rowcolorhead
            \multicolumn{3}{|c|}{\headertitle{FORKLIFTS → SERVER}}\\
            \hline
            \rowcolorhead
            \headertitle{Comando} & \headertitle{Descrizione} & \headertitle{Risposta} \\
            \hline
            \texttt{POS,X,Y,DIR} & Posizione attuale del muletto, considerando la mappa come una matrice:
            \begin{itemize}
                \item X: riga della matrice
                \item Y: colonna ““
                \item DIR: orientamento assoluto secondo enum Orientation
            \end{itemize}

            & -- \\
            \texttt{LIST} & Richiede nuova lista di task\textsubscript{G} da completare & \texttt{LIST,...} \\

            \texttt{PATH,C} & Richiede il percorso migliore per raggiungere il POI\textsubscript{A} della task\textsubscript{G} corrente a partire dalla posizione attuale. Se C=1 rimuove la task\textsubscript{G} e passa alla successiva & \texttt{PATH,...} \\

            \texttt{BASE} & Segnala l'arrivo ad una base & -- \\

            \texttt{ECC} & Segnala il verificarsi di un evento eccezionale & -- \\

            \hline
        \end{tabular}
        \caption{Comandi clients → server | Forklifts}
    \end{table}

    \begin{table}[h!]
        \centering
        \begin{tabular}{|c|p{8cm}|c|}
            \hline
            \rowcolorhead
            \multicolumn{3}{|c|}{\headertitle{USER generico → SERVER}}\\
            \hline
            \rowcolorhead
            \headertitle{Comando} & \headertitle{Descrizione} & \headertitle{Risposta} \\
            \hline
            \texttt{EDIT,T,PAR} & Modifica dati del proprio profilo
            \begin{itemize}
                \item T: NAME/LAST/PWD

                \item PAR: nuova valore per T
            \end{itemize}
             & -- \\

             \texttt{LOGOUT} & Richiesta di disconnessione & -- \\

            \hline


        \end{tabular}
        \caption{Comandi clients → server | User generico}
    \end{table}

    \begin{table}[h!]
        \centering
        \begin{tabular}{|c|p{8cm}|c|}
            \hline
            \rowcolorhead
            \multicolumn{3}{|c|}{\headertitle{MANAGER → SERVER}}\\
            \hline
            \rowcolorhead
            \headertitle{Comando} & \headertitle{Descrizione} & \headertitle{Risposta} \\
            \hline
            \texttt{ADL,P1,P2,…} & Aggiunge una nuova lista di task\textsubscript{G}
            \begin{itemize}

                \item P1..: id dei poi che compongono la lista
            \end{itemize}
            & \texttt{ADL,...} \\

            \texttt{RML,ID} & Richiede la cancellazione della lista con id=ID & \texttt{RML,...} \\

            \hline
        \end{tabular}
        \caption{Comandi clients → server | User Manager (Responsabile)}
    \end{table}

    %\begin{table}[h!]
        %\centering
        \begin{longtable}[h!]{|c|p{8cm}|c|}
            %\endfirsthead
            \hline
            \rowcolorhead
            \multicolumn{3}{|c|}{\headertitle{ADMIN → SERVER}}\\
            \hline
            \rowcolorhead
            \headertitle{Comando} & \headertitle{Descrizione} & \headertitle{Risposta} \\
            \hline
            \endhead

            \texttt{MAP,R,C,SEQ} & Nuova planimetria\textsubscript{G}
            \begin{itemize}
                \item R: num righe
                \item C: “ colonne
                \item SEQ: sequenza di interi corrispondenti all’enum CellType rappresentanti stati di una cella, indicanti la nuova planimetria\textsubscript{G}, elencati per righe
            \end{itemize}
            & \texttt{MAP,...} \\

            \texttt{CELL,X,Y,A[,ID,T,NAME]} & Modifica una cella, la parte tra [ ] è presente solo in caso di POI\textsubscript{A}
            \begin{itemize}
                \item X e Y riga e colonna della matrice

                \item A: numero rappresentante l'azione  da intraprendere, corentemente con CellType
            \end{itemize}

            Solo nel caso di POI\textsubscript{A}:
            \begin{itemize}
                \item ID: id del POI\textsubscript{A}, 0 se in creazione;

                \item T: tipo di POI\textsubscript{A} secondo PoiType

                \item NAME: stringa di caratteri da associare al POI\textsubscript{A}
            \end{itemize}
            & \texttt{CELL,...} \\

            \texttt{ADU,T,NAME,LAST} & aggiunge nuovo utente
            \begin{itemize}
                \item T: tipo (ADMIN o MANAGER)

                \item NAME e LAST: rispettivamente nome e cognome
            \end{itemize}
            & \texttt{ADU,...} \\

            \texttt{RMU,ID} & Rimuove l’utente con id=ID & \texttt{RMU,...} \\

            \texttt{EDU,ID,A,PAR} & Modifica l’utente con id=ID, con
            \begin{itemize}
                \item A: azione da intraprendere tra:

                    \subitem -- NAME: modifica nome

                    \subitem -- LAST: modifica cognome

                    \subitem -- RESET: esegue reset della password
                \item PAR: nuovo valore da assegnare (assente in caso di reset)
            \end{itemize}
            & \texttt{EDU,...} \\

            \texttt{ADF,ID} & Aggiunge nuovo muletto. ID: stringa che si vuole assegnare come identificativo al nuovo muletto (NON sarà più modificabile) & \texttt{ADF,...} \\

            \texttt{RMF,ID} & Rimuove il muletto con id=ID & \texttt{RMF,...} \\

            \texttt{LISTF} & Richiede lista di tutti i muletti registrati & \texttt{LISTF,...} \\

            \texttt{LISTU} & Richiede lista di tutti gli utenti registrati & \texttt{LISTU,...} \\
            \hline
        \hiderowcolors
        \caption{Comandi clients → server | User Admin (Amministratore)}\\
        \showrowcolors
        \end{longtable}

    %\end{table}

\clearpage
\pparagraph{Comandi server → clients}
\label{commands-server-client}
    Tutti i comandi che possono ritornare un esito positivo o negativo hanno 2 varianti:
    \begin{itemize}
        \item \texttt{CMD,OK,MORE}: successo, eventualmente MORE contiene parametri di risposta aggiuntivi
        \item \texttt{CMD,FAIL,MSG}: fallimento, MSG contiene maggiori informazioni sulle cause.
    \end{itemize}

    \begin{table}[h!]
        \centering
        \begin{tabular}{|c|p{8cm}|c|}
            \hline
            \rowcolorhead
            \multicolumn{3}{|c|}{\headertitle{SERVER → CLIENTS generici}}\\
            \hline
            \rowcolorhead
            \headertitle{Comando} & \headertitle{Descrizione} & \headertitle{Risposta} \\
            \hline
            \texttt{MAP,R,C,SEQ} & Indica la planimetria\textsubscript{G}
            \begin{itemize}
                \item R e C: numero di righe e colonne

                \item SEQ: sequenza di interi corrispondenti all’enum CellType rappresentanti stati di una cella, indicanti la nuova planimetria\textsubscript{G}, elencati per righe
            \end{itemize}
            & --\\
            \texttt{POI,N,X,Y,T,ID,NAME} & Rappresenta tutti i poi, la parte da X in poi si ripete per ogni POI\textsubscript{A}
            \begin{itemize}
                \item N: num totale POI\textsubscript{A}

                \item X,Y e T posizione nella matrice e tipo secondo PoiType

                \item ID e NAME: rispettivamente identificativo e nome del POI\textsubscript{A}
            \end{itemize}
            & -- \\

            \hline
        \end{tabular}
        \caption{Comandi server → clients | generici per tutti i client}
    \end{table}

    \begin{table}[h!]
        \centering
        \begin{tabular}{|c|p{8cm}|c|}
            \hline
            \rowcolorhead
            \multicolumn{3}{|c|}{\headertitle{SERVER → FORKLIFT}}\\
            \hline
            \rowcolorhead
            \headertitle{Comando} & \headertitle{Descrizione} & \headertitle{Risposta} \\
            \hline
            \texttt{ALIVE} & Ha lo scopo primario di verificare l’integrità della connessione ed ha come effetto l’ottenimento della nuova posizione. Se l’invio fallisce il muletto corrispondente viene considerato disconnesso, l’oggetto Connection relativo verrà chiuso e distrutto e il muletto dovrà riautenticarsi. Si presuppone che questo non si muova più finché la connessione non viene ristabilita, in quanto i suoi spostamenti sarebbero sconosciuti al server e questo non potrebbe intervenire per evitare eventuali collisioni.

            Ad esso possono seguire ulteriori comandi o risposte a comandi precedente, secondo la sintassi generale per cui separati da ';'
            & \texttt{POS,...} \\

            \texttt{LIST,ID1,ID2…} & Invia lista di task\textsubscript{G} assegnate. ID1.. sono gli id dei POI\textsubscript{A} da raggiungere & -- \\

            \texttt{PATH,SEQ} & Invia il percorso per raggiungere il prossimo POI\textsubscript{A}, composto da mosse successive secondo Move, separate da ','& -- \\

            \texttt{STOP,N} & Richiede lo stop immediato del muletto per N istanti, che verranno contati alle ricezioni dei futuri ALIVE. Se N=0 stop indefinito fino alla ricezione di \texttt{START}. & -- \\
            \texttt{START} & Consente ad un muletto fermato a tempo indeterminato di ripartire & -- \\

            \hline
        \end{tabular}
        \caption{Comandi server → clients | Forklifts (muletti)}
    \end{table}

    \begin{table}[h!]
        \centering
        \begin{tabular}{|c|p{8cm}|c|}
            \hline
            \rowcolorhead
            \multicolumn{3}{|c|}{\headertitle{SERVER → USER generico}}\\
            \hline
            \rowcolorhead
            \headertitle{Comando} & \headertitle{Descrizione} & \headertitle{Risposta} \\
            \hline
            \texttt{UNI,N,ID1,X1,Y1,D1} & Indica le posizioni di tutti i muletti. Da ID1 in poi si ripete per ogni muletto
            \begin{itemize}
                \item N: num totale dei muletti attivi per i quali si sta ricevendo la posizione

                \item IDn: id del muletto n

                \item Xn, Yn e Dn: posizione rispetto alla matrice e orientamento secondo Orientation del muletto IDn.
            \end{itemize}
            Se l'invio di questo fallisce, l'utente viene considerato disconnesso.
            & -- \\

            \texttt{LIST,NF,IDF,N,IDP1,IDP2…} & Indica la lista di task\textsubscript{G} presa in carico da un muletto
            \begin{itemize}
                \item NF: numero totale muletti dei quali si stanno ricevendo le task\textsubscript{G};

                \item IDF: id del muletto a cui ci si sta riferendo

                \item N: numero di task\textsubscript{G} prese in carico

                \item IDP1…: sequenza di id dei POI\textsubscript{A} da raggiungere
            \end{itemize}
            & -- \\


            \hline
        \end{tabular}
        \caption{Comandi server → clients | User generico}
    \end{table}


    \begin{table}[h!]
        \centering
        \begin{tabular}{|c|p{8cm}|c|}
            \hline
            \rowcolorhead
            \multicolumn{3}{|c|}{\headertitle{SERVER → MANAGER}}\\
            \hline
            \rowcolorhead
            \headertitle{Comando} & \headertitle{Descrizione} & \headertitle{Risposta} \\
            \hline
            \texttt{ADL,OK,ID} & Conferma aggiunta nuova lista di task\textsubscript{G}. ID indica l'identificativo della nuova lista & -- \\

            \texttt{ADL,FAIL,MSG} & Segnala errore nella creazione di una nuova lista di task\textsubscript{G}. & -- \\
            \multicolumn{3}{|c|}{Funzionamento analogo per \texttt{RML}}\\


            \hline
        \end{tabular}
        \caption{Comandi server → clients | Utente Manager (responsabile)}
    \end{table}

    \begin{table}[h!]
        \centering
        \begin{tabular}{|c|p{8cm}|c|}
            \hline
            \rowcolorhead
            \multicolumn{3}{|c|}{\headertitle{SERVER → ADMIN}}\\
            \hline
            \rowcolorhead
            \headertitle{Comando} & \headertitle{Descrizione} & \headertitle{Risposta} \\
            \hline
            \texttt{MAP,OK} & Conferma successo modifica mappa & -- \\
            \texttt{MAP,FAIL,MSG} & Modifica mappa fallita & -- \\
            \multicolumn{3}{|c|}{Analogamente a quelli sopra, stesso discorso vale per: \texttt{CELL, RMU, RMF}} \\
            \texttt{ADU,ID,PWD} & In risposta alla creazione di un utente

            \begin{itemize}
                \item ID che rappresenta il nuovo utente

                \item PWD password temporanea per il nuovo utente, il quale è tenuto a cambiarla tempestivamente
            \end{itemize}
            & -- \\

            \texttt{EDU,OK[,PWD]} & Modifica utente avvenuta con successo. PWD contiene la nuova password in caso di richiesta di reset. Anche in questo caso, l'utente una volta ricevuta la password dall'admin è tenuto a reimpostarla.
            & --\\

            \texttt{EDU,FAIL,MSG} & Modifica utente fallita
            & -- \\

            \texttt{ADF,OK,TOKEN} & Aggiunta di un nuovo muletto avvenuta con successo. Il TOKEN serve per la configurazione del nuovo muletto sul dispositivo client che verrà associato alla nuova unità & --\\
            \texttt{ADF,FAIL,MSG} & Aggiunta nuovo muletto fallita (esiste già muletto con l'id richiesto) & -- \\

            \texttt{LISTF,N,ID1,T1,ID2,T2…} & In risposta alla richiesta della lista dei muletti:
            \begin{itemize}
                \item N: num totale muletti

                \item IDn, Tn: rispettivamente id e token del muletto n
            \end{itemize}
            & -- \\

            \texttt{LISTU,N,ID1,UN1,UL1,R1…} & In risposta alla richiesta della lista degli utenti:
            \begin{itemize}
                \item N: num totale utenti

                \item ID1,UN…: rispettivamente: identificativo, nome, cognome e ruolo
            \end{itemize}
            & -- \\


            \hline
        \end{tabular}
        \caption{Comandi server → clients | Utente Admin (amministratore)}
    \end{table}


\begin{comment}
\begin{longtable}[h!]{|p{2cm}|p{8cm}|p{2cm}|}
\hline
\rowcolorhead
\multicolumn{3}{|c|}{\headertitle{FORKLIFTS}}\\
\hline
\rowcolorhead
\headertitle{Comando} & \headertitle{Descrizione} & \headertitle{Risposta} \\
\hline
POS,X,Y,DIR & posizione attuale del muletto, considerando la mappa come una matrice:\newline
\begin{itemize}
\item     X: riga della matrice
\item     Y: colonna ““
\item     DIR: orientamento assoluto secondo enum Orientation
\end{itemize}

& -- \\
LIST & richiede nuova lista di task\textsubscript{G} da completare & LIST... \\




\caption{prova}
\end{longtable}
\end{comment}














 %@Gregg
	\pagebreak

    %\begin{comment}
        	\section{Estendere PORTACS}
In questa sezione verranno riportate tutte quelle informazioni utili ad una semplice e corretta estensione del prodotto PORTACS. Possono essere estensioni legate sia a nuovi algoritmi e framework, più efficaci o efficienti, che a nuove categorie di un sotto-sistemi di PORTACS.














 %Tex - intro

        \subsection{Algoritmo alternativo per il path finding} %@Gregg

        \subsection{Introdurre nuove tipologie di utenti}
\subsubsection{Lato server}
Per introdurre nuove tipologie di utenti, bisogna estendere la classe astratta User e concretizzare in classi concrete, erediterà i campi principali, si può aggiungere altri metodi e attributi, ma è necessario fare l'override del metodo ProcessComunication, che si occupa di gestire la comunicazione, in particolare ricevere i messagi, eseguire quello che viene richiesto e inviare i dati necessari.
 %@Gregg

        \subsubsection{Lato client}
Per introdurre nuovi tipi di utente basterà creare un nuovo package dentro \texttt{Connection} lato Node del sistema, inserendo tutte le funzionalità necessarie al tipo d'utente. \\Successivamente bisognerà implementare le corrispondenti parti nel package \texttt{Services} e \texttt{Component} di Angular per la visualizzazione. \\Infine bisognerà inserire il nuovo tipo di utente dentro la classe di \texttt{Login}. %@TeamFE

        \subsection{Implementare tipi di persistenza alternativi} %@Gregg

        \subsection{Modificare handler nell'algoritmo di gestione delle collisioni}

La struttura fornita dal design pattern\textsubscript{G} \textit{Pipeline}, come anticipato nella \S \ref{collision-details}, consente di concatenare operazioni da eseguirsi sequenzialmente e il cui output di ognuna costituisce l'input della successiva. Per aggiungere un'operazione è necessario implementare l'interfaccia \texttt{Handler<I,O>} specificando i parametri di input (\texttt{I}) e di output (\texttt{O}) del nuovo \texttt{ConcreteHandler}. La logica dell'operazione si costruisce eseguendo l'\textit{Override} del metodo \texttt{Process(I input) : O}. La costruzione della pipeline prevederà l'aggiunta del nuovo \texttt{ConcreteHandler} tramite il metodo \texttt{addHandler} in modo che l'esecuzione, avviata invocando il metodo \texttt{execute}, includa la nuova operazione nella sua sequenza.

L'applicazione di questo design pattern\textsubscript{G} consente di modificare facilmente le operazioni che compongono la sequenza: se necessario, è possibile sostituirle anche tutte, cambiando di fatto l'implementazione dell'algoritmo; mantenendo però intatti i parametri di ingresso e uscita. %@Simone
        \pagebreak
    %\end{comment}



\end{document}
