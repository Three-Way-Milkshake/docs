\subsection{Version Control System e Continuous Integration}

\subsubsection{Git e gitflow}
Git è un sistema di controllo per il versionamento veloce ed efficiente\textsubscript{G}. Gitflow è un workflow che aiuta lo sviluppo software dando delle linee guida sui branch che strutturano le repo\textsubscript{A} e le operazioni per l'implementazione di feature e rilascio di releases.\\
Maggiori informazioni:
\begin{itemize}
    \item \textbf{Git: }\url{https://git-scm.com/};
    \item \textbf{gitflow: }\url{https://www.atlassian.com/git/tutorials/comparing-workflows/gitflow-workflow}.
\end{itemize}

\subsubsection{GitHub}
GitHub è un provider di hosting internet per lo sviluppo di software e il controllo della versione utilizzando Git. Fornisce un intero ecosistema di strumenti (version control, issue tracking, project boards, continuous integration\textsubscript{G} e delivery...) e permette la creazione di account personali o di organizzazioni.
\begin{itemize}
    \item \textbf{Maggiori informazioni su GitHub:} \url{https://github.com/about};
    \item \textbf{\group{} su GitHub: }\url{https://github.com/Three-Way-Milkshake}.
\end{itemize}

\subsubsection{GitHub Actions}
È uno strumento integrato in ogni repo\textsubscript{A} di GitHub, permette di creare singole attività\textsubscript{G} combinabili al fine di realizzare complessi workflow personalizzati. 
Si possono creare nuove \textit{actions} ed eventualmente pubblicarle, o utilizzare il vasto catalogo di automazioni già realizzate dalla community.
\begin{itemize}
    \item \textbf{Documentazione: } \url{https://docs.github.com/en/actions}.
\end{itemize}