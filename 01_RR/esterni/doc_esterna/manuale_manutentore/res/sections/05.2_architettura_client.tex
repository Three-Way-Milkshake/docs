\subsection{Client}
\subsubsection{Diagramma di classe per l'unità}

\begin{figure}[H]
	\centering
	\includegraphics[scale=0.5]{res/images/UML_operatore.png}
	\caption{Diagramma UML delle classi per l'unità}
\end{figure}

Qui utilizziamo due tecnologie: Node per quanto riguarda il package \texttt{connection} e Angular per il resto. Queste due parti del front end comunicano attraverso il package esterno \texttt{Socket.io}, necessario gestire un flusso di dati attraverso i socket.\\
Il package \texttt{services}, come descritto anche nella documentazione di Angular, fa da intermediario tra il package \texttt{connection} e il package \texttt{component}, utilizzando degli Observer in ascolto di uno specifico socket e instradando l'informazione verso l'opportuno component.\\
Il package \texttt{component} permette di visualizzare sullo schermo le informazioni richieste grazie anche ai template di Angular. Ogni classe di questo package serve ad una specifica funzionalità:
\begin{itemize}
	\item \texttt{Map} → visualizza la mappa del magazzino con la posizione in real time dell'unità;
	\item \texttt{StartButton} → mostra un bottone che serve a far partire l'unità;
	\item \texttt{TaskList} → mostra la lista di task che l'operatore dovrà compiere;
	\item \texttt{Arrows} → visualizza le azioni che compie l'unità in real time;
	\item \texttt{ManualDrive} → permette di cambiare guida da manuale ad automatica e viceversa, facendo visualizzare anche i pulsanti da premere per far muovere l'unità manualmente in caso;
	\item \texttt{AdminNotification} → visulizza un pulsante che, se premuto, notifica all'admin un evento eccezionale;
	\item \texttt{ComeBack} → mostra un pulsante alla fine del turno dell'operatore che, se premuto, guida automaticamente il muletto verso la propria base.
\end{itemize}
Il package connection, attraverso le classi \texttt{Index} e \texttt{CommandsToJava}, instaura inoltre una comunicazione TCP Socket con java, permettendo di creare una connessione tra frontend e backend.\\

\subsubsection{Diagramma di classe per l'admin-manager}

\begin{figure}[H]
	\centering
	\includegraphics[scale=0.6]{res/images/UML_admin-manager.png}
	\caption{Diagramma UML delle classi per gli admin e i manager}
\end{figure}
Il diagramma di classe dell'admin-responsabile è molto simile a quello dell'unità: si avvale delle tecnologie Node, con il package \texttt{connection}, ed Angular, con tutti gli altri package.\\
Il contesto dei package è lo stesso dell'unità, di cui però si fa una differenziazione tra le funzionalità del'admin e quelle del manager grazie alla classe \texttt{Login} e al routing di \texttt{AppRouting} presente nel package \texttt{component}: permette all'utente di effettuare il login, "attivando" solamente le funzionalità riferite al tipo di utente loggato.\\
Il package \texttt{generic} contiene classi di funzionalità condivise tra manager e admin.\\
Le classi presenti in component permettono di attivare certe classi riferite al package (e quindi alle funzionalità di uno specifico tipo di utente):
\begin{itemize}
	\item admin:
	\begin{itemize}
		\item \texttt{ListUsers} → aggiunta o rimozione di manager;
		\item \texttt{EventAlert} → visualizzazione eventi eccezionali;
		\item \texttt{ManageMap} → modifica la planimetria e le caratteristiche della mappa;
	\end{itemize}
	\item manager:
	\begin{itemize}
		\item \texttt{TaskLists} → visualizzazione delle liste di task;
		\item \texttt{MangeListsTask} → aggiunta, modifica e rimozione di liste di task;
	\end{itemize}
	\item generic (sia per admin che per manager):
	\begin{itemize}
		\item \texttt{ViewMap} e \texttt{POIList} → visualizzazione in real time di tutte le unità nel magazzino.
	\end{itemize}
\end{itemize}



