\subsection{Server}

\subsubsection{Tecnologie}
\label{tecnologie}

\begin{itemize}
	\item \textbf{Java}: linguaggio di programmazione ad alto livello, orientato agli oggetti e a tipizzazione statica, si appoggia sull'omonima piattaforma software di esecuzione (Java Virtual machine), specificamente progettato per essere il più possibile indipendente dalla piattaforma hardware di esecuzione. La componente server del software è realizzata interamente con questo linguaggio.
	\begin{itemize}
		\item Versione utilizzata: JavaSE-15
		\item Documentazione: \url{https://docs.oracle.com/en/java/javase/15/}
	\end{itemize}
	Per il download e l'installazione si rimanda alla documentazione:
	\begin{itemize}
		\item \url{https://jdk.java.net/java-se-ri/15}
	\end{itemize}

	\item \textbf{JSON}: acronimo di JavaScript Object Notation, è un formato di conservazione e invio di dati. Si basa su oggetti, ovvero coppie chiave/valore, e supporta i tipi booleano, stringa, numero, e lista. È semplice e leggibile ad occhio umano, inoltre non necessita di alcun processo di
	compilazione particolare per essere modificato. Nel software viene utilizzato come persistenza per la conservazione di dati.
	\begin{itemize}
		\item Documentazione: \url{https://www.json.org/json-it.html}
	\end{itemize}

	\item \textbf{Docker}: piattaforma software che permette di creare, testare e distribuire applicazioni. Docker raccoglie il software in unità standardizzate chiamate container che offrono tutto il necessario per la loro corretta esecuzione, incluse librerie, strumenti di sistema, codice e runtime. Le componenti client e server del software sono distribuite in containter instanziabili negli ambienti in cui si vuole eseguire il programma.
	\begin{itemize}
		\item Versione utilizzata: 19.03.*
		\item Documentazione: \url{https://docs.docker.com/}
	\end{itemize}
	Per il download e l'installazione si rimanda al seguente link: \url{https://docs.docker.com/get-docker/}
	
	\begin{itemize}
	\item \textbf{Installazione con Linux}\\
	Per poter installare Docker su Linux è necessario dare dal terminale il comando \texttt{sudo apt install docker}.
		
	\item \textbf{Installazione con Mac}\\Per poter installare Docker su Mac è necessario scaricare il file installante, prestando attenzione alla versione del processore della quale si dispone, a questo link: \url{https://docs.docker.com/docker-for-mac/install/}. 
	\\Una volta scaricato basta avviarlo e procedere con l'installazione. \\Per ulteriori informazioni leggere qui: \url{https://docs.docker.com/docker-for-mac/install/#install-and-run-docker-desktop-on-mac}.
	
	\item \textbf{Installazione con Windows}\\
		Per poter installare Docker su Windows è necessario scaricare il file installante (direttamente da qui: \url{https://desktop.docker.com/win/stable/amd64/Docker\%20Desktop\%20Installer.exe}) e avviarlo, dopo essersi assicurati che sia abilitata l'opzione per la funzionalità di Windows Hyper-V o sarà necessario installare i componenti Windows necessari per WSL 2 e che siano selezionati nella pagina di "Configurazione". Per ulteriori informazioni leggere qui: \url{https://docs.docker.com/docker-for-windows/install/#install-docker-desktop-on-windows}.
	\end{itemize}

		
	\item \textbf{Gradle}: sistema open source per l'automazione dello sviluppo fondato sulle idee di Apache Ant e Apache Maven, introduce un domain-specific language (DSL) basato su Groovy, al posto della modalità XML usata da Apache Maven per dichiarare la configurazione del progetto\textsubscript{G}. Proprio come avviene con Apache Maven, la struttura di Gradle è costituita da un nucleo astratto e da una serie di plugin che ne espandono le funzionalità; al contrario di Maven, però, offre possibilità di definire il meccanismo di costruzione in linguaggio Groovy, nel file build, file che risulterà più leggero dell'equivalente XML e con una notazione più compatta per descrivere le dipendenze.
	\begin{itemize}
		\item Versione utilizzata: 6.7
		\item Documentazione: \url{https://docs.gradle.org/current/userguide/userguide.html}
	\end{itemize}
	Per il download e l'installazione si rimanda alla documentazione:
	\begin{itemize}
		\item \url{https://docs.gradle.org/current/userguide/installation.html}
	\end{itemize}

\end{itemize}

\subsubsection{Librerie e Framework}

\begin{itemize}
	\item \textbf{Spring}: framework\textsubscript{G} open source per lo sviluppo di applicazioni su piattaforma Java. Fornisce una serie completa di strumenti per gestire la complessità dello sviluppo software, fornendo un approccio semplificato ai più comuni problemi di sviluppo e di testing; inoltre, grazie alla sua struttura estremamente modulare, è possibile utilizzarlo nella sua interezza o solo in parte, senza stravolgere l’architettura del progetto\textsubscript{G}.
	\begin{itemize}
		\item Documentazione: \url{https://spring.io/}
	\end{itemize}
	\item \textbf{Gson}: libreria Java che può essere utilizzata per convertire gli oggetti Java nella loro rappresentazione JSON. Può anche essere utilizzato per convertire una stringa JSON in un oggetto Java equivalente. Nel contesto del software, viene utilizzato per il la serializzazione e deserielizzazione dei file Json della persistenza.
	\begin{itemize}
	\item Documentazione: \url{https://github.com/google/gson/blob/master/UserGuide.md}
	\end{itemize}
	\item \textbf{JUnit}: framework\textsubscript{G} di unit testing per il linguaggio di programmazione Java.
	\begin{itemize}
		\item Versione utilizzata: 5.7
		\item Documentazione: \url{https://junit.org/junit5/docs/current/user-guide/}
	\end{itemize}

	\item \textbf{Mockito}: framework\textsubscript{G} open source di test per il linguaggio di programmazione Java. Il framework\textsubscript{G} consente la creazione di oggetti fittizi (Mock Objects) in test di unità automatizzati. I Mock Objects sono oggetti simulati che imitano il comportamento di componenti reali in un ambiente controllato.
		\begin{itemize}
		\item Versione utilizzata: 3.*
		\item Documentazione: \url{https://junit.org/junit5/docs/current/user-guide/}
	\end{itemize}

\end{itemize}