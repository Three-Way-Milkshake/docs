\section{Setup}
Il progetto PORTACS è composto da 3 container Docker differenti, uno rappresentante il backend, un altro creato per l'operatore dell'unità e l'ultimo ideato per l'admin o i manager. Il container per l'unità sarà installato su un muletto, quello per l'admin o i manager su un personal computer e quello rappresentante il backend su un server. I contenitori del client funzioneranno su un browser che supporti bene HTML e CSS.


\subsection{Requisiti di sistema}
Sotto elencati saranno descritti i requisiti minimi del sistema per un corretto funzionamento del software PORTACS.


\subsubsection{Requisiti Hardware}
\begin{itemize}
	\item Client -> unità:
\begin{itemize}
	\item CPU single-core o maggiore;
	\item Memoria Ram >= 2GB;
	\item definizione schermo >= 640 x 480 (Standard Definition);
	\item connessione internet con bassi tempi di risposta.
\end{itemize}
	\item Client -> admin o responsabile:
\begin{itemize}
	\item CPU dual-core o maggiore;
	\item Memoria Ram >= 3GB;
	\item definizione schermo >= 1280 x 720 pixel (High Definition);
	\item connessione internet con bassi tempi di risposta.
\end{itemize}
	\item Server:
\begin{itemize}
	\item CPU dual-core o maggiore;
	\item Memoria Ram >= 4GB;
	\item connessione internet con bassi tempi di risposta.
\end{itemize}
\end{itemize}

\subsubsection{Requisiti Software}
\begin{itemize}
	\item Client:
\begin{itemize}
	\item Docker.
	%\item Chrome Versione 90.0.4430.85 (Build ufficiale) (a 64 bit);
	%\item Node.js;
	%\item Angular;
	
	%\item Windows 10.
\end{itemize}
	\item Server:
\begin{itemize}
	\item Docker.
	%\item Java;
	%\item gradle;
	
	%\item Windows 10.
\end{itemize}
\end{itemize}







\subsection{Installazione}
%TODO



