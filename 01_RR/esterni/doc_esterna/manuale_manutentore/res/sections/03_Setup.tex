\section{Setup}
PORTACS viene distribuito tramite container Docker, per cui i dispositivi sui quali dovrà eseguire avranno minimi requisiti software. È possibile utilizzare i container anche per la fase\textsubscript{G} di sviluppo, altrimenti si possono scaricare ed utilizzare gli strumenti descritti in \ref{tecnologie} per l'esecuzione diretta in locale. PORTACS\textsubscript{A} si divide su tre immagini Docker:
\begin{enumerate}
    \item server;
    \item client muletto (forklift);
    \item client utente (user).
\end{enumerate}


\subsection{Requisiti di sistema}
Sotto elencati saranno descritti i requisiti minimi del sistema per un corretto funzionamento del software PORTACS\textsubscript{A}.


\subsubsection{Requisiti Hardware}
\begin{itemize}
	\item Client → unità:
\begin{itemize}
	\item CPU\textsubscript{A} dual-core o maggiore;
	\item memoria Ram >= 4GB;
	\item connessione con bassi tempi di risposta.
\end{itemize}
	\item Client → admin o responsabile:
\begin{itemize}
	\item CPU\textsubscript{A} dual-core o maggiore;
	\item memoria Ram >= 4GB;
	\item connessione con bassi tempi di risposta.
\end{itemize}
	\item Server:
\begin{itemize}
	\item CPU\textsubscript{A} quad-core o maggiore;
	\item memoria Ram >= 8GB;
	\item connessione con bassi tempi di risposta.
\end{itemize}
\end{itemize}

\subsubsection{Requisiti Software}
    \pparagraph{Esecuzione}
    \begin{itemize}
        \item Docker (v19.03.*);
        \item Google Chrome (v90).
    \end{itemize}

    \pparagraph{Sviluppo}
    Vedi \S\ \ref{tecnologie}

    \begin{comment}
    	\item Client:
    \begin{itemize}
    	\item Docker.
    	%\item Chrome Versione 90.0.4430.85 (Build ufficiale) (a 64 bit);
    	%\item Node.js;
    	%\item Angular;

    	%\item Windows 10.
    \end{itemize}
    	\item Server:
    \begin{itemize}
    	\item Docker.
    	%\item Java;
    	%\item gradle;

    	%\item Windows 10.
    \end{itemize}
    \end{comment}







\subsection{Installazione ed avvio degli applicativi}
    \subsubsection{Server}
    Nella macchina da utilizzare come server andrà scaricata l'immagine di portacs-server con la versione desiderata \href{https://hub.docker.com/r/threewaymilkshake}{docker hub di \group}.
    Per l'avvio, in ambiente Linux e MacOS:
    \begin{itemize}
        \item predisporre una cartella che contenga un file \texttt{config.txt} ed una cartella \texttt{resources} che manterrà la persistenza;
        \item posizionarsi su tale cartella;
        \item eseguire:
    \begin{verbatim}
    docker run -v $(pwd)/resources:/resources \
        --env-file config.txt threewaymilkshake/portacs-server
    \end{verbatim}
    \end{itemize}
    All'interno del file \texttt{config.txt} si possono specificare (come coppie key=value) delle configurazioni diverse da quelle di default per quanto riguardo percorso della persistenza e porta sulla quale il server dovrà esporre il Server Socket per il collegamento. Se non si desiderano modificare i valori di default il file e la relativa parte nel comando possono essere omessi.

    \subsubsection{Client}
    Come per il server, scaricare l'immagine dal docker hub di \group{} relativa al client voluto (forklift o user). Dopodiché sul dispositivo che dovrà fungere da client andranno impostata la configurazione in un file \texttt{config.txt}:
    \begin{itemize}
        \item per i client \textit{user} sarà sufficiente specificare l'indirizzo IP della macchina server così: \texttt{SERVER\_ADDR=ip};
        \item per i client \texttt{forklift}, oltre all'indirizzo come per gli utenti, bisognerà aggiungere altre 2 righe per la configurazione di ogni muletto:
        \begin{itemize}
            \item \texttt{ID=id del muletto};
            \item \texttt{TOKEN=token del muletto}.
        \end{itemize}
        Queste informazioni sono a disposizione degli admin.
    \end{itemize}

    Dopodiché, in entrambi i casi, sarà sufficiente eseguire:
    \begin{verbatim}
        docker run --env-file config.txt threewaymilshake/portacs-client-<type>
    \end{verbatim}
    con \texttt{<type>} tra \texttt{forklift} o \texttt{user}.


