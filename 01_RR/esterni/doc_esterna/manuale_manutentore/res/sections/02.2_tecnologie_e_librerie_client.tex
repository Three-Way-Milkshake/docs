\subsection{Client}

\subsubsection{Tecnologie}

\begin{itemize}
	\item \textbf{Node.js} \\
	Node.js è un runtime system open source multipiattaforma orientato agli eventi per l'esecuzione di codice Javascript. Molti dei suoi moduli base sono scritti in Javascript. 
	\begin{itemize}
		\item Versione utilizzata: 14.15.5
		\item Link per download: \url{https://nodejs.org/it/download/}
	\end{itemize}
	
	\item \textbf{HTML} \\
	HTML è un linguaggio markup per la strutturazione di pagine web. Viene utilizzato insieme a Angular per la costruzione della struttura della web app.

	\item \textbf{CSS} \\
	CSS è un linguaggio utilizzato per definire la formattazione di documenti HTML, XHTML e XML, ad esempio i siti web e le relative pagine web. L'uso del CSS permette la separazione dei conte-nuti delle pagine HTML dal loro layout e permette una programmazione più chiara e facile da utilizzare, sia per gli autori delle pagine stesse sia per gli utenti, garantendo contem-poraneamente anche il riutilizzo di codice ed una sua più facile manutenzione. Viene utilizzato insieme ad Angular per la stilizzazione degli elementi HTML.
	\item \textbf{Typescript}\\
	TypeScript è un linguaggio di programmazione open-source. Si tratta di un super-set di JavaScript che basa le sue caratteristiche su ECMAScript 6. Il linguaggio estende la sintassi di JavaScript in modo che qualunque programma scritto in JavaScript sia anche in grado di funzionare con TypeScript senza nessuna modifica. È progettato per lo sviluppo di grandi applicazioni ed è destinato a essere compilato in JavaScript per poter essere interpretato da qualunque web browser o app. Viene utilizzato insieme ad Angular per la codifica del comportamento della webapp.
	\begin{itemize}
		\item Versione utilizzata: 3.8 o superiore.
	\end{itemize}

\end{itemize}

\subsubsection{Librerie e Framework}

\begin{itemize}
	\item \textbf{Angular} \\
	Angular è un framework open source per lo sviluppo di applicazioni web a single page. Esso si basa sul pattern MVVM. Si basa sul linguaggio di programmazione TypeScript e sulla creazione di componenti che costruiscono la pagina web. Le applicazioni sviluppate in Angular vengono eseguite interamente dal web browser dopo essere state scaricate dal web server (elaborazione lato client). Questo comporta il risparmio di dover spedire indietro la pagina web al web-server ogni volta che c'è una richiesta di azione da parte dell'utente.
	\begin{itemize}
		\item Versione utilizzata: 11.2.0
		\item Link per installazione: \url{https://angular.io/guide/setup-local}
	\end{itemize}
 	\item \textbf{PrimeNG} \\
 	PrimeNG è una libreria per Angular per customizzare i componenti così da creare un'interfaccia utente più accattivante.
	 \begin{itemize}
		\item Link per installazione: \url{https://primefaces.org/primeng/showcase/#/setup}
	\end{itemize}

	
\end{itemize}