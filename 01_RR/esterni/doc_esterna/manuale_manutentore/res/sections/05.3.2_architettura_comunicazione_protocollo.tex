\subsubsection{Protocollo di comunicazione}
\label{comm-protocol}

Ogni stringa può contenere uno o più comandi, separati da ‘;' e ogni comando può avere 0 o più parametri, separati da ',’. \\
\textbf{Esempio sequenza:} \texttt{POS,1,1,0;PATH,1}

\pparagraph{Connessione: identificazione e login}
    Quando un client si connette deve essere identificato come tipo ed autenticato, perciò deve inviare separatamente ed in sequenza:
    \begin{enumerate}
        \item \textbf{TYPE: }\texttt{FORKLIFT} o \texttt{USER};
        \item \textbf{ID: } identificativo personale;
        \item \textbf{PWD/TOKEN: } password o token a seconda che sia rispettivamente un utente o un muletto.
    \end{enumerate}
    Quindi riceverà una risposta tra:
    \begin{itemize}
        \item \textbf{OK } se l'autenticazione è andata a buon fine;
        \item \textbf{FAIL,msgErrore} se è fallita, dove msgErrore conterrà maggiori dettagli sulla causa.
    \end{itemize}
    \subparagraph{Esempio connessione ed autenticazione muletto}
        Dato un muletto con id=f1 e token=abcdef:
        \begin{itemize}
            \item invia: \texttt{FORKLIFT\textbackslash nf1\textbackslash nabcdef};

            \item riceve: \texttt{OK} oppure \texttt{FAIL,messaggioErrore}
        \end{itemize}


        Funzionamento analogo per gli utenti con password al posto del token.

\pparagraph{Enumerazioni}
    Di seguito si farà riferimento più volte ai diversi tipi enum presenti nella logica di business, per cui segue un riassunto:


    \begin{table}[h!]
        \centering
        \begin{tabular}{|c|c|c|c|c|}
            \hline
            \rowcolorhead
            \multicolumn{5}{|c|}{\headertitle{ENUM}}\\
            \hline
            \rowcolorhead
            \headertitle{↓Val \textbackslash{} Enum→} & \headertitle{PoiType} & \headertitle{Move}       & \headertitle{Orientation} & \headertitle{CellType} \\
            0          & LOAD    & GOSTRAIGHT & UP          & OBSTACLE \\
            1          & UNLOAD  & TURNAROUND & RIGHT       & NEUTRAL \\
            2          & EXIT    & TURNRIGHT  & DOWN        & UP \\
            3          & --      & TURNLEFT   & LEFT        & RIGHT \\
            4          & --      & STOP       & --          & DOWN \\
            5          & --      & --         & --          & LEFT \\
            6          & --      & --         & --          & POI \\ [1ex]
            \hline
        \end{tabular}
        \caption{Riepilogo enumerazioni}
    \end{table}
\newcommand{\tabitem}{~~\llap{\textbullet}~~}


\pparagraph{Comandi clients → server}
    \begin{table}[h!]
        \centering
        \begin{tabular}{|c|p{8cm}|c|}
            \hline
            \rowcolorhead
            \multicolumn{3}{|c|}{\headertitle{FORKLIFTS → SERVER}}\\
            \hline
            \rowcolorhead
            \headertitle{Comando} & \headertitle{Descrizione} & \headertitle{Risposta} \\
            \hline
            POS,X,Y,DIR & Posizione attuale del muletto, considerando la mappa come una matrice:
            \begin{itemize}
                \item X: riga della matrice
                \item Y: colonna ““
                \item DIR: orientamento assoluto secondo enum Orientation
            \end{itemize}

            & -- \\
            LIST & Richiede nuova lista di task da completare & LIST,... \\

            PATH,C & Richiede il percorso migliore per raggiungere il POI della task corrente a partire dalla posizione attuale. Se C=1 rimuove la task e passa alla successiva & PATH,... \\



            last & last & last\\ [1ex]


        \end{tabular}
        \caption{Comandi clients → server | Forklifts}
    \end{table}

    \begin{table}[h!]
        \centering
        \begin{tabular}{|c|p{8cm}|c|}
            \hline
            \rowcolorhead
            \multicolumn{3}{|c|}{\headertitle{USER generico → SERVER}}\\
            \hline
            \rowcolorhead
            \headertitle{Comando} & \headertitle{Descrizione} & \headertitle{Risposta} \\
            \hline
            EDIT,T,PAR & Modifica dati del proprio profilo
            \begin{itemize}
                \item T: NAME/LAST

                \item PAR: nuova valore per T
            \end{itemize}
             & -- \\

             LOGOUT & Richiesta di disconnessione & -- \\

            last & last & last\\ [1ex]


        \end{tabular}
        \caption{Comandi clients → server | User generico}
    \end{table}

    \begin{table}[h!]
        \centering
        \begin{tabular}{|c|p{8cm}|c|}
            \hline
            \rowcolorhead
            \multicolumn{3}{|c|}{\headertitle{MANAGER → SERVER}}\\
            \hline
            \rowcolorhead
            \headertitle{Comando} & \headertitle{Descrizione} & \headertitle{Risposta} \\
            \hline
            ADL,P1,P2,… & Aggiunge una nuova lista di task
            \begin{itemize}
                \item ID: identificativo della lista

                \item P1..: id dei poi che compongono la lista
            \end{itemize}
            & ADL,...\\

            RML,ID & Richiede la cancellazione della lista con id=ID & RML,... \\


            last & last & last\\ [1ex]
            \hline


        \end{tabular}
        \caption{Comandi clients → server | User Manager (Responsabile)}
    \end{table}

    %\begin{table}[h!]
        %\centering
        \begin{longtable}[h!]{|c|p{8cm}|c|}
            \endfirsthead
            \hline
            \rowcolorhead
            \multicolumn{3}{|c|}{\headertitle{ADMIN → SERVER}}\\
            \hline
            \rowcolorhead
            \headertitle{Comando} & \headertitle{Descrizione} & \headertitle{Risposta} \\
            \hline
            \endhead

            \texttt{MAP,R,C,SEQ} & Nuova planimetria
            \begin{itemize}
                \item R: num righe
                \item C: “ colonne
                \item SEQ: sequenza di interi corrispondenti all’enum CellType rappresentanti stati di una cella, indicanti la nuova planimetria, elencati per righe
            \end{itemize}
            & \texttt{MAP,...} \\

            \texttt{CELL,X,Y,A[,T,NAME]} & Modifica una cella, la parte tra [ ] è presente solo in caso di POI
            \begin{itemize}
                \item X e Y riga e colonna della matrice

                \item A: numero rappresentante l'azione  da intraprendere, corentemente con CellType
            \end{itemize}

            Solo nel caso di POI:
            \begin{itemize}
                \item T: tipo di POI secondo PoiType

                \item NAME: stringa di caratteri da associare al POI
            \end{itemize}
            & \texttt{CELL,...} \\

            \texttt{ADU,T,NAME,LAST} & aggiunge nuovo utente
            \begin{itemize}
                \item T: ADMIN o MANAGER

                \item NAME e LAST nome e cognome
            \end{itemize}
            & \texttt{ADU,...} \\

            \texttt{RMU,ID} & Rimuove l’utente con id=ID & \texttt{RMU,...} \\

            \texttt{EDU,ID,A,PAR} & Modifica l’utente con id=ID, con
            \begin{itemize}
                \item A: azione da intraprendere tra:

                    \subitem -- NAME: modifica nome

                    \subitem -- LAST: modifica cognome

                    \subitem -- RESET: esegue reset della password
                \item PAR: nuovo valore da assegnare (assente in caso di reset)
            \end{itemize}
            & \texttt{EDU,...} \\

            \texttt{ADF,ID} & Aggiunge nuovo muletto. ID: stringa che si vuole assegnare come identificativo al nuovo muletto (NON sarà più modificabile) & \texttt{ADF,...} \\

            \texttt{RMF,ID} & rimuove il muletto con id=ID & \texttt{RMF,...} \\

            \texttt{LISTF} & Richiede lista di tutti i muletti registrati & \texttt{LISTF,...} \\

            \texttt{LISTU} & Richiede lista di tutti gli utenti registrati & \texttt{LISTU,...} \\

            last & last & last\\ [1ex]
            \hline
        \caption{Comandi clients → server | User Admin (Amministratore)}
        \end{longtable}

    %\end{table}

\pparagraph{Comandi server → clients}
    Tutti i comandi che possono ritornare un esito positivo o negativo hanno 2 varianti:
    \begin{itemize}
        \item \texttt{CMD,OK,MORE}: successo, eventualmente MORE contiene parametri di risposta aggiuntivi
        \item \texttt{CMD,FAIL,MSG}: fallimento, MSG contiene maggiori informazioni sulle cause.
    \end{itemize}
    \begin{table}[h!]
        \centering
        \begin{tabular}{|c|p{8cm}|c|}
            \hline
            \rowcolorhead
            \multicolumn{3}{|c|}{\headertitle{SERVER → CLIENTS generici}}\\
            \hline
            \rowcolorhead
            \headertitle{Comando} & \headertitle{Descrizione} & \headertitle{Risposta} \\
            \hline
            \texttt{MAP,R,C,SEQ} & Indica la planimetria
            \begin{itemize}
                \item R e C: numero di righe e colonne

                \item SEQ: sequenza di interi corrispondenti all’enum CellType rappresentanti stati di una cella, indicanti la nuova planimetria, elencati per righe
            \end{itemize}
            & --\\
            \texttt{POI,N,X,Y,T,ID,NAME} & rappresenta tutti i poi, la parte da X in poi si ripete per ogni POI da rappresentare
            \begin{itemize}
                \item N: num totale POI

                \item X,Y e T posizione nella matrice e tipo secondo PoiType

                \item ID e NAME: rispettivamente identificativo e nome del POI
            \end{itemize}
            & -- \\

            \hline
        \end{tabular}
        \caption{Comandi server → clients | generici per tutti i client}
    \end{table}

    \begin{table}[h!]
        \centering
        \begin{tabular}{|c|p{8cm}|c|}
            \hline
            \rowcolorhead
            \multicolumn{3}{|c|}{\headertitle{SERVER → FORKLIFT}}\\
            \hline
            \rowcolorhead
            \headertitle{Comando} & \headertitle{Descrizione} & \headertitle{Risposta} \\
            \hline
            \texttt{ALIVE} & Ha lo scopo primario di verificare l’integrità della connessione ed ha come effetto l’ottenimento della nuova posizione. Se l’invio fallisce il muletto corrispondente viene considerato disconnesso, l’oggetto Connection relativo verrà chiuso e distrutto e il muletto dovrà riautenticarsi. Si presuppone che questo non si muova più finché la connessione non viene ristabilita, in quanto i suoi spostamenti sarebbero sconosciuti al server e questo non potrebbe intervenire per evitare eventuali collisioni.

            Ad esso possono seguire ulteriori comandi o risposte a comandi precedente, secondo la sintassi generale per cui separati da ';'
            & \texttt{POS,...} \\

            \texttt{LIST,ID1,ID2…} & Invia lista di task assegnate. ID1.. sono gli id dei POI da raggiungere & -- \\

            \texttt{PATH,SEQ} & Invia il percorso per raggiungere il prossimo POI, composto da mosse successive secondo Move & -- \\

            \texttt{STOP,N} & Richiede lo stop immediato del muletto per N istanti, che verranno contati alle ricezioni dei futuri ALIVE. Se N=0 stop indefinito fino alla ricezione di \texttt{START}. & -- \\
            \texttt{START} & consente ad un muletto fermato a tempo indeterminato di ripartire & -- \\

            \hline
        \end{tabular}
        \caption{Comandi server → clients | Forklifts (muletti)}
    \end{table}

    \begin{table}[h!]
        \centering
        \begin{tabular}{|c|p{8cm}|c|}
            \hline
            \rowcolorhead
            \multicolumn{3}{|c|}{\headertitle{SERVER → USER generico}}\\
            \hline
            \rowcolorhead
            \headertitle{Comando} & \headertitle{Descrizione} & \headertitle{Risposta} \\
            \hline
            \texttt{UNI,N,ID1,X1,Y1,D1} & Indica le posizioni di tutti i muletti. Da ID1 in poi si ripete per ogni muletto
            \begin{itemize}
                \item N: num totale dei muletti attivi per i quali si sta ricevendo la posizione

                \item IDn: id del muletto n

                \item Xn, Yn e Dn: posizione rispetto alla matrice e orientamento secondo Orientation del muletto IDn.
            \end{itemize}
            Se l'invio di questo fallisce, l'utente viene considerato disconnesso. & -- \\

            \texttt{LIST,IDF,N,IDP1,IDP2…} & Indica la lista di task presa in carico da un muletto
            \begin{itemize}
                \item IDF: id del muletto a cui ci si sta riferendo

                \item N: numero di task prese in carico

                \item IDP1…: sequenza di id dei POI da raggiungere
            \end{itemize}
            & -- \\


            \hline
        \end{tabular}
        \caption{Comandi server → clients | User generico}
    \end{table}


    \begin{table}[h!]
        \centering
        \begin{tabular}{|c|p{8cm}|c|}
            \hline
            \rowcolorhead
            \multicolumn{3}{|c|}{\headertitle{SERVER → MANAGER}}\\
            \hline
            \rowcolorhead
            \headertitle{Comando} & \headertitle{Descrizione} & \headertitle{Risposta} \\
            \hline
            \texttt{ADL,OK,ID} & Conferma aggiunta nuova lista di task. ID indica l'identificativo della nuova lista & -- \\

            \texttt{ADL,FAIL,MSG} & Segnala errore nella creazione di una nuova lista di task. & -- \\
            \multicolumn{3}{|c|}{Funzionamento analogo per \texttt{RML}}\\


            \hline
        \end{tabular}
        \caption{Comandi server → clients | Utente Manager (responsabile)}
    \end{table}

    \begin{table}[h!]
        \centering
        \begin{tabular}{|c|p{8cm}|c|}
            \hline
            \rowcolorhead
            \multicolumn{3}{|c|}{\headertitle{SERVER → ADMIN}}\\
            \hline
            \rowcolorhead
            \headertitle{Comando} & \headertitle{Descrizione} & \headertitle{Risposta} \\
            \hline
            \texttt{MAP,OK} & Conferma successo modifica mappa & -- \\
            \texttt{MAP,FAIL,MSG} & Modifica mappa fallita & -- \\
            \multicolumn{3}{|c|}{Analogamente a quelli sopra, stesso discorso vale per: \texttt{CELL, RMU, RMF}} \\
            \texttt{ADU,ID,PWD} & In risposta alla creazione di un utente

            \begin{itemize}
                \item ID che rappresenta il nuovo utente

                \item PWD password temporanea per il nuovo utente, il quale è tenuto a cambiarla tempestivamente
            \end{itemize}
            & -- \\




            \hline
        \end{tabular}
        \caption{Comandi server → clients | Utente Admin (amministratore)}
    \end{table}


\begin{comment}
\begin{longtable}[h!]{|p{2cm}|p{8cm}|p{2cm}|}
\hline
\rowcolorhead
\multicolumn{3}{|c|}{\headertitle{FORKLIFTS}}\\
\hline
\rowcolorhead
\headertitle{Comando} & \headertitle{Descrizione} & \headertitle{Risposta} \\
\hline
POS,X,Y,DIR & posizione attuale del muletto, considerando la mappa come una matrice:\newline
\begin{itemize}
\item     X: riga della matrice
\item     Y: colonna ““
\item     DIR: orientamento assoluto secondo enum Orientation
\end{itemize}

& -- \\
LIST & richiede nuova lista di task da completare & LIST... \\




\caption{prova}
\end{longtable}
\end{comment}














