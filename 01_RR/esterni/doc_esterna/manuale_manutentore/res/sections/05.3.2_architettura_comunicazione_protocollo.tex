\subsubsection{Protocollo di comunicazione}
\label{comm-protocol}

Ogni stringa può contenere uno o più comandi, separati da ‘;' e ogni comando può avere 0 o più parametri, separati da ',’.
\textbf{esempio sequenza:} POS,1,1,0;PATH,1

\pparagraph{Connessione: identificazione e login}
    Quando un client si connette deve essere identificato come tipo ed autenticato, perciò deve inviare separatamente ed in sequenza:
    \begin{enumerate}
        \item \textbf{TYPE: }FORKLIFT o USER;
        \item \textbf{ID: } identificativo personale;
        \item \textbf{PWD/TOKEN: } password o token a seconda che sia rispettivamente un utente o un muletto.
    \end{enumerate}
    Quindi riceverà una risposta tra:
    \begin{itemize}
        \item \textbf{OK } se l'autenticazione è andata a buon fine;
        \item \textbf{FAIL,msgErrore} se è fallita, dove msgErrore conterrà maggiori dettagli sulla causa.
    \end{itemize}
    \subparagraph{Esempio connessione ed autenticazione muletto}
        Dato un muletto con id=f1 e token=abcdef:
        \begin{itemize}
            \item invia: FORKLIFT\textbackslash nf1\textbackslash nabcdef;

            \item riceve: OK oppure FAIL,messaggioErrore
        \end{itemize}


        Funzionamento analogo per gli utenti con password al posto del token.

\pparagraph{Enumerazioni}
    Di seguito si farà riferimento più volte ai diversi tipi enum presenti nella logica di business, per cui segue un riassunto:


    \begin{table}[h!]
        \centering
        \begin{tabular}{|c|c|c|c|c|}
            \hline
            \rowcolorhead
            \multicolumn{5}{|c|}{\headertitle{ENUM}}\\
            \hline
            \rowcolorhead
            \headertitle{↓Val \textbackslash{} Enum→} & \headertitle{PoiType} & \headertitle{Move}       & \headertitle{Orientation} & \headertitle{CellType} \\
            0          & LOAD    & GOSTRAIGHT & UP          & OBSTACLE \\
            1          & UNLOAD  & TURNAROUND & RIGHT       & NEUTRAL \\
            2          & EXIT    & TURNRIGHT  & DOWN        & UP \\
            3          & --      & TURNLEFT   & LEFT        & RIGHT \\
            4          & --      & STOP       & --          & DOWN \\
            5          & --      & --         & --          & LEFT \\
            6          & --      & --         & --          & POI \\ [1ex]
            \hline
        \end{tabular}
        \caption{Riepilogo enumerazioni}
    \end{table}
















