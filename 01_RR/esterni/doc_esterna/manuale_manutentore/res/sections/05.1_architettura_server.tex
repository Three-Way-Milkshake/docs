\subsection{Server}

L'architettura della componente server si articola in una 3-layer architecture, in cui si identificano i seguenti layer:
\begin{itemize}
	\item \textbf{Communication layer}
	\item \textbf{Business layer}
	\item \textbf{Persistence layer}
\end{itemize}

\begin{figure}[H]
	\centering
	\includegraphics[scale=0.22]{res/diagrams/server/server_complessivo_minimal.jpg}
	\caption{Visione complessiva dell'architettura del server}
\end{figure}

Le sezioni che seguono illustrano la struttura di ogni layer.

\subsubsection{Communication layer}

\begin{figure}[H]
	\centering
	\includegraphics[scale=0.55]{res/diagrams/server/server_communication.jpg}
	\caption{Visione di dettaglio del Communication Layer}
\end{figure}


\subsubsection{Business layer}

Nel Business layer risiede il nucleo di elaborazione dei dati ricevuti dal layer superiore: i dominii principali di cui si occupa sono:
\begin{itemize}
	\item gestione della mappa e path finding;
	\item gestione dell'autenticazione dei client;
	\item gestione delle tasks e dei POI;
	\item rilevazione e gestione delle collisioni.
\end{itemize}  
Per facilitare la consultazione, lo studio di questo layer si concentra separatamente sui package di cui si compone. Per una visione dall'alto, riferirsi al diagramma complessivo all'inizio della sezione 5.1.





\paragraph{Clients}
\subparagraph*{ }

\begin{figure}[H]
	\centering
	\includegraphics[scale=0.40]{res/diagrams/server/server_pack_clients.jpg}
	\caption{Visione di dettaglio del package Clients}
\end{figure}

Il server conserva nelle classi \texttt{UsersList} e \texttt{ForkliftList} le liste degli utenti e dei muletti (client) connessi con i relativi token di autenticazione. In particolare, la gerarchia dei client prevede una prima suddivisione suddivisione tra \texttt{Forklift} e \texttt{User} (muletto e utente), gli \texttt{User} si specializzano ulteriormente in \texttt{Manager} (responsabile) e \texttt{Admin} (amministratore).

I \texttt{Forklift} di caratterizzano dagli attributi:
\begin{itemize}
	\item \texttt{Position}: rappresenta la posizione e orientamento attuali del muletto nella mappa;
	\item \texttt{TaskSequence}: una sequenza di task da compiere;
	\item \texttt{Move}: una lista di mosse atte a raggiungere il prossimo POI (e quindi evadere la prossima task).
\end{itemize} 

La classe \texttt{Engine} è il cuore del motore di calcolo: essa esegue su un thread dedicato e tramite un timer scandisce l'esecuzione temporizzata dell'elaborazione. In particolare, interroga periodicamente \texttt{UsersList} e \texttt{ForkliftList} con i seguenti obiettivi:
\begin{itemize}
	\item ricevere le nuove posizioni dai muletti;
	\item inviare le nuove informazioni agli utenti per la visualizzazione nel monitor real-time;
	\item processare eventuali altre richieste (calcolo percorso, aggiunta task, modifica mappa).
\end{itemize}
Dopodichè la \texttt{ForkliftList} viene utilizzata dal modulo di rilevazione gestione delle collisioni.

In questo layer si concentra l'utilizzo del framework Spring, utilizzato per gestire le dipendenze: alcune classi di utilizzo frequente e condiviso come \texttt{UsersList}, \texttt{ForkliftList} e \texttt{TaskSequenceList} (quest'ultima contenente tutte le liste di task inserite dal responsabile) vengono istanziate tramite \textit{Dependency Injection} sfruttando il meccanismo dei \textit{Bean} di Spring.




\paragraph{Mappa}
\subparagraph*{ }

\begin{figure}[H]
	\centering
	\includegraphics[scale=0.60]{res/diagrams/server/server_pack_map.jpg}
	\caption{Visione di dettaglio del package Map}
\end{figure}


La classe \texttt{WarehouseMap} contiene la rappresentazione della planimetria del magazzino: essa è rappresentata tramite una matrice di \texttt{CellType}, campo di tipo enumerazione che esprime le caratteristiche di ogni frazione spaziale. Alla mappa è associata una lista di POI, e per ognuno la relativa locazione. 
Si osserva l'applicazione di alcuni design pattern:
\begin{itemize}
	\item \textbf{observer}: tramite la libreria \texttt{PropertyChangeSupport} e \texttt{PropertyChangeListener} di Java viene applicato il pattern \textit{observer}, definendo la \texttt{WarehouseMap} come Subject, e i \texttt{Client} come \textit{Observer}: essi verranno notificati ad ogni cambiamento della stessa in modo che possano comunicare ai client esterni le modifiche, e possano essere aggiornate le interfacce grafiche che visualizzano la mappa.
	\item \textbf{Strategy}: per l’algoritmo di path finding attualmente viene implementata una strategia di tipo \textit{breadth-first}, ma l’impostazione del pattern permette di aggiungere e variare dinamicamente eventuali altre implementazioni aggiunte in futuro. \texttt{WarehouseMap} assume il ruolo di \textit{context}, e i beneficiari sono i \texttt{Forklift}, i quali richiederanno il percorso ogni qualvolta si renderà necessario.
\end{itemize}





\paragraph{Collisioni}
\subparagraph*{ }

\begin{figure}[H]
	\centering
	\includegraphics[scale=0.50]{res/diagrams/server/server_pack_collision.jpg}
	\caption{Visione di dettaglio del package Collision}
\end{figure}


\subsubsection{Persistence layer}

\begin{figure}[H]
	\centering
	\includegraphics[scale=0.50]{res/diagrams/server/server_persistency.jpg}
	\caption{Visione di dettaglio del Persistence Layer}
\end{figure}

L'accesso a questo layer è regolato da 3 interfacce che gestiscono la persistenza delle tre tipologie di dati che vengono salvati: 
\begin{itemize}
	\item le credenziali di autenticazione degli utenti;
	\item i token di autenticazione dei muletti;
	\item la rappresentazione della mappa.
\end{itemize}

Ogni interfaccia si rivolge alla relativa componente del layer superiore che conserva a runtime i dati impiegati nell'esecuzione. La presenza delle interfacce favorisce il disaccoppiamento tra i moduli e permette di estendere a tipi di persistenza alternativi. Attualmente è implementato il salvataggio dei dati su file di tipo .json, viene fatto uso della libreria standard java.io e GSON per gestire l'interazione con questo tipo di tecnologia.



