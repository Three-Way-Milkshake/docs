\section{Introduzione}




\subsection{Scopo del documento}
Lo scopo di questo documento è presentare tutte le informazioni necessarie al mantenimento e all'estensione del software PORTACS\textsubscript{A}, mostrando nel dettaglio l'architettura del sistema e l'organizzazione del codice sorgente.\\
In questo documento saranno presentate le varie tecnologie usate, sia lato front end che back end, come anche le varie librerie e framework\textsubscript{G}. Verrà inoltre mostrato il sistema di versionamento utilizzato e la Continuous Integration applicata.





\subsection{Scopo del prodotto}

Il capitolato\textsubscript{G} C5 propone un progetto\textsubscript{G} in cui viene richiesto lo sviluppo di un software per il monitoraggio in tempo reale di unità che si muovono in uno spazio definito. All'interno di questo spazio, creato dall’utente per riprodurre le caratteristiche di un ambiente reale, le unità dovranno essere in grado di circolare in autonomia, o sotto il controllo dell’utente, per raggiungere dei punti di interesse posti nella mappa.  La circolazione è sottoposta a vincoli di viabilità e ad ostacoli propri della topologia dell’ambiente, deve evitare le collisioni con le altre unità e prevedere la gestione di situazioni critiche nel traffico.




\subsection{Riferimenti}



\subsubsection{Normativi}

\begin{itemize}
	\item \textsc{Norme di progetto\textsubscript{G} v3.0.0 }: per qualsiasi convenzione sulla nomenclatura degli elementi presenti all’interno del documento;

	\item Regolamento progetto\textsubscript{G} didattico: \\ {\url{https://www.math.unipd.it/~tullio/IS-1/2020/Dispense/P1.pdf}};
	\item Model-View Patterns: \\ {\url{https://medium.com/@anshul.vyas380/mvc-pattern-3b5366e60ce4}};
    \item SOLID Principles: \\ \href{https://www.digitalocean.com/community/conceptual\_articles/s-o-l-i-d-the-first-five-principles-of-object-oriented-design}{\texttt{https://www.digitalocean.com/community/conceptual\_articles/}\\ \texttt{s-o-l-i-d-the-first-five-principles-of-object-oriented-design}};
	\item Diagrammi delle classi: \\ {\url{https://www.math.unipd.it/~rcardin/swea/2021/Diagrammi delle Classi_4x4.pdf}};
	\item Diagrammi dei package: \\ {\url{https://www.math.unipd.it/~rcardin/swea/2021/Diagrammi dei Package_4x4.pdf}};
	\item Diagrammi di sequenza: \\ {\url{https://www.math.unipd.it/~rcardin/swea/2021/Diagrammi di Sequenza_4x4.pdf}};
	\item Design Pattern Creazionali: \\ {\url{https://refactoring.guru/design-patterns/creational-patterns}};
	\item Design Pattern Strutturali: \\ {\url{https://refactoring.guru/design-patterns/structural-patterns}};
	\item Design Pattern Comportamentali: \\ {\url{https://refactoring.guru/design-patterns/behavioral-patterns}}.
\end{itemize}



\subsubsection{Informativi}
\begin{itemize}
	\item \textsc{\href{https://github.com/Three-Way-Milkshake/docs/wiki/Glossario}{Glossario}}: per la definizione dei termini (pedice G) e degli acronimi (pedice A) evidenziati nel documento;
	\item Capitolato d'appalto C5-PORTACS: \\
{\url{https://www.math.unipd.it/~tullio/IS-1/2020/Progetto/C5.pdf}}
	\item Software Engineering - Iam Sommerville - $10^{th}$ Edition.
	\item Angular: \\ {\url{https://angular.io/}};
	\item Node.js: \\ {\url{https://nodejs.org/en/}};
	\item PrimeNG: \\ {\url{https://www.primefaces.org/primeng/}};
	\item Java: \\ {\url{https://www.java.com/it/}};
	\item Spring: \\ {\url{https://spring.io/}} \\ \url{https://start.spring.io/};
	\item Docker: \\ {\url{https://www.docker.com/}} \\ \url{https://dockertutorial.it/}.

\end{itemize}