\section{Requisiti}
\subsection{Introduzione}
In questa sezione vengono riportati i requisito\textsubscript{G}, strutturati secondo la loro classificazione per tipologia, ovvero requisito\textsubscript{G} funzionali, requisito\textsubscript{G} prestazionali, requisito\textsubscript{G} di qualità e requisito\textsubscript{G} di vincolo.
\subsection{Requisiti funzionali}
\renewcommand{\arraystretch}{1.5}
\rowcolors{2}{pari}{dispari}
\begin{longtable}{ 
		>{}p{0.145\textwidth} 
		>{}p{0.5425\textwidth}
		>{\centering}p{0.22\textwidth} }
	\rowcolorhead
	\centering \headertitle{Codice} &
	\centering \headertitle{Descrizione} &	
	\centering \headertitle{\normalfont \textbf{Fonte}}	
	\endfirsthead	
	\endhead
RF-1-O &
Un utente deve effettuare il login alla piattaforma tramite il suo codice identificativo e password
& UC1
\tabularnewline
RF-2-O &
Il processo di login dell’utente non va a buon fine se il codice inserito e/o la password non sono corretti o non sono presenti nel sistema
& UC1.1
\tabularnewline
RF-3-O &
L’amministratore può creare l’account di un responsabile
& UC2
\tabularnewline
RF-3.1-O &
La registrazione di un nuovo utente necessita del nome del lavoratore
& UC2.1.1
\tabularnewline
RF-3.2-O &
La registrazione di un nuovo utente necessita del cognome del lavoratore
& UC2.1.2
\tabularnewline
RF-4-O &
Il server centrale permette all'amministratore di gestire gli account inseriti
& UC3
\tabularnewline
RF-4.1-O &
Il server centrale permette la modifica di un utente già registrato
& UC3.1
\tabularnewline
RF-4.1.1-O
&
Il server permette all'amministratore e al responsabile di modificare il nome
& UC3.1.1
\tabularnewline
RF-4.1.1.1-O &
L’amministratore può modificare il campo nome di un account esistente
& UC3.1.1
\tabularnewline
RF-4.1.1.2-O &
L’amministratore può modificare il campo nome del proprio account
& UC3.1.1
\tabularnewline
RF-4.1.1.3-O &
Il responsabile può modificare il campo nome del proprio account
& UC3.1.1
\tabularnewline
RF-4.1.2-O &
Il server permette all'amministratore e al responsabile di modificare il cognome
& UC3.1.2
\tabularnewline
RF-4.1.2.1-O &
L’amministratore può modificare il campo cognome di un account esistente
& UC3.1.2
\tabularnewline
RF-4.1.2.2-O &
L’amministratore può modificare il campo cognome del proprio account
& UC3.1.2
\tabularnewline
RF-4.1.2.3-O &
Il responsabile può modificare il campo cognome del proprio account
& UC3.1.2
\tabularnewline
RF-4.1.3-O &
Il server permette all'amministratore e al responsabile di modificare la password
& UC3.1.3
\tabularnewline
RF-4.1.3.1-O &
L’amministratore può modificare il campo password del proprio account
& UC3.1.3
\tabularnewline
RF-4.1.3.2-O &
Il responsabile può modificare il campo password del proprio account
& UC3.1.3
\tabularnewline
RF-4.1.4-O &
L'amministratore può mandare un segnale al server per resettare la password di un preciso account di un responsabile
&
UC3.1.4
\tabularnewline
RF-4.2-O &
L'amminisrtatore può eliminare un utente precedentemente registrato
&UC3.2
\tabularnewline
RF-4.3-O&
Il server centrale permette all'amministratore di visualizzare la lista di tutti gli account inseriti con i loro relativi dati pubblici
&UC3.3
\tabularnewline
RF-5-O&
Il responsabile si occupa della gestione della task
&UC4
\tabularnewline
RF-5.1-O&
Il responsabile può inserire una nuova lista ordinata di task
&UC4.1
\tabularnewline
RF-5.1.2-O&
Il server permette al responsabile di aggiungere una task singola con il proprio POI di scarico per soddisfarla
&UC4.1.1
\tabularnewline
RF-5.1.3-O&
Il server permette al responsabile la rimozione di una task già aggiunta nella lista che si sta creando
&UC4.1.2
\tabularnewline
RF-5.1.4-O&
Il server necessita della conferma della lista che si è creata, così che la possa prendere in carico
&UC4.1.3
\tabularnewline
RF-5.2-O&
Il server permette l'eliminazione di una lista di task
&UC4.2
\tabularnewline
RF-6-O&
Il server centrale permette all'amministratore e al responsabile di fare il logout dall'applicativo
&UC5
\tabularnewline
RF-7-O&
Il server centrale permette la visualizzazione della mappa all’amministratore e ai responsabili
&UC6
\tabularnewline
RF-7.1-O&
Il server centrale permette la visualizzazione di tutti i tipi di POI nella mappa all'amministratore e ai responsabili
&UC6
\tabularnewline
RF-7.2-O&
Il server centrale permette la visualizzazione delle caratteristiche delle zone di percorrenza (senso di marcia, numero massimo di unità che possono transitare) all’amministratore e ai responsabili
&UC6
\tabularnewline
RF-7.2.1-O&
Il server centrale permette la visualizzazione delle zone non transitabili all’amministratore e ai responsabili
&UC6
\tabularnewline
RF-8-O&
Il server centrale permette la visualizzazione della posizione dei muletti in real-time sulla mappa
&UC6.1
\tabularnewline
RF-9-F&
Il server centrale permette la visualizzazione della posizione delle persone in real-time sulla mappa
&Capitolato
\tabularnewline
RF-10-O&
Il server centrale permette all'amministratore la visualizzazione di una notifica in caso della segnalazione da parte di un operatore di un evento eccezionale
&
UC6.2
\tabularnewline
RF-11
-
O
&
L’amministratore autenticato può accedere all’interfaccia per gestire la mappa
&
UC7
\tabularnewline
RF-11.1
-
O
&
L’amministratore può modificare planimetria del magazzino
&
UC7.2
\tabularnewline
RF-11.2
-
O
&
L’amministratore può modificare la percorrenza del magazzino
&
UC7.3
\tabularnewline
RF-12
-
O
&
L’amministratore può gestire i POI
&
UC7.4
\tabularnewline
RF-12.1
-
O
&
L'amministratore può modificare la posizione di un POI già esistente
&
UC7.4.1
\tabularnewline
RF-12.2
-
O
&
L'amministratore può inserire un nuovo POI
&
UC7.4.2
\tabularnewline
RF-12.2.1
-
O
&
Inserendo un nuovo POI, l'amministratore dovrà specificare la sua posizione nella mappa
&
UC7.4.3
\tabularnewline
RF-12.2.2
-
O
&
Inserendo un nuovo POI, l'amministratore dovrà specificare il suo codice identificativo
&
UC7.4.4
\tabularnewline
RF-12.2.3
-
O
&
Inserendo un nuovo POI, l'amministratore dovrà specificare il tipo di POI inserito (carico, scarico, base)
&
UC7.4.5
\tabularnewline
RF-12.3
-
O
&
L'amministratore può eliminare un POI
&
UC7.4.6
\tabularnewline
RF-13
-
O
&
La User Interface di una specifica unità attiva implementa una mappa contente i relativi POI presenti nella lista delle task da soddisfare, numerati secondo la lista
&
UC8.1
\tabularnewline
RF-14
-
O
&
La User Interface implementa sotto alla mappa una lista ordinata contenente la task rimanenti da eseguire dell’operatore che sta usando l’unità
&
UC8.2
\tabularnewline
RF-15
-
O
&
La mappa mostra il prossimo task da soddisfare (POI da raggiungere)
&
UC8.3
\tabularnewline
RF-15.1
-
O
&
Nella mappa specifica dell’unità verrà evidenziato con un colore diverso il prossimo POI da raggiungere
&
UC8.3
\tabularnewline
RF-16
-
O
&
L’operatore segnala al server centrale la conclusione dell’incarico attraverso la user interface
&
UC9
\tabularnewline
RF-17
-
O
&
La User Interface che rappresenterà ogni singola unità dovrà prevedere le 4 frecce direzionali che indicano il suggerimento del sistema
&
UC10
\tabularnewline
RF-17.1
-
O
&
Il server centrale permette all’operatore la visualizzazione di direzione e spostamento del muletto a cui è a bordo, in caso in cui nel muletto sia attiva la guida automatica
&
UC10
\tabularnewline
RF-18
-
O
&
Nella user interface è presente un pulsante che permette di passare dalla guida manuale alla guida autonoma dell’unità
&
UC11.1
\tabularnewline
RF-18.1
-
O
&
La User Interface del controllo manuale permette di passare alla guida autonoma
&
UC11.1, verbale\_esterno\_1
\tabularnewline
RF-19
-
O
&
Nella user interface è presente un pulsante che permette di passare dalla guida autonoma alla guida manuale dell’unità
&
UC11.2
\tabularnewline
RF-19.1
-
O
&
La User Interface del controllo automatico permette di passare alla guida manuale
&
UC11.2, verbale\_esterno\_1
\tabularnewline
RF-20
-
O
&
Nella user interface è presente un pulsante che permette di segnalare al server un evento eccezionale
&
UC11.3
\tabularnewline
RF-21
-
O
&
Nella user interface comparirà un pulsante per il ritorno alla base dell’unità se l'operatore avrà concluso tutte le task assegnategli
&
UC11.5
\tabularnewline
RF-22
-
O
&
La User Interface che rappresenterà ogni singola unità dovrà prevedere le 4 frecce direzionali che permettono gli spostamenti manuali ed i pulsanti di start/stop
&
UC11.4, verbale\_esterno\_1
\tabularnewline
RF-23
-
D
&
Il pannello permette di visualizzare l’indicatore di velocità attuale (che avrà come massimo la velocità massima anagrafica)
&
Capitolato
\tabularnewline
RF-24
-
O
&
Il server centrale pilota e coordina tutte le unità per evitare incidenti e ingorghi
&
Capitolato
\tabularnewline
RF-25
-
F
&
il server centrale fornisce il percorso migliore ad ogni unità tramite algoritmi di ricerca operativa
&
Capitolato
\tabularnewline
RF-26
-
O
&
Il server centrale permette al responsabile di visualizzare la lista di tutti i POI con il proprio tipo (carico, scarico, base) presenti nella mappa
&
UC12
\tabularnewline
RF-27
-
O
&
Il server centrale permette all'amministratore di visualizzare la lista di tutti i POI con il proprio tipo (carico, scarico, base) presenti nella mappa
&
UC12
\tabularnewline
RF-28
-
O
&
Il responsabile ha a disposizione un pulsante per poter vedere una lista completa delle task non ancora prese in carico
&
UC13.1
\tabularnewline
RF-29
-
O
&
Il responsabile ha a disposizione un pulsante per poter vedere una lista completa delle task assegnate
&
UC13.2
\tabularnewline
RF-30
-
O
&
L'amministratore ha a disposizione un'interfaccia su cui può gestire le unità
&
UC14
\tabularnewline
RF-30.1
-
O
&
L'amministratore può aggiungere una nuova unità
&
UC14.1
\tabularnewline
RF-30.2
-
O
&
L'amministratore può eliminare un'unità
&
UC14.2
\tabularnewline
RF-31
-
O
&
Il server centrale necessita di conoscere la posizione nella mappa di una determinata unità
&
UC15
\tabularnewline
RF-32
-
O
&
Il server invia il percorso tra due POI quando un’unità deve partire da un punto nella mappa per raggiungere il prossimo POI per soddisfare la successiva task nella sua lista
&
UC16
\tabularnewline
RF-33
-
O
&
La geolocalizzazione va simulata
&
Capitolato
\tabularnewline
RF-34
-
O
&
L'applicativo propone una mappatura in tempo reale della posizione georeferenziata delle unità
&
Capitolato
\tabularnewline
RF-35
-
F
&
L'applicativo propone una mappatura in tempo reale della posizione georeferenziata delle persone
&
Capitolato
\tabularnewline
RF-36
-
O
&
Il server centrale deve prevedere ed evitare le collisioni
&
Capitolato
\tabularnewline
RF-37
-
O
&
Ogni unità deve rispettare i vincoli dimensionali delle zone
&
Capitolato
\tabularnewline
RF-38
-
O
&
Tutte le unità, quando sono in movimento, devono viaggiare tutte alla stessa velocità che rimane costante
&
Capitolato
\tabularnewline
RF-38.1
-
F
&
L’applicativo permette di gestire il cambiamento della velocità di un’unità
&
Capitolato
\tabularnewline
RF-38.2
-
D
&
Ogni unità ha una velocità di crociera
&
Capitolato
\tabularnewline
RF-38.3
-
D
&
Ogni unità ha una velocità massima
&
Capitolato
\tabularnewline
RF-39
-
O
&
Ogni unità ha un suo identificativo
&
Capitolato
\tabularnewline
RF-40
-
O
&
Il server centrale conosce la posizione di ogni singola unità
&
Capitolato
\tabularnewline
RF-41
-
O
&
Il server centrale conosce la direzione di ogni singola unità
&
Capitolato
\tabularnewline
RF-42
-
D
&
Il server centrale conosce la velocità di ogni singola unità
&
Capitolato
\tabularnewline
RF-43
-
O
&
Ogni unità ha una lista di task da risolvere ogni volta che fa carico
&
Capitolato
\tabularnewline
RF-44
-
O
&
Ogni task è collegata ad un POI da raggiungere
&
Capitolato
\tabularnewline
RF-45
-
O
&
Ogni POI può essere di carico o scarico o base
&
Decisione interna
\tabularnewline
RF-46
-
O
&
Ci devono essere più di un POI di scarico
&
Decisione interna
\tabularnewline
RF-47
-
O
&
Ci deve essere almeno un POI di base
&
Decisione interna
\tabularnewline
RF-48
-
O
&
Ci deve essere almeno un POI di carico
&
Decisione interna
\tabularnewline
RF-49
-
F
&
Ci possono essere più POI di base
&
Decisione interna
\tabularnewline
RF-50
-
F
&
Ci possono essere più POI di carico
&
Decisione interna
\tabularnewline
RF-51
-
O
&
Ogni unità parte da una base. La sua partenza dalla base determina l'inizio del turno di un operatore
&
Decisione interna
\tabularnewline
RF-52
-
O
&
Ogni unità torna ad una base quando termina il turno dell’operatore
&
Decisione interna
\tabularnewline
RF-53
-
O
&
Ogni unità passa per un’area di carico prima di iniziare la sequenza di scarichi (tasks)
&
Decisione interna
\tabularnewline
RF-54
-
O
&
Ogni unità torna ad un'area di carico se ha scaricato tutta la merce (completato i task) e il turno dell’operatore non è terminato
&
Decisione interna
\tabularnewline
RF-55
-
O
&
Il server centrale conosce ogni spostamento (in avanti, indietro, a destra e a sinistra) di ogni singola unità
&
Capitolato
\tabularnewline
RF-56
-
O
&
Il server centrale conosce la fermata di ogni singola unità
&
Capitolato
\tabularnewline
RF-57
-
O
&
Il server centrale conosce la partenza di ogni singola unità
&
Capitolato
\tabularnewline
RF-58
-
O
&
Ci deve uno e un solo account registrato con il ruolo di amministratore
&
Decisione interna
\tabularnewline
RF-59
-
O
&
Ci deve essere almeno un account registrato con il ruolo di responsabile
&
Decisione interna
\tabularnewline
RF-60
-
F
&
Ci possono essere più account registrati con il ruolo di responsabile
&
Decisione interna
\tabularnewline
RF-61
-
O
&
Ci devono essere più account registrati con il ruolo di operatore
&
Decisione interna
\tabularnewline
\caption{Tabella Requisiti Funzionali\label{ Tabella Requisiti Funzionali}}
\end{longtable}
\subsection{Requisiti prestazionali}
In questo progetto\textsubscript{G} non sono stati rilevati alcuni requisito\textsubscript{G} prestazionali per quanto riguarda i requisito\textsubscript{G} obbligatori.

\subsection{Requisiti di qualità}
\renewcommand{\arraystretch}{1.5}
\rowcolors{2}{pari}{dispari}
\begin{longtable}{ 
		>{}p{0.135\textwidth} 
		>{}p{0.5425\textwidth}
		>{\centering}p{0.2\textwidth} }
	\rowcolorhead
	\centering\headertitle{Codice} &
	\centering \headertitle{Descrizione} &	
	\centering \headertitle{\normalfont \textbf{Fonte}}	
	\endfirsthead	
	\endhead
RQ-1-O & Disporre di diagrammi UML\textsubscript{A} relativi agli use cases di progetto\textsubscript{G} & Capitolato\tabularnewline
RQ-2-O & Disporre di uno schema design relativo alla base dati (se ritenuta necessaria) & Capitolato\tabularnewline
RQ-3-O & Rendere disponibile la documentazione delle API\textsubscript{A} che saranno realizzate & Capitolato\tabularnewline
RQ-4-O & Rendere disponibile la lista dei bug\textsubscript{G} risolti durante la fase\textsubscript{G} di sviluppo & Capitolato\tabularnewline
RQ-5-O & Fornire il codice prodotto in formato sorgente utilizzando sistemi di versionamento del codice, quali GitHub o Bitbucket & Capitolato\tabularnewline
RQ-6-O & Rilasciare il codice sorgente di quanto realizzato & Capitolato\tabularnewline
RQ-7-O & Rendere disponibile il Docker file (\#1) con la componente applicativa, rappresentante il motore di calcolo & Capitolato\tabularnewline 
RQ-8-O & Fornire il Docker file (\#2) con la componente applicativa rappresentante il visualizzatore/monitor real-time (in base all'implementazione, potrebbe essere incorporato nel \#1) & Capitolato\tabularnewline
RQ-9-O & Erogare il Docker file (\#3), da istanziare N volte, rappresentante la singola unità & Capitolato\tabularnewline
RQ-10-F & Rendere disponibile il Docker file (\#4), da istanziare N volte, rappresentante il singolo pedone & Capitolato\tabularnewline 
\caption{Tabella Requisiti di Qualità\label{ Tabella Requisiti di Qualità}}
\end{longtable}
\subsection{Requisiti di vincolo}
\renewcommand{\arraystretch}{1.5}
\rowcolors{2}{pari}{dispari}
\begin{longtable}{ 
		>{}p{0.135\textwidth} 
		>{}p{0.5425\textwidth}
		>{\centering}p{0.2\textwidth} }
	\rowcolorhead
	\centering\headertitle{Codice} &
	\centering \headertitle{Descrizione} &	
	\centering \headertitle{\normalfont \textbf{Fonte}}	
	\endfirsthead	
	\endhead
RV-1-O & Ogni zona di percorrenza ha un numero massimo di unità che possono percorrerla in parallelo (dimensione della zona) 
& Capitolato \tabularnewline
RV-2-O & Ogni zona di percorrenza ha un modo in cui può essere percorsa (senso unico, doppio senso)
& Capitolato \tabularnewline
RV-3-O & Tutto quello che si muove all'interno è censito dal server, in quanto non esiste unità non riconosciuta e controllata dal server e non c'è nulla che il server non conosca e da cui non riceva dati
& Capitolato \tabularnewline
RV-4-O & Versione di Chrome da utilizzare è la 88.0
& Decisione interna \tabularnewline
\caption{Tabella Requisiti di vincolo\label{ Tabella Requisiti di vincolo}}
\end{longtable}
\pagebreak
\subsection{Tracciamento}
\subsubsection{Fonti - Requisiti}
\renewcommand{\arraystretch}{1.5}
\rowcolors{2}{pari}{dispari}
\begin{longtable}{ 
		>{}p{0.22\textwidth} 
		>{}p{0.25\textwidth} }
	\rowcolorhead
	\headertitle{Fonte} &
	\headertitle{\normalfont \textbf{Requisiti}}	
	\endfirsthead	
	\endhead
Capitolato &	
	RV-1-O	\newline
	RV-2-O	\newline
	RV-3-F	\newline
	RV-5-O	\newline
	RV-6-O	\newline
	RV-7-O	\newline
	RV-8-O	\newline
	RV-9-O	\newline
	RV-10-F	\newline
	RV-11-D	\newline
	RV-12-D	\newline
	RV-13-O	\newline
	RV-14-O	\newline
	RV-15-O	\newline
	RV-16-D	\newline
	RV-17-O	\newline
	RV-18-O	\newline
	RV-29-O	\newline
	RV-30-O	\newline
	RV-31-O	\newline
	RF-17-F	\newline
	RF-31-D	\newline
	RF-32-O	\newline
	RF-33-F	\newline
	RQ-1-O	\newline
	RQ-2-O	\newline
	RQ-3-O	\newline
	RQ-4-O	\newline
	RQ-5-O	\newline
	RQ-6-O	\newline
	RQ-7-O	\newline
	RQ-8-O	\newline
	RQ-9-O	\newline
	RQ-10-F	\tabularnewline
Decisione interna &	RV-4-O	\newline
	RV-19-O	\newline
	RV-20-O	\newline
	RV-21-O	\newline
	RV-22-O	\newline
	RV-23-F	\newline
	RV-24-F	\newline
	RV-25-O	\newline
	RV-26-O	\newline
	RV-27-O	\newline
	RV-28-O	\newline
	RV-32-O	\newline
	RV-33-O	\newline
	RV-34-O	\newline
	RV-35-O	\tabularnewline
\textsc{Verbale Esterno 1} &	RF-26.1-O	\newline
	RF-27.1-O	\newline
	RF-30-O	\tabularnewline
UC1 &	RF-1-O	\newline
	RF-2-O	\tabularnewline
UC2 &	RF-3-O	\newline
	RF-4-O	\newline
	RF-4.1-O	\newline
	RF-4.2-O	\newline
	RF-4.3-O	\newline
	RF-5-O	\tabularnewline
UC3 &	RF-6-O	\newline
	RF-6.1-O	\newline
	RF-6.1.1-O	\newline
	RF-6.1.2-O	\newline
	RF-6.1.3-O	\newline
	RF-6.2-O	\tabularnewline
UC4 &	RF-7-O	\newline
	RF-8-O	\newline
	RF-8.1-O	\newline
	RF-8.2-O	\newline
	RF-8.3-O	\newline
	RF-9-O	\newline
	RF-10-O	\tabularnewline
UC5 &	RF-11-O	\newline
	RF-12-O	\newline
	RF-13-O	\newline
	RF-14-O	\tabularnewline
UC6 &	RF-15-O	\newline
	RF-15.1-O	\newline
	RF-15.2-O	\newline
	RF-15.2.1-O	\newline
	RF-16-O	\newline
	RF-18-O	\tabularnewline
UC7 &	RF-19-O	\newline
	RF-19.1-O	\newline
	RF-19.2-O	\newline
	RF-20-O	\newline
	RF-20.1-O	\newline
	RF-20.2-O	\newline
	RF-20.2.1-O	\newline
	RF-20.2.2-O	\newline
	RF-20.2.3-O	\newline
	RF-20.3-O	\tabularnewline
UC8 &	RF-20-O	\newline
	RF-22-O	\newline
	RF-23-O	\newline
	RF-23.1-O	\tabularnewline
UC9 &	RF-24-O	\tabularnewline
UC10 &	RF-25-O	\newline
	RF-25.1-O	\tabularnewline
UC11 &	RF-26-O	\newline
	RF-26.1-O	\newline
	RF-27-O	\newline
	RF-27.1-O	\newline
	RF-28-O	\newline
	RF-29-O	\newline
	RF-30-O	\tabularnewline
UC12 &	RF-34-O	\newline
	RF-35-O	\tabularnewline
UC13 &	RF-36-O	\tabularnewline
UC14 &	RF-37-O	\newline
	RF-37.1-O	\newline
	RF-37.2-O	\tabularnewline
\caption{Tabella Fonti - Requisiti\label{ Tabella Fonti - Requisiti}}
\end{longtable}.
\pagebreak
\subsubsection{Requisiti - Fonti}
\renewcommand{\arraystretch}{1.5}
\rowcolors{2}{pari}{dispari}
\begin{longtable}{ 
		>{}p{0.2\textwidth} 
		>{}p{0.35\textwidth} }
	\rowcolorhead
	\headertitle{Requisito} &
	\headertitle{\normalfont \textbf{Fonti}}	
	\endfirsthead	
	\endhead
RF-1-O	&	UC1	\tabularnewline
RF-2-O	&	UC1.1	\tabularnewline
RF-3-O	&	UC2	\tabularnewline
RF-4-O	&	UC2	\tabularnewline
RF-4.1-O	&	UC2.1.1	\tabularnewline
RF-4.2-O	&	UC2.1.2	\tabularnewline
RF-4.3-O	&	UC2.1.3	\tabularnewline
RF-5-O	&	UC2.3	\tabularnewline
RF-6-O	&	UC3	\tabularnewline
RF-6.1-O	&	UC3.1	\tabularnewline
RF-6.1.1-O	&	UC3.1.1	\tabularnewline
RF-6.1.2-O	&	UC3.1.2	\tabularnewline
RF-6.1.3-O	&	UC3.1.3	\tabularnewline
RF-6.2-O	&	UC3.2	\tabularnewline
RF-7-O	&	UC4	\tabularnewline
RF-8-O	&	UC4.1	\tabularnewline
RF-8.1-O	&	UC4.2	\tabularnewline
RF-8.2-O	&	UC4.3	\tabularnewline
RF-8.3-O	&	UC4.4	\tabularnewline
RF-9-O	&	UC4.5	\tabularnewline
RF-10-O	&	UC4.6	\tabularnewline
RF-11-O	&	UC5	\tabularnewline
RF-12-O	&	UC5	\tabularnewline
RF-13-O	&	UC5	\tabularnewline
RF-14-O	&	UC5	\tabularnewline
RF-15-O	&	UC6	\tabularnewline
RF-15.1-O	&		\tabularnewline
RF-15.2-O	&	UC6	\tabularnewline
RF-15.2.1-O	&	UC6	\tabularnewline
RF-16-O	&	UC6.1	\tabularnewline
RF-17-F	&	Capitolato	\tabularnewline
RF-18-O	&	UC6.2	\tabularnewline
RF-19-O	&	UC7	\tabularnewline
RF-19.1-O	&	UC7.2	\tabularnewline
RF-19.2-O	&	UC7.3	\tabularnewline
RF-20-O	&	UC7.4	\tabularnewline
RF-20.1-O	&	UC7.4.1	\tabularnewline
RF-20.2-O	&	UC7.4.2	\tabularnewline
RF-20.2.1-O	&	UC7.4.3	\tabularnewline
RF-20.2.2-O	&	UC7.4.4	\tabularnewline
RF-20.2.3-O	&	UC7.4.5	\tabularnewline
RF-20.3-O	&	UC7.4.6	\tabularnewline
RF-20-O	&	UC8.1	\tabularnewline
RF-22-O	&	UC8.2	\tabularnewline
RF-23-O	&	UC8.3	\tabularnewline
RF-23.1-O	&	UC8.3	\tabularnewline
RF-24-O	&	UC9	\tabularnewline
RF-25-O	&	UC10	\tabularnewline
RF-25.1-O	&	UC10	\tabularnewline
RF-26-O	&	UC11.1	\tabularnewline
RF-26.1-O	&	UC11.1, \textsc{Verbale Esterno 1}	\tabularnewline
RF-27-O	&	UC11.2	\tabularnewline
RF-27.1-O	&	UC11.2, \textsc{Verbale Esterno 1}	\tabularnewline
RF-28-O	&	UC11.3	\tabularnewline
RF-29-O	&	UC11.5	\tabularnewline
RF-30-O	&	UC11.4, \textsc{Verbale Esterno 1}	\tabularnewline
RF-31-D	&	Capitolato	\tabularnewline
RF-32-O	&	Capitolato	\tabularnewline
RF-33-F	&	Capitolato	\tabularnewline
RF-34-O	&	UC12	\tabularnewline
RF-35-O	&	UC12	\tabularnewline
RF-36-O	&	UC13	\tabularnewline
RF-37-O	&	UC14	\tabularnewline
RF-37.1-O	&	UC14.1	\tabularnewline
RF-37.2-O	&	UC14.2	\tabularnewline
RQ-1-O	&	Capitolato	\tabularnewline
RQ-2-O	&	Capitolato	\tabularnewline
RQ-3-O	&	Capitolato	\tabularnewline
RQ-4-O	&	Capitolato	\tabularnewline
RQ-5-O	&	Capitolato	\tabularnewline
RQ-6-O	&	Capitolato	\tabularnewline
RQ-7-O	&	Capitolato	\tabularnewline
RQ-8-O	&	Capitolato	\tabularnewline
RQ-9-O	&	Capitolato	\tabularnewline
RQ-10-F	&	Capitolato	\tabularnewline
RV-1-O	&	Capitolato	\tabularnewline
RV-2-O	&	Capitolato	\tabularnewline
RV-3-F	&	Capitolato	\tabularnewline
RV-4-O	&	Decisione interna	\tabularnewline
RV-5-O	&	Capitolato	\tabularnewline
RV-6-O	&	Capitolato	\tabularnewline
RV-7-O	&	Capitolato	\tabularnewline
RV-8-O	&	Capitolato	\tabularnewline
RV-9-O	&	Capitolato	\tabularnewline
RV-10-F	&	Capitolato	\tabularnewline
RV-11-D	&	Capitolato	\tabularnewline
RV-12-D	&	Capitolato	\tabularnewline
RV-13-O	&	Capitolato	\tabularnewline
RV-14-O	&	Capitolato	\tabularnewline
RV-15-O	&	Capitolato	\tabularnewline
RV-16-D	&	Capitolato	\tabularnewline
RV-17-O	&	Capitolato	\tabularnewline
RV-18-O	&	Capitolato	\tabularnewline
RV-19-O	&	Decisione interna	\tabularnewline
RV-20-O	&	Decisione interna	\tabularnewline
RV-21-O	&	Decisione interna	\tabularnewline
RV-22-O	&	Decisione interna	\tabularnewline
RV-23-F	&	Decisione interna	\tabularnewline
RV-24-F	&	Decisione interna	\tabularnewline
RV-25-O	&	Decisione interna	\tabularnewline
RV-26-O	&	Decisione interna	\tabularnewline
RV-27-O	&	Decisione interna	\tabularnewline
RV-28-O	&	Decisione interna	\tabularnewline
RV-29-O	&	Capitolato	\tabularnewline
RV-30-O	&	Capitolato	\tabularnewline
RV-31-O	&	Capitolato	\tabularnewline
RV-32-O	&	Decisione interna	\tabularnewline
RV-33-O	&	Decisione interna	\tabularnewline
RV-34-F	&	Decisione interna	\tabularnewline
RV-35-O	&	Decisione interna	\tabularnewline
\caption{Tabella Requisiti - Fonti\label{ Tabella Requisiti - Fonti}}
\end{longtable}.
\subsubsection{Riepilogo requisiti}
\renewcommand{\arraystretch}{1.5}
\rowcolors{2}{pari}{dispari}
\begin{longtable}{ 
		>{\centering}p{0.15\textwidth} 
		>{\centering}p{0.15\textwidth}
		>{\centering}p{0.15\textwidth}
		>{\centering}p{0.15\textwidth}
		>{\centering}p{0.1\textwidth} }
	\rowcolorhead
	\headertitle{Tipologia} &
	\centering \headertitle{Obbligatorio} &	
	\centering \headertitle{Facoltativo} &	
	\centering \headertitle{Desiderabile} &	
	\headertitle{\normalfont \textbf{Totale}}	
	\endfirsthead	
	\endhead
Funzionale & 62 & 2 & 1 & 65\tabularnewline
Di Qualità & 9 & 1 & 0 & 10\tabularnewline
Di Vincolo & 27 & 5 & 3 & 35\tabularnewline
\caption{Tabella Riepilogo dei Requisiti\label{ Tabella Riepilogo dei Requisiti}}
\end{longtable}.
