\section{Introduzione}
\subsection{Scopo del documento}
Il seguente documento ha lo scopo di elencare in modo formale e dettagliato tutti i usecase\textsubscript{G} e i requisito\textsubscript{G} dedotti dall'analisi del capitolato\textsubscript{G} C5 portacs\textsubscript{A} presentato dalla azienda Sanmarco Informatica.

\subsection{Scopo del prodotto}
Il capitolato\textsubscript{G} C5 propone un progetto\textsubscript{G} in cui viene richiesto lo sviluppo di un software per il monitoraggio in tempo reale di unità che si muovono in uno spazio definito. All'interno di questo spazio, creato dall'utente per riprodurre le caratteristiche di un ambiente reale, le unità dovranno essere in grado di circolare in autonomia, o sotto il controllo dell'utente, per raggiungere dei punti di interesse posti nella mappa. La circolazione è sottoposta a vincoli di viabilità e ad ostacoli propri della topologia dell'ambiente, il server inoltre deve evitare le collisioni tra le unità e prevedere la gestione di situazioni critiche nel traffico.

\subsection{Riferimenti}
\subsubsection{Normativi}
\begin{itemize}
\item \textsc{Norme di progetto\textsubscript{G} v1.0.0 }: per qualsiasi convenzione sulla nomenclatura degli elementi presenti all'interno del documento;
\item Regolamento progetto\textsubscript{G} didattico - slide del corso di Ingegneria del Software: \\ \url{https://www.math.unipd.it/~tullio/IS-1/2020/Dispense/P1.pdf}
\item Specifica sui usecase\textsubscript{G} - slide del corso di Ingegneria del Software: \\ \url{https://www.math.unipd.it/\%7Ercardin/swea/2021/Diagrammi\%20Use\%20Case_4x4.pdf}
\end{itemize}
\subsubsection{Informativi}
\begin{itemize}
\item \textsc{\href{https://github.com/Three-Way-Milkshake/docs/wiki/Glossario}{Glossario}}: per la definizione dei termini (pedice G) e degli acronimi (pedice A) evidenziati nel documento;
\item Capitolato d'appalto C5-PORTACS: \\
{\url{https://www.math.unipd.it/~tullio/IS-1/2020/Progetto/C5.pdf}}
\item Software Engineering - Iam Sommerville - $10^{th}$ Edition
\item \textsc{Verbale Esterno 1 v1.0.0}


\end{itemize}