\usepackage[toc,acronym]{glossaries}
\makeglossaries

\newglossaryentry{wiki}{name={wiki}, plural={wikis},%
	description={Termine di origine hawaiana che significa veloce, con cui si identifica un tipo di sito internet che permette la creazione e la modifica di pagine multimediali attraverso un'interfaccia semplice}}

\newglossaryentry{vmodel}{name={Modello a V},%
	description={Modello che illustra le relazioni tra ogni fase del ciclo di vita del software con la relativa fase di testing ad essa associata}}

\newglossaryentry{usecase}{name={caso d'uso}, plural={casi d'uso},%
	description={Un caso d'uso è un insieme di \glspl{scenario} che hanno in comune uno scopo finale (obiettivo) per un utente (\gls{attore})}}

\newglossaryentry{techbase}{name={technology baseline},%
	description={Motiva le tecnologie, i framework, e le librerie selezionate per la realizzazione del prodotto. Ne dimostra adeguatezza e fattibilità, tramite un proof of concept coerente con gli obiettivi}}

\newglossaryentry{task}{name={task}, plural={tasks},%
	description={Nel contesto del capitolato, con questo termine si identifica un compito da svolgere da parte di un unità (muletto) che consiste nel raggiungere un \acrshort{POI} e caricare o scaricare la merce}}

\newglossaryentry{stakeholder}{name={stakeholder}, plural={stakeholders},%
	description={Tutti coloro che a vario titolo hanno influenza sul prodotto e sul progetto. Sono stakeholders il committente, il proponente, gli utenti, il team di sviluppo}}

\newglossaryentry{sistematico}{name={sistematico},%
	description={Modo di lavorare metodico e rigoroso, che conosce, usa ed evolve le best practice di dominio}}

\newglossaryentry{shell}{name={shell},%
	description={Interprete dei comandi}}

\newglossaryentry{security}{name={security},%
	description={Non vulnerabilità rispetto a intrusioni}}

\newglossaryentry{scenario}{name={scenario}, plural={scenari},%
	description={Nell'ambito dello sviluppo software, sequenza di passi che descrive l'interazione tra l'\gls{attore} e il sistema, e le elaborazioni necessarie per soddisfare la richiesta dell'\gls{attore}}}

\newglossaryentry{safety}{name={safety},%
	description={Sicurezza rispetto a malfunzionamenti}}

\newglossaryentry{risorsa}{name={risorsa}, plural={risorse},%
	description={Mezzo o capacità disponibile, nello sviluppo software ad esempio le persone, il tempo, il denaro, gli strumenti necessari allo sviluppo del progetto. Le attività di progetto consumano le risorse disponibili}}

\newglossaryentry{revisione}{name={revisione}, plural={revisioni},%
	description={Esame o controllo, per lo più periodico, inteso a verificare il grado dell'efficienza, della funzionalità, della corrispondenza a determinati requisiti. Nell'ambito del corso di Ingegneria del Software, la revisione di avanzamento identifica il momento in cui il team consegna e presenta gli artefatti sviluppati durante il periodo appena trascorso}}

\newglossaryentry{requisito}{name={requisito}, plural={requisiti},%
	description={Esistono due interpretazioni principali di un requisito \begin{itemize}\item dal lato del bisogno(ovvero il cliente/utente) è la capacità necessaria a un utente per risolvere un problema o raggiungere un obiettivo\item dal lato della soluzione (ovvero lo sviluppatore) è la capacità che deve essere posseduta (o condizione che deve essere soddisfatta) da un sistema per adempiere a un obbligo \end{itemize}}}

\newglossaryentry{repository}{name={repository},%
	description={Posizione di archiviazione per pacchetti software}}

\newglossaryentry{quantificabile}{name={quantificabile},%
	description={Che permette di misurare l'efficienza e l'efficacia del suo agire}}

\newglossaryentry{proofconcept}{name={proof of concept},%
	description={Dimostratore eseguibile. Il suo codice può (ma non deve) essere usa-e-getta}}

\newglossaryentry{progetto}{name={progetto}, plural={progetti},%
	description={Insieme di attività che devono raggiungere determinati obiettivi a partire da determinate specifiche; hanno una data d'inizio e una data di fine fissate, dispongono di risorse limitate e consumano tali risorse nel loro svolgersi}}

\newglossaryentry{prodbase}{name={product baseline},%
	description={Illustra la baseline architetturale del prodotto, coerente con la \acrshort{tb}. Rappresenta il design definitivo}}

\newglossaryentry{precondizione}{name={precondizione}, plural={precondizioni},%
	description={Condizioni che devono essere soddisfatte affinché si verifichino determinati eventi successivi}}

\newglossaryentry{postcondizione}{name={postcondizione}, plural={postcondizioni},%
	description={Condizioni che devono verificarsi dopo determinati eventi}}

\newglossaryentry{planimetria}{name={planimetria}, plural={planimetrie},%
	description={Rappresentazione in piano del magazzino che ne evidenzia le caratteristiche (aree non transitabili, zone di percorrenza, punti di interesse)}}

\newglossaryentry{periodo}{name={periodo}, plural={periodi},%
	description={Nel contesto del documento, l'intervallo di tempo che intercorre tra due revisioni successive}}

\newglossaryentry{percorrenza}{name={percorrenza}, plural={percorrenze},%
	description={Nel contesto del capitolato, i vincoli relativi alle zone transitabili: \begin{itemize} \item sensi di marcia \item numero massimo di unità che vi possono transitare \end{itemize}}}

\newglossaryentry{modellosviluppo}{name={modello di sviluppo},%
	description={Principio teorico che indica il metodo da seguire nel progettare e nello scrivere un programma. Esistono tre principali modelli di sviluppo, ossia sequenziale, \gls{incrementale} ed evolutivo.}}

\newglossaryentry{milestone}{name={milestone},%
	description={Pietra miliare, istante temporale su cui si misura l'avanzamento del progetto. Viene fissata nel futuro per stabilire un avanzamento atteso. Una buona milestone è delimitata per ampiezza ed ambizioni, misurabile in termini di tempo e impegno necessario per raggiungerla, traducibile in compiti assegnabili ai membri del team e coerente con le scadenze del progetto}}

\newglossaryentry{kanban}{name={kanban},%
	description={Framework usato per implementare modelli di sviluppo Agili e DevOps. Richiede comunicazione in tempo reale tra i componenti de team e totale trasparenza. I work item vengono visualizzati su una Kanban Board, permettendo ai membri del team di vedere lo stato di ogni attività in ogni momento. Per maggiori dettagli \url{https://www.atlassian.com/agile/kanban/kanban-vs-scrum}}}

\newglossaryentry{incrementale}{name={incrementale},%
	description={Che procede per incrementi, ossia migliorie che, partendo da un impianto base, portano al raggiungimento dell'obiettivo producendo valore e limitando al massimo la regressione a stati già attraversati}}

\newglossaryentry{gitpush}{name={git-push},%
	description={Azione di git che permette di caricare le proprie modifiche locali sul server remoto condiviso del repository}}

\newglossaryentry{gitpull}{name={pull},%
	description={Comando di \gls{git} che permette di aggiornare il proprio \acrshort{repo} locale con i cambiamenti remoti}}

\newglossaryentry{gitcommit}{name={commit}, plural={commits},%
	description={Comando di \gls{git} che permette di salvare e versionare le modifiche attuate ai file in locale}}

\newglossaryentry{git}{name={git},%
	description={Sistema di controllo del versionamento distribuito}}

\newglossaryentry{gantt}{name={Gantt},%
	description={Diagramma per la pianificazione di un progetto. Esplicita le attività volte al suo svolgimento e per ognuna ne identifica data di inizio e di fine a seconda delle stime effettuate. Facilità la visualizzazione delle connessioni tra le attività e lo stato di avanzamento del progetto}}

\newglossaryentry{form}{name={form},%
	description={Interfaccia di un programma che consente a un utente di inserire e inviare uno o più dati}}

\newglossaryentry{fase}{name={fase}, plural={fasi},%
	description={Segmento temporale contiguo che che presuppone uno stazionamento in uno stato o in una transizione di ciclo di vita}}

\newglossaryentry{efficienza}{name={efficienza},%
	description={Misura dell'abilità di raggiungere l'obiettivo impiegando le risorse in maniera ottimale}}

\newglossaryentry{efficacia}{name={efficacia},%
	description={Misura della capacità di raggiungere l'obiettivo prefissato}}

\newglossaryentry{disciplinato}{name={disciplinato},%
	description={Che segue le regole che si è dato}}

\newglossaryentry{designpattern}{name={design pattern}, plural={design patterns},%
	description={Soluzione progettuale generale ad un problema ricorrente}}

\newglossaryentry{cruscotto}{name={cruscotto}, plural={cruscotti},%
	description={Interfaccia utente grafica che fornisce viste indicatori chiave di prestazione rilevanti per un particolare obiettivo o processo aziendale}}

\newglossaryentry{combobox}{name={combobox},%
	description={Controllo grafico (widget) che permette all'utente di effettuare una scelta scrivendola in una casella di testo o selezionandola da un elenco}}

\newglossaryentry{capitolato}{name={capitolato}, plural={capitolati},%
	description={Documento tecnico redatto dal cliente in cui vengono specificati i vincoli contrattuali (prezzo e scadenze) per lo sviluppo di un determinato prodotto software. Viene presentato in un bando d'appalto per trovare qualcuno che possa svolgere il lavoro richiesto}}

\newglossaryentry{bug}{name={bug},%
	description={Problema che porta al malfunzionamento del software, per esempio producendo un risultato inatteso o errato, tipicamente dovuto a un errore nella scrittura del codice sorgente di un programma}}

\newglossaryentry{bestpractice}{name={best practice}, plural={best practices},%
	description={Modo di fare noto, che abbia mostrato di garantire i migliori risultati in circostanze note e specifiche}}

\newglossaryentry{bash}{name={bash},%
	description={G:bash:bash:X:\gls{shell} Unix ed un linguaggio interpretato scritto da Brian Fox per il progetto GNU come sostituto del software gratuito per la shell Bourne}}

\newglossaryentry{baseline}{name={baseline},%
	description={Istantanea dello stato di avanzamento del progetto. Misura i risultati ottenuti e li fissa in una versione consolidata del prodotto. Può essere associata ad un rifermento temporale (\gls{milestone})}}

\newglossaryentry{attore}{name={attore}, plural={attori},%
	description={Ruolo interpretato da un utente (persona o sistema esterno) nei confronti del sistema. \begin{itemize} \item \textbf{attore primario}: attore richiede l'assistenza del sistema \item \textbf{attore secondario}: attore di cui è il sistema stesso a richiedere l'intervento \end{itemize}}}

\newglossaryentry{attivita}{name={attività},%
	description={Insieme di una o più azioni il cui completamento porta ad un avanzamento nel complesso}}

\newglossaryentry{artefatto}{name={artefatto}, plural={artefatti},%
	description={Sottoprodotto che viene realizzato durante lo sviluppo software. Sono artefatti i casi d'uso, i diagrammi delle classi, i modelli UML, il codice sorgente e la documentazione varia, che aiutano a descrivere la funzione, l'architettura e la progettazione del software}}

\newglossaryentry{agile}{name={agile},%
	description={Approccio stutturato e iterativo alla gestione dei progetti e alla sviluppo di software. Riconosce la volatilità dello sviluppo del prodotto e propone un metodo per permettere ai team di rispondere in maniera rapida a cambiamenti non pianificati e di produrre valore fin dalle prime iterazioni. Per maggiori dettagli \url{https://www.atlassian.com/agile/kanban/kanban-vs-scrum}}}

\newacronym{vcs}{VCS}{Version Control System}

\newacronym{uml}{UML}{Unified Modeling Language}

\newacronym{tb}{TB}{\gls{techbase}}

\newacronym{spa}{spa}{X}

\newacronym{rest}{REST}{REpresentational State Transfer}

\newacronym{repo}{repo}{\gls{repository}}

\newacronym{ram}{RAM}{Random Access Memory}

\newacronym{portacs}{PORTACS}{\acrshort{POI} Oriented Real Time Anti Collision System}

\newacronym{poc}{PoC}{Proof of Concept}

\newacronym{pb}{PB}{\gls{prodbase}}

\newacronym{jit}{JiT}{Just in Time}

\newacronym{faas}{FaaS}{Functions as a Service}

\newacronym{eg}{e.g.}{Example given, si usa per indicare che ciò che segue è un esempio}

\newacronym{cpu}{CPU}{Central Processing Unit}

\newacronym{cms}{CMS}{Content Management System}

\newacronym{baas}{BaaS}{Backend as a Service}

\newacronym{api}{API}{Application Programming Interface}

\newacronym{POI}{POI}{Point Of Interest}

