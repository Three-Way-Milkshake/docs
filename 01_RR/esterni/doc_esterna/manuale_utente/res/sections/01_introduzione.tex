\section{Introduzione}
\subsection{Scopo del documento}
    Il presente documento ha lo scopo di illustrare tutte le funzionalità del progetto PORTACS. In questo modo l'utente finale avrà tutte le indicazioni per un corretto utilizzo del software.


\subsection{Scopo del prodotto}

    Il capitolato C5 propone un progetto in cui viene richiesto lo sviluppo di un software per il monitoraggio in tempo reale di unità che si muovono in uno spazio definito. All'interno di questo spazio, creato dall'utente per riprodurre le caratteristiche di un ambiente reale, le unità dovranno essere in grado di circolare in autonomia, o sotto il controllo dell'utente, per raggiungere dei punti di interesse posti nella mappa. La circolazione è sottoposta a vincoli di viabilità e ad ostacoli propri della topologia dell'ambiente, dove è necessario evitare le collisioni tra unità e prevedere la gestione di situazioni critiche nel traffico.

\subsection{Riferimenti}
\label{ref}
    \subsubsection{Normativi}
    \begin{itemize}
    	\item \textsc{Norme di progetto}: per qualsiasi convenzione sulla nomenclatura degli elementi presenti all'interno del documento;
    	
    \end{itemize}

    \subsubsection{Informativi}
\begin{itemize}
	\item \textsc{\href{https://github.com/Three-Way-Milkshake/docs/wiki/Glossario}{Glossario}}: per la definizione dei termini (pedice G) e degli acronimi (pedice A) evidenziati nel documento;
	\item Capitolato d'appalto C5-PORTACS: \\
{\url{https://www.math.unipd.it/~tullio/IS-1/2020/Progetto/C5.pdf}}
	\item Software Engineering - Iam Sommerville - $10^{th}$ Edition.
	\item Angular: \\ {\url{https://angular.io/}};
	\item Node.js: \\ {\url{https://nodejs.org/en/}};
	\item PrimeNG: \\ {\url{https://www.primefaces.org/primeng/}};
	\item Java: \\ {\url{https://www.java.com/it/}};
	\item Spring: \\ {\url{https://spring.io/}};
	\item Docker: \\ {\url{https://www.docker.com/}}.
	
\end{itemize}