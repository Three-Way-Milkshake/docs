\section{Istruzione utilizzo utente operatore}

La seguente sezione fornirà indicazioni utili per il corretto utilizzo del software nel caso l'utente interessato sia l'operatore.
\subsection{Guida automatica dell'unità}
\begin{itemize}
    \item Premere sul pulsante "Start" per iniziare il movimento del muletto verso il primo POI della lista;
    \item i movimenti effettuati sono visualizzati nella mappa e nelle frecce.
\end{itemize}
\subsection{Passaggio a guida manuale/automatica}
\begin{itemize}
    \item Premere sul pulsante "Cambio guida";
    \item i comandi verranno cambiati in base allo stato di guida;
\end{itemize}
\subsection{Guida manuale dell'unità}
\begin{itemize}
    \item Premere sul pulsante "Metti in guida manuale";
    \item premere "Start" per iniziare il movimento del muletto;
    \item premere nella frecce visualizzate per spostare l'unità nella direzione desiderata;
    \item premere "Stop" per fermarsi;
\end{itemize}
\subsection{Completamento task}
\begin{itemize}
    \item Una volta raggiunto il POI, viene visualizzato un bottono "Task completata";
    \item premere su esso per notificare lo scarico delle merci e per continuare il proprio percorso.
\end{itemize}

\subsection{Segnalazione evento eccezionale}
\begin{itemize}
    \item Premere sul pulsante "Evento eccezionale" per notificare l'amministratore dell'avvenimento di un evento non previsto.
\end{itemize}