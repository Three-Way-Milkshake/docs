\section{Valutazioni per il miglioramento}
L'obiettivo di questa sezione è la valutazione atta al miglioramento dell'intero processo produttivo legato al progetto\textsubscript{G} in corso. Risulta necessario trovare un modo per affrontare i problemi che possono sorgere durante il lavoro, così da poter proporre soluzioni efficienti per la loro risoluzione. E` inoltre necessario tenere traccia dei problemi riscontrati e delle loro soluzioni, così che essi non vengano ripetuti.
Più in dettaglio si valuteranno i problemi legati a:
\begin{itemize}
    \item organizzazione: qualsiasi problema inerente all'organizzazione e alla collaborazione del gruppo;
    \item ruoli: qualsiasi problema legato allo svolgimento di un ruolo;
    \item strumenti: qualsiasi problema riscontrato nell'utilizzo di determinati strumenti.
\end{itemize}

Una difficoltà rilevante in queste valutazioni è il fatto che sono gestite dal gruppo stesso, quindi si tratta di un'autovalutazione. Ogni singolo membro deve esternare i propri problemi individuali e quelli di gruppo per permettere una celere risoluzione e favorire un lavoro più efficiente.
Tale sezione mira quindi a migliorare costantemente la qualità di prodotto, infatti verrà aggiornata durante l'intero periodo\textsubscript{G} di progetto\textsubscript{G} man mano che si verificheranno problemi.
Vi è inoltre una sezione riguardante i  rischi all'interno del \textsc{Piano di Progetto} con la loro descrizione e relativa soluzione a completamento di questa parte sui possibili problemi.
\subsection{Valutazioni sull'organizzazione}
\renewcommand{\arraystretch}{1.5}
\rowcolors{2}{pari}{dispari}
\begin{longtable}{
    >{}p{0.5\textwidth}
        >{}p{0.5\textwidth}
}
\rowcolorhead
\centering \headertitle{Problema} &
\centering \headertitle{Soluzione}
\endfirsthead
\endhead
Durante i primi periodi si ha avuto difficoltà a comunicare con tutti i membri del gruppo, avendo difficoltà a organizzare gli incontri e a ricevere risposta per domande o chiarificazioni sul proprio lavoro & Si è deciso di utilizzare come sistema di comunicazione ufficiale Slack così, oltre ad avere diversi topic di conversazione, si ha un promemoria automatico per l'avviso di nuove riunioni \\

Difficoltà nel rispettare le scadenze dei lavori assegnati; probabile causa la scarsa esperienza di pianificazione e quindi erronea stima del tempo impiegato per un determinato lavoro & Come soluzione si è deciso di rispettare di più le scadenze, lavorando più del periodo\textsubscript{G} passato e di stimare le scadenze con più cura. \\
\caption{Tabella Problemi di organizzazione}
    \end{longtable}.

\subsection{Valutazioni sui ruoli}

\subsubsection{Analista}
\renewcommand{\arraystretch}{1.5}
\rowcolors{2}{pari}{dispari}
\begin{longtable}{
    >{}p{0.5\textwidth}
        >{}p{0.5\textwidth}
}
\rowcolorhead
\centering \headertitle{Problema} &
\centering \headertitle{Soluzione}
\endfirsthead
\endhead
Riscontrata difficoltà nell'individuazione dei requisti per la creazione dell'\textsc{Analisi dei Requisiti}. Si è individuato il problema come conseguenza principale dell'inesperienza sull'argomento e della difficoltà nell'affrontarlo singolarmente. & Si è passati ad un lavoro più collettivo sfruttando i mezzi di comunicazione appositi. \\

Difficoltà nella creazione degli schemi dei casi d'uso, probabilmente causa della poca esperienza. & Si è deciso di lavorare più in gruppo per comprendere meglio l'argomento. \\
\caption{Tabella Problemi Analista}
    \end{longtable}


\subsubsection{Verificatore}
\renewcommand{\arraystretch}{1.5}
\rowcolors{2}{pari}{dispari}
\begin{longtable}{
    >{}p{0.5\textwidth}
        >{}p{0.5\textwidth}
}
\rowcolorhead
\centering \headertitle{Problema} &

\centering \headertitle{Soluzione}
\endfirsthead
\endhead
Difficoltà nell'analisi approfondita dei documenti per verificarne correttezza e completezza. Questo è causato probabilmente dallo scarso tempo dedicato all'attività di verifica. & Si è deciso di dedicare più tempo all'attività di verifica cosicché i Verificatori potranno correggere in modo più approfondito. \\
\caption{Tabella problemi verificatore}
    \end{longtable}.

\subsection{Valutazioni sugli strumenti}

\subsubsection{\LaTeX}
\renewcommand{\arraystretch}{1.5}
\rowcolors{2}{pari}{dispari}
\begin{longtable}{
    >{}p{0.5\textwidth}
        >{}p{0.5\textwidth}
}
\rowcolorhead
\centering \headertitle{Problema} &

\centering \headertitle{Soluzione}
\endfirsthead
\endhead
Difficoltà nell'apprendimento dello strumento e quindi nella scrittura di documenti. & Si è ricordato ai Verificatori di controllare oltre alla correttezza del contenuto dei documenti, anche la corretta impaginazione. \\
\caption{Tabella problemi \LaTeX}
    \end{longtable}.
