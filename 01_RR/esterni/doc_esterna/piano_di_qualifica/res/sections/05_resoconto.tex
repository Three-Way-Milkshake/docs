
\section{Resoconto {attività} di verifica}
In questa sezione possiamo vedere gli esiti delle attività\textsubscript{G} di verifica.
Il nostro cruscotto\textsubscript{G} è presente al seguente indirizzo:\\ \url{https://sites.google.com/view/three-way-milkshake-dashboard}.\\ (Nota: Nel caso di problemi di visualizzazione, utilizzare un account non unipd o una finestra in incognito).
\subsection{Osservazioni}
\subsubsection{Avvio}
Durante il macro periodo di avvio, il numero di riunioni interne è stato esiguo e il numero di riunioni esterne è stato nullo. Di conseguenza, il valore di \textbf{SRI} (Scarto Riunioni Interne) rientra nel range dei valori accettabili, mentre la \textbf{SRE} (Scarto Riunioni Esterne) non è stato possibile calcolarla e quindi la \textbf{REI} (Rapporto riunioni Esterne e Interne) risulta non superata. \\
Analogamente anche la \textbf{RRL} (Rapporto tempo Riunioni e Lavoro individuale) risulta non superata, in quanto il tempo impiegato nel lavoro individuale è stato proporzionalmente molto maggiore rispetto a quello trascorso nelle riunioni. \\
Sebbene il valore della metrica \textbf{RTPI} (Rapporto Tempo Preventivato totale e Individuale) risulta superato, e quindi questo sta ad indicare che il tempo preventivato è spartito equamente tra tutti i componenti del gruppo, il valore del \textbf{RTEI} (Rapporto Tempo Effettivo totale e Individuale) risulta non superato da due componenti del gruppo. Nonostante ciò, la \textbf{DTEP} (Differenza Tempo Effettivo e Preventivato) risulta superata da tutti i membri del gruppo.\\
Tutte le altre metriche non sono state calcolate.\\\\
In generale da questi dati, risulta chiaro che il tempo trascorso riunioni interne è stato troppo esiguo rispetto al lavoro individuale, il quale dev'essere oltretutto preventivato e quindi spartito in modo più consono tra i vari componenti del gruppo.

\subsubsection{Analisi dei requisiti}
Durante il macro periodo di analisi dei requisiti, a fronte di un numero maggiore di tempo trascorso riunioni interne, il valore di \textbf{SRI} (Scarto Riunioni Interne) è risultato comunque superato. È stata fatta una prima (e unica) riunione con il proponente, che ha permesso di calcolare il valore di \textbf{SRE} (Scarto Riunioni Esterne), ma il valore della metrica (Rapporto riunioni Esterne e Interne) risulta sempre non superata.\\
Anche il valore della \textbf{RRL} (Rapporto tempo Riunioni e Lavoro individuale) risulta non superato, in quanto il tempo impiegato nel lavoro individuale è stato proporzionalmente molto maggiore rispetto a quello trascorso nelle riunioni, e il valore registrato è ancora peggiore rispetto a quello registrato durante il macro periodo di avvio. \\
Ancora una volta il valore della metrica \textbf{RTPI} (Rapporto Tempo Preventivato totale e Individuale) risulta superato, ma, anche questa volta, il valore del \textbf{RTEI} (Rapporto Tempo Effettivo totale e Individuale) risulta non superato da un solo componente del gruppo e, analogamente, anche la \textbf{DTEP} (Differenza Tempo Effettivo e Preventivato) risulta non superata da un solo componente del gruppo. \\
È stato calcolato per la prima volta anche il valore \textbf{IG} (Indice di Gulpease), che risulta superato in toto.\\
Tutte le altre metriche non sono state calcolate.\\\\
In generale, questi dati mettono in risalto come il tempo preventivato deve essere calcolato e spartito in modo più consono tra i vari componenti del gruppo considerando anche il tempo trascorso nelle riunioni interne, e che il numero di riunioni esterne risulta essere ancora troppo basso.

\subsubsection{Progettazione Architetturale}
Durante il macro periodo di progettazione architetturale, sono state incrementate le riunioni con il proponente e internamente sono state predilette le riunioni in sottogruppi tra i componenti del gruppo. Così facendo,
il valore di \textbf{SRI} (Scarto Riunioni Interne) è risultato superato, così come il valore di \textbf{SRE} (Scarto Riunioni Esterne). Nonostante ciò, il valore della metrica (Rapporto riunioni Esterne e Interne) risulta sempre non superata, addirittura con un valore peggiore rispetto a quello riscontrato nel macro periodo precedente, visto che il tempo trascorso nelle riunioni interne è ancora molto maggiore rispetto a quello nelle riunioni esterne.\\
Grazie alla suddivisione del lavoro in sottogruppi e ad una preventivazione migliore del tempo impiegato individualmente, il valore della \textbf{RRL} (Rapporto tempo Riunioni e Lavoro individuale) e il valore del \textbf{RTEI} (Rapporto Tempo Effettivo totale e Individuale), in questo macro periodo, risultano entrambi superati con un valore accettabile. \\
La preventivazione del tempo individuale è stata migliore, in quanto il valore della metrica \textbf{RTPI} risulta sempre superato, ma non perfetta come dimostra il valore della \textbf{DTEP} (Differenza Tempo Effettivo e Preventivato) risulta comunque non superata da metà dei componenti del gruppo.\\
Grazie alla creazione e all'implementazione del cruscotto, sono state calcolate nuove utili metriche che hanno permesso di valutare meglio il nostro andamento. \\
La \textbf{PDDWT} (Percentuale Discostamento DoneWorking in Tempo) e la \textbf{PDDWR} (Percentuale Discostamento DoneWorking in Ritardo) risultano entrambe superate. Risulta non superata la \textbf{PDDWA} (Percentuale Discostamento DoneWorking in Anticipo). Allo stesso modo la \textbf{PDDVT} (Percentuale Discostamento DoneVerifying in Tempo) e la \textbf{PDDVR} (Percentuale Discostamento DoneVerifying in Ritardo) risultano entrambe superate e invece risulta non superata la \textbf{PDDVA} (Percentuale Discostamento DoneVerifying in Anticipo). \\ Questo porta la \textbf{PDTT} (Percentuale Discostamento Totale in Tempo) e la \textbf{PDTR} (Percentuale Discostamento Totale in Ritardo) a risultare entrambe superate, mentre non superata la \textbf{PDTA} (Percentuale Discostamento Totale in Anticipo).\\ I valori risultati mostrano che molti task hanno completato il loro ciclo, prima della data di scadenza prefissata, e questo è indicatore di una cattiva gestione dei tempi, perché il tempo che inizialmente era stato convogliato in task risolvibili in meno tempo, è stato tolto a task più onerose.\\
Il valore di \textbf{IG} (Indice di Gulpease), che risulta ancora superato completamente.\\
Tutte le altre metriche non sono state calcolate.\\\\
In generale, per il prossimo macro periodo sarà necessario preventivare in maniera ancora migliore il tempo da spartire ai vari componenti del gruppo e per i vari task.

\subsubsection{Progettazione di Dettaglio e Codifica}
Durante il macro periodo di progettazione dettaglio e codifica, sono state necessarie molte riunioni interne per poter procedere con la realizzazione del progetto.
Il valore di \textbf{SRI} (Scarto Riunioni Interne) è risultato superato, così come il valore di \textbf{SRE} (Scarto Riunioni Esterne). Nonostante ciò, il valore della metrica (Rapporto riunioni Esterne e Interne) risulta non superata, in quanto abbiamo avuto molte riunioni interne.\\
Grazie alla suddivisione del lavoro in sottogruppi e ad una preventivazione migliore del tempo impiegato individualmente, il valore della \textbf{RRL} (Rapporto tempo Riunioni e Lavoro individuale) risulta superato con valore accettabile. , in questo macro periodo, risultano entrambi superati con un valore accettabile. \\
Il valore della metrica \textbf{RTPI}(Rapporto Tempo Preventivato totale e Individuale) e il valore del \textbf{RTEI} (Rapporto Tempo Effettivo totale e Individuale) risultano entrambi superati.
Questa volta il valore della \textbf{DTEP} (Differenza Tempo Effettivo e Preventivato) risulta superata da tutti i componenti del gruppo.\\
La \textbf{PDDWT} (Percentuale Discostamento DoneWorking in Tempo), la \textbf{PDDWA} (Percentuale Discostamento DoneWorking in Anticipo) e la \textbf{PDDWR} (Percentuale Discostamento DoneWorking in Ritardo) risultano entrambe non superate. Risulta non superata . Diversamente la \textbf{PDDVT} (Percentuale Discostamento DoneVerifying in Tempo), la \textbf{PDDVA} (Percentuale Discostamento DoneVerifying in Anticipo) e la \textbf{PDDVR} (Percentuale Discostamento DoneVerifying in Ritardo) risultano superate. \\ Questo porta la \textbf{PDTT} (Percentuale Discostamento Totale in Tempo) e la \textbf{PDTR} (Percentuale Discostamento Totale in Ritardo) a risultare entrambe superate, mentre non superata la \textbf{PDTA} (Percentuale Discostamento Totale in Anticipo).\\ I valori risultati mostrano che molti task hanno completato il loro ciclo, prima della data di scadenza prefissata, e questo è indicatore di una cattiva gestione dei tempi, perché il tempo che inizialmente era stato convogliato in task risolvibili in meno tempo, è stato tolto a task più onerose.\\
Il valore di \textbf{IG} (Indice di Gulpease) risulta ancora superato completamente.\\
Riguardo alle metriche di prodotto.\\
Il valore di PROS (Percentuale Requisiti Obbligatori Soddisfatti) è stata superata.\\I valori di DEP (Depth of Hierarchies), LEV (Level of Nesting), PAR (Parametri per metodo) e CCL (Complessità Ciclomatica) sono state superate in quanto rientrano tutti in valori accettabili.\\La metrica di Code Coverage non è stata superata, in quanto dobbiamo ancora implementare molti test.\\PST (Percentuale di Superamento Test), CS (Completezza Software), A (Affidabilità Software), C (numero Click necessari), S(Numero Secondi necessari) sono state tutte superate, poiché rientrano in valori accettabili.
