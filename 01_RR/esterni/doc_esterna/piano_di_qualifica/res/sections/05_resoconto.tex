
\section{Resoconto {attività} di verifica}
In questa sezione possiamo vedere gli esiti delle attività\textsubscript{G} di verifica durante la fase\textsubscript{G} di Avvio e Analisi dei Requisiti, e quelle in corso nella fase\textsubscript{G} di Progettazione Architetturale.\\
Il nostro cruscotto\textsubscript{G} è presente al seguente indirizzo:\\ \url{https://sites.google.com/view/three-way-milkshake-dashboard}.\\ (PS: Nel caso di problemi di visualizzazione, utilizzare un account non unipd o una finestra in incognito).
\subsection{Osservazioni}
\subsubsection{Avvio e Analisi dei Requisiti}
Dati gli esiti delle attività\textsubscript{G} di verifica delle fasi di Avvio e Analisi dei Requisiti, è preferibile:
\begin{itemize}
	\item aumentare la quantità di riunioni esterne col proponente;
	\item diminuire la quantità di riunioni interne e aumentare la quantità di lavoro individuale.
\end{itemize}
\subsubsection{Progettazione Architetturale}
Dati gli esiti delle attività di verifica della fase di Progettazione Architetturale, possiamo notare:
\begin{itemize}
	\item il rapporto riunioni esterne interne (REI) è ancora insufficente;
	\item il rapporto riunioni interne e lavoro individuale (RRL) è ora accettabile;
	\item il tempo effettivo e preventivato totale e individuale (RTEI e RTPI) per i vari membri del gruppo è accettabile, ma la differenza tra il tempo effettivo e preventivato (DTEP) di 3 componenti del gruppo risulta troppo alto, indice che hanno lavorato più del preventivato;
	\item le metriche di distribuzione lavoro effettivo e preventivato (DLE e DLP) risultano anch'esse entro i limiti accettabili;
	\item infine le metriche relative alle tempistiche di completamento dei ticket sono accettabili, tranne le metriche relative ai ticket completati in anticipo, poichè ci sono meno ticket completati in anticipo del previsto.
\end{itemize}