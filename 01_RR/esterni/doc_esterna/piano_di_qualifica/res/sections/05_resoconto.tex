
\section{Resoconto \gls{attivita}\textsubscript{G} di verifica}
\subsection{Verifica della Documentazione}
Vengono riportati i risultati ottenuti al termine della procedura di verifica riguardo alle metriche sulla documentazione.
Per verificare quanto leggibili sono i documenti abbiamo utilizzato l'indice di Gulpease, calcolato attraverso uno script redatto in python.
\subsubsection{Metriche di Leggibilità della documentazione}
\begin{itemize}
	\item Esiti dell'indice di Gulpease (IG)
\end{itemize}
\begin{table}[H]
	\begin{center}
		\caption{Tabella dei valori Gulpease}
		\begin{tabular}{ccc}
			\rowcolorhead
			\headertitle{Nome Documento} & \headertitle{Valore Gulpease} & \headertitle{Esito}\\
			
			\textsc{Analisi dei Requisiti} v1.0.0 & 66 & Superato\\
			\textsc{Glossario} v1.0.0 & 62 & Superato\\
			\textsc{Norme di Progetto} v1.0.0 & 70 & Superato\\
			\textsc{Piano di Qualifica} v1.0.0 & 74 & Superato\\
			\textsc{Piano di Progetto} v1.0.0 & 56 & Superato\\
			\textsc{Studio di Fattibilità} v1.0.0 & 58 & Superato\\
			\textsc{Verbale Esterno 1} & 55 & Superato\\
			\textsc{Verbale Interno 1} & 55 & Superato\\
			\textsc{Verbale Interno 2} & 53 & Superato\\
			\textsc{Verbale Interno 3} & 67 & Superato\\
			\textsc{Verbale Interno 4} & 65 & Superato\\
			\textsc{Verbale Interno 5} & 73 & Superato\\
			
		\end{tabular}
		
	\end{center}
\end{table}
\subsection{Verifica dei Processi}
Vengono riportati i risultati ottenuti al termine della procedura di verifica riguardo alle metriche sui processi.
\subsubsection{Metriche di Miglioramento continuo}
\begin{itemize}
	\item Scarto Riunioni (SR)
	\item Totale Riunioni (TR)
	\item Totale Riunioni Interne (TRI)
	\item Totale Riunioni Esterne (TRE)
	\item Scarto Ticket (ST)
	\item Differenza To Verify (DV)
	\item Uniformità del lavoro nel tempo (UL)
\end{itemize}
\subsubsection{Metriche di Corretta Pianificazione}
\begin{itemize}
	\item Budget at Completion (BAC)
	\item Earned Value (EV)
	\item Planned Value (PV)
	\item Schedule Variance (SV)
	\item Actual Cost (AC)
\end{itemize}
\subsubsection{Metriche di Implementazione Requisiti Obbligatori}
\begin{itemize}
	\item Percentuale Requisiti Obbligatori Soddisfatti (PROS)
\end{itemize}
\pagebreak
\subsection{Verifica del Prodotto}
Vengono riportati i risultati ottenuti al termine della procedura di verifica riguardo alle metriche sul prodotto, ovvero tutte le metriche riguardanti il codice ed il software.
Queste metriche non sono ancora state calcolate poichè dobbiamo ancora sviluppare il codice, quindi le abbiamo solo elencate.
\subsubsection{Metriche di Manutenzione e comprensione del codice}
\begin{itemize}
	\item Coupling Between Objects (CBO)
	\item Depth of hierarchies (DEP)
	\item Level of nesting (LEV)
	\item Parametri per metodo (PAR)
	\item Rapporto Codice Commenti (RCC)
	\item Leggibilità Software (L)
\end{itemize}
\subsubsection{Metriche di Copertura del codice}
\begin{itemize}
	\item Code Coverage (CC)
\end{itemize}
\subsubsection{Metriche di Superamento test}
\begin{itemize}
	\item Percentuale Superamento Test (PST)
\end{itemize}
\subsubsection{Metriche di Conformità}
\begin{itemize}
	\item Completezza del Software (CS)
\end{itemize}
\subsubsection{Metriche di Robustezza}
\begin{itemize}
	\item Affidabilità del software (A)
\end{itemize}
\subsubsection{Metriche di Usabilità}
\begin{itemize}
	\item Numero di tocchi/click necessari (C)
	\item Numero di secondi necessari (S)
\end{itemize}