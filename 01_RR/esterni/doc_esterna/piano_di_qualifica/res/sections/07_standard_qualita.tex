\section{Standard di qualità}
\subsection{ISO/IEC 12207}
\subsection{ISO/IEC 9126}
ISO/IEC 9126 è uno standard internazionale per valutare la qualità del software.\\
Questo standard fornisce un modello di qualitè e 3 tipologie di metriche, queste 4 sezioni vengono riportate di seguito.
\subsubsection{Metriche per la qualità interna}
Definisce metriche applicabili al codice sorgente non eseguibile. Idealmente, la qualità interna determina la qualità esterna.\\
Viene rilevata tramite \textbf{analisi statica}.
\subsubsection{Metriche per la qualità esterna}
Definisce metriche applicabili al software in esecuzione che ne misurano i comportamenti tramite test. Idealmente, la qualità esterna determina la qualità in uso.\\
Viene rilevata tramite \textbf{analisi dinamica}.
\subsubsection{Metriche per la qualità in uso}
Definisce metriche applicabili solo quando il prodotto è finito e utilizzato in condizioni reali.
\subsubsection{Modello della qualità del software}
\subsection{ISO/IEC 25010:2011}