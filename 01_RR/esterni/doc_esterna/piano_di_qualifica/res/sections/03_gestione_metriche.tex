\section{Gestione amministrativa}
\subsection{Misure e metriche in dettaglio}
In questa sezione vengono descritte nel dettaglio le varie metriche utilizzate, accompagnate dalle relative modalità di calcolo.
Le soglie di accettabilità sono riportate nella tabella 2.4.1 "Tabella delle Metriche", valori inferiori ai limiti accettabili sono considerati negativi e il prodotto o processo dovrà essere sottoposto ad ulteriori indagini e verifiche.
\subsection{Metriche per i processi}
Per tenere traccia delle metriche per i processi, è stato utilizzato un foglio Google Sheet, così che ogni membro del gruppo possa inserire i dati relativi al lavoro proprio e collettivo nelle apposite tabelle. Inoltre esso permette di calcolare in automatico i valori e visualizzarli sotto forma di grafico.

\subsubsection{Scarto Riunioni Interne (SRI)}
Questa metrica mostra la differenza fra il tempo preventivato e il tempo effettivo delle riunioni interne in minuti. In questo modo si può vedere se la pianificazione è corretta, oppure se serve un controllo.
\[\frac{\sum_{i=1}^{num\_riunioni\_interne}min\_durata\_preventivata_i-min\_durata\_effettiva_i}{num\_riunioni\_interne}\]
con $i = {numero\_della\_riunione\_interna}$.
\subsubsection{Scarto Riunioni Esterne (SRE)}
Con questo calcolo si può trovare la differenza tra il tempo preventivato e il tempo effettivo delle riunioni esterne in minuti, così da controllare se la pianificazione è corretta.
\[\frac{\sum_{i=1}^{num\_riunioni\_esterne}min\_durata\_preventivata_i-min\_durata\_effettiva_i}{num\_riunioni\_esterne}\]
con $i = {numero\_della\_riunione\_esterna}$.
\subsubsection{Rapporto riunioni Esterne e Interne (REI)}
Si tratta del rapporto tra il tempo totale impiegato nelle riunioni esterne e quello nelle riunioni interne. Serve per raggiungere un equilibrio negli incontri del gruppo.
\[\frac{\sum_{i=1}^{num\_riunioni\_esterne} durata_i}{\sum_{i=1}^{num\_riunioni\_interne} durata_i}\]

Se il valore calcolato tende a:
\begin{itemize}
    \item 1: vi è una distribuzione equa del tempo impiegato nelle riunioni interne e esterne;
    \item 0: il tempo impiegato nelle riunioni esterne è molto inferiore rispetto a quello delle riunioni interne;
    \item $+\infty$: il tempo impiegato nelle riunioni esterne è molto superiore rispetto a quello delle riunioni interne.
\end{itemize}
\subsubsection{Rapporto tempo Riunioni e Lavoro individuale (RRL)}
Indica il rapporto tra le ore dedicate alle riunioni, quindi al lavoro collettivo, e quelle dedicate al lavoro individuale.
\[\dfrac{\sum_{i=1}^{num\_riunioni\_totali} durataRiunioni_i}{\sum_{i=1}^{num\_persone\_gruppo} durataLavoro_i}\]
Offre una visione sulla distribuzione del lavoro collettivo e individuale.
\subsubsection{Rapporto Tempo Effettivo totale e Individuale (RTEI)}
Indica il rapporto tra i minuti di lavoro effettivamente spesi da ogni membro e il tempo di lavoro totale del gruppo.
\[tot\_ore\_effettive\_persona / tot\_ore
\]
Questa metrica viene calcolata per ogni membro del gruppo.
\subsubsection{Rapporto Tempo Preventivato totale e Individuale (RTPI)}
Indica il rapporto tra i minuti di lavoro preventivato per svolgere i propri compiti da parte di ogni membro e il tempo di lavoro totale preventivato dal gruppo.
\[tot\_min\_preventivati\_persona / tot\_min\_preventivato\]
Questa metrica deve essere calcolata per ogni membro del gruppo.
\subsubsection{Differenza Tempo Effettivo e Preventivato (DTEP)}
Questa metrica mostra la discrepanza tra il tempo effettivo impiegato allo svolgimento dei compiti e quello preventivato precedentemente, per ogni membro del gruppo.
\[tempo\_effettivo_i - tempo\_preventivato_i\]
con $i \in {componenti\_del\_gruppo}$
\subsubsection{Distribuzione Lavoro Preventivato (DLP)}
Mostra se la pianificazione del lavoro preventivata è bilanciata, ovverosia distribuita in modo equo all'interno del gruppo.
\[\sqrt{\frac{\sum_{i=1}^{n\_componenti}(lavoro_i-media\_lavoro)^2}{n\_componenti}}\]
con:\\
$lavoro_i$ = lavoro individuale preventivato;\\
$media\_lavoro$ = media lavoro preventivato;\\
$n\_componenti$ = numero totale dei componenti (6).
\\Se il risultato tende a:
\begin{itemize}
	\item 0: significa che il lavoro è uniformemente distribuito;
	\item $+\infty$: il lavoro è distribuito in modo poco uniforme.
\end{itemize}
\subsubsection{Distribuzione Lavoro Effettivo (DLE)}
Mostra quanto sia distribuito in modo uniforme il lavoro effettuato, così da poter adattare le future organizzazioni dei compiti.
\[\sqrt{\frac{\sum_{i=1}^{n\_componenti}(lavoro_i-media\_lavoro)^2}{n\_componenti}}\]

Se il risultato tende a:
\begin{itemize}
	\item 0: significa che il lavoro è uniformemente distribuito;
	\item $+\infty$: il lavoro è distribuito in modo poco uniforme.
\end{itemize}

\subsubsection{Percentuale Discostamento Totale (in Tempo) (PDTT)}
Indica la percentuale dei compiti completati in tempo rispetto al numero totale. Per completati in tempo si intendono le task\textsubscript{G} che hanno terminato il loro ciclo, ovverosia che sono state verificate esattamente alla data di scadenza prefissata.
\[\frac{n\_compiti\_risolti\_intempo}{tot\_num\_compiti}\]
\subsubsection{Percentuale Discostamento Totale (in Ritardo) (PDTR)}
Indica la percentuale dei compiti completati in ritardo rispetto al numero totale. Per completati in ritardo si intendono le task\textsubscript{G} che hanno completato il loro ciclo, ovverosia che sono state verificate dopo la data di scadenza prefissata.
\[\frac{n\_compiti\_risolti\_inritardo}{tot\_num\_compiti}\]
\subsubsection{Percentuale Discostamento Totale (in Anticipo) (PDTA)}
Indica la percentuale dei compiti completati in anticipo rispetto al numero totale. Per completati in anticipo si intendono le task\textsubscript{G} che hanno completato il loro ciclo, ovverosia che sono state verificate prima della data di scadenza prefissata.
\[\frac{n\_compiti\_risolti\_inanticipo}{tot\_num\_compiti}\]

\subsubsection{Percentuale Discostamento DoneWorking (in Tempo) (PDDWT)}
Indica la percentuale di compiti risolti, ma non ancora verificati, rispetto al numero totale. In questo caso si intendono solo i compiti completati esattamente alla data di scadenza prefissata.
\[\frac{n\_compiti\_risoltiDW\_intempo}{tot\_num\_compiti}\]
\subsubsection{Percentuale Discostamento DoneWorking (in Ritardo) (PDDWR)}
Indica la percentuale di compiti risolti, ma non ancora verificati, rispetto al numero totale. In questo caso si intendono solo i compiti completati dopo la data di scadenza prefissata.
\[\frac{n\_compiti\_risoltiDW\_inritardo}{tot\_num\_compiti}\]
\subsubsection{Percentuale Discostamento DoneWorking (in Anticipo) (PDDWA)}
Indica la percentuale di compiti risolti, ma non ancora verificati, rispetto al numero totale. In questo caso si intendono solo i compiti completati prima della data di scadenza prefissata.
\[\frac{n\_compiti\_risoltiDW\_inanticipo}{tot\_num\_compiti}\]
\subsubsection{Percentuale Discostamento DoneVerifying (in Tempo) (PDDVT)}
Indica la percentuale dei compiti verificati, rispetto al numero totale. In questo caso si intendono solo le operazioni di verifica concluse esattamente alla data di scadenza prefissata.
\[\frac{n\_compiti\_risoltiDV\_intempo}{tot\_num\_compiti}\]

\subsubsection{Percentuale Discostamento DoneVerifying (in Ritardo) (PDDVR)}
Indica la percentuale dei compiti verificati, rispetto al numero totale. In questo caso si intendono solo le operazioni di verifica concluse dopo la data di scadenza prefissata.
\[\frac{n\_compiti\_risoltiDV\_inritardo}{tot\_num\_compiti}\]
\subsubsection{Percentuale Discostamento DoneVerifying (in Anticipo) (PDDVA)}
Indica la percentuale dei compiti verificati, rispetto al numero totale. In questo caso si intendono solo le operazioni di verifica concluse prima della data di scadenza prefissata.
\[\frac{n\_compiti\_risoltiDV\_inanticipo}{tot\_num\_compiti}\]
\subsection{Metriche per la documentazione}

\subsubsection{Indice di Gulpease (IG)}
Indica la leggibilità di un testo, tarato sulla lingua italiana. Differentemente da indici di lingua straniera, ha il vantaggio di controllare la lunghezza delle parole anziché il numero di sillabe per parola, semplificandone il calcolo automatico.
Nel calcolo vengono ignorati frontespizio, registro modifiche, elenco figure, elenco tabelle, tabelle e figure; in modo da poter valutare appieno la leggibilità del contenuto testuale dei documenti.
Il valore risultante è compreso tra 0 e 100, dove ad un indice più alto corrisponde una maggiore leggibilità.
Le soglie dei valori dell'indice di Gulpease sono:
\begin{itemize}
    \item inferiore a 80, il documento è difficile da leggere per chi ha la licenza elementare;
    \item inferiore a 60, il documento è difficile da leggere per chi possiede la licenza media;
    \item inferiore a 40, il documento è difficile da leggere per chi ha un diploma superiore.
\end{itemize}
\[
    89+ \frac{300\cdot (num\_frasi) - 10\cdot (num\_lettere)}{num\_parole}
\]

\subsection{Metriche per il software}
Questa sezione contiene le metriche che si cercherà di applicare al software prodotto. A causa dell'inesperienza del gruppo, tali valori sono una dichiarazione di intenti per la qualità del software e potrebbero essere rivisti con le successive revisioni.

\subsubsection{Percentuale Requisiti Obbligatori Soddisfatti (PROS)}
Indica la quantità di requisiti obbligatori soddisfatti rispetto al totale, così da poterli monitorare in ogni istante.
\[\frac{requisiti\_obbligatori\_soddisfatti}{requisiti\_obbligatori\_totali}\]

\subsubsection{Coupling Between Objects (CBO)}
Indica l'accoppiamento tra classi e oggetti; due classi si dicono accoppiate se una utilizza metodi o variabili dell'altra.

\subsubsection{Depth of hierarchies(DEP)}
Indica la profondità delle gerarchie nel codice sviluppato. Va limitato questo valore in modo da limitare l'accoppiamento. Preferibilmente le classi dovranno dipendere solo da classi astratte e potranno implementare una o più interfacce. In ogni caso non deve venire usata l'ereditarietà multipla.

\subsubsection{Level of nesting (LEV)}
Questa metrica indica il livello di annidamento nei vari metodi presenti nel codice prodotto. Questo valore deve essere il più basso possibile, sia per una questione di leggibilità del codice, che di manutenibilità.


\subsubsection{Parametri per metodo (PAR)}
Indica il numero di parametri presenti nei metodi sviluppati nel codice. Un numero troppo elevato potrebbe indicare una complessità troppo elevata del metodo.

\subsubsection{Attributi per classe (ATT)}
Considera il numero totale di attributi per ogni classe. Un valore elevato potrebbe indicare che la classe si fa carico di una quantità eccessiva di responsabilità, in questo caso si può optare per incapsulare parte di essa in un'altra classe.

\subsubsection{Metodi per classe (MET)}
Rappresenta il numero di metodi per classe. Se troppo elevato, potrebbe indicare che questa classe svolge troppi compiti, sarà quindi preferibile scomporla in più classi.

\subsubsection{Rapporto Codice Commenti (RCC)}
Indica il rapporto tra le linee di codice e le linee di commento all'interno dei file. Questo rapporto aiuta a stimare la manutenibilità del codice. Un rapporto troppo basso indica una carenza di informazioni necessarie alla comprensione del codice scritto
\[\frac{linee\_commento}{linee\_codice}\]

\subsubsection{Complessità Ciclomatica (CCL)}
Questa metrica è utilizzata per stimare la complessità di funzioni, metodi o classi di un programma. Questo valore rappresenta quanto complesso è un metodo tramite la misura del numero di cammini linearmente indipendenti che attraversano il grafo di flusso di controllo. Un valore troppo elevato porta ad un'eccessiva complessità del codice, che comporta difficile manutenzione. Al contrario, un valore ridotto potrebbe indicare una scarsa efficienza\textsubscript{G} dei metodi. Per calcolarlo si rappresenta il programma con un grafo dove i  nodi (\textbf{N}) sono i gruppi indivisibili di istruzioni e un arco (\textbf{E}) connette due nodi se le istruzioni di uno dei nodi possono essere eseguite direttamente dopo l'esecuzione delle istruzioni dell'altro nodo. Quindi il valore interessato è:
\[E-N+2P\]
dove P è il numero di componenti connesse.
\subsubsection{Code Coverage (CC)}
Indica la quantità di codice che viene effettivamente eseguito durante i test; aiuta a valutare la completezza di questi. Maggiore sarà la copertura del codice, maggiore sarà la possibilità che eventuali errori vengano individuati e risolti. Un valore troppo basso indica un'insufficiente verifica della correttezza del codice.
\[\frac{linee\_codice\_verificate}{linee\_codice\_totali}\]

\subsubsection{Percentuale Superamento Test (PST)}
La seguente metrica indica la percentuale di test superati correttamente.
\[\frac{n\_test\_superati}{n\_totale\_test\_implementati}\]

\subsubsection{Completezza del Software(CS)}
Viene specificata la completezza del software. Questo rapporto serve per capire a che percentuale di completamento del software ci si trova.
\[C = \frac{funzionalita\_implementate }{funzionalita\_totali}\]
Se il valore calcolato è:
\begin{itemize}
    \item 1, allora sono state implementate tutte le funzionalità;
    \item 0, non sono state implementate nessuna delle funzionalità.
\end{itemize}

\subsubsection{Affidabilità del software (A)}
Viene specificata l'abilità del software di resistere a malfunzionamenti.
\[A = \frac{numero\_di\_errori}{numero\_di\_test\_eseguiti}\]

\subsubsection{Numero di tocchi/Click necessari (C)}
Viene specificata la facilità con la quale l'utente riesce a raggiungere ciò che vuole attraverso il conteggio del numero di tocchi o click necessari al suo raggiungimento. Più il valore è basso, più è facile per l'utente interagire con il sistema.\\
Si considera per esempio la capacità dell'operatore di visualizzare la propria lista delle task\textsubscript{G}.

\subsubsection{Numero di Secondi necessari (S)}
Viene specificata la rapidità con la quale l'utente riesce a raggiungere ciò che vuole attraverso il conteggio dei secondi necessari al suo raggiungimento.\\
Si considera la capacità dell'operatore di visualizzare la propria lista delle task\textsubscript{G}.


\subsection{Comunicazione e risoluzione delle anomalie}

Questa attività\textsubscript{G} è finalizzata alla tempestiva individuazione e risoluzione delle anomalie, ovverosia le deviazioni del piano prefissato.\\ Rappresentano un'anomalia:
\begin{itemize}
    \item violazioni delle norme tipografiche prefissate;
    \item presenza di contenuti non inerenti con l'argomento trattato;
    \item mancato rispetto dei valori contenuti in questo documento;
    \item incongruenze tra il prodotto e le funzionalità descritte nell'\textsc{Analisi dei Requisiti}.
\end{itemize}
Nel caso venga individuata una nuova anomalia, deve essere segnalata tempestivamente, nella modalità descritta nelle \textsc{Norme di Progetto}. In questo modo il Responsabile sarà informato dell'anomalia e sarà possibile gestirla in maniera corretta.
