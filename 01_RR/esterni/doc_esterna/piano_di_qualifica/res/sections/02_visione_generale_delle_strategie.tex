\section{Visione generale delle strategie di gestione della qualità}
In questa sezione vengono illustrati gli obiettivi fissati dal gruppo per garantire la qualità di processo e di prodotto nella realizzazione del progetto.
Al fine di monitorare costantemente lo stato e il raggiungimento degli obiettivi, sono stati adottati standard e metriche adeguate, le quali verranno illustrate in dettaglio nelle sezioni successive.
Sia gli obiettivi che le metriche sono identificati univocamente da un codice alfanumerico in modo da renderli facilmente tracciabili e quindi controllabili costantemente.

\subsection{Qualità di processo}
Vista l’importanza della qualità di processo per ottenere un prodotto valido nei tempi prestabili si è deciso di usare gli standard ISO/IEC 12207 e ISO/IEC 25010:2011, semplificandoli e riadattandoli in base alle esigenze. 

\subsection{Qualità del prodotto}
Per valutare la qualità del prodotto, il gruppo Three Way Milkshake ha deciso di avvalersi dello standard ISO/IEC 9126 descritto nell'appendice \S\ D. Tale standard definisce i criteri di applicazione delle metriche descritte nella sezione \S\ 2.4, utilizzate per valutare il livello del raggiungimento degli obiettivi descritti.
I prodotti realizzati sono:
\begin{itemize}
    \item \textbf{documentazione}: che deve essere leggibile e priva di errori ortografici, sintattici, logici e semantici;
    \item \textbf{software}: 
    \begin{itemize}
        \item deve possedere tutti i requisiti obbligatori descritti nell'\textsc{Analisi dei Requisiti};
        \item deve essere leggibile, comprensibile e mantenibile;
        \item deve essere ampiamente testato e robusto.
    \end{itemize}
\end{itemize}

\subsection{Tabella Obiettivi}
Viene presentata in seguito la tabella degli obiettivi di qualità prefissati e le relative metriche di misura.



\renewcommand{\arraystretch}{1.5}
\rowcolors{2}{pari}{dispari}
\begin{longtable}{ 
		>{}p{0.135\textwidth} 
		>{}p{0.18\textwidth}
        >{}p{0.27\textwidth}
        >{\centering}p{0.30\textwidth} }
        
	\rowcolorhead
	\centering \headertitle{Codice} &
	\centering \headertitle{Nome} &	
    \centering \headertitle{Descrizione} &
    \centering \headertitle{Metriche}	
	\endfirsthead	
    \endhead
    
        01 & Miglioramento continuo & Capacità del processo di misurare e migliorare le proprie capacità & \textbf{SR}: Scarto Riunioni \newline
                         \textbf{TR}: Totale Riunioni \newline
                        \textbf{TRI}: Totale Riunioni Interne \newline
                        \textbf{TRE}: Totali Riunioni Esterne \newline
                        \textbf{ST}: Scarto Ticket \newline
                        \textbf{DV}: Differenza To Verify \newline
                         \textbf{UL}: Uniformità del lavoro nel tempo \tabularnewline
        02 & Corretta pianificazione & La pianificazione del progetto non deve discostarsi dall'effettivo andamento &\textbf{ BAC}: Budget at Completion \newline 
                                            \textbf{EV}: Earned Value \newline
                                            \textbf{PV}: Planned Value \newline
                                            \textbf{SV}: Schedule Variance \newline
                                            \textbf{AC}: Actual Cost \tabularnewline

        03 & Leggibilità della documentazione & I documenti devono essere leggibili e comprensibili da persone con licenza di scuola media/superiore & \textbf{IG}: Indice di Gulpease \tabularnewline

        04 & Implementazione Requisiti Obbligatori & Devono venire implementati tutti i requisiti obbligatori descritti dall'\textsc{Analisi dei Requisiti} & \textbf{PROS}: Percentuale Requisiti Obbligatori Soddisfatti \tabularnewline

        05 & Manutenzione e comprensione del codice & Il codice deve essere quanto più comprensibile e mantenibile & \textbf{CBO}: Coupling Between Objects \newline \textbf{DEP}: Depth of hierarchies \newline \textbf{LEV}: Level of nesting \newline \textbf{PAR}: Parametri per metodo \newline \textbf{RCC}: Rapporto Codice Commenti \newline \textbf{L}: Leggibilità del software \tabularnewline

        06 & Copertura del codice & Il codice dovrà essere testato in ogni sua parte per garantire le funzionalità previste dai requisiti & \textbf{CC}: Code Coverage \tabularnewline

        07 & Superamento test & La Percentuale di superamento dei test dovrà essere $\geq 80$ del totale & \textbf{PST}: Percentuale superamento test \tabularnewline

        08 & Conformità & Il prodotto dovrà essere conforme ai requisiti, implementando le funzionalità richieste & \textbf{CS}: Completezza del Software \tabularnewline

        09 & Robustezza & Il prodotto dovrà far fronte a situazioni anomale gestendole senza arrestare la sua esecuzione & \textbf{A}: Affidabilità del software \tabularnewline

        10 & Usabilità & Il prodotto dovrà essere il più semplice possibile da utilizzare & \textbf{C}: Numero di tocchi/click necessari \newline \textbf{S}: Numero di secondi necessari\tabularnewline
        \caption{Tabella Obiettivi}
    \end{longtable}.



\subsection{Metriche}
Al fine del raggiungimento degli obiettivi di qualità è necessario che il processo di verifica produca dei risultati quantificabili, così da poterli confrontare con gli obiettivi fissati a priori. Per questo vengono prefissate delle metriche e i valori di sufficienza minimi necessari, che indicano se i livelli qualitativi di processo e di prodotto sono in linea con gli obiettivi prefissati. La seguente tabella riporta le metriche utilizzate, le rispettive soglie di valori preferibili e accettabili e i relativi obiettivi, così da monitorare e controllare gli obiettivi raggiunti e monitorarne i progressi.

%sistemare
\renewcommand{\arraystretch}{1.5}
\rowcolors{2}{pari}{dispari}
\begin{longtable}{ 
		>{\centering}p{0.1\textwidth} 
		>{}p{0.18\textwidth}
        >{\centering}p{0.20\textwidth}
        >{\centering}p{0.20\textwidth}
        >{}p{0.12\textwidth} }
        
	\rowcolorhead
	\centering \headertitle{Codice} &
	\centering \headertitle{Nome} &	
    \centering \headertitle{Valori Preferibili} &
    \centering \headertitle{Valori Accettabili}	&
    \centering \headertitle{Obiettivi}	
	\endfirsthead	
    \endhead

            SR & Scarto Riunioni & 2 & 3 & 01 \\
            
            TR & Totale Riunioni & 2 & 3 & 01 \tabularnewline
            
            TRI & Totale Riunioni Interne & 2 & 3 & 01 \\
            
            TRE & Totale Riunioni Esterne & 2 & 3 & 01 \\

            ST & Scarto Ticket & 2 & 3 & 01 \\

            DV & Differenza To Verify & 2 & 3 & 01 \\

            DD & Differenza Done Verify & 2 & 3 & 01 \\

            UL & Uniformità del lavoro nel tempo &2 & 2 & 01 \\

            BAC & Budget at Completion & preventivo & $preventivo-5\%\leq BAC \leq preventivo+5\%$ & 02 \\

            EV & Earned Value & $EV \geq 0$ & $EV \geq 0$ & 02 \\

            PV & Planned Value & $PV \geq 0$ & $PV \geq 0$ & 02 \\

            SV & Schedule Variance & $SV \geq 0$ & $SV = 0$ & 02 \\

            AC & Actual Cost & $0 \leq AC \leq PV$ & $0 \leq AC \leq budget$ & 02 \\

            IG & Indice di Gulpease & $70 \leq IG \leq 100$ & $50 \leq IG \leq 100$ & 03 \\

            PROS & Percentuale Requisiti Obbligatori Soddisfatti & 100\% & 100\% & 04 \\

            CBO & Coupling Between Objects & $0\leq CBO \leq 1$ & $0\leq CBO \leq 6$ & 05 \\

            DEP & Depth og hierarchies & $DEP \leq 2$ & $DEP \leq 3$ & 05 \\

            LEV & Level of nesting & $1\leq LEV \leq 3$ & $1\leq CBO \leq 6$ & 05 \\

            PAR & Parametri per metodo & $PAR \leq 4$ & $PAR \leq 6$ & 05 \\

            RCC & Rapporto Codice Commenti & $RCC \geq 0.4$ & $RCC \geq 0.2$  & 05 \\

            CC & Code Coverage & 100\% & 75\% & 06 \\

            ST & Percentuale superamento test & 80\% & 75\% & 07 \\

            CS & Completezza del Software & $CS=1$ & $CS=1$ & 08 \\

            A & Affidabilità del Software & $A=0$ & $A < 0.15$ & 09 \\

           C & Numero di tocchi/click necessari & $C<4$ & $C<6$& 10 \\

           S & Numero di secondi necessari & $S<15$ & $S<40$ & 10 \\

           L & Leggibilità del Software & $L<0.15$ & $L>0.10$ & 05 \\
           \caption{Tabella delle Metriche}
        \end{longtable}.

\subsection{Scadenze temporali}

\subsection{Risorse}
