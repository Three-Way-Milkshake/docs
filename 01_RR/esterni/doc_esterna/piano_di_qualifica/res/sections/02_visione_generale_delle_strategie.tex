\section{Visione generale delle strategie di gestione della qualità}
In questa sezione vengono illustrati gli obiettivi fissati dal gruppo per garantire la qualità di processo e di prodotto nella realizzazione del progetto\textsubscript{G}.
Al fine di monitorare costantemente lo stato e il raggiungimento degli obiettivi, sono stati adottati standard e metriche adeguate, le quali verranno illustrate in dettaglio nelle sezioni successive.
Sia gli obiettivi che le metriche sono identificati univocamente da un codice alfanumerico in modo da renderli facilmente tracciabili e quindi controllabili costantemente.

\subsection{Qualità di processo}
Vista l'importanza della qualità di processo per ottenere un prodotto valido nei tempi prestabiliti, si è deciso di adottare gli standard ISO/IEC 12207 e ISO/IEC 25010:2011, semplificandoli e riadattandoli in base alle esigenze. Viene riportata una descrizione di tali standard nell'appendice \S\ D.

\subsection{Qualità del prodotto}
Per valutare la qualità del prodotto, il gruppo Three Way Milkshake ha deciso di avvalersi dello standard ISO/IEC 9126 descritto nell'appendice \S\ D. Questo definisce i criteri di applicazione delle metriche descritte nella sezione \S\ 2.4, utilizzate per valutare il livello del raggiungimento degli obiettivi descritti nella tabella 2.3.1.
I prodotti realizzati sono:
\begin{itemize}
    \item \textbf{documentazione}: deve essere leggibile e priva di errori ortografici, sintattici, logici e semantici;
    \item \textbf{software}:
    \begin{itemize}
        \item deve possedere tutti i requisiti obbligatori descritti nell'\textsc{Analisi dei Requisiti};
        \item deve essere leggibile, comprensibile e mantenibile;
        \item deve essere ampiamente testato e robusto.
    \end{itemize}
\end{itemize}
\pagebreak
\subsection{Tabella Obiettivi}
Viene presentata in seguito la tabella degli obiettivi di qualità prefissati e le relative metriche di misura.



\renewcommand{\arraystretch}{1.5}
\rowcolors{2}{pari}{dispari}
\begin{longtable}{
		>{}p{0.1\textwidth}
		>{}p{0.2\textwidth}
        >{}p{0.5\textwidth}
        >{\centering}p{0.30\textwidth} }

	\rowcolorhead
	\centering \headertitle{Codice} &
	\centering \headertitle{Nome} &
    \centering \headertitle{Descrizione} 
	\endfirsthead
    \endhead

        01 & Miglioramento continuo & Capacità del processo di misurare e migliorare le proprie capacità 
                         \tabularnewline

        02 & Leggibilità della documentazione & I documenti devono essere leggibili e comprensibili da persone con licenza di scuola media/superiore  \tabularnewline

        03 & Implementazione Requisiti Obbligatori & Devono venire implementati tutti i requisiti obbligatori descritti dall'\textsc{Analisi dei Requisiti} \tabularnewline

        04 & Manutenzione e comprensione del codice & Il codice deve essere quanto più comprensibile e mantenibile  \tabularnewline

        05 & Copertura del codice & Il codice dovrà essere testato per buona parte per garantire le funzionalità previste dai requisiti \tabularnewline

        06 & Conformità & Il prodotto dovrà essere conforme ai requisiti, implementando le funzionalità richieste \tabularnewline

        07 & Robustezza & Il prodotto dovrà far fronte a situazioni anomale gestendole senza arrestare la sua esecuzione  \tabularnewline

        08 & Usabilità & Il prodotto dovrà essere il più semplice possibile da utilizzare \tabularnewline
        \caption{Tabella Obiettivi}
    \end{longtable}


\pagebreak
