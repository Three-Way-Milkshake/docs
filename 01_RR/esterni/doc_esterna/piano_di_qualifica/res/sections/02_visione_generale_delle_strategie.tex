\section{Visione generale delle strategie di gestione della qualità}
In questa sezione vengono illustrati gli obiettivi fissati dal gruppo per garantire la qualità di processo e di prodotto nella realizzazione del progetto.
Al fine di monitorare costantemente lo stato e il raggiungimento degli obiettivi, sono state adottate standard e metriche adeguate, le quali verranno illustrate in dettagli nelle sezioni successive.
Sia gli obiettivi che le metriche sono identificati univocamente da un codice alfanumerico in modo da renderli facilmente tracciabili e quindi controllabili costantemente.

\subsection{Qualità di processo}
Vista l’importanza della qualità di processo per ottenere un prodotto valido nei tempi prestabili si è deciso di usare gli standard ISO/IEC 12207 e ISO/IEC 25010:2011, semplificandoli e riadattandoli in base alle esigenze. 

\subsection{Qualità del prodotto}
Per valutare la qualità del prodotto, il gruppo Three Way Milkshake ha deciso di avvalersi dello standard ISO/IEC 9126 descritto nell'appendice \S\ D. Tale standard definisce i criteri di applicazione delle metriche descritte nella --, utilizzate per valutare il livello del raggiungimento degli obiettivi descritti.
I prodotti realizzati sono:
\begin{itemize}
    \item documentazione: che deve essere leggibile e priva di errori ortografici, sintattici, logici e semantici;
    \item software: 
    \begin{itemize}
        \item deve possedere tutti i requisiti obbligatori descritti nell'\textsc{Analisi dei Requisiti};
        \item deve essere leggibile, comprensibile e mantenibile;
        \item deve essere ampiamente testato e robusto.
    \end{itemize}
\end{itemize}

\subsection{Tabella Obiettivi}
Viene presentata in seguito la tabella degli obiettivi di qualità prefissati e le relative metriche di misura.

\subsection{Metriche}
Al fine del raggiungimento degli obiettivi di qualità è necessario che il processo di verifica produca dei risultati quantificabili, così da poterli confrontare con gli obiettivi fissati a priori. Per questo vengono prefissate delle metriche e i valori di sufficienza minimi necessari, che indicano se i livelli qualitativi di processo e di prodotto sono in linea con gli obiettivi prefissati. La seguente tabella riporta le metriche utilizzate, le rispettive soglie di valori preferibili e accetabili e i relativi obiettivi, così da monitorare e controllare gli obiettivi raggiunti e monitorarne i progressi.

%sistemare
\begin{table}[H]
	\begin{center}
		\caption{Tabella delle metriche}
		\begin{tabular}{p{0.25\linewidth} p{0.25\linewidth}c!{\color[HTML]{9b240a}\vrule width 0.05cm}cc}
			\rowcolorhead
			\headertitle{Codice} & \headertitle{Nome}         & \headertitle{Valori Preferibili} & \headertitle{Valori Accettabili} & \headertitle{Obiettivi} \\

            SR & Scarto Riunioni & 2 & 3 & Ob1 \\
            
            TR & Totale Riunioni & 2 & 3 & Ob1 \\
            
            TRI & Totale Riunioni Interne & 2 & 3 & Ob1 \\
            
            TRE & Totale Riunioni Esterne & 2 & 3 & Ob1 \\

            ST & Scarto Ticket & 2 & 3 & Ob1 \\

            DV & Differenza To Verify & 2 & 3 & Ob1 \\

            DD & Differenza Done Verify & 2 & 3 & Ob1 \\

            UL & Uniformità del lavoro nel tempo &2 & 2 & ob2 \\

            BAC & Budget at Completion & preventivo & $preventivo-5\%\leq BAC \leq preventivo+5\%$ & ob \\

            EV & Earned Value & $EV \geq 0$ & $EV \geq 0$ & ob \\

            PV & Planned Value & $PV \geq 0$ & $PV \geq 0$ & ob \\

            SV & Schedule Variance & $SV \geq 0$ & $SV = 0$ & ob \\

            AC & Actual Cost & $0 \leq AC \leq PV$ & $0 \leq AC \leq budget$ & ob \\

            IG & Indice di Gulpease & $70 \leq IG \leq 100$ & $50 \leq IG \leq 100$ & ob \\

            PROS & Requisiti Obbligatori Soddisfatti & 100\% & 100\% & ob \\

            CBO & Coupling Between Objects & $0\leq CBO \leq 1$ & $0\leq CBO \leq 6$ & ob \\

            DEP & Depth og hierarchies & $DEP \leq 2$ & $DEP \leq 3$ & ob \\

            LEV & Level of nesting & $1\leq LEV \leq 3$ & $1\leq CBO \leq 6$ & ob \\

            PAR & Parametri per metodo & $PAR \leq 4$ & $PAR \leq 6$ & ob \\

            RCC & Rapporto Codice Commenti & $RCC \geq 0.4$ & $RCC \geq 0.2$  & ob \\

            CC & Code Coverage & 100\% & 75\% & ob \\

            CS & Completezza del Software & $CS=1$ & $CS=1$ & ob \\

            A & Affidabilità del Software & $A=0$ & $A < 0.15$ & ob \\

           C & Numero di tocchi/click necessari & $C<4$ & $C<6$& ob \\

           S & Numero di secondi necessari & $S<15$ & $S<40$ & ob \\

           L & Leggibilità del Software & $L<0.15$ & $L>0.10$ & ob \\
		\end{tabular}

	\end{center}
\end{table}

\subsection{Scadenze temporali}

\subsection{Risorse}
