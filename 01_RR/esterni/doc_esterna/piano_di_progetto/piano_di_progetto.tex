\documentclass[a4paper]{article}

%Tutti gli usepackage vanno qui

\usepackage{geometry}
\usepackage[italian]{babel}
\usepackage[utf8]{inputenc}
\usepackage[T1]{fontenc}
\usepackage[normalem]{ulem}
\usepackage{tgschola}
%\usepackage{tgbonum}
\usepackage{tabularx}
\usepackage{longtable}
\usepackage{hyperref}
\usepackage{enumitem}
\usepackage[toc]{appendix}
\hypersetup{
	colorlinks=true,
	linkcolor=blue,
	filecolor=magenta,
	urlcolor=blue,
}
% Numerazione figure
\let\counterwithout\relax
\let\counterwithin\relax
\usepackage{chngcntr}

\counterwithin{table}{subsection}
\counterwithin{figure}{subsection}

\usepackage[bottom]{footmisc}
\usepackage{fancyhdr}
\setcounter{secnumdepth}{4}
\usepackage{amsmath, amssymb}
\usepackage{array}
\usepackage{graphicx}

\usepackage{ifthen}

%\usepackage{float}
\usepackage{layouts}
\usepackage{url}
\usepackage{comment}
\usepackage{float}
\usepackage{eurosym}

\usepackage{lastpage}
\usepackage{layouts}
\usepackage{float}
\usepackage{eurosym}

%Comandi di impaginazione uguale per tutti i documenti
\pagestyle{fancy}
\lhead{\includegraphics[scale=0.04]{../../../../latex/images/logoTWM.png}}
%Titolo del documento
\rhead{\doctitle{}}
%\rfoot{\thepage}
\cfoot{Pagina \thepage\ di \pageref{LastPage}}
\setlength{\headheight}{35pt}
\setcounter{tocdepth}{5}
\setcounter{secnumdepth}{5}
\renewcommand{\footrulewidth}{0.4pt}

% multirow per tabelle
\usepackage{multirow}

% Permette tabelle su più pagine
%\usepackage{longtable}


% colore di sfondo per le celle
\usepackage[table]{xcolor}

%COMANDI TABELLE
\newcommand{\rowcolorhead}{\rowcolor[HTML]{9b240a}} %intestazione
% check for missing commands
\newcommand{\headertitle}[1]{\textbf{\color{white}#1}} %titolo colonna
\definecolor{pari}{HTML}{FFDBCB}
\definecolor{dispari}{HTML}{F1F7FD}

% comandi glossario
\newcommand{\glo}{$_{G}$}
\newcommand{\glosp}{$_{G}$ }


%label custom
\makeatletter
\newcommand{\uclabel}[2]{%
	\protected@write \@auxout {}{\string \newlabel {#1}{{#2}{\thepage}{#2}{#1}{}} }%
	\hypertarget{#1}{#2}
}
\makeatother

%riportare pezzi di codice
\definecolor{codegray}{gray}{0.9}
\newcommand{\code}[1]{\colorbox{codegray}{\texttt{#1}}}



% Configurazione della pagina iniziale
\newcommand{\doctitle}{Verbale interno 15}
\newcommand{\docdate}{26 Febbraio 2021}
\newcommand{\rev}{1.0.0}
\newcommand{\stato}{Approvato}
\newcommand{\uso}{Interno}
\newcommand{\approv}{Tessari Andrea}
\newcommand{\red}{Crivellari Alberto}
\newcommand{\ver}{De Renzis Simone}
\newcommand{\dest}{Three Way Milkshake\\ Prof. Vardanega Tullio\\ Prof. Cardin Riccardo}
\newcommand{\describedoc}{Verbale del meeting del 2021-02-26 del gruppo Three Way Milkshake}
 % modifica questo file
\makeindex

\usepackage{hyperref}
\hypersetup{
    colorlinks=true,
    linkcolor=blue,
    urlcolor=blue,
    hyperfootnotes=false
}
\usepackage{multicol}
\usepackage{pgfplots}
\usepackage{verbatim}
\usepackage{pgf-pie}
\usepackage{ragged2e}
\setlength{\columnseprule}{1pt}

\usepackage[toc,acronym]{glossaries}
\makeglossaries

\newglossaryentry{wiki}{name={wiki}, plural={wikis},%
	description={Termine di origine hawaiana che significa veloce, con cui si identifica un tipo di sito internet che permette la creazione e la modifica di pagine multimediali attraverso un'interfaccia semplice}}

\newglossaryentry{usecase}{name={caso d'uso}, plural={casi d'uso},%
	description={Un caso d'uso è un insieme di scenari (sequenze di azioni) che hanno in comune uno scopo finale (obiettivo) per un utente (attore)}}

\newglossaryentry{techbase}{name={technology baseline},%
	description={Motiva le tecnologie, i framework, e le librerie selezionate per la realizzazione del prodotto. Ne dimostra adeguatezza e fattibilità, tramite un proof of concept coerente con gli obiettivi}}

\newglossaryentry{stakeholder}{name={stakeholder}, plural={stakeholders},%
	description={Tutti coloro che a vario titolo hanno influenza sul prodotto e sul progetto}}

\newglossaryentry{sistematico}{name={sistematico},%
	description={Modo di lavorare metodico e rigoroso, che conosce, usa ed evolve le best practice di dominio}}

\newglossaryentry{shell}{name={shell},%
	description={Interprete dei comandi}}

\newglossaryentry{security}{name={security},%
	description={Non vulnerabilità rispetto a intrusioni}}

\newglossaryentry{safety}{name={safety},%
	description={Sicurezza rispetto a malfunzionamenti}}

\newglossaryentry{repository}{name={repository},%
	description={Posizione di archiviazione per pacchetti software (esempio}}

\newglossaryentry{quantificabile}{name={quantificabile},%
	description={Che permette di misurare l’efficienza e l’efficacia del suo agire}}

\newglossaryentry{proofconcept}{name={proof of concept},%
	description={Dimostratore eseguibile. Il suo codice può (ma non deve) essere usa-e-getta}}

\newglossaryentry{prodbase}{name={product baseline},%
	description={G}}

\newglossaryentry{precondizione}{name={precondizione}, plural={precondizioni},%
	description={Condizioni che devono essere soddisfatte perché si verifichino determinati eventi successivi}}

\newglossaryentry{postcondizione}{name={postcondizione}, plural={postcondizioni},%
	description={Condizioni che devono verificarsi dopo determinati eventi}}

\newglossaryentry{gitpush}{name={push},%
	description={Comando di \gls{git} che permette di caricare le proprie modifiche locali sul server remoto condiviso del repository}}

\newglossaryentry{gitpull}{name={pull},%
	description={Comando di \gls{git} che permette di aggiornare il proprio \acrshort{repo} locale con i cambiamenti remoti}}

\newglossaryentry{gitcommit}{name={commit}, plural={commits},%
	description={Comando di \gls{git} che permette di salvare e versionare le modifiche attuate ai file in locale}}

\newglossaryentry{git}{name={git},%
	description={Sistema di controllo del versionamento distribuito}}

\newglossaryentry{efficienza}{name={efficienza},%
	description={Misura dell'abilità di raggiungere l'obiettivo impiegando le risorse minime indispensabili}}

\newglossaryentry{efficacia}{name={efficacia},%
	description={Misura della capacità di raggiungere l'obiettivo prefissato}}

\newglossaryentry{disciplinato}{name={disciplinato},%
	description={Che segue le regole che si è dato}}

\newglossaryentry{designpattern}{name={design pattern}, plural={design patterns},%
	description={Una soluzione progettuale generale ad un problema ricorrente}}

\newglossaryentry{bestpractice}{name={best practice}, plural={best practices},%
	description={Modo di fare noto, che abbia mostrato di garantire i migliori risultati in circostanze note e specifiche}}

\newglossaryentry{bash}{name={bash},%
	description={\gls{shell} Unix ed un linguaggio interpretato scritto da Brian Fox per il progetto GNU come sostituto del software gratuito per la shell Bourne}}

\newacronym{vcs}{VCS}{Version Control System}

\newacronym{uml}{UML}{Unified Modeling Language}

\newacronym{tb}{TB}{\gls{techbase}}

\newacronym{repo}{repo}{\gls{repository}}

\newacronym{ram}{RAM}{Random Access Memory}

\newacronym{poc}{PoC}{Proof of Concept}

\newacronym{pb}{PB}{\gls{prodbase}}

\newacronym{jit}{JiT}{Just in Time}

\newacronym{eg}{e.g.}{Example given, si usa per indicare che ciò che segue è un esempio}

\newacronym{cpu}{CPU}{Central Processing Unit}


\begin{document}
	\thispagestyle{empty}
\begin{titlepage}
	\begin{center}
		
		\includegraphics[scale = 0.17]{../../../../latex/images/logoTWM.png}\\[0.7cm]
		

		\noindent\rule{\textwidth}{1pt} \\[0.4cm]
		\Huge \textbf{\doctitle} \\[0.1cm]
		\ifthenelse{\equal{\docdate}{ }}{ }{ \huge \textbf{\docdate} \\[0.1cm] }
		
		\noindent\rule{\textwidth}{1pt}\\[0.7cm]
		
		\large \textbf{Three Way Milkshake - Progetto "PORTACS"} \\[0.4cm] 
                \texttt{threewaymilkshake@gmail.com} \\[0.4cm]
                
		
        
        
        \large

        \begin{tabular}{r|l}
                        \textbf{Versione} & \rev{} \\
                        \textbf{Stato} & \stato{} \\
                        \textbf{Uso} & \uso{} \\                         
                        \textbf{Approvazione} & \approv{} \\                      
                        \textbf{Redazione} & \red{} \\ 
                        \textbf{Verifica} &  \ver{} \\                         
                        \textbf{Destinatari} & \parbox[t]{5cm}{ \dest{} }
                \end{tabular} 
                \\[0.3cm]
                \large \textbf{Descrizione} \\ \describedoc{} 
               

	\end{center}
\end{titlepage}
	\pagebreak

	% Registro delle modifiche
	\section*{Registro delle modifiche}

\newcommand{\changelogTable}[1]{
	
	
	\renewcommand{\arraystretch}{1.5}
	\rowcolors{2}{pari}{dispari}
	\begin{longtable}{ 
			>{\centering}p{0.07\textwidth} 
			>{}p{0.21\textwidth}
			>{\centering}p{0.17\textwidth}
			>{\centering}p{0.13\textwidth} 
			>{\centering}p{0.17\textwidth} 
			>{\centering}p{0.13\textwidth} }
		\rowcolorhead
		\headertitle{Vers.} &
		\centering \headertitle{Descrizione} &	
		\headertitle{Redazione} &
		\headertitle{Data red.} & 
		\headertitle{Verifica} &
		\headertitle{Data ver.}
		\endfirsthead	
		\endhead
		
		#1
		
	\end{longtable}
	\vspace{-2em}
	
}


\newcommand{\approvingTable}[1]{ 
	
	
	\renewcommand{\arraystretch}{1.5}
	\rowcolors{2}{pari}{dispari}
	\begin{longtable}{ 
			>{\centering}p{0.07\textwidth} 
			>{\centering}p{0.415\textwidth}
			>{\centering}p{0.13\textwidth}
			>{\centering}p{0.322\textwidth}  }
		\rowcolorhead
		\headertitle{Vers.} &
		\centering \headertitle{Descrizione} &	
		\headertitle{Data appr.} &
		\headertitle{Approvazione}
		\endfirsthead	
		\endhead
		
		#1
		
	\end{longtable}
	\vspace{-2em}
	
}
	\approvingTable{
	1.0.0 & Approvazione del verbale & 2021-04-18 & Greggio Nicolò
}

\changelogTable{
	0.1.0 & Stesura e verifica del verbale & Crivellari Alberto & 2021-04-15 & De Renzis Simone & 2021-04-18
} % modifica questo file
	\end{longtable}
	\pagebreak

	% indice
	{
        \hypersetup{linkcolor=black}
        \tableofcontents
        \pagebreak

        % indice delle figure
        \listoffigures
        \pagebreak

        % indice delle tabelle
        \listoftables
        \pagebreak
    }


	% contenuto del documento, ogni sezione in un file
	\section{Introduzione}




\subsection{Scopo del documento}
Lo scopo di questo documento è presentare tutte le informazioni necessarie al mantenimento e all'estensione del software PORTACS, mostrando nel dettaglio l'architettura del sistema e l'organizzazione del codice sorgente.\\
In questo documento saranno presentate le varie tecnologie usate, sia lato front end che back end, come anche le varie librerie e framework. Verrà inoltre mostrato il sistema di versionamento utilizzato e la Continuous Integration applicata.





\subsection{Scopo del prodotto}

Il capitolato\textsubscript{G} C5 propone un progetto\textsubscript{G} in cui viene richiesto lo sviluppo di un software per il monitoraggio in tempo reale di unità che si muovono in uno spazio definito. All'interno di questo spazio, creato dall’utente per riprodurre le caratteristiche di un ambiente reale, le unità dovranno essere in grado di circolare in autonomia, o sotto il controllo dell’utente, per raggiungere dei punti di interesse posti nella mappa.  La circolazione è sottoposta a vincoli di viabilità e ad ostacoli propri della topologia dell’ambiente, deve evitare le collisioni con le altre unità e prevedere la gestione di situazioni critiche nel traffico.




\subsection{Riferimenti}



\subsubsection{Normativi}

\begin{itemize}
	\item \textsc{Norme di progetto\textsubscript{G} v3.0.0 }: per qualsiasi convenzione sulla nomenclatura degli elementi presenti all’interno del documento;
	
	\item Regolamento progetto\textsubscript{G} didattico: \\ {\url{https://www.math.unipd.it/~tullio/IS-1/2020/Dispense/P1.pdf}};
	\item Model-View Patterns: \\ {\url{https://www.math.unipd.it/~rcardin/sweb/2020/L02.pdf}};
	\item SOLID Principles: \\ {\url{https://www.math.unipd.it/~rcardin/sweb/2020/L04.pdf}};
	\item Diagrammi delle classi: \\ {\url{https://www.math.unipd.it/~rcardin/swea/2021/Diagrammi delle Classi_4x4.pdf}};
	\item Diagrammi dei package: \\ {\url{https://www.math.unipd.it/~rcardin/swea/2021/Diagrammi dei Package_4x4.pdf}};
	\item Diagrammi di sequenza: \\ {\url{https://www.math.unipd.it/~rcardin/swea/2021/Diagrammi di Sequenza_4x4.pdf}};
	\item Design Pattern Creazionali: \\ {\url{https://www.math.unipd.it/~rcardin/swea/2021/Design Pattern Creazionali_4x4.pdf}};
	\item Design Pattern Strutturali: \\ {\url{https://www.math.unipd.it/~rcardin/swea/2021/Design Pattern Strutturali_4x4.pdf}};
	\item Design Pattern Comportamentali: \\ {\url{https://www.math.unipd.it/~rcardin/swea/2021/Design Pattern Comportamentali_4x4.pdf}}.
\end{itemize}



\subsubsection{Informativi}
\begin{itemize}
	\item \textsc{\href{https://github.com/Three-Way-Milkshake/docs/wiki/Glossario}{Glossario}}: per la definizione dei termini (pedice G) e degli acronimi (pedice A) evidenziati nel documento;
	\item Capitolato d'appalto C5-PORTACS: \\
{\url{https://www.math.unipd.it/~tullio/IS-1/2020/Progetto/C5.pdf}}
	\item Software Engineering - Iam Sommerville - $10^{th}$ Edition.
\end{itemize}
	\pagebreak

	\section{Analisi dei rischi}



\subsection{Rischi tecnologici}



\subsection{Rischi organizzativi}



\subsection{Rischi interpersonali}


	\pagebreak

	\section{Pianificazione}

\subsection{Modello di sviluppo}

Il modello di sviluppo scelto è il modello \textbf{incrementale}. Esso si adatta al sistema di revisioni a cui vanno incontro gli artefatti prodotti nel corso del progetto. Il modello incrementale infatti prevede che Analisi dei requisiti e Progettazione architetturale si svolgano una volta sola: queste fasi servono studiare il problema e a strutturarne la soluzione. Se si dovesse tornare su queste fasi sarà solo per raffinarne i contenuti in base a nuove evidenze individuate nelle fasi successive.

La progettazione di dettaglio e la codifica invece si svilupperanno attraverso cicli di incremento atti a integrare il sistema di nuove funzionalità: si partirà dal soddisfacimento dei requisiti più importanti, per poi eventualmente incrementare con requisiti secondari e opzionali. 

Queste modalità permettono, gettate le basi del prodotto, di accrescerne le funzionalità producendo valore fin da subito, in modo da avere riscontro quasi immediato sull'operato e poterne indirizzare gli sviluppi successivi in base ai feedback ricevuti (anche dal proponente) e alle risorse disponibili.

Il team adotterà anche alcune tecniche tipiche dello sviluppo Agile: viene fatto uso di una Kanban board, strumento che permette di pianificare in dettaglio e visualizzare gli obiettivi a cui ciascun membro del team si dedica. Questa tecnica riflette la modalità con cui il team si organizza nel contesto di un incremento: un meeting a cadenza settimanale permette di pianificare l'avanzamento e stabilire le future assegnazioni, in modo da affrontare eventuali ritardi o difficoltà prima che possano causare problemi allo sviluppo complessivo.



\subsection{Scadenze}

Il gruppo stabilisce di affrontare le revisioni di avanzamento nelle seguenti date:
\begin{itemize}
	\item \textbf{Revisione dei Requisiti}: 18 gennaio 2021
	\item \textbf{Revisione di Progettazione}: 8 marzo 2021 
	\item \textbf{Revisione di Qualifica}: 9 aprile 2021
	\item \textbf{Revisione di Accettazione}: 10 maggio 2021	
\end{itemize}





\subsection{Fasi}

A fronte del modello di sviluppo scelto e delle scadenze fissate, lo sviluppo procederà attraverso le seguenti fasi:
\begin{itemize}
	\item \textbf{Avvio}
	\item \textbf{Analisi dei requisiti}
	\item \textbf{Progettazione architetturale}
	\item \textbf{Progettazione di dettaglio e codifica}
	\item \textbf{Validazione e collaudo}
\end{itemize}

Di seguito vengono descritte le attività da realizzarsi nel contesto di ogni fase.

\subsubsection{Avvio}

\textit{Dal 12/11/2020 al 13/12/2020}
\\\\
Questa fase inizia in corrispondenza del primo seminario tecnologico tenuto da una delle aziende proponenti e termina con la scelta del capitolato a cui il gruppo intende avanzare la propria offerta nella relativa gara d'appalto.
Durante questo periodo, vengono svolte le seguenti attività:
\begin{itemize}
	\item \textbf{Visione dei seminari}: i seminari tecnologici costituiscono un fattore importante nel contesto della scelta del capitolato, fanno luce sui requisiti e sulla fattibilità dei progetti;
	\item \textbf{Norme di progetto}: inizia in questa fase la definizione delle norme che il gruppo intende adottare. Si studiano e testano gli strumenti che permetteranno l'organizzazione interna, il tracciamento, la stesura dei documenti e il loro versionamento, la gestione dei meeting e la loro calendarizzazione;
	\item \textbf{Studio di fattibilità}: l'analisi del materiale di ogni capitolato permette ai gruppi di farsi una prima idea sulle preferenze nella scelta del capitolato. Con il termine della visione dei seminari, si effettua uno studio approfondito di ogni progetto e si redige il documento studio di fattibilità, nel quale si esprime la propria preferenza definitiva;
	\item \textbf{Piano di progetto}: inizia la definizione delle fasi in cui si articola il progetto e la redazione del documento piano di progetto;
\end{itemize}  





\subsubsection{Analisi dei requisiti}



\subsubsection{Progettazione architetturale}



\subsubsection{Progettazione di dettaglio e codifica}



\subsubsection{Validazione e collaudo}




	\pagebreak

	\section{Preventivo}
Questa sezione fornisce una stima dei costi che il gruppo dovrà sostenere nelle varie fasi che interessano lo svolgimento del progetto. Verranno utilizzate le seguenti abbreviazioni per descrivere l'utilizzo delle risorse da parte del team:
\begin{itemize}
	\item \textbf{R}: Responsabile
	\item \textbf{V}: Verificatore
	\item \textbf{An}: Analista
	\item \textbf{Am}: Amministratore
	\item \textbf{Pr}: Programmatore
	\item \textbf{Pt}: Progettista
\end{itemize}


\subsection{Avvio}

\subsubsection{Prospetto orario}
Di seguito viene illustrato l'utilizzo della risorsa tempo (espresso in ore) dei vari componenti del gruppo nella fase di Avvio:

\begin{table}[H]
\begin{center}
\begin{tabular}{c
	!{\color[HTML]{9b240a}\vrule width 1pt}
	cccccc
	!{\color[HTML]{9b240a}\vrule width 1pt}	
	c}
\rowcolorhead
\headertitle{Nome} & \headertitle{R} & \headertitle{V} & \headertitle{An} & \headertitle{Am} & \headertitle{Pr} & \headertitle{Pt} & \headertitle{Tot} \\

Chiarello Sofia & 2 & 2 & 5 & 0 & 0 & 0 & 9\\
Crivellari Alberto & 2 & 2 & 5 & 0 & 0 & 0 & 9\\
De Renzis Simone & 4 & 1 & 2 & 3 & 0 & 0 & 10\\
Greggio Nicolò & 2 & 1 & 2 & 5 & 0 & 0 & 10\\
Tessari Andrea & 2 & 2 & 5 & 0 & 0 & 0 & 9\\
Zuccolo Giada & 2 & 2 & 5 & 0 & 0 & 0 & 9\\
\end{tabular}
\caption[Occupazione oraria Avvio]{Per ogni componente, i ruoli ricoperti e la relativa occupazione oraria nella fase di Avvio}
\end{center}
\end{table}


\pgfplotsset{width=10cm,compat=1.17}
\begin{figure}[H]
	\centering
	\begin{tikzpicture}
		\begin{axis}[
			enlarge y limits=0.30,
			enlarge x limits=0.10,
			ybar stacked,
			width=10cm,
			bar width=0.7cm,
			%every node near coord/.style={text width=3 cm},
			%nodes near coords,
			%every node near coord/.append style={color=black},
			%enlargelimits=0.25,
			legend style={font=\footnotesize,at={(1.2,1)},
				cells={anchor=west},
				anchor=north,legend columns=1},
			%ylabel={\#participants},
			symbolic x coords={Chiarello Sofia, Crivellari Alberto, De Renzis Simone, Greggio Nicolò, Tessari Andrea, Zuccolo Giada},
			xtick=data,
			x tick label style={font=\footnotesize,text width=1.7cm,align=center},
			]
			\addplot+[ybar,fill=red, draw=black] plot coordinates 
			%Responsabile
				{(Chiarello Sofia,2) 
				(Crivellari Alberto,2) 
				(De Renzis Simone,4) 
				(Greggio Nicolò,2) 
				(Tessari Andrea,2) 
				(Zuccolo Giada,2) };
			\addplot+[ybar,fill=pink, draw=black] plot coordinates 
			%Verificatore
				{(Chiarello Sofia,2) 
				(Crivellari Alberto,2) 
				(De Renzis Simone,1) 
				(Greggio Nicolò,1) 
				(Tessari Andrea,2) 
				(Zuccolo Giada,2) };
			\addplot+[ybar,fill=green, draw=black] plot coordinates 
			%Analista
				{(Chiarello Sofia,5) 
				(Crivellari Alberto,5) 
				(De Renzis Simone,2) 
				(Greggio Nicolò,2) 
				(Tessari Andrea,5) 
				(Zuccolo Giada,5) };
			\addplot+[ybar,fill=yellow, draw=black] plot coordinates
			%Amministratore
				{(Chiarello Sofia,0)
				(Crivellari Alberto,0) 
				(De Renzis Simone,3) 
				(Greggio Nicolò,5)
				(Tessari Andrea,0)
				(Zuccolo Giada,0) };
			\addplot+[ybar,fill=cyan, draw=black] plot coordinates 
			%Programmatore
				{(Chiarello Sofia,0) 
				(Crivellari Alberto,0) 
				(De Renzis Simone,0) 
				(Greggio Nicolò,0) 
				(Tessari Andrea,0) 
				(Zuccolo Giada,0) };
			\addplot+[ybar,fill=orange, draw=black] plot coordinates
			%Progettista
				{(Chiarello Sofia,0) 
				(Crivellari Alberto,0) 
				(De Renzis Simone,0)
				(Greggio Nicolò,0) 
				(Tessari Andrea,0) 
				(Zuccolo Giada,0) };
			\legend{Responsabile \\ Verificatore \\ Analista \\ Amministratore \\ Programmatore \\ Progettista \\}
		\end{axis}
	\end{tikzpicture}
	\caption[Istogramma distribuzione oraria Avvio]{Istogramma che visualizza la ripartizione delle ore nella fase di Avvio} 
\end{figure}






\subsubsection{Prospetto economico}
Il costo derivante dalle ore impiegate dai componenti nella fase di Avvio è descritto di seguito, calcolandone il totale.

\begin{table}[H]
{\setlength{\parindent}{0cm}
\begin{minipage}{.43\textwidth}
	\begin{tabular}{ccc}
	\rowcolorhead
	\headertitle{Ruolo} & \headertitle{Ore} & \headertitle{Costo(€)}\\
	Responsabile & 14 & 420\\
	Verificatore & 10 & 150\\
	Analista & 24 & 600\\
	Amministratore & 8 & 160\\
	Programmatore & 0 & 0\\
	Progettista & 0 & 0\\
	\hline
	\textbf{Totale} & \textbf{56} & \textbf{1330}\\
	\end{tabular}
\end{minipage}% This must go next to `\end{minipage}`
\begin{minipage}{.57\textwidth}
  \begin{tikzpicture}
\pie [rotate = 180,
    sum = auto,
    before number=\pgfsetfillopacity{0.0},
    %text = legend, 
    radius = 2.7,
    color = {red, pink, green, yellow}]
    {    
    150/Verificatore,    
    160/Amministratore,
    420/Responsabile,
    600/Analista
    }
\end{tikzpicture} 
\end{minipage} }
\caption[Prospetto economico della fase di Avvio]{Per ogni ruolo, il complessivo delle ore impiegate dai membri e il relativo ammontare in denaro. Il diagramma a torta visualizza la composizione dei costi nella fase di Avvio}
\end{table}



\subsection{Analisi dei requisiti}

\subsubsection{Prospetto orario}
Di seguito viene illustrato l'utilizzo della risorsa tempo (espresso in ore) dei vari componenti del gruppo nella fase di Analisi dei Requisiti:

\begin{table}[H]
	\begin{center}
		\begin{tabular}{c
				!{\color[HTML]{9b240a}\vrule width 1pt}
				cccccc
				!{\color[HTML]{9b240a}\vrule width 1pt}	
				c}
			\rowcolorhead
			\headertitle{Nome} & \headertitle{R} & \headertitle{V} & \headertitle{An} & \headertitle{Am} & \headertitle{Pr} & \headertitle{Pt} & \headertitle{Tot} \\
			
			Chiarello Sofia & 0 & 6 & 18 & 1 & 0 & 0 & 25\\
			Crivellari Alberto & 0 & 18 & 4 & 2 & 0 & 0 & 24\\
			De Renzis Simone & 12 & 5 & 5 & 3 & 0 & 0 & 25\\
			Greggio Nicolò & 5 & 5 & 5 & 12 & 0 & 0 & 27\\
			Tessari Andrea & 8 & 5 & 4 & 8 & 0 & 0 & 25\\
			Zuccolo Giada & 0 & 6 & 18 & 1 & 0 & 0 & 25\\
		\end{tabular}
		\caption[Occupazione oraria Analisi dei Requisiti]{Per ogni componente, i ruoli ricoperti e la relativa occupazione oraria nella fase di Analisi dei Requisiti}
	\end{center}
\end{table}


\pgfplotsset{width=10cm,compat=1.17}
\begin{figure}[H]
	\centering
	\begin{tikzpicture}
		\begin{axis}[
			enlarge y limits=0.05,
			enlarge x limits=0.10,
			ybar stacked,
			width=10cm,
			bar width=0.7cm,
			%every node near coord/.style={text width=3 cm},
			%nodes near coords,
			%every node near coord/.append style={color=black},
			%enlargelimits=0.25,
			legend style={font=\footnotesize,at={(1.2,1)},
				cells={anchor=west},
				anchor=north,legend columns=1},
			%ylabel={\#participants},
			symbolic x coords={Chiarello Sofia, Crivellari Alberto, De Renzis Simone, Greggio Nicolò, Tessari Andrea, Zuccolo Giada},
			xtick=data,
			x tick label style={font=\footnotesize,text width=1.7cm,align=center},
			]
			\addplot+[ybar,fill=red, draw=black] plot coordinates 
			%Responsabile
			{(Chiarello Sofia,0) 
				(Crivellari Alberto,0) 
				(De Renzis Simone,12) 
				(Greggio Nicolò,5) 
				(Tessari Andrea,8) 
				(Zuccolo Giada,0) };
			\addplot+[ybar,fill=pink, draw=black] plot coordinates 
			%Verificatore
			{(Chiarello Sofia,6) 
				(Crivellari Alberto,18) 
				(De Renzis Simone,5) 
				(Greggio Nicolò,5) 
				(Tessari Andrea,5) 
				(Zuccolo Giada,6) };
			\addplot+[ybar,fill=green, draw=black] plot coordinates 
			%Analista
			{(Chiarello Sofia,18) 
				(Crivellari Alberto,4) 
				(De Renzis Simone,5) 
				(Greggio Nicolò,5) 
				(Tessari Andrea,4) 
				(Zuccolo Giada,18) };
			\addplot+[ybar,fill=yellow, draw=black] plot coordinates
			%Amministratore
			{(Chiarello Sofia,1)
				(Crivellari Alberto,2) 
				(De Renzis Simone,3) 
				(Greggio Nicolò,12)
				(Tessari Andrea,8)
				(Zuccolo Giada,1) };
			\addplot+[ybar,fill=cyan, draw=black] plot coordinates 
			%Programmatore
			{(Chiarello Sofia,0) 
				(Crivellari Alberto,0) 
				(De Renzis Simone,0) 
				(Greggio Nicolò,0) 
				(Tessari Andrea,0) 
				(Zuccolo Giada,0) };
			\addplot+[ybar,fill=orange, draw=black] plot coordinates
			%Progettista
			{(Chiarello Sofia,0) 
				(Crivellari Alberto,0) 
				(De Renzis Simone,0)
				(Greggio Nicolò,0) 
				(Tessari Andrea,0) 
				(Zuccolo Giada,0) };
			\legend{Responsabile \\ Verificatore \\ Analista \\ Amministratore \\ Programmatore \\ Progettista \\}
		\end{axis}
	\end{tikzpicture}
	\caption[Istogramma distribuzione oraria Analisi dei Requisiti]{Istogramma che visualizza la ripartizione delle ore nella fase di Analisi dei Requisiti} 
\end{figure}


\subsubsection{Prospetto economico}
Il costo derivante dalle ore impiegate dai componenti nella fase di Analisi dei Requisiti è descritto di seguito, calcolandone il totale.

\begin{table}[H]
	{\setlength{\parindent}{0cm}
		\begin{minipage}{.43\textwidth}
			\begin{tabular}{ccc}
				\rowcolorhead
				\headertitle{Ruolo} & \headertitle{Ore} & \headertitle{Costo(€)}\\
				Responsabile & 25 & 750\\
				Verificatore & 45 & 675\\
				Analista & 54 & 1350\\
				Amministratore & 27 & 540\\
				Programmatore & 0 & 0\\
				Progettista & 0 & 0\\
				\hline
				\textbf{Totale} & \textbf{151} & \textbf{3315}\\
			\end{tabular}
		\end{minipage}% This must go next to `\end{minipage}`
		\begin{minipage}{.57\textwidth}
			\begin{tikzpicture}
				\pie [rotate = 180,
				sum = auto, 
				before number=\pgfsetfillopacity{0.0},
				%text = legend, 
				radius = 2.7,
				color = {red, pink, green, yellow}]
				{    
					540/Verificatore,    
					675/Amministratore,
					750/Responsabile,
					1350/Analista
				}
			\end{tikzpicture} 
	\end{minipage} }
	\caption[Prospetto economico della fase di Analisi dei Requisiti]{Per ogni ruolo, il complessivo delle ore impiegate dai membri e il relativo ammontare in denaro. Il diagramma a torta visualizza la composizione dei costi nella fase di Analisi dei Requisiti}
\end{table}

\subsection{Progettazione architetturale}

\subsubsection{Prospetto orario}
Di seguito viene illustrato l'utilizzo della risorsa tempo (espresso in ore) dei vari componenti del gruppo nella fase di Progettazione Architetturale:

\begin{table}[H]
	\begin{center}
		\begin{tabular}{c
				!{\color[HTML]{9b240a}\vrule width 1pt}
				cccccc
				!{\color[HTML]{9b240a}\vrule width 1pt}	
				c}
			\rowcolorhead
			\headertitle{Nome} & \headertitle{R} & \headertitle{V} & \headertitle{An} & \headertitle{Am} & \headertitle{Pr} & \headertitle{Pt} & \headertitle{Tot} \\
			
			Chiarello Sofia & 1 & 2 & 4 & 0 & 8 & 15 & 30\\
			Crivellari Alberto & 1 & 6 & 0 & 0 & 20 & 3 & 30\\
			De Renzis Simone & 4 & 2 & 0 & 2 & 7 & 14 & 29\\
			Greggio Nicolò & 1 & 2 & 0 & 5 & 7 & 13 & 28\\
			Tessari Andrea & 2 & 2 & 0 & 4 & 20 & 2 & 30\\
			Zuccolo Giada & 1 & 2 & 4 & 0 & 8 & 15 & 30\\
		\end{tabular}
		\caption[Occupazione oraria Progettazione Architetturale]{Per ogni componente, i ruoli ricoperti e la relativa occupazione oraria nella fase di Progettazione Architetturale}
	\end{center}
\end{table}


\pgfplotsset{width=10cm,compat=1.17}
\begin{figure}[H]
	\centering
	\begin{tikzpicture}
		\begin{axis}[
			enlarge y limits=0.05,
			enlarge x limits=0.10,
			ybar stacked,
			width=10cm,
			bar width=0.7cm,
			%every node near coord/.style={text width=3 cm},
			%nodes near coords,
			%every node near coord/.append style={color=black},
			%enlargelimits=0.25,
			legend style={font=\footnotesize,at={(1.2,1)},
				cells={anchor=west},
				anchor=north,legend columns=1},
			%ylabel={\#participants},
			symbolic x coords={Chiarello Sofia, Crivellari Alberto, De Renzis Simone, Greggio Nicolò, Tessari Andrea, Zuccolo Giada},
			xtick=data,
			x tick label style={font=\footnotesize,text width=1.7cm,align=center},
			]
			\addplot+[ybar,fill=red, draw=black] plot coordinates 
			%Responsabile
			{(Chiarello Sofia,1) 
				(Crivellari Alberto,1) 
				(De Renzis Simone,4) 
				(Greggio Nicolò,1) 
				(Tessari Andrea,2) 
				(Zuccolo Giada,1) };
			\addplot+[ybar,fill=pink, draw=black] plot coordinates 
			%Verificatore
			{(Chiarello Sofia,2) 
				(Crivellari Alberto,2) 
				(De Renzis Simone,2) 
				(Greggio Nicolò,2) 
				(Tessari Andrea,2) 
				(Zuccolo Giada,2) };
			\addplot+[ybar,fill=green, draw=black] plot coordinates 
			%Analista
			{(Chiarello Sofia,4) 
				(Crivellari Alberto,0) 
				(De Renzis Simone,0) 
				(Greggio Nicolò,0) 
				(Tessari Andrea,0) 
				(Zuccolo Giada,4) };
			\addplot+[ybar,fill=yellow, draw=black] plot coordinates
			%Amministratore
			{(Chiarello Sofia,0)
				(Crivellari Alberto,0) 
				(De Renzis Simone,2) 
				(Greggio Nicolò,5)
				(Tessari Andrea,4)
				(Zuccolo Giada,0) };
			\addplot+[ybar,fill=cyan, draw=black] plot coordinates 
			%Programmatore
			{(Chiarello Sofia,8) 
				(Crivellari Alberto,20) 
				(De Renzis Simone,7) 
				(Greggio Nicolò,7) 
				(Tessari Andrea,20) 
				(Zuccolo Giada,8) };
			\addplot+[ybar,fill=orange, draw=black] plot coordinates
			%Progettista
			{(Chiarello Sofia,15) 
				(Crivellari Alberto,3) 
				(De Renzis Simone,14)
				(Greggio Nicolò,13) 
				(Tessari Andrea,2) 
				(Zuccolo Giada,15) };
			\legend{Responsabile \\ Verificatore \\ Analista \\ Amministratore \\ Programmatore \\ Progettista \\}
		\end{axis}
	\end{tikzpicture}
	\caption[Istogramma distribuzione oraria Progettazione Architetturale]{Istogramma che visualizza la ripartizione delle ore nella fase di Progettazione Architetturale} 
\end{figure}


\subsubsection{Prospetto economico}
Il costo derivante dalle ore impiegate dai componenti nella fase di Progettazione Architetturale è descritto di seguito, calcolandone il totale.

\begin{table}[H]
	{\setlength{\parindent}{0cm}
		\begin{minipage}{.43\textwidth}
			\begin{tabular}{ccc}
				\rowcolorhead
				\headertitle{Ruolo} & \headertitle{Ore} & \headertitle{Costo(€)}\\
				Responsabile & 10 & 300\\
				Verificatore & 16 & 240\\
				Analista & 8 & 200\\
				Amministratore & 11 & 220\\
				Programmatore & 70 & 1050\\
				Progettista & 62 & 1364\\
				\hline
				\textbf{Totale} & \textbf{177} & \textbf{3374}\\
			\end{tabular}
		\end{minipage}% This must go next to `\end{minipage}`
		\begin{minipage}{.57\textwidth}
		\begin{tikzpicture}
			\pie [rotate = 270,
			sum = auto, 
			before number=\pgfsetfillopacity{0.0},
			%text = legend, 
			radius = 2.7,
			color = {red, pink, green, yellow, cyan, orange}]
			{
				300/Responsabile,
				240/Verificatore,
				200/Analista,
				220/Amministratore,
				1050/Programmatore,
				1364/Progettista
			}
		\end{tikzpicture} 
	\end{minipage} }
	\caption[Prospetto economico della fase di Analisi dei Requisiti]{Per ogni ruolo, il complessivo delle ore impiegate dai membri e il relativo ammontare in denaro. Il diagramma a torta visualizza la composizione dei costi nella fase di Analisi dei Requisiti}
\end{table}


\subsection{Progettazione di dettaglio e codifica dei requisiti obbligatori}



\subsubsection{Prospetto orario}
Di seguito viene illustrato l'utilizzo della risorsa tempo (espresso in ore) dei vari componenti del gruppo nella fase di Progettazione e Codifica dei Requisiti Obbligatori:

\begin{table}[H]
	\begin{center}
		\begin{tabular}{c
				!{\color[HTML]{9b240a}\vrule width 1pt}
				cccccc
				!{\color[HTML]{9b240a}\vrule width 1pt}	
				c}
			\rowcolorhead
			\headertitle{Nome} & \headertitle{R} & \headertitle{V} & \headertitle{An} & \headertitle{Am} & \headertitle{Pr} & \headertitle{Pt} & \headertitle{Tot} \\
			
			Chiarello Sofia & 0 & 15 & 2 & 0 & 22 & 6 & 45\\
			Crivellari Alberto & 0 & 20 & 0 & 1 & 5 & 20 & 46\\
			De Renzis Simone & 2 & 14 & 0 & 0 & 21 & 8 & 45\\
			Greggio Nicolò & 1 & 14 & 0 & 3 & 21 & 5 & 44\\
			Tessari Andrea & 1 & 15 & 0 & 0 & 5 & 25 & 46\\
			Zuccolo Giada & 0 & 14 & 2 & 2 & 23 & 4 & 45\\
		\end{tabular}
		\caption[Occupazione oraria Progettazione e Codifica dei Requisiti Obbligatori]{Per ogni componente, i ruoli ricoperti e la relativa occupazione oraria nella fase di Progettazione e Codifica dei Requisiti Obbligatori}
	\end{center}
\end{table}


\pgfplotsset{width=10cm,compat=1.17}
\begin{figure}[H]
	\centering
	\begin{tikzpicture}
		\begin{axis}[
			enlarge y limits=0.05,
			enlarge x limits=0.10,
			ybar stacked,
			width=10cm,
			bar width=0.7cm,
			%every node near coord/.style={text width=3 cm},
			%nodes near coords,
			%every node near coord/.append style={color=black},
			%enlargelimits=0.25,
			legend style={font=\footnotesize,at={(1.2,1)},
				cells={anchor=west},
				anchor=north,legend columns=1},
			%ylabel={\#participants},
			symbolic x coords={Chiarello Sofia, Crivellari Alberto, De Renzis Simone, Greggio Nicolò, Tessari Andrea, Zuccolo Giada},
			xtick=data,
			x tick label style={font=\footnotesize,text width=1.7cm,align=center},
			]
			\addplot+[ybar,fill=red, draw=black] plot coordinates 
			%Responsabile
			{(Chiarello Sofia,0) 
				(Crivellari Alberto,0) 
				(De Renzis Simone,2) 
				(Greggio Nicolò,1) 
				(Tessari Andrea,1) 
				(Zuccolo Giada,0) };
			\addplot+[ybar,fill=pink, draw=black] plot coordinates 
			%Verificatore
			{(Chiarello Sofia,15) 
				(Crivellari Alberto,20) 
				(De Renzis Simone,14) 
				(Greggio Nicolò,14) 
				(Tessari Andrea,15) 
				(Zuccolo Giada,14) };
			\addplot+[ybar,fill=green, draw=black] plot coordinates 
			%Analista
			{(Chiarello Sofia,2) 
				(Crivellari Alberto,0) 
				(De Renzis Simone,0) 
				(Greggio Nicolò,0) 
				(Tessari Andrea,0) 
				(Zuccolo Giada,2) };
			\addplot+[ybar,fill=yellow, draw=black] plot coordinates
			%Amministratore
			{(Chiarello Sofia,0)
				(Crivellari Alberto,1) 
				(De Renzis Simone,0) 
				(Greggio Nicolò,3)
				(Tessari Andrea,0)
				(Zuccolo Giada,2) };
			\addplot+[ybar,fill=cyan, draw=black] plot coordinates 
			%Programmatore
			{(Chiarello Sofia,22) 
				(Crivellari Alberto,5) 
				(De Renzis Simone,21) 
				(Greggio Nicolò,21) 
				(Tessari Andrea,5) 
				(Zuccolo Giada,23) };
			\addplot+[ybar,fill=orange, draw=black] plot coordinates
			%Progettista
			{(Chiarello Sofia,6) 
				(Crivellari Alberto,20) 
				(De Renzis Simone,8)
				(Greggio Nicolò,5) 
				(Tessari Andrea,25) 
				(Zuccolo Giada,4) };
			\legend{Responsabile \\ Verificatore \\ Analista \\ Amministratore \\ Programmatore \\ Progettista \\}
		\end{axis}
	\end{tikzpicture}
	\caption[Istogramma distribuzione oraria Progettazione e Codifica dei Requisiti Obbligatori]{Istogramma che visualizza la ripartizione delle ore nella fase di Progettazione e Codifica dei Requisiti Obbligatori} 
\end{figure}


\subsubsection{Prospetto economico}
Il costo derivante dalle ore impiegate dai componenti nella fase di Progettazione e Codifica dei Requisiti Obbligatori è descritto di seguito, calcolandone il totale.

\begin{table}[H]
	{\setlength{\parindent}{0cm}
		\begin{minipage}{.43\textwidth}
			\begin{tabular}{ccc}
				\rowcolorhead
				\headertitle{Ruolo} & \headertitle{Ore} & \headertitle{Costo(€)}\\
				Responsabile & 4 & 120\\
				Verificatore & 92 & 1380\\
				Analista & 4 & 100\\
				Amministratore & 6 & 120\\
				Programmatore & 97 & 1455\\
				Progettista & 68 & 1496\\
				\hline
				\textbf{Totale} & \textbf{271} & \textbf{4671}\\
			\end{tabular}
		\end{minipage}% This must go next to `\end{minipage}`
		\begin{minipage}{.57\textwidth}
			\begin{tikzpicture}
				\pie [rotate = 270,
				sum = auto, 
				before number=\pgfsetfillopacity{0.0},
				%text = legend, 
				radius = 2.7,
				color = {red, pink, green, yellow, cyan, orange}]
				{
					120/Responsabile,
					1380/Verificatore,
					100/Analista,
					120/Amministratore,
					1455/Programmatore,
					1496/Progettista
				}
			\end{tikzpicture} 
	\end{minipage} }
	\caption[Prospetto economico della fase di Progettazione e Codifica dei Requisiti Obbligatori]{Per ogni ruolo, il complessivo delle ore impiegate dai membri e il relativo ammontare in denaro. Il diagramma a torta visualizza la composizione dei costi nella fase di Progettazione e Codifica dei Requisiti Obbligatori}
\end{table}





\subsection{Progettazione di dettaglio e codifica dei requisiti desiderabili e facoltativi}



\subsubsection{Prospetto orario}
Di seguito viene illustrato l'utilizzo della risorsa tempo (espresso in ore) dei vari componenti del gruppo nella fase di Progettazione e Codifica dei Requisiti Desiderabili e Facoltativi:

\begin{table}[H]
	\begin{center}
		\begin{tabular}{c
				!{\color[HTML]{9b240a}\vrule width 1pt}
				cccccc
				!{\color[HTML]{9b240a}\vrule width 1pt}	
				c}
			\rowcolorhead
			\headertitle{Nome} & \headertitle{R} & \headertitle{V} & \headertitle{An} & \headertitle{Am} & \headertitle{Pr} & \headertitle{Pt} & \headertitle{Tot} \\
			
			Chiarello Sofia & 0 & 3 & 1 & 0 & 5 & 1 & 10\\
			Crivellari Alberto & 0 & 6 & 0 & 0 & 4 & 0 & 10\\
			De Renzis Simone & 1 & 3 & 0 & 0 & 6 & 0 & 10\\
			Greggio Nicolò & 0 & 3 & 0 & 1 & 6 & 0 & 10\\
			Tessari Andrea & 0 & 3 & 0 & 0 & 7 & 0 & 10\\
			Zuccolo Giada & 0 & 3 & 1 & 0 & 5 & 1 & 10\\
		\end{tabular}
		\caption[Occupazione oraria Progettazione e Codifica dei Requisiti Desiderabili e Facoltativi]{Per ogni componente, i ruoli ricoperti e la relativa occupazione oraria nella fase di Progettazione e Codifica dei Requisiti Desiderabili e Facoltativi}
	\end{center}
\end{table}


\pgfplotsset{width=10cm,compat=1.17}
\begin{figure}[H]
	\centering
	\begin{tikzpicture}
		\begin{axis}[
			enlarge y limits=0.05,
			enlarge x limits=0.10,
			ybar stacked,
			width=10cm,
			bar width=0.7cm,
			%every node near coord/.style={text width=3 cm},
			%nodes near coords,
			%every node near coord/.append style={color=black},
			%enlargelimits=0.25,
			legend style={font=\footnotesize,at={(1.2,1)},
				cells={anchor=west},
				anchor=north,legend columns=1},
			%ylabel={\#participants},
			symbolic x coords={Chiarello Sofia, Crivellari Alberto, De Renzis Simone, Greggio Nicolò, Tessari Andrea, Zuccolo Giada},
			xtick=data,
			x tick label style={font=\footnotesize,text width=1.7cm,align=center},
			]
			\addplot+[ybar,fill=red, draw=black] plot coordinates 
			%Responsabile
			{(Chiarello Sofia,0) 
				(Crivellari Alberto,0) 
				(De Renzis Simone,1) 
				(Greggio Nicolò,0) 
				(Tessari Andrea,0) 
				(Zuccolo Giada,0) };
			\addplot+[ybar,fill=pink, draw=black] plot coordinates 
			%Verificatore
			{(Chiarello Sofia,3) 
				(Crivellari Alberto,6) 
				(De Renzis Simone,3) 
				(Greggio Nicolò,3) 
				(Tessari Andrea,3) 
				(Zuccolo Giada,3) };
			\addplot+[ybar,fill=green, draw=black] plot coordinates 
			%Analista
			{(Chiarello Sofia,1) 
				(Crivellari Alberto,0) 
				(De Renzis Simone,0) 
				(Greggio Nicolò,0) 
				(Tessari Andrea,0) 
				(Zuccolo Giada,1) };
			\addplot+[ybar,fill=yellow, draw=black] plot coordinates
			%Amministratore
			{(Chiarello Sofia,0)
				(Crivellari Alberto,0) 
				(De Renzis Simone,0) 
				(Greggio Nicolò,1)
				(Tessari Andrea,0)
				(Zuccolo Giada,0) };
			\addplot+[ybar,fill=cyan, draw=black] plot coordinates 
			%Programmatore
			{(Chiarello Sofia,5) 
				(Crivellari Alberto,4) 
				(De Renzis Simone,6) 
				(Greggio Nicolò,6) 
				(Tessari Andrea,7) 
				(Zuccolo Giada,5) };
			\addplot+[ybar,fill=orange, draw=black] plot coordinates
			%Progettista
			{(Chiarello Sofia,1) 
				(Crivellari Alberto,0) 
				(De Renzis Simone,0)
				(Greggio Nicolò,0) 
				(Tessari Andrea,0) 
				(Zuccolo Giada,1) };
			\legend{Responsabile \\ Verificatore \\ Analista \\ Amministratore \\ Programmatore \\ Progettista \\}
		\end{axis}
	\end{tikzpicture}
	\caption[Istogramma distribuzione oraria Progettazione e Codifica dei Requisiti Desiderabili e Facoltativi]{Istogramma che visualizza la ripartizione delle ore nella fase di Progettazione e Codifica dei Requisiti Desiderabili e Facoltativi} 
\end{figure}


\subsubsection{Prospetto economico}
Il costo derivante dalle ore impiegate dai componenti nella fase di Progettazione e Codifica dei Requisiti Desiderabili e Facoltativi è descritto di seguito, calcolandone il totale.

\begin{table}[H]
	{\setlength{\parindent}{0cm}
		\begin{minipage}{.43\textwidth}
			\begin{tabular}{ccc}
				\rowcolorhead
				\headertitle{Ruolo} & \headertitle{Ore} & \headertitle{Costo(€)}\\
				Responsabile & 1 & 30\\
				Verificatore & 21 & 315\\
				Analista & 1 & 50\\
				Amministratore & 1 & 20\\
				Programmatore & 33 & 495\\
				Progettista & 2 & 44\\
				\hline
				\textbf{Totale} & \textbf{60} & \textbf{954}\\
			\end{tabular}
		\end{minipage}% This must go next to `\end{minipage}`
		\begin{minipage}{.57\textwidth}
			\begin{tikzpicture}
				\pie [rotate = 270,
				sum = auto, 
				before number=\pgfsetfillopacity{0.0},
				%text = legend, 
				radius = 2.7,
				color = {red, pink, green, yellow, cyan, orange}]
				{
					30/Responsabile,
					315/Verificatore,
					50/Analista,
					20/Amministratore,
					495/Programmatore,
					44/Progettista
				}
			\end{tikzpicture} 
	\end{minipage} }
	\caption[Prospetto economico della fase di Progettazione e Codifica dei Requisiti Desiderabili e Facoltativi]{Per ogni ruolo, il complessivo delle ore impiegate dai membri e il relativo ammontare in denaro. Il diagramma a torta visualizza la composizione dei costi nella fase di Progettazione e Codifica dei Requisiti Desiderabili e Facoltativi}
\end{table}


\subsection{Validazione e collaudo}

\subsubsection{Prospetto orario}
Di seguito viene illustrato l'utilizzo della risorsa tempo (espresso in ore) dei vari componenti del gruppo nella fase di Verifica e Collaudo:

\begin{table}[H]
	\begin{center}
		\begin{tabular}{c
				!{\color[HTML]{9b240a}\vrule width 1pt}
				cccccc
				!{\color[HTML]{9b240a}\vrule width 1pt}	
				c}
			\rowcolorhead
			\headertitle{Nome} & \headertitle{R} & \headertitle{V} & \headertitle{An} & \headertitle{Am} & \headertitle{Pr} & \headertitle{Pt} & \headertitle{Tot} \\
			
			Chiarello Sofia & 0 & 11 & 2 & 0 & 9 & 3 & 25\\
			Crivellari Alberto & 0 & 25 & 0 & 0 & 0 & 0 & 25\\
			De Renzis Simone & 2 & 12 & 0 & 0 & 8 & 3 & 25\\
			Greggio Nicolò & 0 & 14 & 0 & 2 & 6 & 3 & 25\\
			Tessari Andrea & 0 & 20 & 0 & 1 & 0 & 3 & 24\\
			Zuccolo Giada & 0 & 12 & 2 & 0 & 8 & 3 & 25\\
		\end{tabular}
		\caption[Occupazione oraria Verifica e Collaudo]{Per ogni componente, i ruoli ricoperti e la relativa occupazione oraria nella fase di Verifica e Collaudo}
	\end{center}
\end{table}


\pgfplotsset{width=10cm,compat=1.17}
\begin{figure}[H]
	\centering
	\begin{tikzpicture}
		\begin{axis}[
			enlarge y limits=0.05,
			enlarge x limits=0.10,
			ybar stacked,
			width=10cm,
			bar width=0.7cm,
			%every node near coord/.style={text width=3 cm},
			%nodes near coords,
			%every node near coord/.append style={color=black},
			%enlargelimits=0.25,
			legend style={font=\footnotesize,at={(1.2,1)},
				cells={anchor=west},
				anchor=north,legend columns=1},
			%ylabel={\#participants},
			symbolic x coords={Chiarello Sofia, Crivellari Alberto, De Renzis Simone, Greggio Nicolò, Tessari Andrea, Zuccolo Giada},
			xtick=data,
			x tick label style={font=\footnotesize,text width=1.7cm,align=center},
			]
			\addplot+[ybar,fill=red, draw=black] plot coordinates 
			%Responsabile
			{(Chiarello Sofia,0) 
				(Crivellari Alberto,0) 
				(De Renzis Simone,2) 
				(Greggio Nicolò,0) 
				(Tessari Andrea,0) 
				(Zuccolo Giada,0) };
			\addplot+[ybar,fill=pink, draw=black] plot coordinates 
			%Verificatore
			{(Chiarello Sofia,11) 
				(Crivellari Alberto,25) 
				(De Renzis Simone,12) 
				(Greggio Nicolò,14) 
				(Tessari Andrea,20) 
				(Zuccolo Giada,12) };
			\addplot+[ybar,fill=green, draw=black] plot coordinates 
			%Analista
			{(Chiarello Sofia,2) 
				(Crivellari Alberto,0) 
				(De Renzis Simone,0) 
				(Greggio Nicolò,0) 
				(Tessari Andrea,0) 
				(Zuccolo Giada,2) };
			\addplot+[ybar,fill=yellow, draw=black] plot coordinates
			%Amministratore
			{(Chiarello Sofia,0)
				(Crivellari Alberto,0) 
				(De Renzis Simone,0) 
				(Greggio Nicolò,2)
				(Tessari Andrea,1)
				(Zuccolo Giada,0) };
			\addplot+[ybar,fill=cyan, draw=black] plot coordinates 
			%Programmatore
			{(Chiarello Sofia,9) 
				(Crivellari Alberto,0) 
				(De Renzis Simone,8) 
				(Greggio Nicolò,6) 
				(Tessari Andrea,0) 
				(Zuccolo Giada,8) };
			\addplot+[ybar,fill=orange, draw=black] plot coordinates
			%Progettista
			{(Chiarello Sofia,3) 
				(Crivellari Alberto,0) 
				(De Renzis Simone,3)
				(Greggio Nicolò,3) 
				(Tessari Andrea,3) 
				(Zuccolo Giada,3) };
			\legend{Responsabile \\ Verificatore \\ Analista \\ Amministratore \\ Programmatore \\ Progettista \\}
		\end{axis}
	\end{tikzpicture}
	\caption[Istogramma distribuzione oraria Verifica e Collaudo]{Istogramma che visualizza la ripartizione delle ore nella fase di Verifica e Collaudo} 
\end{figure}


\subsubsection{Prospetto economico}
Il costo derivante dalle ore impiegate dai componenti nella fase di Verifica e Collaudo è descritto di seguito, calcolandone il totale.

\begin{table}[H]
	{\setlength{\parindent}{0cm}
		\begin{minipage}{.43\textwidth}
			\begin{tabular}{ccc}
				\rowcolorhead
				\headertitle{Ruolo} & \headertitle{Ore} & \headertitle{Costo(€)}\\
				Responsabile & 2 & 60\\
				Verificatore & 94 & 1410\\
				Analista & 4 & 100\\
				Amministratore & 3 & 60\\
				Programmatore & 31 & 465\\
				Progettista & 15 & 330\\
				\hline
				\textbf{Totale} & \textbf{149} & \textbf{2425}\\
			\end{tabular}
		\end{minipage}% This must go next to `\end{minipage}`
		\begin{minipage}{.57\textwidth}
			\begin{tikzpicture}
				\pie [rotate = 270,
				sum = auto, 
				before number=\pgfsetfillopacity{0.0},
				%text = legend, 
				radius = 2.7,
				color = {red, pink, green, yellow, cyan, orange}]
				{
					60/Responsabile,
					1410/Verificatore,
					100/Analista,
					60/Amministratore,
					465/Programmatore,
					330/Progettista
				}
			\end{tikzpicture} 
	\end{minipage} }
	\caption[Prospetto economico della fase di Verifica e Collaudo]{Per ogni ruolo, il complessivo delle ore impiegate dai membri e il relativo ammontare in denaro. Il diagramma a torta visualizza la composizione dei costi nella fase di Verifica e Collaudo}
\end{table}



\subsection{Riepilogo}

\subsubsection{Totale Ore}
Di seguito viene illustrato l'utilizzo totale della risorsa tempo (espresso in ore) dei vari componenti del gruppo:

\begin{table}[H]
	\begin{center}
		\begin{tabular}{c
				!{\color[HTML]{9b240a}\vrule width 1pt}
				cccccc
				!{\color[HTML]{9b240a}\vrule width 1pt}	
				c}
			\rowcolorhead
			\headertitle{Nome} & \headertitle{R} & \headertitle{V} & \headertitle{An} & \headertitle{Am} & \headertitle{Pr} & \headertitle{Pt} & \headertitle{Tot} \\
			
			Chiarello Sofia & 3 & 39 & 32 & 1 & 44 & 25 & 144\\
			Crivellari Alberto & 3 & 77 & 9 & 3 & 29 & 23 & 144\\
			De Renzis Simone & 25 & 37 & 7 & 8 & 42 & 25 & 144\\
			Greggio Nicolò & 9 & 39 & 7 & 28 & 40 & 21 & 144\\
			Tessari Andrea & 13 & 47 & 9 & 13 & 32 & 30 & 144\\
			Zuccolo Giada & 3 & 39 & 32 & 3 & 44 & 23 & 144\\
		\end{tabular}
		\caption[Occupazione oraria totale]{Per ogni componente, i ruoli ricoperti e la relativa occupazione oraria per tutta la durata del lavoro}
	\end{center}
\end{table}


\pgfplotsset{width=10cm,compat=1.17}
\begin{figure}[H]
	\centering
	\begin{tikzpicture}
		\begin{axis}[
			enlarge y limits=0.05,
			enlarge x limits=0.10,
			ybar stacked,
			width=10cm,
			bar width=0.7cm,
			%every node near coord/.style={text width=3 cm},
			%nodes near coords,
			%every node near coord/.append style={color=black},
			%enlargelimits=0.25,
			legend style={font=\footnotesize,at={(1.2,1)},
				cells={anchor=west},
				anchor=north,legend columns=1},
			%ylabel={\#participants},
			symbolic x coords={Chiarello Sofia, Crivellari Alberto, De Renzis Simone, Greggio Nicolò, Tessari Andrea, Zuccolo Giada},
			xtick=data,
			x tick label style={font=\footnotesize,text width=1.7cm,align=center},
			]
			\addplot+[ybar,fill=red, draw=black] plot coordinates 
			%Responsabile
			{(Chiarello Sofia,3) 
				(Crivellari Alberto,3) 
				(De Renzis Simone,25) 
				(Greggio Nicolò,9) 
				(Tessari Andrea,13) 
				(Zuccolo Giada,3) };
			\addplot+[ybar,fill=pink, draw=black] plot coordinates 
			%Verificatore
			{(Chiarello Sofia,39) 
				(Crivellari Alberto,77) 
				(De Renzis Simone,37) 
				(Greggio Nicolò,39) 
				(Tessari Andrea,47) 
				(Zuccolo Giada,39) };
			\addplot+[ybar,fill=green, draw=black] plot coordinates 
			%Analista
			{(Chiarello Sofia,32) 
				(Crivellari Alberto,9) 
				(De Renzis Simone,7) 
				(Greggio Nicolò,7) 
				(Tessari Andrea,9) 
				(Zuccolo Giada,32) };
			\addplot+[ybar,fill=yellow, draw=black] plot coordinates
			%Amministratore
			{(Chiarello Sofia,1)
				(Crivellari Alberto,3) 
				(De Renzis Simone,8) 
				(Greggio Nicolò,28)
				(Tessari Andrea,13)
				(Zuccolo Giada,3) };
			\addplot+[ybar,fill=cyan, draw=black] plot coordinates 
			%Programmatore
			{(Chiarello Sofia,44) 
				(Crivellari Alberto,29) 
				(De Renzis Simone,42) 
				(Greggio Nicolò,40) 
				(Tessari Andrea,32) 
				(Zuccolo Giada,44) };
			\addplot+[ybar,fill=orange, draw=black] plot coordinates
			%Progettista
			{(Chiarello Sofia,25) 
				(Crivellari Alberto,23) 
				(De Renzis Simone,25)
				(Greggio Nicolò,21) 
				(Tessari Andrea,30) 
				(Zuccolo Giada,23) };
			\legend{Responsabile \\ Verificatore \\ Analista \\ Amministratore \\ Programmatore \\ Progettista \\}
		\end{axis}
	\end{tikzpicture}
	\caption[Istogramma distribuzione oraria totale]{Istogramma che visualizza la ripartizione delle ore per tutta la durata del lavoro} 
\end{figure}


\subsubsection{Prospetto economico}
Questa tabella mostra il costo totale per ogni ruolo all'interno del team. Viene evidenziato il totale.

\begin{table}[H]
	{\setlength{\parindent}{0cm}
		\begin{minipage}{.43\textwidth}
			\begin{tabular}{ccc}
				\rowcolorhead
				\headertitle{Ruolo} & \headertitle{Ore} & \headertitle{Costo(€)}\\
				Responsabile & 56 & 1680\\
				Verificatore & 278 & 4170\\
				Analista & 96 & 2400\\
				Amministratore & 56 & 1120\\
				Programmatore & 231 & 3465\\
				Progettista & 147 & 3234\\
				\hline
				\textbf{Totale} & \textbf{864} & \textbf{16069}\\
			\end{tabular}
		\end{minipage}% This must go next to `\end{minipage}`
		\begin{minipage}{.57\textwidth}
			\begin{tikzpicture}
				\pie [rotate = 270,
				sum = auto, 
				before number=\pgfsetfillopacity{0.0},
				%text = legend, 
				radius = 2.7,
				color = {red, pink, green, yellow, cyan, orange}]
				{
					1680/Responsabile,
					4170/Verificatore,
					2400/Analista,
					1120/Amministratore,
					3465/Programmatore,
					3234/Progettista
				}
			\end{tikzpicture} 
	\end{minipage} }
	\caption[Prospetto economico complessivo]{Per ogni ruolo, il complessivo delle ore impiegate dai membri e il relativo ammontare in denaro. Il diagramma a torta visualizza la composizione dei costi complessivi}
\end{table}


\subsubsection{Ore rendicontate}


Questa tabella descrive il numero di ore rendicontate di ogni componente (sono cioè escluse le ore dedicate all'Avvio e all'Analisi dei Requisiti):

\begin{table}[H]
	\begin{center}
		\begin{tabular}{c
				!{\color[HTML]{9b240a}\vrule width 1pt}
				cccccc
				!{\color[HTML]{9b240a}\vrule width 1pt}	
				c}
			\rowcolorhead
			\headertitle{Nome} & \headertitle{R} & \headertitle{V} & \headertitle{An} & \headertitle{Am} & \headertitle{Pr} & \headertitle{Pt} & \headertitle{Tot} \\
			
			Chiarello Sofia & 1 & 31 & 9 & 0 & 44 & 25 & 110\\
			Crivellari Alberto & 57 & 0 & 1 & 3 & 29 & 23 & 111\\
			De Renzis Simone & 9 & 31 & 0 & 2 & 42 & 25 & 109\\
			Greggio Nicolò & 2 & 33 & 0 & 11 & 40 & 21 & 107\\
			Tessari Andrea & 3 & 40 & 0 & 5 & 32 & 30 & 110\\
			Zuccolo Giada & 1 & 31 & 9 & 2 & 44 & 23 & 110\\
		\end{tabular}
		\caption[Occupazione oraria totale rendicontata]{Per ogni componente, i ruoli ricoperti e la relativa occupazione oraria rendicontata totale}
	\end{center}
\end{table}


\pgfplotsset{width=10cm,compat=1.17}
\begin{figure}[H]
	\centering
	\begin{tikzpicture}
		\begin{axis}[
			enlarge y limits=0.05,
			enlarge x limits=0.10,
			ybar stacked,
			width=10cm,
			bar width=0.7cm,
			%every node near coord/.style={text width=3 cm},
			%nodes near coords,
			%every node near coord/.append style={color=black},
			%enlargelimits=0.25,
			legend style={font=\footnotesize,at={(1.2,1)},
				cells={anchor=west},
				anchor=north,legend columns=1},
			%ylabel={\#participants},
			symbolic x coords={Chiarello Sofia, Crivellari Alberto, De Renzis Simone, Greggio Nicolò, Tessari Andrea, Zuccolo Giada},
			xtick=data,
			x tick label style={font=\footnotesize,text width=1.7cm,align=center},
			]
			\addplot+[ybar,fill=red, draw=black] plot coordinates 
			%Responsabile
			{(Chiarello Sofia,1) 
				(Crivellari Alberto,1) 
				(De Renzis Simone,9) 
				(Greggio Nicolò,2) 
				(Tessari Andrea,3) 
				(Zuccolo Giada,1) };
			\addplot+[ybar,fill=pink, draw=black] plot coordinates 
			%Verificatore
			{(Chiarello Sofia,31) 
				(Crivellari Alberto,27) 
				(De Renzis Simone,31) 
				(Greggio Nicolò,33) 
				(Tessari Andrea,40) 
				(Zuccolo Giada,31) };
			\addplot+[ybar,fill=green, draw=black] plot coordinates 
			%Analista
			{(Chiarello Sofia,9) 
				(Crivellari Alberto,0) 
				(De Renzis Simone,0) 
				(Greggio Nicolò,0) 
				(Tessari Andrea,0) 
				(Zuccolo Giada,9) };
			\addplot+[ybar,fill=yellow, draw=black] plot coordinates
			%Amministratore
			{(Chiarello Sofia,0)
				(Crivellari Alberto,1) 
				(De Renzis Simone,2) 
				(Greggio Nicolò,11)
				(Tessari Andrea,5)
				(Zuccolo Giada,2) };
			\addplot+[ybar,fill=cyan, draw=black] plot coordinates 
			%Programmatore
			{(Chiarello Sofia,44) 
				(Crivellari Alberto,29) 
				(De Renzis Simone,42) 
				(Greggio Nicolò,40) 
				(Tessari Andrea,32) 
				(Zuccolo Giada,44) };
			\addplot+[ybar,fill=orange, draw=black] plot coordinates
			%Progettista
			{(Chiarello Sofia,25) 
				(Crivellari Alberto,23) 
				(De Renzis Simone,25)
				(Greggio Nicolò,21) 
				(Tessari Andrea,30) 
				(Zuccolo Giada,23) };
			\legend{Responsabile \\ Verificatore \\ Analista \\ Amministratore \\ Programmatore \\ Progettista \\}
		\end{axis}
	\end{tikzpicture}
	\caption[Istogramma distribuzione oraria rendicontata totale]{Istogramma che visualizza la ripartizione delle ore rendicontate per tutta la durata del lavoro} 
\end{figure}


\subsubsection{Prospetto economico ore rendicontate}
Questa tabella mostra il costo totale rendicontato per ogni ruolo all'interno del team. Viene evidenziato il totale.

\begin{table}[H]
	{\setlength{\parindent}{0cm}
		\begin{minipage}{.43\textwidth}
			\begin{tabular}{ccc}
				\rowcolorhead
				\headertitle{Ruolo} & \headertitle{Ore} & \headertitle{Costo(€)}\\
				Responsabile & 17 & 510\\
				Verificatore & 223 & 3345\\
				Analista & 18 & 450\\
				Amministratore & 21 & 420\\
				Programmatore & 231 & 3465\\
				Progettista & 147 & 3234\\
				\hline
				\textbf{Totale} & \textbf{657} & \textbf{11424}\\
			\end{tabular}
		\end{minipage}% This must go next to `\end{minipage}`
		\begin{minipage}{.57\textwidth}
			\begin{tikzpicture}
				\pie [rotate = 270,
				sum = auto, 
				before number=\pgfsetfillopacity{0.0},
				%text = legend, 
				radius = 2.7,
				color = {red, pink, green, yellow, cyan, orange}]
				{
					510/Responsabile,
					3345/Verificatore,
					450/Analista,
					420/Amministratore,
					3465/Programmatore,
					3234/Progettista
				}
			\end{tikzpicture} 
	\end{minipage} }
	\caption[Prospetto economico complessivo delle ore rendicontate]{Per ogni ruolo, il complessivo delle ore rendicontate impiegate dai membri e il relativo ammontare in denaro. Il diagramma a torta visualizza la composizione dei costi complessivi rendicontati}
\end{table}


\subsection{Conclusione}
Il lavoro prevede costi per \textbf{11424 €}, tenendo conto solamente delle ore rendicontate.

	\pagebreak

	\section{Consuntivo}

La sezione che segue espone le spese effettivamente sostenute, registrate al termine delle fasi di Avvio e di Analisi dei Requisti. In relazione alle spese preventivate, il periodo chiuderà in:
\begin{itemize}
	\item \textbf{positivo} se il preventivo supera il consuntivo;
	\item \textbf{pari} se il preventivo e il consuntivo collimano;
	\item \textbf{negativo} se il consuntivo supera il preventivo.
\end{itemize}


\subsection{Avvio}

\begin{table}[H]
	\begin{center}
		\begin{tabular}{c
				!{\color[HTML]{9b240a}\vrule width 1pt}
				cccccc
				!{\color[HTML]{9b240a}\vrule width 1pt}	
				c}
			\rowcolorhead
			\headertitle{Nome} & \headertitle{R} & \headertitle{V} & \headertitle{An} & \headertitle{Am} & \headertitle{Pr} & \headertitle{Pt} & \headertitle{Tot} \\
			
			Chiarello Sofia & 1 & 0 & 4 & 3 & 0 & 0 & 8\\
			Crivellari Alberto & 2 & 0 & 1 & 2 & 0 & 0 & 5\\
			De Renzis Simone & 3 & 0 & 2 & 4 & 0 & 0 & 9\\
			Greggio Nicolò & 2 & 0 & 2 & 5 & 0 & 0 & 9\\
			Tessari Andrea & 2 & 0 & 1 & 3 & 0 & 0 & 6\\
			Zuccolo Giada & 1 & 0 & 4 & 2 & 0 & 0 & 7\\
		\end{tabular}
		\caption[Consuntivo fase di Avvio]{Per ogni componente, le ore effettivamente spese nella fase di Avvio}
	\end{center}
\end{table}




\subsection{Analisi dei requisiti}

\begin{table}[H]
	\begin{center}
		\begin{tabular}{c
				!{\color[HTML]{9b240a}\vrule width 1pt}
				cccccc
				!{\color[HTML]{9b240a}\vrule width 1pt}	
				c}
			\rowcolorhead
			\headertitle{Nome} & \headertitle{R} & \headertitle{V} & \headertitle{An} & \headertitle{Am} & \headertitle{Pr} & \headertitle{Pt} & \headertitle{Tot} \\
			
			Chiarello Sofia & 3 & 4 & 15 & 3 & 0 & 0 & 25\\
			Crivellari Alberto & 4 & 12 & 4 & 5 & 0 & 0 & 25\\
			De Renzis Simone & 12 & 4 & 5 & 5 & 0 & 0 & 26\\
			Greggio Nicolò & 5 & 2 & 5 & 25 & 0 & 0 & 37\\
			Tessari Andrea & 8 & 9 & 4 & 3 & 0 & 0 & 24\\
			Zuccolo Giada & 3 & 6 & 13 & 3 & 0 & 0 & 25\\
		\end{tabular}
		\caption[Consuntivo fase di Analisi dei Requisiti]{Per ogni componente, le ore effettivamente spese nella fase di Analisi dei Requisiti}
	\end{center}
\end{table}



\subsection{Totale}

\begin{table}[H]
	\centering
	\begin{tabular}{ccc}
		\rowcolorhead
		\headertitle{Ruolo} & \headertitle{Ore} & \headertitle{Costo(\euro{})}\\
		Responsabile & 46 (+7) & 1380 (+210) \\
		Verificatore & 37 (-18) & 555 (-270)\\
		Analista & 60 (-18) & 1500 (-450)\\				
		Amministratore & 63 (+28) & 1260 (+560)\\
		Programmatore & 0 (+0) & 0 (+0)\\
		Progettista & 0 (+0) & 0 (+0)\\
		\hline		
		\textbf{Totale consuntivo} & \textbf{206} & \textbf{4695}\\
		\textbf{Totale preventivo} & \textbf{207} & \textbf{4645}\\
		\textbf{Differenza} & \textbf{-1} & \textbf{50}\\
	\end{tabular}
	\caption[Confronto tra preventivo e consuntivo]{Per ogni ruolo, il totale delle ore effettivamente impiegate, con lo scostamento dal preventivo}
\end{table}


\subsection{Preventivo a finire}
Il preventivo a finire comprende i costi consuntivi di tutte le attività terminate più i costi previsti per le attività da eseguire. Consistendo il periodo di attività non rendicontate, il preventivo a finire coinciderà con il preventivo presentato nella sezione \S 4.



\subsection{Conclusioni}

Il periodo si chiude in \textbf{positivo}, permettendo al gruppo un risparmio di \textbf{50 \euro{}}. Tuttavia sono osservabili rilevanti discostamenti tra i costi preventivati e quelli realmente sostenuti, in particolare per quanto riguarda i ruoli di:
\begin{itemize}
	\item \textbf{Verificatore}: la verifica della documentazione si è svolta in maniera snella e senza inconvenienti;
	\item \textbf{Analista}: nonostante il periodo prevedesse ampie attività di analisi, i lavori si sono svolti con più efficienza del previsto;
	\item \textbf{Amministratore}: questo ruolo ha richiesto più tempo di quanto preventivato, in particolare nella messa a punto di attività di automatizzazione come nel caso del \textsc{Glossario}.
\end{itemize}
Le osservazioni ricavate da questo periodo verranno tenute in considerazione nello svolgimento delle prossime fasi, nelle corso delle quali si valuterà se attuare delle correzioni al preventivo presentato.

	\pagebreak

	\appendix
	\section{Riscontro rischi}

\newcolumntype{M}[1]{>{\centering\arraybackslash}m{#1}}
\renewcommand{\arraystretch}{1.5}
\rowcolors{2}{pari}{dispari}
\begin{longtable} { 
		>{}M{0.15\textwidth} 
		>{}p{0.70\textwidth} 
		}
	\rowcolorhead
	\centering\headertitle{Codice} &
	\centering\headertitle{Riscontro}
	\endfirsthead	
	\endhead
	\caption{Organigramma di accettazione} \endhead	
	
	RIS\_T - 3 & La documentazione del capitolato è risultata insufficiente per una completa comprensione del problema e non è stato possibile derivarne direttamente dei requisiti dettagliati. \'E stato quindi tenuto un incontro con il proponente, il quale ha chiarito con precisione il dominio del problema e i suoi vincoli principali. La disponibilità del proponente ha permesso un intenso scambio di domande e risposte che è continuato per tutta la durata della fase di Analisi, attraverso il quale i dubbi del gruppo sono stati risolti.		
	\tabularnewline	
	RIS\_O - 2 & Complice anche il periodo di festività, alcuni membri non hanno partecipato alle riunioni, si sono presentati in ritardo o sono dovuti uscire in anticipo a causa di impegni personali. In tutti questi casi il gruppo era comunque stato avvisato per tempo, e l'assenza o i ritardi erano giustificati. I membri interessati hanno potuto consultare i verbali prodotti in seguito alle agli incontri per rimanere aggiornati sulle decisioni prese.
	
		
\end{longtable}




\newpage

\section{Organigramma}



\subsection{Redazione}
\newcolumntype{M}[1]{>{\centering\arraybackslash}m{#1}}
\renewcommand{\arraystretch}{1.5}
\rowcolors{2}{pari}{dispari}
\begin{longtable} { 
		>{}M{0.25\textwidth} 
		>{\centering}M{0.25\textwidth} 
		>{\centering}M{0.35\textwidth}    }
	\rowcolorhead
	\centering\headertitle{Nominativo} &
	\centering\headertitle{Data di Redazione} &	
	\centering\headertitle{Firma} 
	\endfirsthead	
	\endhead
	\caption{Organigramma di accettazione} \endhead	
	Chiarello Sofia & 2021-01-09 & \includegraphics[scale = 0.06]{../../../../latex/images/signatures/sofia.png}	
	\tabularnewline	
	Crivellari Alberto & 2021-01-09 & \includegraphics[scale = 0.31]{../../../../latex/images/signatures/alberto.png}	
	\tabularnewline	
	De Renzis Simone & 2021-01-09 & \includegraphics[scale = 0.13]{../../../../latex/images/signatures/simone.png}	
	\tabularnewline	
	Greggio Nicolò & 2021-01-09 & \includegraphics[scale = 0.10]{../../../../latex/images/signatures/nicolo.png}	
	\tabularnewline	
	Tessari Andrea & 2021-01-09 & \includegraphics[scale = 0.13]{../../../../latex/images/signatures/simone.png}	
	\tabularnewline		
	Zuccolo Giada & 2021-01-09 & \includegraphics[scale = 0.15]{../../../../latex/images/signatures/giada.png}	
\end{longtable}



\subsection{Approvazione}


\renewcommand{\arraystretch}{1.5}
\rowcolors{2}{pari}{dispari}
\begin{longtable} { 
		>{}M{0.25\textwidth} 
		>{\centering}M{0.25\textwidth} 
		>{\centering}M{0.35\textwidth}    }
	\rowcolorhead
	\centering\headertitle{Nominativo} &
	\centering\headertitle{Data di Redazione} &	
	\centering\headertitle{Firma} 
	\endfirsthead	
	\endhead
	\caption{Organigramma di accettazione} \endhead	
	De Renzis Simone & 2021-01-09 & \includegraphics[scale = 0.13]{../../../../latex/images/signatures/simone.png}	
	\tabularnewline	
	Vardanega Tullio &  & 
	\tabularnewline	
	Cardin Riccardo &  & 	
\end{longtable}

\newpage


\subsection{Accettazione dei componenti}


\renewcommand{\arraystretch}{1.5}
\rowcolors{2}{pari}{dispari}
\begin{longtable} { 
		>{}M{0.25\textwidth} 
		>{\centering}M{0.25\textwidth} 
		>{\centering}M{0.35\textwidth}    }
	\rowcolorhead
	\centering\headertitle{Nominativo} &
	\centering\headertitle{Data di Redazione} &	
	\centering\headertitle{Firma} 
	\endfirsthead	
	\endhead
	\caption{Organigramma di accettazione} \endhead	
	Chiarello Sofia & 2021-01-09 & \includegraphics[scale = 0.06]{../../../../latex/images/signatures/sofia.png}	
	\tabularnewline	
	Crivellari Alberto & 2021-01-09 & \includegraphics[scale = 0.31]{../../../../latex/images/signatures/alberto.png}	
	\tabularnewline	
	De Renzis Simone & 2021-01-09 & \includegraphics[scale = 0.13]{../../../../latex/images/signatures/simone.png}	
	\tabularnewline	
	Greggio Nicolò & 2021-01-09 & \includegraphics[scale = 0.10]{../../../../latex/images/signatures/nicolo.png}	
	\tabularnewline	
	Tessari Andrea & 2021-01-09 & \includegraphics[scale = 0.13]{../../../../latex/images/signatures/simone.png}	
	\tabularnewline		
	Zuccolo Giada & 2021-01-09 & \includegraphics[scale = 0.15]{../../../../latex/images/signatures/giada.png}	
\end{longtable}




\subsection{Componenti}

\renewcommand{\arraystretch}{1.5}
\rowcolors{2}{pari}{dispari}
\begin{longtable} { 
		>{}p{0.22\textwidth} 
		>{\centering}p{0.22\textwidth} 
		>{}p{0.43\textwidth}    }
	\rowcolorhead
	\headertitle{Nominativo} &
	\centering\headertitle{Matricola} &	
	\headertitle{Indirizzo di posta elettronica} 
	\endfirsthead	
	\endhead
	\caption{Organigramma di accettazione} \endhead	
	Chiarello Sofia & 1187024 & 
	sofia.chiarello@studenti.unipd.it
	\tabularnewline	
	Crivellari Alberto & 1170913 & alberto.crivellari.2@studenti.unipd.it
	\tabularnewline	
	De Renzis Simone & 1187510 & simone.derenzis@studenti.unipd.it	
	\tabularnewline	
	Greggio Nicolò & 1193398 & nicolo.greggio.1@studenti.unipd.it	
	\tabularnewline	
	Tessari Andrea & 1188322 & andrea.tessari.3@studenti.unipd.it 
	\tabularnewline		
	Zuccolo Giada & 1193485 & giada.zuccolo@studenti.unipd.it	
\end{longtable}



	\pagebreak


	%\glsaddall
	\printglossary[type=\acronymtype,title=Lista degli Acronimi]
	\printglossary[title=Glossario dei Termini]
\end{document}
