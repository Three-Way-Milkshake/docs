\section{Pianificazione}

\subsection{Modello di sviluppo}

Il modello di sviluppo\textsubscript{G} scelto è il modello \textbf{incrementale}. Esso si adatta al sistema di revisioni\textsubscript{G} a cui vanno incontro gli artefatti\textsubscript{G} prodotti nel corso del progetto\textsubscript{G}. Il modello incrementale\textsubscript{G} infatti prevede che Analisi dei Requisiti e Progettazione Architetturale si svolgano una volta sola: queste attività\textsubscript{G} servono a studiare il problema e a strutturarne la soluzione. Si tornerà su queste attività\textsubscript{G} per raffinarne i contenuti in base a nuove evidenze individuate nei periodi successivi.

La progettazione di dettaglio e la codifica invece si svilupperanno attraverso cicli di incremento atti a integrare il sistema di nuove funzionalità: si partirà dal soddisfacimento dei requisiti\textsubscript{G} obbligatori, per poi eventualmente incrementare con requisiti\textsubscript{G} desiderabili e facoltativi.

Queste modalità permettono, gettate le basi del prodotto, di accrescerne le funzionalità producendo valore fin da subito, in modo da avere riscontro quasi immediato sull'operato e poterne indirizzare gli sviluppi successivi in base ai feedback ricevuti, anche dal proponente, e alle risorse\textsubscript{G} disponibili.

Il team adotterà anche alcune tecniche tipiche dello sviluppo Agile: viene fatto uso di una Kanban board, strumento che permette di pianificare in dettaglio e visualizzare gli obiettivi a cui ciascun membro del team si dedica. Questa tecnica riflette la modalità con cui il team si organizza nel contesto di un incremento: un meeting a cadenza settimanale permette di pianificare l'avanzamento e stabilire le future assegnazioni, in modo da affrontare eventuali ritardi o difficoltà prima che possano causare problemi allo sviluppo complessivo.



\subsection{Scadenze}

Il gruppo stabilisce di affrontare le revisione\textsubscript{G} di avanzamento nelle seguenti date:
\begin{itemize}
	\item \textbf{Revisione dei Requisiti}: 18 Gennaio 2021
	\item \textbf{Revisione di Progettazione}: 8 Marzo 2021 
	\item \textbf{Revisione di Qualifica}: 9 Aprile 2021
	\item \textbf{Revisione di Accettazione}: 10 Maggio 2021	
\end{itemize}


\subsection{Macro periodi}

A fronte del modello di sviluppo\textsubscript{G} scelto e delle scadenze fissate, lo sviluppo procederà attraverso i seguenti macro periodi:
\begin{itemize}
	\item \textbf{Avvio}
	\item \textbf{Analisi dei Requisiti}
	\item \textbf{Progettazione Architetturale}
	\item \textbf{Progettazione di Dettaglio e Codifica}
	\item \textbf{Validazione e Collaudo}
\end{itemize}



\subsection{Organizzazione del documento}

Le sezioni che seguono descrivono la pianificazione, la concretizzazione e il riscontro del progetto\textsubscript{G}. Ogni macro periodo\textsubscript{G} di lavoro si articola in periodi di durata circoscritta: una diagramma di Gantt mostra ad alto livello l'organizzazione del macro periodo\textsubscript{G}. Per ogni periodo\textsubscript{G} è prevista la seguente caratterizzazione:
\begin{itemize}
	\item \textbf{Pianificazione preventiva}: la pianificazione realizzata a priori, ossia all'inizio del progetto\textsubscript{G}. Sono esplicitate le attività\textsubscript{G} che interessano il periodo\textsubscript{G}, per ogni attività\textsubscript{G} una breve descrizione, un preventivo delle ore previsto per portarla a termine e il ruolo interessato. Le ore impiegate e i costi sono sintetizzati in un preventivo "a priori" che va a contribuire al preventivo iniziale.
	
	\item \textbf{Pianificazione di periodo}: la pianificazione realizzata al termine del periodo\textsubscript{G} precedente: è aggiornata sulla base dell'evoluzione del progetto\textsubscript{G} e, se necessario, è ricalibrata nei tempi. La pianificazione è scandita da un diagramma di Gantt. Fornisce inoltre un preventivo dettagliato "di periodo" che esplicita, per ogni componente del gruppo, le ore e il costo dei ruoli da ognuno ricoperti\footnote{i macroperiodi di Avvio e Analisi dei Requisiti non presentano questa parte in quando introdotta successivamente allo svolgimento delle stesse.}.
	
	\item \textbf{Riscontro di fine periodo}: contiene il consuntivo e la differenza in ore e costi del periodo\textsubscript{G} trascorso (confrontato con la pianificazione di periodo) e il preventivo a finire. Il preventivo a finire è ottenuto sommando le spese fin'ora sostenute (consuntivo periodi precedenti e quello appena trascorso), il preventivo di periodo\textsubscript{G} del prossimo periodo\textsubscript{G} e i preventivi a priori dei periodi successivi al prossimo. In relazione alle spese preventivate, il periodo\textsubscript{G} chiude in: \textbf{positivo} se il preventivo supera il consuntivo; \textbf{pari} se il preventivo e il consuntivo collimano; \textbf{negativo} se il consuntivo supera il preventivo.
	
\end{itemize}

La sezione finale prevede il preventivo iniziale totale, ottenuto come somma dei singoli preventivi a priori. Corrisponde al preventivo con cui il gruppo si candida alla Revisione dei Requisiti.
