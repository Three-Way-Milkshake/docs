\section{Introduzione}




\subsection{Scopo del documento}

Nel contesto della realizzazione del progetto\textsubscript{G} PORTACS\textsubscript{A} da parte del gruppo Three Way Milkshake, il documento risponde alle seguenti esigenze: 
\begin{itemize}
	\item analizzare i rischi che possono emergere durante lo sviluppo, elaborando strategie per mitigarne gli effetti;
	\item pianificare il lavoro istanziando delle attivita\textsubscript{G} a partire dal modello di sviluppo\textsubscript{G} scelto e fissandone le scadenze;
	\item fornire una valutazione preventiva delle risorsa\textsubscript{G} necessarie a ciascuna fase\textsubscript{G} in termini di ore di lavoro;
	\item esporre le spese sostenute nelle fase\textsubscript{G} già attraversate;
	\item verbalizzare i rischi effettivamente riscontrati.
\end{itemize}



\subsection{Scopo del prodotto}

Il capitolato\textsubscript{G} C5 propone un progetto\textsubscript{G} in cui viene richiesto lo sviluppo di un software per il monitoraggio in tempo reale di unità che si muovono in uno spazio definito. All’interno di questo spazio, creato dall’utente per riprodurre le caratteristiche di un ambiente reale, le unità dovranno essere in grado di circolare in autonomia, o sotto il controllo dell’utente, per raggiungere dei punti di interesse posti nella mappa.  La circolazione è sottoposta a vincoli di viabilità e ad ostacoli propri della topologia dell’ambiente, deve evitare le collisioni con le altre unità e prevedere la gestione di situazioni critiche nel traffico.




\subsection{Riferimenti}



\subsubsection{Normativi}

\begin{itemize}
	\item \textsc{Norme di progetto\textsubscript{G} v1.0.0 }: per qualsiasi convenzione sulla nomenclatura degli elementi presenti all’interno del documento;
	\item Specifica tecnico-economica e organigramma: \\ {\url{https://www.math.unipd.it/~tullio/IS-1/2020/Progetto/RO.html}} %solo per piano di progetto\textsubscript{G}
	\item Regolamento progetto\textsubscript{G} didattico - slide del corso di Ingegneria del Software: \\ {\url{https://www.math.unipd.it/~tullio/IS-1/2020/Dispense/P1.pdf}}
\end{itemize}



\subsubsection{Informativi}
\begin{itemize}
	\item \textsc{\href{https://github.com/Three-Way-Milkshake/docs/wiki/Glossario}{Glossario}}: per la definizione dei termini (pedice G) e degli acronimi (pedice A) evidenziati nel documento;
	\item Capitolato d'appalto C5-PORTACS: \\
{\url{https://www.math.unipd.it/~tullio/IS-1/2020/Progetto/C5.pdf}}
	\item Software Engineering - Iam Sommerville - $10^{th}$ Edition
	\item Slide L05 \\
	{\url{https://www.math.unipd.it/~tullio/IS-1/2020/Dispense/L05.pdf}}%solo per piano di progetto\textsubscript{G}
	\item Slide L06 \\
	{\url{https://www.math.unipd.it/~tullio/IS-1/2020/Dispense/L06.pdf}}%solo per piano di progetto\textsubscript{G}
	\item Slide L07 \\
	{\url{https://www.math.unipd.it/~tullio/IS-1/2020/Dispense/L07.pdf}}%solo per piano di progetto\textsubscript{G}
\end{itemize}