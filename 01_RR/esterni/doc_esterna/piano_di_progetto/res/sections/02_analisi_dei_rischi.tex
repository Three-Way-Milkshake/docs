\section{Analisi dei rischi}

In un progetto\textsubscript{G} di queste dimensioni è possibile incontrare problemi di varia natura. Per arginare i rischi si possono gestire attentamente 4 attivita\textsubscript{G}:

\begin{itemize}
	\item \textbf{Individuazione dei rischi}: individuare i fattori che possono introdurre criticità nello svolgimento del progetto\textsubscript{G};
	\item \textbf{Analisi dei rischi}: esaminare i fattori di rischio stimando la probabilità che la criticità si manifesti, l'impatto che ha e le sue conseguenze nel progetto\textsubscript{G};
	\item \textbf{Pianificazione per il controllo}: pianificare delle misure atte a impedire il verificarsi del problema e ad arginarne le conseguenze;
	\item \textbf{Monitoraggio dei rischi}: controllare attivamente e in modo costante i fattori di rischio al fine di prevenirne o intercettarne in modo tempestivo gli effetti.
\end{itemize}


\subsection{Rischi tecnologici}



%------------------------------------RIS\_T \_ 1---------------------------------------

\renewcommand{\arraystretch}{1.5}
\rowcolors{2}{pari}{dispari}
\begin{longtable} { 
		>{\raggedright}p{0.33\textwidth} 
		>{\raggedright}p{0.33\textwidth} 
		>{\raggedright}p{0.33\textwidth}    }
		
		\caption{RIS\_T - 1} \endhead	


	\textbf{Nome}: \\ Novità del problema e delle tecnologie
	& \textbf{Codice}: \\ RIS\_T - 1  
	& \textbf{Occorrenza}: Alta \\ \textbf{Pericolosità}: Media
	
	\tabularnewline
	
	\textbf{Descrizione}: \\ Il capitolato\textsubscript{G} non pone vincoli sull'utilizzo delle tecnologie da adottare. Se da un lato questo permette libertà nell'implementazione, dall'altro può causare disorientamento in studenti con poca esperienza. Vista la novità del problema da trattare, le tecnologie da impiegare potranno risultare nuove per molti.
	& 
	\textbf{Rilevamento}: \\ Il responsabile si occuperà di censire le conoscenze e competenze dei membri del gruppo, al fine di individuare particolari lacune. I membri, qualora dovessero riscontrare difficoltà, lo comunicheranno al resto del gruppo. 	
	&  
	\textbf{Piano di contingenza}: \\ Dopo un'esplorazione generale delle tecnologie che si prestano a risolvere il problema richiesto, ci si confronterà con il proponente per confermare la bontà delle scelte adottate. I membri che hanno più esperienza guideranno lo studio di queste tecnologie.

\end{longtable}

\newpage

%------------------------------------RIS\_T \_ 2---------------------------------------


\renewcommand{\arraystretch}{1.5}
\rowcolors{2}{pari}{dispari}
\begin{longtable} { 
		>{\raggedright}p{0.33\textwidth} 
		>{\raggedright}p{0.33\textwidth} 
		>{\raggedright}p{0.33\textwidth}    }
	
	\caption{RIS\_T - 2} \endhead	
	
	
	\textbf{Nome}: \\ Malfunzionamento dei dispositivi
	& \textbf{Codice}: \\ RIS\_T - 2  
	& \textbf{Occorrenza}: Bassa \\ \textbf{Pericolosità}: Bassa
	
	\tabularnewline
	
	\textbf{Descrizione}: \\ I computer dei componenti del gruppo di lavoro possono andare incontro a guasti software o hardware. Questo può compromettere parte del lavoro svolto o rallentarne l'avanzamento.
	& 
	\textbf{Rilevamento}: \\ Il membro interessato dal guasto avviserà tempestivamente il gruppo se l'imprevisto dovesse causare difficoltà nel proseguimento del lavoro o se parte di esso fosse stato perso.
		
	&  
	\textbf{Piano di contingenza}: \\ \'E caldamente consigliato mantenere una copia di backup del lavoro in corso di svolgimento. L'interessato dal guasto si adopererà con urgenza a ripristinare il funzionamento del proprio dispositivo. Se non fosse possibile recuperare il lavoro svolto, esso verrà suddiviso tra i membri ed elaborato nuovamente.
	
\end{longtable}



\renewcommand{\arraystretch}{1.5}
\rowcolors{2}{pari}{dispari}
\begin{longtable} { 
		>{\raggedright}p{0.33\textwidth} 
		>{\raggedright}p{0.33\textwidth} 
		>{\raggedright}p{0.33\textwidth}    }
	
	\caption{RIS\_T - 3} \endhead	
	
	
	\textbf{Nome}: \\ Difficoltà nella compresione dei requisito\textsubscript{G}
	& \textbf{Codice}: \\ RIS\_T - 3  
	& \textbf{Occorrenza}: Alta \\ \textbf{Pericolosità}: Bassa
	
	\tabularnewline
	
	\textbf{Descrizione}: \\ Il documento di descrizione del capitolato\textsubscript{G} fornisce una definizione generale del problema ma risulta poco dettagliato in termini di requisito\textsubscript{G} da soddisfare. Inoltre il gruppo può trovare difficoltà nella comprensione dei requisito\textsubscript{G} di un prodotto di cui non si figura ancora l'implementazione.
	& 
	\textbf{Rilevamento}: \\ Il gruppo non riesce a definire con precisione quali requisito\textsubscript{G} siano richiesti dal prodotto e ha difficoltà a definire con precisione il comportamento degli attore\textsubscript{G} coinvolti.
	
	&  
	\textbf{Piano di contingenza}: \\ Verrà fissato un incontro sincrono con il proponente al fine di definire i requisito\textsubscript{G} del prodotto. La comunicazione continuerà in maniera asincrona per trovare conferma delle assunzioni fatte durante l'Analisi dei Requisiti.
	
\end{longtable}

\newpage

\subsection{Rischi organizzativi}

%------------------------------------RIS\_O \_ 1---------------------------------------

\renewcommand{\arraystretch}{1.5}
\rowcolors{2}{pari}{dispari}
\begin{longtable} { 
		>{\raggedright}p{0.33\textwidth} 
		>{\raggedright}p{0.33\textwidth} 
		>{\raggedright}p{0.33\textwidth}    }
	
	\caption{RIS\_O - 1} \endhead	
	
	
	\textbf{Nome}: \\ Organizzazione e preventivazione attivita\textsubscript{G}
	& \textbf{Codice}: \\ RIS\_O - 1
	& \textbf{Occorrenza}: Alta \\ \textbf{Pericolosità}: Media
	
	\tabularnewline
	
	\textbf{Descrizione}: \\ Preventivare le ore necessarie a svolgere le attivita\textsubscript{G} future è difficile se non si ha maturato esperienza nello sviluppo di progetto\textsubscript{G} complessi. \\
	\textbf{Rilevamento}: \\ I costi preventivati possono non corrispondere alle ore effettivamente spese, e accumulare ritardi può compromettere la buona riuscita del lavoro.
	& 
	\textbf{Piano di contingenza}: \\ L'organizzazione delle attivita\textsubscript{G} future sarà sufficientemente generale da permettere di essere raffinata quando le attivita\textsubscript{G} si fanno più prossime, e sufficientemente precisa da dettare le scadenze oltre le quali i ritardi diventano critici. Il Responsabile di Progetto guiderà il lavoro riferendosi costantemente al 
	&  
	cruscotto\textsubscript{G} di progetto\textsubscript{G}. Significativi discostamenti tra preventivo e costi effettivamente sostenuti verranno comunicati con tempestività al proponente.
	
\end{longtable}



\renewcommand{\arraystretch}{1.5}
\rowcolors{2}{pari}{dispari}
\begin{longtable} { 
		>{\raggedright}p{0.33\textwidth} 
		>{\raggedright}p{0.33\textwidth} 
		>{\raggedright}p{0.33\textwidth}    }
	
	\caption{RIS\_O - 2 } \endhead	
	
	
	\textbf{Nome}: \\ Impegni esterni 
	& \textbf{Codice}: \\ RIS\_O - 2
	& \textbf{Occorrenza}: Alta \\ \textbf{Pericolosità}: Media
	
	\tabularnewline
	
	\textbf{Descrizione}: \\ La disponibilità dei membri del gruppo nella partecipazione al lavoro risente degli impegni esterni a cui essi sono soggetti.
	
	\textbf{Rilevamento}: \\ I componenti potrebbero non essere presenti ai meeting o non svolgere i compiti assegnati entro le scadenze fissate.
	
	& 
	\textbf{Piano di contingenza}: \\ Ogni membro presterà massimo impegno a rispettare le scadenze fissate. Il componente interessato comunicherà gli impegni che eventualmente impediscono di rispettare le scadenze per i propri compiti. Il Responsabile di Progetto provvederà a distribuire le attivita\textsubscript{G} tra altri membri, in modo da non ritardare	
	& 
	l'avanzamento del lavoro. Gli incontri se possibile vengono fissati con discreto anticipo valutando la possibilità della partecipazione di tutti. 
	In caso di criticità non risolvibili internamente, verà interpellato il professor Vardanega Tullio.
	
	
\end{longtable}


\renewcommand{\arraystretch}{1.5}
\rowcolors{2}{pari}{dispari}
\begin{longtable} { 
		>{\raggedright}p{0.33\textwidth} 
		>{\raggedright}p{0.33\textwidth} 
		>{\raggedright}p{0.33\textwidth}    }
	
	\caption{RIS\_O - 3} \endhead	
	
	
	\textbf{Nome}: \\ Emergenza sanitaria
	& \textbf{Codice}: \\ RIS\_O - 3
	& \textbf{Occorrenza}: Media \\ \textbf{Pericolosità}: Media
	
	\tabularnewline
	
	\textbf{Descrizione}: \\ La situazione di emergenza attualmente in atto complica la gestione e l'organizzazione del lavoro. Non è possibile tenere incontri in presenza: le riunioni da remoto possono risultare meno efficaci per comunicare e curare i rapporti tra i membri del gruppo. Inoltre il difficile periodo\textsubscript{G} può impattare negativamente sulla produttività dei membri del gruppo.
	&
	\textbf{Rilevamento}: \\ La comunicazione risulta così complessa da causare incomprensioni importanti o rallentare in modo considerevole il lavoro. I membri sono interessati da problemi personali legati all'emergenza sanitaria che ostacolano lo svolgimento dei propri compiti.
	& 
	\textbf{Piano di contingenza}: \\ Verranno sperimentati vari strumenti di comunicazione sincrona per migliorare il più possibile la qualità degli incontri, pur riconoscendo i limiti di questi mezzi. I membri sono invitati a comunicare liberamente se eventuali problemi personali causati dalla situazione emergenziale hanno ostacolato l'adempimento dei propri compiti: essi verranno distribuiti dal Responsabile di Progetto per alleviare il carico di lavoro.

	
\end{longtable}



\subsection{Rischi interpersonali}



\renewcommand{\arraystretch}{1.5}
\rowcolors{2}{pari}{dispari}
\begin{longtable} { 
		>{\raggedright}p{0.33\textwidth} 
		>{\raggedright}p{0.33\textwidth} 
		>{\raggedright}p{0.33\textwidth}    }
	
	\caption{RIS\_I - 1} \endhead	
	
	
	\textbf{Nome}: \\ Divergenze tra membri del gruppo di lavoro
	& \textbf{Codice}: \\ RIS\_I - 1
	& \textbf{Occorrenza}: Media \\ \textbf{Pericolosità}: Media
	
	\tabularnewline
	
	\textbf{Descrizione}: \\ I membri del gruppo, se sottoposti a situazioni stressanti come può essere lo svolgimento di un lavoro impegnativo, potrebbero trovare difficoltà nella cooperazione e generare contrasti all'interno del team.
	\\
	
	\textbf{Rilevamento}: \\ Il lavoro subisce rallentamenti a causa di conflitti tra i componenti.
	& 
	\textbf{Piano di contingenza}: \\ Ogni membro del gruppo è tenuto a tenere un atteggiamento aperto al dialogo e al compromesso, conscio del fatto che la buona riuscita del progetto\textsubscript{G} è imprescindibile da una stretta collaborazione interna. Dopo ogni revisione\textsubscript{G} si terrà un'attività di verifica in cui ogni membro avrà l'opportunità di esporre 
	&  
	in modo costruttivo eventuali critiche nel lavoro o nel comportamento degli altri componenti. Se i conflitti risultassero impossibili da gestire internamente, verrà interpellato il professor Vardanega Tullio.
	
\end{longtable}











