\section{Consuntivo}

La sezione che segue espone le spese effettivamente sostenute, registrate al termine delle fase\textsubscript{G} di Avvio e di Analisi dei Requisti. In relazione alle spese preventivate, il periodo\textsubscript{G} chiuderà in:
\begin{itemize}
	\item \textbf{positivo} se il preventivo supera il consuntivo;
	\item \textbf{pari} se il preventivo e il consuntivo collimano;
	\item \textbf{negativo} se il consuntivo supera il preventivo.
\end{itemize}


\subsection{Avvio}

\begin{table}[H]
	\begin{center}
		\begin{tabular}{c
				!{\color[HTML]{9b240a}\vrule width 1pt}
				cccccc
				!{\color[HTML]{9b240a}\vrule width 1pt}	
				c}
			\rowcolorhead
			\headertitle{Nome} & \headertitle{R} & \headertitle{V} & \headertitle{An} & \headertitle{Am} & \headertitle{Pr} & \headertitle{Pt} & \headertitle{Tot} \\
			
			Chiarello Sofia & 1 & 0 & 4 & 3 & 0 & 0 & 8\\
			Crivellari Alberto & 2 & 0 & 1 & 2 & 0 & 0 & 5\\
			De Renzis Simone & 3 & 0 & 2 & 4 & 0 & 0 & 9\\
			Greggio Nicolò & 2 & 0 & 2 & 5 & 0 & 0 & 9\\
			Tessari Andrea & 2 & 0 & 1 & 3 & 0 & 0 & 6\\
			Zuccolo Giada & 1 & 0 & 4 & 2 & 0 & 0 & 7\\
		\end{tabular}
		\caption[Consuntivo fase\textsubscript{G} di Avvio]{Per ogni componente, le ore effettivamente spese nella fase\textsubscript{G} di Avvio}
	\end{center}
\end{table}




\subsection{Analisi dei requisiti}

\begin{table}[H]
	\begin{center}
		\begin{tabular}{c
				!{\color[HTML]{9b240a}\vrule width 1pt}
				cccccc
				!{\color[HTML]{9b240a}\vrule width 1pt}	
				c}
			\rowcolorhead
			\headertitle{Nome} & \headertitle{R} & \headertitle{V} & \headertitle{An} & \headertitle{Am} & \headertitle{Pr} & \headertitle{Pt} & \headertitle{Tot} \\
			
			Chiarello Sofia & 3 & 4 & 15 & 3 & 0 & 0 & 25\\
			Crivellari Alberto & 4 & 12 & 4 & 5 & 0 & 0 & 25\\
			De Renzis Simone & 12 & 4 & 5 & 5 & 0 & 0 & 26\\
			Greggio Nicolò & 5 & 2 & 5 & 25 & 0 & 0 & 37\\
			Tessari Andrea & 8 & 9 & 4 & 3 & 0 & 0 & 24\\
			Zuccolo Giada & 3 & 6 & 13 & 3 & 0 & 0 & 25\\
		\end{tabular}
		\caption[Consuntivo fase\textsubscript{G} di Analisi dei Requisiti]{Per ogni componente, le ore effettivamente spese nella fase\textsubscript{G} di Analisi dei Requisiti}
	\end{center}
\end{table}



\subsection{Totale}

\begin{table}[H]
	\centering
	\begin{tabular}{ccc}
		\rowcolorhead
		\headertitle{Ruolo} & \headertitle{Ore} & \headertitle{Costo(\euro{})}\\
		Responsabile & 46 (+7) & 1380 (+210) \\
		Verificatore & 37 (-18) & 555 (-270)\\
		Analista & 60 (-18) & 1500 (-450)\\				
		Amministratore & 63 (+28) & 1260 (+560)\\
		Programmatore & 0 (+0) & 0 (+0)\\
		Progettista & 0 (+0) & 0 (+0)\\
		\hline		
		\textbf{Totale consuntivo} & \textbf{206} & \textbf{4695}\\
		\textbf{Totale preventivo} & \textbf{207} & \textbf{4645}\\
		\textbf{Differenza} & \textbf{-1} & \textbf{50}\\
	\end{tabular}
	\caption[Confronto tra preventivo e consuntivo]{Per ogni ruolo, il totale delle ore effettivamente impiegate, con lo scostamento dal preventivo}
\end{table}


\subsection{Preventivo a finire}
Il preventivo a finire comprende i costi consuntivi di tutte le attivita\textsubscript{G} terminate più i costi previsti per le attivita\textsubscript{G} da eseguire. Consistendo il periodo\textsubscript{G} di attivita\textsubscript{G} non rendicontate, il preventivo a finire coinciderà con il preventivo presentato nella sezione \S 4.



\subsection{Conclusioni}

Il periodo\textsubscript{G} si chiude in \textbf{negativo}, costringendo il gruppo ad una spesa supplementare di \textbf{50 \euro{}}. Nonostante le ore spese siano inferiori a quelle preventivate, sono osservabili rilevanti discostamenti nella distribuzione delle stesse tra i ruoli. Questo è valido in particolare per i ruoli di:
\begin{itemize}
	\item \textbf{Verificatore}: la verifica della documentazione si è svolta in maniera snella e senza inconvenienti;
	\item \textbf{Analista}: nonostante il periodo\textsubscript{G} prevedesse ampie attivita\textsubscript{G} di analisi, i lavori si sono svolti con più velocità del previsto;
	\item \textbf{Amministratore}: questo ruolo ha richiesto più tempo di quanto preventivato, in particolare nella messa a punto di attivita\textsubscript{G} di automatizzazione come nel caso del \textsc{Glossario}.
\end{itemize}
Le osservazioni ricavate da questo periodo\textsubscript{G} verranno tenute in considerazione nello svolgimento delle prossime fase\textsubscript{G}, nelle corso delle quali si valuterà se attuare delle correzioni al preventivo presentato.
