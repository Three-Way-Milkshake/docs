PoC - Incremento 1.A & Prima implementazione di studio per la comunicazione via socket tra le componenti in Node.js e il server in Java. & 25 & Programmatore
\tabularnewline 
PoC - Incremento 1.B & Simulazione della mappa che visualizza le unità sulla base della posizione inviata dalle stesse. & 15 & Programmatore
\tabularnewline 
PoC - Incremento 1.C & Definizione semplificata di un algoritmo per la ricerca del miglior percorso per raggiungere la destinazione. & 5 & Programmatore
\tabularnewline 
PoC - Incremento 1.D & Ideazione e implementazione di sistema base per la rilevazione e gestione delle collisioni tra le unità. & 25 & Programmatore
\tabularnewline 
PoC - Incremento 2 & Realizzazione del backend dell'unità in Node.js e collegamento con il server. & 5 & Programmatore
\tabularnewline 
PoC - Incremento 3.A & Presentazione della mappa del magazzino tramite interfaccia realizzata in Angular.js. & 20 & Programmatore
\tabularnewline 
PoC - Incremento 3.B & Aggiunta pannello di guida dell'unità nell'interfaccia grafica, e finestra di visualizzazione delle task\textsubscript{G} delle unità & 5 & Programmatore
\tabularnewline 
PoC - Incremento 4 & Collegamento tra server, frontend e unità. & 20 & Programmatore
\tabularnewline 
Lettera di Presentazione & Avviene la stesura della lettera con cui il gruppo si candida alla Revisione di Progettazione. & 1 & Responsabile
\tabularnewline 
\caption{Pianificazione di periodo\textsubscript{G} - Progettazione Architetturale - Periodo 2}