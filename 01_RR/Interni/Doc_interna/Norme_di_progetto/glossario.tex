\usepackage[toc,acronym]{glossaries}
\makeglossaries

\newglossaryentry{wiki}{name={wiki}, plural={wikis},%
	description={Termine di origine hawaiana che significa veloce, con cui si identifica un tipo di sito internet che permette la creazione e la modifica di pagine multimediali attraverso un'interfaccia semplice}}

\newglossaryentry{usecase}{name={caso d'uso}, plural={casi d'uso},%
	description={Un caso d'uso è un insieme di scenari (sequenze di azioni) che hanno in comune uno scopo finale (obiettivo) per un utente (attore)}}

\newglossaryentry{techbase}{name={technology baseline},%
	description={Motiva le tecnologie, i framework, e le librerie selezionate per la realizzazione del prodotto. Ne dimostra adeguatezza e fattibilità, tramite un proof of concept coerente con gli obiettivi}}

\newglossaryentry{stakeholder}{name={stakeholder}, plural={stakeholders},%
	description={Tutti coloro che a vario titolo hanno influenza sul prodotto e sul progetto}}

\newglossaryentry{sistematico}{name={sistematico},%
	description={Modo di lavorare metodico e rigoroso, che conosce, usa ed evolve le best practice di dominio}}

\newglossaryentry{security}{name={security},%
	description={Non vulnerabilità rispetto a intrusioni}}

\newglossaryentry{safety}{name={safety},%
	description={Sicurezza rispetto a malfunzionamenti}}

\newglossaryentry{repository}{name={repository},%
	description={Posizione di archiviazione per pacchetti software}}

\newglossaryentry{quantificabile}{name={quantificabile},%
	description={Che permette di misurare l’efficienza e l’efficacia del suo agire}}

\newglossaryentry{proofconcept}{name={proof of concept},%
	description={Dimostratore eseguibile. Il suo codice può (ma non deve) essere usa-e-getta}}

\newglossaryentry{prodbase}{name={product baseline},%
	description={G}}

\newglossaryentry{precondizione}{name={precondizione}, plural={precondizioni},%
	description={Condizioni che devono essere soddisfatte perché si verifichino determinati eventi successivi}}

\newglossaryentry{postcondizione}{name={postcondizione}, plural={postcondizioni},%
	description={Condizioni che devono verificarsi dopo determinati eventi}}

\newglossaryentry{gitpush}{name={push},%
	description={Comando di \gls{git} che permette di caricare le proprie modifiche locali sul server remoto condiviso del repository}}

\newglossaryentry{gitpull}{name={pull},%
	description={Comando di \gls{git} che permette di aggiornare il proprio \acrshort{repo} locale con i cambiamenti remoti}}

\newglossaryentry{gitcommit}{name={commit}, plural={commits},%
	description={Comando di \gls{git} che permette di salvare e versionare le modifiche attuate ai file in locale}}

\newglossaryentry{git}{name={git},%
	description={Sistema di controllo del versionamento distribuito}}

\newglossaryentry{efficienza}{name={efficienza},%
	description={Misura dell'abilità di raggiungere l'obiettivo impiegando le risorse minime indispensabili}}

\newglossaryentry{efficacia}{name={efficacia},%
	description={Misura della capacità di raggiungere l'obiettivo prefissato}}

\newglossaryentry{disciplinato}{name={disciplinato},%
	description={Che segue le regole che si è dato}}

\newglossaryentry{designpattern}{name={design pattern}, plural={design patterns},%
	description={Una soluzione progettuale generale ad un problema ricorrente}}

\newglossaryentry{bestpractice}{name={best practice}, plural={best practices},%
	description={Modo di fare noto, che abbia mostrato di garantire i migliori risultati in circostanze note e specifiche}}

\newglossaryentry{bash}{name={bash},%
	description={È una shell Unix ed un linguaggio interpretato scritto da Brian Fox per il progetto GNU come sostituto del software gratuito per la shell Bourne}}

\newacronym{vcs}{VCS}{Version Control System}

\newacronym{uml}{UML}{Unified Modeling Language}

\newacronym{tb}{TB}{\gls{techbase}}

\newacronym{repo}{repo}{\gls{repository}}

\newacronym{ram}{RAM}{Random Access Memory}

\newacronym{poc}{PoC}{Proof of Concept}

\newacronym{pb}{PB}{\gls{prodbase}}

\newacronym{jit}{JiT}{Just in Time}

\newacronym{eg}{e.g.}{Example given, si usa per indicare che ciò che segue è un esempio}

\newacronym{cpu}{CPU}{Central Processing Unit}

