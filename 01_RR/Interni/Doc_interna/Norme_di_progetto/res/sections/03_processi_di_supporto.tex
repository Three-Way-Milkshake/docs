\section{Processi di Supporto}
\label{supporto}
\subsection{Documentazione}
    \subsubsection{Scopo}
        Lo scopo del processo di documentazione è regolamentare la creazione e la gestione dei documenti e fissare le modalità di stesura ed approvazione degli stessi.
    \subsubsection{Aspettative}
        Avere un approccio condiviso ed uniforme per la stesura e l'aggiornamento dei documenti all'interno del gruppo di lavoro è fondamentale per rendere la documentazione uno strumento costruttivo e di supporto, e non un mera formalità aggiuntiva.
        Inoltre fornire un aspetto uniforme attraverso tutti i documenti facilita qualunque lettore.
    \subsubsection{Descrizione}
        Il gruppo \group si doterà di due categorie di documentazione:
        \begin{itemize}
            \item \textbf{formale: }documenti interni o esterni che rispetteranno strettamente i vincoli descritti in seguito e che saranno interamente pubblici, realizzati in \LaTeX{} aderendo ad un template condiviso;
            \item \textbf{informale: }documenti interni che potranno svolgere diverse funzioni, tra cui:
            \begin{itemize}
                \item  raccolta appunti e ordini del giorno per riunioni;
                \item  raccolta argomenti delle discussioni delle riunioni, per tracciare l'evoluzione delle stesse e favorire la stesura dei verbali in seguito;
                \item  creazione di wiki per condivisione di materiale utile riguardo l'uso di tecnologie o strumenti a supporto di qualsiasi attività.
            \end{itemize}
            Questi documenti saranno realizzati sfruttando \href{https://www.atlassian.com/software/confluence}{Confluence} per garantire semplicità, accentramento e condivisione real-time.

        \end{itemize}
    \subsubsection{Ciclo di vita dei documenti}
        Ogni documento formale si redige ed incrementa tramite queste attività:
        \begin{itemize}
            \item \textbf{stesura: }la scrittura del documento in sé, riguarda sia la creazione di nuove parti che l'aggiornamento di queste. Uno o più redattori si occupano di ciò;
            \item \textbf{verifica: }eseguita da uno o più verificatori, necessariamente diversi dai redattori, consiste nel controllo della correttezza sintattica, semantica, grammaticale ed ortografica e della conformità del documento rispetto alle suo scopo. Nel caso in cui si rendano necessarie modifiche sostanziali, i verificatori notificheranno il responsabile che provvederà a riportare il documento in stesura e solleciterà i redattori affinché apportino le correzioni richieste;
            \item \textbf{approvazione: }quando i verificatori riporteranno la completa correttezza ed aderenza ai requisiti del documento, il responsabile provvederà all'approvazione finale ed al rilascio di una nuova versione dello stesso.
        \end{itemize}
        Adottando un approccio incrementale, queste attività possono ripetersi.
    \subsubsection{Struttura dei documenti}


    \paragraph{singole attività...}
    \subsubsection{Strumenti}
    ...

\subsection{Versionamento} %TODO corretto averli qui o meglio in processi organizzativi?
    \subsubsection{Scopo}
    Lo scopo del processo di fornitura è di determinare
    \subsubsection{Aspettative}
    ...
    \subsubsection{Descrizione}
    ...
    \subsubsection{Attività}
    ...
    \paragraph{singole attività...}
    \subsubsection{Strumenti}
    ...

\subsection{Verifica}
\label{verifica}
    \subsubsection{Scopo}
    Lo scopo del processo di fornitura è di determinare
    \subsubsection{Aspettative}
    ...
    \subsubsection{Descrizione}
    ...
    \subsubsection{Attività}
    ...
    \paragraph{singole attività...}
    \subsubsection{Strumenti}
    ...

\subsection{Validazione}
    \subsubsection{Scopo}
    Lo scopo del processo di fornitura è di determinare
    \subsubsection{Aspettative}
    ...
    \subsubsection{Descrizione}
    ...
    \subsubsection{Attività}
    ...
    \paragraph{singole attività...}
    \subsubsection{Strumenti}
    ...