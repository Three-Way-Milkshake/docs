\section{Processi primari}
\subsection{Fornitura}
    \subsubsection{Scopo}
        Il processo di fornitura sostanzialmente si occupa della gestione dei rapporti con il cliente.
        Il suo scopo quindi è quello di determinare strumenti e competenze utili e necessarie alla realizzazione del prodotto e di assicurare la conformità di questo con le richieste del proponente. Si rende necessaria quindi la produzione di documenti che descrivano le intenzioni e le modalità che il gruppo si prefigge di seguire al fine di soddisfare il cliente.
    \subsubsection{Aspettative}
        Il confronto diretto e frequente tra fornitore e proponente è senza dubbio utile ad entrambe le parti, affinché ambedue soddisfino i loro obiettivi in tempi desiderabili.
    \subsubsection{Descrizione}
        % todo breve desc
        Il processo di fornitura si compone \hfoot{Standard ISO 12207 \S\ 5.2} di 7 attività, definite come segue:
    \subsubsection{Attività}
        \pparagraph{Inizializzazione}
            Il gruppo dovrà effettuare collettivamente una valutazione di tutti i capitolati proposti e formalizzarla in uno \textsc{studio di fattibilità}, il quale, per ogni capitolato, darà una breve descrizione dello stesso, delle finalità, delle tecnologie, degli aspetti positivi e delle criticità, proponendo poi delle conclusioni che saranno il frutto delle riflessioni interne ed indicando la scelta definitiva dei membri.

        \pparagraph{Preparazione della risposta}
            Il gruppo preparerà una lettera di presentazione indirizzata al committente ed al proponente del capitolato scelto, per candidarsi alla fornitura del prodotto indicando un sunto del preventivo dei costi.

        \pparagraph{Contratto}
            %5.2.3 Contract. This activity consists of the following tasks:
            %5.2.3.1 The supplier shall negotiate and enter into a contract with the acquire organization to provide the software product or service.
            %5.2.3.2 The supplier may request modification to the contract as part of the change control mechanism.

        \pparagraph{Pianificazione}
        \label{pianificazione}
            Il gruppo dovrà fornire dei documenti che illustrino la gestione del lavoro e mostrino come verranno assicurati qualità e conformità del prodotto. Nello specifico realizzerà:
            \begin{itemize}
                \item un \textsc{piano di progetto}, contenente \footnote{\url{https://www.math.unipd.it/~tullio/IS-1/2020/Dispense/FC2.pdf} \S\ 3}:
                    \begin{itemize}
                        \item \textbf{pianificazione macroscopica (a lungo periodo):}
                            \subitem -- scadenze, fissate all'indietro;
                            \subitem -- analisi dei rischi;
                            \subitem -- preventivo dei costi.
                        \item \textbf{pianificazione dettagliata (a breve):}
                            \subitem -- attività, fissate in avanti;
                            \subitem -- preventivo minuto, alla luce del consuntivo di periodo precedente;
                            \subitem -- riscontro dei rischi ed aggiornamento delle misure di mitigazione.
                    \end{itemize}

                    che andrà così strutturato: \footnote{\url{https://www.math.unipd.it/~tullio/IS-1/2020/Dispense/L06.pdf} \S\ 25}:
                    \subitem -- Introduzione (scopo e struttura);
                    \subitem -- Organizzazione del progetto;
                    \subitem -- Analisi dei Rischi;
                    \subitem -- Risorse disponibili (tempo e persone);
                    \subitem -- Suddivisione del lavoro (work breakdown);
                    \subitem -- Calendario delle attività;
                    \subitem -- Meccanismi di controllo e di rendicontazione.

                \item un \textsc{piano di qualifica}, contenente \footnote{\url{https://www.math.unipd.it/~tullio/IS-1/2020/Dispense/FC2.pdf} \S\ 4}:
                    \subitem -- obiettivi quantitativi di qualità;
                    \subitem -- cruscotto di misurazione;
                    \subitem -- analisi degli scostamenti e misure correttive.
            \end{itemize}

        \pparagraph{Esecuzione e controllo}
            In questa attività il gruppo \group dovrà implementare ed eseguire i piani delineati al punto \ref{pianificazione} e sviluppare il prodotto in accordo con il \hyperref[sviluppo]{processo di sviluppo}.

        \pparagraph{Revisione e valutazione}
            Il gruppo dovrà coordinare le revisioni delle attività svolte e gestire la comunicazione con il committente ed il proponente. Si dovrà inoltre aver cura di operare in accordo con quanto scritto negli altri processi.


        \pparagraph{Consegna e completamento}
            Il fornitore dovrà consegnare il prodotto in accordo con quanto specificato nel contratto. A seguito della consegna del prodotto, il gruppo \group non si farà carico delle mansioni di supporto ed assistenza.
    \subsubsection{Strumenti}
        %todo rimuovere visto che niente in più?
        Oltre a quelli definiti nei processi di \hyperref[supporto]{supporto} e \hyperref[organizzativi]{organizzativi} non è stata individuata la necessità di ulteriori particolari strumenti.
        % todo rivedere se vero poi che non si usa altro

\subsection{Sviluppo}
\label{sviluppo}
    \subsubsection{Scopo}
        Il processo di sviluppo comprende tutte quelle attività che portano alla costruzione del prodotto finale.
    \subsubsection{Aspettative}
        Questo processo dev'essere attuato secondo quanto pattuito con proponente e committente, rispettando i loro requisiti e soddisfacendo le loro aspettative. Tutto ciò naturalmente rispettando le norme definite in questo documento
    \subsubsection{Descrizione}
        %todo scrivere quante attività descriviamo
        Nel processo di sviluppo si individuano \hfootiso{5.3} diverse attività, riassumibili come segue.

    \subsubsection{Attività}
        \paragraph{Analisi dei requisiti}
            \ssubparagraph{Descrizione}
                Gli analisti devono stabilire, raccogliere e documentare tutti i requisiti stilando un documento che fornirà una base precisa su cui i progettisti si potranno fondare. Dovrà contenere, in accordo con quanto richiesto dal cliente, la raccolta dei casi d'uso, rappresentati anche tramite diagrammi UML, ed il tracciamento di tutti i requisiti individuati

            \ssubparagraph{Nomenclatura casi d'uso}
                Ogni caso d'uso è univocamente identificato da un codice, secondo lo schema:
                $$\text{UC[caso].[sottoCaso].[sottoSottoCaso]}$$ dove caso, sottoCaso e sottoSottoCaso sono numeri progressivi che partono da 1. Segue poi

                ..standard UML 2.0 \footnote{\url{https://www.omg.org/spec/UML/2.0/Superstructure/PDF} \S\ 16} \footnote{\url{https://www.math.unipd.it/~rcardin/swea/2021/Diagrammi\%20Use\%20Case_4x4.pdf}}

        \paragraph{Progettazione}
            ...
        \paragraph{Codifica}
            ...
    \subsubsection{Strumenti}
        ...