\section{Processi primari}
\subsection{Fornitura}
    \subsubsection{Scopo}
        Il processo di fornitura sostanzialmente si occupa della gestione dei rapporti con il cliente.
        Il suo scopo quindi è quello di determinare strumenti e competenze utili e necessarie alla realizzazione del prodotto e di assicurare la conformità di questo con le richieste del proponente. Si rende necessaria quindi la produzione di documenti che descrivano le intenzioni e le modalità che il gruppo si prefigge di seguire al fine di soddisfare il cliente.
    \subsubsection{Aspettative}
        Il confronto diretto e frequente tra fornitore e proponente è senza dubbio utile ad entrambe le parti, affinché ambedue soddisfino i loro obiettivi in tempi desiderabili.
    \subsubsection{Descrizione}
        % todo breve desc
        Il processo di fornitura si compone \footnote{Standard ISO 12207, \hyperref[ref]{vedi riferimenti}}
    \subsubsection{Attività}
        \pparagraph{Inizializzazione}
            Il gruppo dovrà effettuare collettivamente una valutazione di tutti i capitolati proposti e formalizzarla in uno studio di fattibilità, il quale, per ogni capitolato, darà una breve descrizione dello stesso, delle finalità, delle tecnologie, degli aspetti positivi e delle criticità, proponendo poi delle conclusioni che saranno il frutto delle riflessioni interne ed indicando la scelta definitiva dei membri.

        \pparagraph{Presentazione}
            Il gruppo preparerà una lettera di presentazione indirizzata al committente ed al proponente del capitolato scelto, per candidarsi alla fornitura del prodotto indicando un preventivo dei costi.

        \pparagraph{Pianificazione}

            \subparagraph{Piano di Qualifica}
                ...
        \pparagraph{Collaudo e Consegna del Prodotto}
            ...
            A seguito della consegna del prodotto, il gruppo \group
    \subsubsection{Strumenti}
        ...

\subsection{Sviluppo}
    \subsubsection{Scopo}
        Lo scopo del processo di ...
    \subsubsection{Aspettative}
        ...
    \subsubsection{Descrizione}
        ...
    \subsubsection{Attività}
        ...
        \paragraph{Analisi dei Requisiti}
            ...
        \paragraph{Progettazione}
            ...
        \paragraph{Codifica}
            ...
    \subsubsection{Strumenti}
        ...