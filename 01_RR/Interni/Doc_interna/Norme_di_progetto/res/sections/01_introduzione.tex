\section{Introduzione}
\subsection{Scopo del documento}
    Questo documento ha lo scopo di fissare e definire tutte le regole, convenzioni e buone pratiche utili a formare un way of working condiviso alla base da tutti i componenti del gruppo per assicurare una collaborazione efficiente ed efficace. Si discuteranno inoltre i vari strumenti che verranno adottati per facilitare lo sviluppo del progetto e per promuovere un'organizzazione adeguata.
    Si ritiene inoltre che la definizione ed il mantenimento per incremento di un documento condiviso all'interno del gruppo di lavoro, che definisca e raccolga quanto descritto in maniera formale e centralizzata, possa favorire, in un contesto dove i membri possano variare, l'inserimento di nuovi componenti facilitandone l'ambientamento. Pur non essendo questo il contesto di lavoro attuale, è comunque una buona pratica da sperimentare e consolidare.

\subsection{Scopo del prodotto}
    % TODO add descrizione condivisa C5
    scopodelprodotto

\subsection{Termini, Abbreviazioni ed altri Documenti}
    Tutti i termini che necessitano di una spiegazione, per fornire un'adeguata comprensione, o perché possono causare ambiguità nel contesto, sono definiti nel glossario alla fine del documento. Ogni occorrenza di questi collega alla voce corrispondente. Analogamente vale lo stesso discorso per le abbreviazioni e gli acronimi. Le definizioni delle voci nel glossario e nella lista degli acronimi presentano inoltre collegamenti alle pagine dove vengono utilizzati, il che permette una comoda navigazione bidirezionale tra termini e significati corrispondenti. Le voci di glossario saranno seguite da una G pedice mentre gli acronimi da una A (e.g.: voce di glossario\textsubscript{G} ; acronimo\textsubscript{A}).
    Inoltre quando si farà riferimento ad un altro documento, il nome di questo sarà in maiuscoletto (e.g.: \textsc{esempio nome documento}).

\subsection{Riferimenti}
\label{ref}
    % todo ? dividere in normativi informativi
    %\subsubsection{Riferimenti Normativi}
    %\subsubsection{Riferimenti Informativi}
        %TODO add links a riferimenti
        \begin{itemize}
            \item Standard ISO 12207:  \url{https://www.math.unipd.it/~tullio/IS-1/2009/Approfondimenti/ISO_12207-1995.pdf}
            \item \url{https://www.math.unipd.it/~tullio/IS-1/2020/Dispense/L03.pdf}
            \item \url{https://www.math.unipd.it/~tullio/IS-1/2020/Dispense/L06.pdf}
            \item \url{https://www.math.unipd.it/~tullio/IS-1/2020/Dispense/FC2.pdf}
            \item \url{https://www.omg.org/spec/UML/2.0/Superstructure/PDF}
            \item \url{https://www.math.unipd.it/~rcardin/swea/2021/Diagrammi\%20Use\%20Case_4x4.pdf}
            \item \url{https://www.math.unipd.it/~rcardin/swea/2021/SOLID\%20Principles\%20of\%20Object-Oriented\%20Design_4x4.pdf}
            \item \url{https://www.math.unipd.it/~tullio/IS-1/2020/Dispense/L09.pdf}

        \end{itemize}

\pagebreak