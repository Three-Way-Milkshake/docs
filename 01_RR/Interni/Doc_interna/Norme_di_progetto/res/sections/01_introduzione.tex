\section{Introduzione}
\subsection{Scopo del documento}
    Questo documento ha lo scopo di fissare e definire tutte le regole, convenzioni e buone pratiche utili a formare un way of working condiviso alla base da tutti i componenti del gruppo per assicurare una collaborazione efficiente ed efficace. Si discuteranno inoltre i vari strumenti che verranno adottati per facilitare lo sviluppo del progetto e per promuovere un'organizzazione adeguata.

\subsection{Scopo del prodotto}
    % TODO add descrizione condivisa C5
    scopodelprodotto

\subsection{Termini, Abbreviazioni ed altri Documenti}
    Tutti i termini che necessitano di una spiegazione, per fornire un'adeguata comprensione, o perché possono causare ambiguità nel contesto, sono definiti nel glossario alla fine del documento. Ogni occorrenza di questi collega alla voce corrispondente. Analogamente vale lo stesso discorso per le abbreviazioni e gli acronimi. Le definizioni delle voci nel glossario e nella lista degli acronimi presentano inoltre collegamenti alle pagine dove vengono utilizzati, il che permette una comoda navigazione bidirezionale tra termini e significati corrispondenti.
    Inoltre quando si farà riferimento ad un altro documento, il nome di questo sarà in maiuscoletto (e.g.: \textsc{})

\subsection{Riferimenti}
    \subsubsection{Riferimenti Normativi}
        %TODO add links a riferimenti
        \begin{itemize}
            \item Qui vengono elencati i rif norm
        \end{itemize}
    \subsubsection{Riferimenti Informativi}
        %TODO add riferimenti
        \begin{itemize}
            \item Qui vengono elencati i rif info
        \end{itemize}

\pagebreak