\subsection{Capitolato C2 - Emporio Lambda}


\subsubsection{Informazioni generali}

\begin{itemize}
	\item{\textbf{Nome:}} Emporio Lambda
	\item{\textbf{Proponente:}} Red Babel
	\item{\textbf{Committente:}} Prof. Tullio Vardanega, Prof. Riccardo Cardin
\end{itemize}



\subsubsection{Descrizione del capitolato}

L'idea alla base di Emporio Lambda è quella di costruire una piattaforma di e-commerce che si basi interamente su tecnologie serverless.


\subsubsection{Finalità del progetto}

Il prodotto in questione dovrà fornire due insiemi di funzionalità principali: uno orientato ai clienti (pagina principale, lista e descrizione prodotti, carrello degli acquisti, pagamento, gestione account) e l'altro alle funzionalità di back office (tutto ciò che può servire agli impiegati, ossia gestione di contabilità, inventario, ordini, giacenza, distribuzione, spedizioni...).
Il tutto eseguito tramite architetture serverless che possono incorporare BaaS (Backend as a Service) di terze parti, o che contengono codice proprietario eseguito su container effimeri (che hanno una durata limitata ad una singola invocazione) su piattaforme Faas (Functions as a Service). A ciò si deve aggiungere un approccio SPA (Single Page Applications). Tramite tutte queste caratteristiche elencate, il sistema nel suo complesso dovrebbe beneficiare di una riduzione in termini di costi di operazioni, complessità e tempi di consegna.


\subsubsection{Tecnologie interessate}

Il linguaggio principale che dev'essere adottato è Typescript. Segue un elenco di altre tecnoglogie a supporto, obbligatorie o consigliate dal proponente:
\begin{itemize}
    \item \textbf{AWS Lambda:} piattaforma di calcolo serverless basata su eventi, parte della suite di servizi web forniti da Amazon;
    \item \textbf{CloudFormation:} strumento consigliato per il rilascio di infrastrutture AWS;
    \item \textbf{Serverless Framework:} framework Node.js da utilizzare per la parte back end;
    \item \textbf{Next.js:} framework da usare per la parte front end;
    \item \textbf{CloudWatch o Datadog:} per implementare il sistema di monitoring;
    \item \textbf{Contentful:} CMS suggerito per l'implementazione di una parte opzionale.
\end{itemize}


\subsubsection{Aspetti positivi}

Emporio Lambda rappresenta sicuramente una realtà familiare a tutti, in quanto sono sempre più numerose e diffuse le piattaforme per gli acquisti online. Inoltre offre l'opportunità di lavorare con tecnologie nuove ed adottando paradigmi diversi a quelli tradizionali, potrebbe sicuramente fornire delle ottime opportunità di apprendimento.


\subsubsection{Criticità}

I requisiti imposti sembrano essere molto vincolanti, e le possibilità di scelte libere attuabili in fase di sviluppo appaiono limitate. Visto così superficialmente e solo come idea, non suscita molto interesse nè sembra portare margine di apprendimento vasto come altre offerte.


\subsubsection{Conclusioni}

Questo capitolato non ha suscitato interesse nel gruppo fin dai primi momenti. Non essendoci inoltre stato un seminario di approfondimento, non è stato possibile rivedere in chiave diversa le tematiche coinvolte, dunque l'opinione generale interna è rimasta invariata. Per cui il gruppo si orienta verso un'altra scelta.




