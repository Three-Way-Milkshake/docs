\section{Capitolato C5 - Portacs}
\subsection{Informazioni generali}
	\begin{itemize}
	\item \textbf{Nome:} \textit{Portacs;}
	\item \textbf{Proponente:} \textit{San Marco Informatica;}
	\item \textbf{Committente:}  \textit{Prof. Tullio Vardanega e Prof. Riccardo Cardin;}
	\end{itemize}
\subsection{Descrizione del capitolato}
Questo capitolato si concentra sulla realizzazione di un software che coordini lo spostamento di varie unità in una determinata griglia di movimento.
Ogni unità (che può rappresentare un robot, un muletto o un'automobile) ha un punto di partenza nella griglia, una velocità massima e una lista di punti denominati "Points Of Interest" (POI) che deve raggiungere.\\
La scacchiera ha i seguenti vincoli:
\begin{itemize}
	\item Percorsi definiti e con possibilità di percorrenze in parallelo e/o sensi unici;
	\item Definizione dei POI.
\end{itemize}
Le varie unità avranno:
\begin{itemize}
	\item Identificativo di sistema;
	\item Velocità massima;
	\item Posizione iniziale;
	\item Lista ordinata dei POI da attraversare.
\end{itemize}

\subsection{Finalità del progetto}
Tramite la realizzazione di questo capitolato, si acquisiranno le seguenti competenze in diversi ambiti:
\begin{itemize}
	\item Real time monitoring \& analysis;
	\item Predictivity e real time decision making;
	\item Introduzione alle problematiche del mondo della logistica e ottimizzazione delle performance nelle consegne in magazzini.
\end{itemize}
Inoltre il progetto è finalizzato alla consegna dei seguenti documenti:
\begin{itemize}
	\item Codice sorgente di quanto realizzato
	\item Docker file con la componente applicativa, rappresentante il motore di calcolo;
	\item Docker file con la componente applicativa, rappresentante il visualizzatore/monitor real time;
	\item Docker file, da istanziare N volte, rappresentante la singola unità;
	\item Docker file, da istanziare N volte, rappresentante il singolo pedone (facoltativo).
\end{itemize}
\subsection{Tecnologie interessate}
Le tecnologie interessate sono:
\begin{itemize}
	\item \textbf{Diagrammi UML};
	\item \textbf{Github} o \textbf{Bitbucket};
	\item \textbf{Docker}.
\end{itemize}
\subsection{Aspetti positivi}
L'azienda mette a disposizione figure di diverso livello in modo tale da poter coprire nella maniera più appropriata le esigenze degli studenti.
Inoltre il codice prodotto sarà reso disponibile al pubblico con licenza libera su Github o BitBucket alla fine del progetto.
\subsection{Criticità e fattori di rischio}
Può risultare difficoltoso individuare un algoritmo che risolvi il problema del Path Finding.
\subsection{Conclusioni}
La disponibilità dell'azienda assieme alla curiosità scaturita dopo il seminario offerto hanno favorito la scelta di questo capitolato.
Inoltre questo progetto aiuta a mettere in pratica le proprie conoscenze di ricerca operativa ed è infine molto utile acquisire competenze nel mondo Docker.