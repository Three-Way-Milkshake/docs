\documentclass{article}
\usepackage[utf8]{inputenc}

\title{studio di fattibilita c3}
\author{crivealby }
\date{December 2020}

\begin{document}

\maketitle

\section{Introduction}
\pagebreak
\subsection{Capitolato C3 - GDP - Gathering Detection Platform}


\subsubsection{Informazioni generali}

\begin{itemize}
	\item{\textbf{Nome:}} GDP - Gathering Detection Platform
	\item{\textbf{Proponente:}} SyncLab
	\item{\textbf{Committente:}} Prof. Tullio Vardanega, Prof. Riccardo Cardin
\end{itemize}



\subsubsection{Descrizione del capitolato}

Il software GDP (Gathering Detection Platform) consiste in una piattaforma che rappresenta mediante visualizzazione grafica zone potenzialmente a rischio di assembramento e previrne di nuove, attraverso sensoristica e varie sorgenti dati.


\subsubsection{Finalità del progetto}

Il software finale prevede l'acquisizione di informazione da sensoristica e altre sorgenti:
\begin{itemize}
	\item{\textbf{Sensoristica:}}
	    \begin{itemize}
	        \item telecamere
	        \item dispositivi contapersone
	        \item etc.
	    \end{itemize}
	\item{\textbf{Sorgenti varie e eterogenee:}}
	    \begin{itemize}
	        \item flussi di prenotazioni Uber
	        \item orari dei mezzi di trasporto con capienze medie per corsia (autobus, metro, treno)
	        \item etc.
	\end{itemize}
\end{itemize}
Gli utilizzatori del software potranno vedere una rappresentazione delle zone a rischio (attuali o possibilmente in futuro), attraverso heat-map.
Attraverso heat-map possono accedere alla situazione globale dei vari flussi:
\begin{itemize}
    \item in tempo reale, con bassa latenza
    \item flussi previsti in futuro.
    \item flussi vecchi raccolti e storicizzati.
\end{itemize}


\subsubsection{Tecnologie interessate}

\begin{itemize}
	\item{\textbf{Java, Angular:}} linguaggi suggeriti dal proponente per lo sviluppo del server back-end e della componente Web Application del sistema;
	\item{\textbf{\textit{framework} Leaflet:}} framework da utilizzare per la gestione delle mappe, ad esempio heatmap;
	\item{\textbf{protocolli asincroni:}} per la comunicazione tra le varie componenti;  
	\item{\textbf{\textit{pattern} Publisher/Subscriber e protocollo MQTT:}}\newline (Message Queue Telemetry Transport), consigliato per essere open, di facile implementazione, di ampia diffusione in applicazioni IoT e M2M.
\end{itemize}

Inoltre il proponente specifica i seguenti requisiti minimi:
\begin{itemize}
	\item \textbf{\textit{responsive}} Sempre risposta a una richiesta di servizio (anche in caso di guasto);
	\item \textbf{\textit{resilient}} Servizi ripristinabili a seguito di guasti;
	\item \textbf{\textit{elastic}} Servizi scalati in base alla domanda;
	\item \textbf{\textit{message-driven}} Servizi devono rispondere al mondo, non controllarlo.
\end{itemize}
Un sistema che soddisfa questi requisiti è definito un Sistema Reattivo.
	
	
\subsubsection{Aspetti positivi}

Il progetto si pone in un contesto molto attuale, di grande interesse del mercato e generale.
Anche le tecnologie impiegate sono molto attuali e valide, il cui studio porterebbe a competenze molto utili in ambito lavorativo.



\subsubsection{Criticità}

Nonostante il capitolato abbia riscosso un buon interesse da parte del gruppo, le tecnologie da impiegare, in quanto alcune nuove e interessanti rappresentano una difficoltà.
In particolare le criticità principali sono:
\begin{itemize}
    \item \textbf{Leaflet:} La tecnologia relativa alle heat-map e framework Leaflet;
    \item \textbf{Linguaggio Angular:} mai affrontato, anche se penso la formazione riguardante Angular sia un carico leggero di lavoro;
    \item \textbf{Protocolli asincroni, Pattern Publisher/Subscriber e protocollo MQTT: } anche queste tecnologie, mai affrontate finora, rappresentano un carico di lavoro.
\end{itemize}


\subsubsection{Conclusioni}

Nonostante l'elevato interesse e attenzione che questi sistemi hanno attirato, il gruppo non ha preso in considerazione questo capitolato in quanto non era più disponibile tra le scelte possibili.






\end{document}
