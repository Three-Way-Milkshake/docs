\section{Introduzione}
\subsection{Scopo del documento}
Questo documento ha lo scopo di illustrare le motivazioni che ci hanno portato alla scelta del capitolato C5 - \textit{PORTACS: piattaforma di controllo mobilità autonoma}, commentando le varie proposte e analizzando le tecnologie usate.

\subsection{Glossario}
E' consigliato leggere questo documento con l'ausilio del glossario che ha lo scopo di definire parole che potrebbero risultare ambigue. Tali termini verranno evidenziati in questo file attraverso la disposizione di una "G" a pedice della stessa alla sua prima occorrenza, ad esempio: parola$_G$.
Il glossario, con tutte le parole definite, è presente nel file denominato "Glossario".

\subsection{Riferimenti}
\subsubsection{Normativi}
\begin{itemize}
  \item Norme di Progetto

\end{itemize}
\subsubsection{Informativi}
\begin{itemize}
  \item Documentazioni base e materiale video aggiuntivo si possono trovare  \href{https://www.math.unipd.it/~tullio/IS-1/2020/Progetto/Capitolati.html}{qui}
  \item Capitolato d'appalto 1 \\
  BlockCOVID: supporto digitale al contrasto della pandemia
  \item Capitolato d'appalto 2 \\
  EmporioLambda: piattaforma di e-commerce in stile Serverless
  \item Capitolato d'appalto 3 \\
  GDP: Gathering Detection Platform
  \item Capitolato d'appalto 4 \\
  HD Viz: visualizzazione di dati multidimensionali
  \item Capitolato d'appalto 5 \\
  PORTACS: piattaforma di controllo mobilità autonoma
  \item Capitolato d'appalto 6 \\
  RGP: Realtime Gaming Platform
  \item Capitolato d'appalto 7 \\
  SSD: soluzioni di sincronizzazione desktop
\end{itemize}