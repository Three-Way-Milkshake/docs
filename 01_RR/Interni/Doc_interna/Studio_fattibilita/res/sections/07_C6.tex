\subsection{Capitolato C6 - RGP: Realtime Gaming Platform}
\subsubsection{Informazioni generali}
	\begin{itemize}
	\item \textbf{Nome:} \textit{RGP: Realtime Gaming Platform;}
	\item \textbf{Proponente:} \textit{Zero12 s.r.l.;}
	\item \textbf{Committente:}  \textit{Prof. Tullio Vardanega e Prof. Riccardo Cardin;}
	\end{itemize}
\subsubsection{Descrizione del capitolato}
Il capitolato proposto prevede la realizzazione di un videogioco a scorrimento verticale, fruibile da dispositivi mobile, con la possibilità di giocare in real time multiplayer.
La modalità del gioco è simile ad Aero Fighters, mentre la grafica è scelta del gruppo di lavoro o fornita da Zero12.
Le modalità di gioco sono:
\begin{itemize}
	\item singleplayer
	\item multiplayer
\end{itemize}
La sfida tra più giocatori rappresenta il cuore del progetto ed è anche la componente di sviluppo principale.
Il gioco è ad eliminazione, l'ultimo giocatore non eliminato vince.
Durante la partita deve essere possibile vedere, in tempo reale, i movimenti del rivale ed è necessario sincronizzare eventuali nemici e power-up in modo tale che la sfida sia la medesima.
L'interazione tra giocatori diversi è puramente visiva.
La modalità singleplayer consiste in una serie infinita di livelli a difficoltà crescente. Il gioco termina quando l'utente ha concluso le vite oppure se non ha raccolto power-up sufficienti a mantenere il proprio oggetto attivo.

\subsubsection{Finalità del progetto}
Il progetto è finalizzato allo sviluppo di un'applicazione mobile superando dei vincoli quali:
\begin{itemize}
	\item Ricerca delle tecnologie AWS per capire quale si può adattare meglio ad un gioco con requisiti di realtime, raccogliendo le motivazioni che supportano la scelta di una tecnologia rispetto ad un'altra.
	\item Implementazione della componente server-side.
	\item Implementazione del gioco per piattaforma mobile.
\end{itemize}
\subsubsection{Tecnologie interessate}
Le tecnologie interessate sono:
\begin{itemize}
	\item \textbf{AWS} GameLift, Appsync oppure altre architetture serverless;
	\item \textbf{NodeJs};
	\item \textbf{Swift/SwiftUI} oppure \textbf{Kotlin};
	\item \textbf{GIT};
\end{itemize}
\subsubsection{Aspetti positivi}
Questo progetto aiuta sia a familiarizzare con lo sviluppo di applicazioni per Android e per Ios che a creare connessioni in real time tra due o più dispositivi attraverso i servizi di Amazon AWS. L'azienda inoltre prevede corsi di formazioni sui vari servizi offerti dll'AWS.
\subsubsection{Criticità e fattori di rischio}
L'ambiente AWS è molto vasto ed è critica la scelta iniziale su quale servizio basarsi. Una scelta errata può portare ad una pessima connessione tra dispositivi e, di conseguenza, ad un non soddisfacimento del requisito di "real time" della parte multiplayer del videogioco.
\subsubsection{Conclusioni}
Anche se i corsi di formazioni proposti risultano allettanti, la maggior parte del gruppo di lavoro non è entusiasmato all'idea di realizzare un videogioco.