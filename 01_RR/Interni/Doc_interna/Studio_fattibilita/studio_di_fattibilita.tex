\documentclass[a4paper]{article}

%Tutti gli usepackage vanno qui

\usepackage{geometry}
\usepackage[italian]{babel}
\usepackage[utf8]{inputenc}
\usepackage[T1]{fontenc}
\usepackage[normalem]{ulem}
\usepackage{tgschola}
%\usepackage{tgbonum}
\usepackage{tabularx}
\usepackage{longtable}
\usepackage{hyperref}
\usepackage{enumitem}
\usepackage[toc]{appendix}
\hypersetup{
	colorlinks=true,
	linkcolor=blue,
	filecolor=magenta,
	urlcolor=blue,
}
% Numerazione figure
\let\counterwithout\relax
\let\counterwithin\relax
\usepackage{chngcntr}

\counterwithin{table}{subsection}
\counterwithin{figure}{subsection}

\usepackage[bottom]{footmisc}
\usepackage{fancyhdr}
\setcounter{secnumdepth}{4}
\usepackage{amsmath, amssymb}
\usepackage{array}
\usepackage{graphicx}

\usepackage{ifthen}

%\usepackage{float}
\usepackage{layouts}
\usepackage{url}
\usepackage{comment}
\usepackage{float}
\usepackage{eurosym}

\usepackage{lastpage}
\usepackage{layouts}
\usepackage{float}
\usepackage{eurosym}

%Comandi di impaginazione uguale per tutti i documenti
\pagestyle{fancy}
\lhead{\includegraphics[scale=0.04]{../../../../latex/images/logoTWM.png}}
%Titolo del documento
\rhead{\doctitle{}}
%\rfoot{\thepage}
\cfoot{Pagina \thepage\ di \pageref{LastPage}}
\setlength{\headheight}{35pt}
\setcounter{tocdepth}{5}
\setcounter{secnumdepth}{5}
\renewcommand{\footrulewidth}{0.4pt}

% multirow per tabelle
\usepackage{multirow}

% Permette tabelle su più pagine
%\usepackage{longtable}


% colore di sfondo per le celle
\usepackage[table]{xcolor}

%COMANDI TABELLE
\newcommand{\rowcolorhead}{\rowcolor[HTML]{9b240a}} %intestazione
% check for missing commands
\newcommand{\headertitle}[1]{\textbf{\color{white}#1}} %titolo colonna
\definecolor{pari}{HTML}{FFDBCB}
\definecolor{dispari}{HTML}{F1F7FD}

% comandi glossario
\newcommand{\glo}{$_{G}$}
\newcommand{\glosp}{$_{G}$ }


%label custom
\makeatletter
\newcommand{\uclabel}[2]{%
	\protected@write \@auxout {}{\string \newlabel {#1}{{#2}{\thepage}{#2}{#1}{}} }%
	\hypertarget{#1}{#2}
}
\makeatother

%riportare pezzi di codice
\definecolor{codegray}{gray}{0.9}
\newcommand{\code}[1]{\colorbox{codegray}{\texttt{#1}}}



% Configurazione della pagina iniziale
\newcommand{\doctitle}{Verbale interno 15}
\newcommand{\docdate}{26 Febbraio 2021}
\newcommand{\rev}{1.0.0}
\newcommand{\stato}{Approvato}
\newcommand{\uso}{Interno}
\newcommand{\approv}{Tessari Andrea}
\newcommand{\red}{Crivellari Alberto}
\newcommand{\ver}{De Renzis Simone}
\newcommand{\dest}{Three Way Milkshake\\ Prof. Vardanega Tullio\\ Prof. Cardin Riccardo}
\newcommand{\describedoc}{Verbale del meeting del 2021-02-26 del gruppo Three Way Milkshake}
 % modifica questo file
\makeindex

\usepackage{hyperref}


\begin{document}
	\thispagestyle{empty}
\begin{titlepage}
	\begin{center}
		
		\includegraphics[scale = 0.17]{../../../../latex/images/logoTWM.png}\\[0.7cm]
		

		\noindent\rule{\textwidth}{1pt} \\[0.4cm]
		\Huge \textbf{\doctitle} \\[0.1cm]
		\ifthenelse{\equal{\docdate}{ }}{ }{ \huge \textbf{\docdate} \\[0.1cm] }
		
		\noindent\rule{\textwidth}{1pt}\\[0.7cm]
		
		\large \textbf{Three Way Milkshake - Progetto "PORTACS"} \\[0.4cm] 
                \texttt{threewaymilkshake@gmail.com} \\[0.4cm]
                
		
        
        
        \large

        \begin{tabular}{r|l}
                        \textbf{Versione} & \rev{} \\
                        \textbf{Stato} & \stato{} \\
                        \textbf{Uso} & \uso{} \\                         
                        \textbf{Approvazione} & \approv{} \\                      
                        \textbf{Redazione} & \red{} \\ 
                        \textbf{Verifica} &  \ver{} \\                         
                        \textbf{Destinatari} & \parbox[t]{5cm}{ \dest{} }
                \end{tabular} 
                \\[0.3cm]
                \large \textbf{Descrizione} \\ \describedoc{} 
               

	\end{center}
\end{titlepage}
	\pagebreak	
	
	% Registro delle modifiche
	\section*{Registro delle modifiche}

\newcommand{\changelogTable}[1]{
	
	
	\renewcommand{\arraystretch}{1.5}
	\rowcolors{2}{pari}{dispari}
	\begin{longtable}{ 
			>{\centering}p{0.07\textwidth} 
			>{}p{0.21\textwidth}
			>{\centering}p{0.17\textwidth}
			>{\centering}p{0.13\textwidth} 
			>{\centering}p{0.17\textwidth} 
			>{\centering}p{0.13\textwidth} }
		\rowcolorhead
		\headertitle{Vers.} &
		\centering \headertitle{Descrizione} &	
		\headertitle{Redazione} &
		\headertitle{Data red.} & 
		\headertitle{Verifica} &
		\headertitle{Data ver.}
		\endfirsthead	
		\endhead
		
		#1
		
	\end{longtable}
	\vspace{-2em}
	
}


\newcommand{\approvingTable}[1]{ 
	
	
	\renewcommand{\arraystretch}{1.5}
	\rowcolors{2}{pari}{dispari}
	\begin{longtable}{ 
			>{\centering}p{0.07\textwidth} 
			>{\centering}p{0.415\textwidth}
			>{\centering}p{0.13\textwidth}
			>{\centering}p{0.322\textwidth}  }
		\rowcolorhead
		\headertitle{Vers.} &
		\centering \headertitle{Descrizione} &	
		\headertitle{Data appr.} &
		\headertitle{Approvazione}
		\endfirsthead	
		\endhead
		
		#1
		
	\end{longtable}
	\vspace{-2em}
	
}
	\approvingTable{
	1.0.0 & Approvazione del verbale & 2021-04-18 & Greggio Nicolò
}

\changelogTable{
	0.1.0 & Stesura e verifica del verbale & Crivellari Alberto & 2021-04-15 & De Renzis Simone & 2021-04-18
} % modifica questo file       
	\end{longtable}
	\pagebreak
		
	% indice
	\tableofcontents	
	\pagebreak
	
	% indice delle figure
	\listoffigures
	\pagebreak
	
	% indice delle tabelle
	\listoftables
	\pagebreak
		
	% contenuto del documento, ogni sezione in un file
	
	\section{Introduzione}




\subsection{Scopo del documento}
Lo scopo di questo documento è presentare tutte le informazioni necessarie al mantenimento e all'estensione del software PORTACS, mostrando nel dettaglio l'architettura del sistema e l'organizzazione del codice sorgente.\\
In questo documento saranno presentate le varie tecnologie usate, sia lato front end che back end, come anche le varie librerie e framework. Verrà inoltre mostrato il sistema di versionamento utilizzato e la Continuous Integration applicata.





\subsection{Scopo del prodotto}

Il capitolato\textsubscript{G} C5 propone un progetto\textsubscript{G} in cui viene richiesto lo sviluppo di un software per il monitoraggio in tempo reale di unità che si muovono in uno spazio definito. All'interno di questo spazio, creato dall’utente per riprodurre le caratteristiche di un ambiente reale, le unità dovranno essere in grado di circolare in autonomia, o sotto il controllo dell’utente, per raggiungere dei punti di interesse posti nella mappa.  La circolazione è sottoposta a vincoli di viabilità e ad ostacoli propri della topologia dell’ambiente, deve evitare le collisioni con le altre unità e prevedere la gestione di situazioni critiche nel traffico.




\subsection{Riferimenti}



\subsubsection{Normativi}

\begin{itemize}
	\item \textsc{Norme di progetto\textsubscript{G} v3.0.0 }: per qualsiasi convenzione sulla nomenclatura degli elementi presenti all’interno del documento;
	
	\item Regolamento progetto\textsubscript{G} didattico: \\ {\url{https://www.math.unipd.it/~tullio/IS-1/2020/Dispense/P1.pdf}};
	\item Model-View Patterns: \\ {\url{https://www.math.unipd.it/~rcardin/sweb/2020/L02.pdf}};
	\item SOLID Principles: \\ {\url{https://www.math.unipd.it/~rcardin/sweb/2020/L04.pdf}};
	\item Diagrammi delle classi: \\ {\url{https://www.math.unipd.it/~rcardin/swea/2021/Diagrammi delle Classi_4x4.pdf}};
	\item Diagrammi dei package: \\ {\url{https://www.math.unipd.it/~rcardin/swea/2021/Diagrammi dei Package_4x4.pdf}};
	\item Diagrammi di sequenza: \\ {\url{https://www.math.unipd.it/~rcardin/swea/2021/Diagrammi di Sequenza_4x4.pdf}};
	\item Design Pattern Creazionali: \\ {\url{https://www.math.unipd.it/~rcardin/swea/2021/Design Pattern Creazionali_4x4.pdf}};
	\item Design Pattern Strutturali: \\ {\url{https://www.math.unipd.it/~rcardin/swea/2021/Design Pattern Strutturali_4x4.pdf}};
	\item Design Pattern Comportamentali: \\ {\url{https://www.math.unipd.it/~rcardin/swea/2021/Design Pattern Comportamentali_4x4.pdf}}.
\end{itemize}



\subsubsection{Informativi}
\begin{itemize}
	\item \textsc{\href{https://github.com/Three-Way-Milkshake/docs/wiki/Glossario}{Glossario}}: per la definizione dei termini (pedice G) e degli acronimi (pedice A) evidenziati nel documento;
	\item Capitolato d'appalto C5-PORTACS: \\
{\url{https://www.math.unipd.it/~tullio/IS-1/2020/Progetto/C5.pdf}}
	\item Software Engineering - Iam Sommerville - $10^{th}$ Edition.
\end{itemize} % andrea
	\pagebreak
	
	\input{res/sections/02_C1.tex} % simone
	\pagebreak

	\input{res/sections/03_C2.tex} % modifica questo file
	\pagebreak
	
	\input{res/sections/04_C3.tex} % modifica questo file
	\pagebreak
	
	\input{res/sections/05_C4.tex} % modifica questo file
	\pagebreak
	
	\input{res/sections/06_C5.tex} % modifica questo file
	\pagebreak
	
	\section{Capitolato C6 - RGP: Realtime Gaming Platform}
\subsection{Informazioni generali}
	\begin{itemize}
	\item \textbf{Nome:} \textit{RGP: Realtime Gaming Platform;}
	\item \textbf{Proponente:} \textit{Zero12 s.r.l.;}
	\item \textbf{Committente:}  \textit{Prof. Tullio Vardanega e Prof. Riccardo Cardin;}
	\end{itemize}
\subsection{Descrizione del capitolato}
Il capitolato proposto prevede la realizzazione di un videogioco a scorrimento verticale, fruibile da dispositivi mobile, con la possibilità di giocare in real time multiplayer.
La modalità del gioco è simile ad Aero Fighters, mentre la grafica è scelta del gruppo di lavoro o fornita da Zero12.
Le modalità di gioco sono:
\begin{itemize}
	\item singleplayer
	\item multiplayer
\end{itemize}
La sfida tra più giocatori rappresenta il cuore del progetto ed è anche la componente di sviluppo principale.
Il gioco è ad eliminazione, l'ultimo giocatore non eliminato vince.
Durante la partita deve essere possibile vedere, in tempo reale, i movimenti del rivale ed è necessario sincronizzare eventuali nemici e power-up in modo tale che la sfida sia la medesima.
L'interazione tra giocatori diversi è puramente visiva.
La modalità singleplayer consiste in una serie infinita di livelli a difficoltà crescente. Il gioco termina quando l'utente ha concluso le vite oppure se non ha raccolto power-up sufficienti a mantenere il proprio oggetto attivo.

\subsection{Finalità del progetto}
Il progetto è finalizzato allo sviluppo di un'applicazione mobile superando dei vincoli quali:
\begin{itemize}
	\item Ricerca delle tecnologie AWS per capire quale si può adattare meglio ad un gioco con requisiti di realtime, raccogliendo le motivazioni che supportano la scelta di una tecnologia rispetto ad un'altra.
	\item Implementazione della componente server-side.
	\item Implementazione del gioco per piattaforma mobile.
\end{itemize}
\subsection{Tecnologie interessate}
Le tecnologie interessate sono:
\begin{itemize}
	\item \textbf{AWS} GameLift, Appsync oppure altre architetture serverless;
	\item \textbf{NodeJs}; 
	\item \textbf{Swift/SwiftUI} oppure \textbf{Kotlin};
	\item \textbf{GIT};
\end{itemize}
\subsection{Aspetti positivi}
Questo progetto aiuta sia a familiarizzare con lo sviluppo di applicazioni per Android e per Ios che a creare connessioni in real time tra due o più dispositivi attraverso i servizi di Amazon AWS. L'azienda inoltre prevede corsi di formazioni sui vari servizi offerti dll'AWS.
\subsection{Criticità e fattori di rischio}
L'ambiente AWS è molto vasto ed è critica la scelta iniziale su quale servizio basarsi. Una scelta errata può portare ad una pessima connessione tra dispositivi e, di conseguenza, ad un non soddisfacimento del requisito di "real time" della parte multiplayer del videogioco.
\subsection{Conclusioni}
Anche se i corsi di formazioni proposti risultano allettanti, la maggior parte del gruppo di lavoro non è entusiasmato all'idea di realizzare un videogioco. % modifica questo file
	\pagebreak
	
	\subsection{Capitolato C7 - Soluzioni di sincronizzazione Desktop}
\subsubsection{Informazioni generali}
	\begin{itemize}
	\item \textbf{Nome:} Soluzioni di sincronizzazione Desktop;
	\item \textbf{Proponente:} Zextras;
	\item \textbf{Committente:}  Prof. Vardanega Tullio e Prof. Cardin Riccardo;
	\end{itemize}
\subsubsection{Descrizione del capitolato}
Il capitolato\textsubscript{G} richiede di sviluppare un algoritmo di sincronizzazione Desktop solido ed efficiente\textsubscript{G} in grado di garantire il salvataggio in cloud del lavoro, in modo da poter raggiungere in qualsiasi momento e da qualsiasi dispositivo il proprio lavoro. Inoltre deve essere sviluppata un'interfaccia multipiattaforma per l'uso dell'algoritmo nei più importanti sistemi operativi esistenti (MacOs, Windows, Linux) senza richiedere all'utente l'installazione di ulteriori prodotti per il funzionamento.
\subsubsection{Finalità del progetto}
L'obiettivo è quello di creare questo algoritmo di sincronizzazione che funzioni in più piattaforme in grado di garantire il salvataggio in cloud del lavoro e contemporaneamente la sincronizzazione dei cambiamenti presenti in cloud. Inoltre viene richiesto l'utilizzo dell’algoritmo sviluppato per richiedere e fornire i cambiamenti ai contenuti in sincronizzazione verso il prodotto Zextras Drive.
L'algoritmo deve avere le seguenti principali funzionalità:
\begin{itemize}
\item {Configurazione ed autenticazione dell’utente};
\item {Gestione di cosa sincronizzare e di cosa ignorare nelle cartelle cloud};
\item {Gestione di cosa sincronizzare e di cosa ignorare nelle cartelle locali};
\item {Sincronizzazione costante dei cambiamenti, siano essi locali o remoti};
\item {Possibilità di modifica delle preferenze a posteriori};
\item {Sistema di notifica utente dei cambiamenti}.
\end{itemize}
\subsubsection{Tecnologie interessate}
L'azienda esprime la necessità di sviluppare l'algoritmo per i più importanti sistemi operativi esistenti (MacOs, Windwos, Linux) e consiglia l'utilizzo:
\begin{itemize}
\item \textbf{\href{https://wiki.qt.io/About_Qt}{Qt Framework}:} strumento basato sul linguaggio C++ orientato ad oggetti, da utilizzare per creare l'interfaccia poiché fortemente supportato e documentato;
\item \textbf{Python:} strumento consigliato per lo sviluppo della Business Logic, linguaggio ad alto livello adatto allo sviluppo di applicazioni distribuite.
\end{itemize}
\subsubsection{Aspetti positivi}
Il progetto\textsubscript{G} si pone in un contesto molto utilizzato sia da utenti privati che da aziende. Inoltre le tecnologie consigliate fanno già parte delle conoscenze di gran parte dei membri del gruppo e ci sono molti esempi, anch'essi ben conosciuti, a cui ispirarsi per la creazione dell'algoritmo (Dropbox, Google Drive, ect).
\subsubsection{Criticità e fattori di rischio}
Il confronto con tecnologie già esistenti di questo tipo può essere svantaggioso poiché si potrebbe creare un prodotto molto più inefficiente. Inoltre le richieste sono numerose e abbastanza complesse.
\subsubsection{Conclusioni}
Si è deciso di puntare su altri capitolato\textsubscript{G} poiché questo non ha suscitato grande interesse per quanto riguarda la tematica che affronta e il settore interessato. % modifica questo file
	\pagebreak	
	
	% ecc, aggiungere le sezioni successive creando i relativi file
	
\end{document}
