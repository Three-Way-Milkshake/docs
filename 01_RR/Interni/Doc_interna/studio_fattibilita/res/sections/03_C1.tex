\subsection{Capitolato C1 - BlockCOVID}


\subsubsection{Informazioni generali}

\begin{itemize}
	\item{\textbf{Nome:}} BlockCOVID
	\item{\textbf{Proponente:}} Imola Informatica
	\item{\textbf{Committente:}} Prof. Vardanega Tullio, Prof. Cardin Riccardo
\end{itemize}



\subsubsection{Descrizione del capitolato}

Il software BlockCOVID si pone l'obiettivo di fornire un'infrastruttura che consenta il tracciamento del personale e della pulizia delle postazioni di lavoro all'interno di un'azienda. Mira infatti a favorire l'attuazione e amministrazione delle misure necessarie per la tutela della salute dei lavoratori nel contesto della pandemia da COVID-19.



\subsubsection{Finalità del progetto}

Il prodotto finale prevede due modalità di tracciamento:
\begin{itemize}
	\item{\textbf{Tracciamento delle presenze:}} registrazione e monitoraggio in tempo reale delle presenze all'interno di un ambiente di lavoro (laboratorio informatico). Gli utenti devono avere la possibilità, in base alle disponibilità di postazioni, di effettuare prenotazioni tramite un applicativo per dispositivi mobili \textit{Android} e \textit{IOS}. Tramite lo stesso dispotivo segnaleranno la propria presenza una volta occupata la postazione ed eventualmente la sanificazione della stessa dopo l'uso. L'amministratore deve poter creare e gestire la struttura dell'ambiente, e visualizzarne in tempo reale lo stato di occupazione e sanificazione delle postazioni; 
	\item{\textbf{Tracciamento della pulizia delle postazioni:}} dedicato agli addetti alle pulizie, il prodotto deve consentire di visualizzare le postazioni non sanificate e segnalare le postazioni sanificate in seguito alla pulizia delle stesse.
\end{itemize}



\subsubsection{Tecnologie interessate}

\textit{Java} (versione 8 o superiori), \textit{Python} o \textit{Node.js}.
\begin{itemize}
	\item{\textbf{Java, Python, Node.js:}} linguaggi suggeriti dal proponente per lo sviluppo del server back-end che amministra l'infrastruttura di tracciamento;
	\item{\textbf{Protocolli asincroni:}} da utilizzare per gestire la comunicazione tra le componenti del sistema: i dispositivi mobili degli utenti, dell'amministratore e degli addetti alle pulizie con il server centrale;
	\item{\textbf{Blockchain:}} sistema che assicura la registrazione dei dati relativi alle sanificazioni conferendone valore legale;  
	\item{\textbf{Kubernetes IAAS, PAAS, Openshift, Rancher:}} piattaforme suggerite dal proponente per il rilascio delle componenti del server e la gestione della scalabilità orizzontale.
\end{itemize}

Inoltre il proponente specifica i seguenti requisiti minimi:
\begin{itemize}
	\item Fornire delle API che utilizzino tecnologie \textit{Rest} o \textit{gRPC} nell'ambito della comunicazione tra server e dispositivi;
	\item Studiare la tecnologia \textit{RFID} o eventuali alternative da utilizzare per certificare la presenza di persone nelle postazioni, analizzandone il consumo energetico;
	\item Effettuare test di integrazione e unità (copertura $\geq 80\%$) delle componenti applicative;
	\item Assicurare l'integrità del sistema tramite test end-to-end.
\end{itemize}
	
	
	
\subsubsection{Aspetti positivi}

Il progetto si pone in un contesto quanto mai attuale e di grande interesse generale. Trova infatti soluzione a problematiche diffuse e richieste da molte realtà, fornendo un caso di studio molto interessante per quanto riguarda i sistemi distribuiti e real-time. Le tecnologie impiegate e richieste sono valide, e il loro studio consentirebbe ai membri del gruppo di acquisire competenze facilmente spendibili nel mondo del lavoro.



\subsubsection{Criticità}

I requisiti del proponente sono dettagliati e appaiono abbastanza esigenti; le tecnologie da impiegare, per quanto interessanti, configurano un carico di lavoro consistente.

In generale, il capitolato ha riscosso un discreto interesse da parte del gruppo ma non si è configurato come prima scelta a favore di capitolati riguardanti temi considerati più interessanti, tecnologie più stimolanti e carichi di lavoro più sostenibili. 



\subsubsection{Conclusioni}

Nonostante l'elevato interesse e attenzione che questi sistemi hanno attirato, il gruppo ha deciso di procedere a favore di altri capitolati.
Per quanto attuale, il progetto si presuppone essere completato ed eventualmente utilizzato in un periodo in cui le problematiche legate alla pandemia dovrebbero, auspicabilmente, risultare fortemente mitigate dall'avvento delle campagne vaccinali. Non è comunque da escludere che, visto l'enorme impatto di questa emergenza, forme di tracciamento non debbano continuare ad essere applicate anche in condizioni di normalità.




