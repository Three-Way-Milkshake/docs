\section{Verbale della riunione}
\subsection{Presentazione studi personali su documenti}
Nella prima della riunione è stato affrontato il tema dei contenuti e forma dei vari documenti da presentare in ingresso alla RR. Ogni componente si è occupato di reperire informazioni rilevanti all'argomento a lui assegnato ed ha quindi esposto agli altri quanto appreso di modo da permettere a tutti di avere una conoscenza di base ed un punto di partenza omogeneo su tutti i documenti, parallelizzando questa raccolta di informazioni in modo da ottimizzare i tempi.
Si è giunti alla conclusione che i documenti da presentare, oltre agli studi di fattibilità sui vari capitolati, sono i seguenti:
\begin{itemize}
    \item \textsc{Norme di Progetto};
    \item \textsc{Piano di Progetto};
    \item \textsc{Piano di Qualifica};
    \item \textsc{Analisi dei Requisiti};
    \item \textsc{Glossario}.
\end{itemize}
Al termine delle presentazioni di tutti componenti è stata raggiunta una discreta comprensione di base dei contenuti e della struttura di massima di questi documenti.
Si è poi distribuito fra i vari componenti il lavoro di inizio impostazione e stesura dei vari documenti.

\subsection{I Ruoli di Progetto}
Sono stati discussi in maniera approfondita i vari ruoli di progetto, le differenze, i compiti, le rotazioni, le presenze ed i cambiamenti durante tutto il percorso di progetto. Si sono sollevati quindi diversi dubbi che verranno presentati al prossimo incontro utile con il Prof. Vardanega per contribuire a degli spunti di riflessione che potranno favorire tutti i gruppi. Le questioni che al momento generano maggior confusione all'interno del gruppo sono:
\begin{itemize}
    \item alcuni ruoli sono mancanti (non si presentano) prima della RR (e.g.: programmatori, progettisti);
    \item funzionamento e modalità di rotazione;
    \item chi produce non può essere verificatore dello stesso documento;
    \begin{itemize}
        \item formalizzazione di queste scelte.
    \end{itemize}
\end{itemize}

\subsection{Stesura, Verifica, Validazione ed Approvazione dei Documenti}
È stato fissato un procedimento di massima generale per il ciclo di vita dei documenti. Ogni documento passerà per le seguenti fasi, così composte:
\begin{itemize}
    \item \textbf{Stesura:}
    \begin{itemize}
        \item Produzione del contenuto del documento\\Questo sarà compito di una o più persone in funzione della complessità ed importanza del documento;
    \end{itemize}
    \item \textbf{Verifica:}
    \begin{itemize}
        \item Controllo approfondito del documento\\
		Questo compito impegnerà una o più persone in funzione della complessità ed importanza del documento e servirà per verificare la correttezza dei contenuti e la loro impostazione.\\Per quanto riguarda correttezza sintattica, ortografica e grammaticale, un primo controllo potrà essere delegato a qualche automazione, ancora da decretare;
    \end{itemize}
    \item \textbf{Validazione ed Approvazione:}
    \begin{itemize}
        \item Effettuato da una persona che verificherà l'adempimento di tutti i requisiti richiesti per il documento in questione e ne sancirà la definitiva approvazione e conseguente primo rilascio ufficiale.
    \end{itemize}
\end{itemize}

\subsection{Strumenti di Comunicazione}
Dopo aver comunicato per varie settimane tramite un gruppo Telegram dedicato al progetto, è sorta la necessità di utilizzare uno strumento più adatto alla coordinazione di un gruppo di progetto, che fosse isolato da contesti che coinvolgono altri scopi che possono essere fonti di distrazione e che raccolga funzionalità ed integrazioni che possono migliorare il workflow di progetto. Si è scelto quindi di adottare lo strumento Slack, che permette innanzitutto di creare la separazione necessaria dall'ambiente esterno al lavoro, poi consente la creazione di diversi canali, i cui membri possono anche essere un sottoinsieme del gruppo, così da coordinare e separare lo svolgimento dei compiti. Al momento si è deciso di realizzare un canale generale, ed un canale per ogni documento per la RR. Per le riunioni si continuerà ad utilizzare Google Meet e per rapide comunicazioni informali il gruppo Telegram. Per tutti i contatti esterni ci si interfaccerà tramite la mail del gruppo \href{mailto:threewaymilkshake@gmail.com}{threewaymilkshake@gmail.com}.\\
Si è scelto, infine, di adottare Atlassian Confluence per la produzione ed il mantenimento di Wiki e raccolta risorse che possono rivelarsi utili a tutti i componenti e per costruire una Knowledge Base.

\subsection{Integrazioni ed Automazioni}
È risultata immediata a tutti la necessità di automatizzare compiti di gestione organizzativa potenzialmente ripetitivi, e la volontà di collegare gli strumenti il più possibile. Si è quindi dedicato un canale Slack alle notifiche automatiche, che Jira e Github possono inviare quando vengono effettuate delle modifiche rispettivamente alla board delle issue ed al repository della documentazione. Queste notifiche Slack sono inoltre interattive e permettono di collegarsi direttamente con lo strumento in questione ed intraprendere ulteriori azioni se necessario, senza lasciare il client di comunicazione, oppure aprire il contesto nel servizio di riferimento.\\
È stata studiata la possibilità di utilizzo di "smart commits" per collegare le issue di riferimento al lavoro svolto, di modo da aggiungere commenti in automatico su Jira e far cambiare transizioni alle task, sempre in maniera automatica. Questa pratica è documentata nella wiki alla pagina git.