\section{Verbale della riunione}
\subsection{Nome del gruppo}
Come prima decisione è stato affrontato il tema del nome del gruppo: varie proposte da parte dei componenti si sono risolte, dopo un breve dibattito, in una votazione che ha decretato il nome finale. Il logo non è stato deciso contestualmente, riservandosi qualche giorno per produrre e valutare nuove proposte.

\subsection{Way Of Working}
Sono state fissate alcune norme per consolidare il Way Of Working: gli incontri tra i membri avverranno secondo riunioni dalla durata massima di un'ora e mezza, al fine di mantenere un'alto grado di concentrazione per l'intera durata del meeting, e massimizzare l'utilizzo del tempo per le decisioni da compiere

\subsection{Strumenti di collaborazione e sviluppo}
Si è poi proceduto a riflettere sugli strumenti che il gruppo avrebbe preferito adottare per agevolare il coordinamento dei compiti da svolgere, la gestione del codice e delle sue versioni, e la redazione della documentazione.
Gli strumenti designati sono elencati di seguito:

\begin{itemize}
	\item \textbf{Git} per il versionamento del codice e della documentazione;
	\item \textbf{GitHub} per la gestione su piattaforma web delle repository dedicate alla documentazione e al codice: viene istituito un account \textit{GitHub Team} a cui i membri del gruppo possono accedere tramite i propri account personali e collaborare in un unico spazio;
	\item \textbf{Slack} come mezzo di comunicazione asincrona tra i membri del gruppo, che, grazie alla possibilità di creare canali per argomento, permette la discussione e il confronto su tematiche precise e con i soli membri interessati;
	\item \textbf{Jira} come tool di collaborazione adatto a suddividersi compiti e fissare scadenze, particolarmente incentrato sullo sviluppo software  e integrabile con \textit{GitHub} e \textit{Slack}; 
	\item \textbf{\LaTeX\ } come linguaggio per la redazione dei documenti: sebbene non conosciuto da tutti è stato confermato come strumento da utilizzare per la sua flessibilità e completezza.
\end{itemize}

\subsection{Preferenze sul capitolato}
Successivamente, il gruppo ha trattato la scelta del capitolato da affrontare, lasciando spazio ad ogni componente di esprimere le proprie preferenze personali. Si sono analizzati i punti di forza e le criticità di ogni capitolato, delineando quella che sarà la preferenza del gruppo: il capitolato C5, \textsc{portacs}.

\subsection{Prossime mosse}
L'ultima parte della riunione ha visto i membri coinvolti in un breve confronto su quali fossero le mosse successive da affrontare per lo sviluppo del progetto: in particolare come strutturare la documentazione e quali gli aspetti decisivi da includere nei documenti da produrre. Al riguardo, ogni componente si vedrà assegnato un tema da approfondire e da discutere al prossimo incontro.






