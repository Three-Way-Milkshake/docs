\section{Qualità del prodotto}
Per valutare  la qualità del prodotto, il gruppo Three Way Milkshake ha deciso di avvalersi dello standard ISO/IEC$_G$ 9126.\\
Questo modello è mirato a  migliorare l'organizzazione e i processi di una società software.\\
Di seguito verrà descritto il modello della qualità del software, in:
\begin{itemize}
	\item \textbf{Funzionalità}
	\item \textbf{Affidabilità}
	\item \textbf{Efficienza}
	\item \textbf{Usabilità}
	\item \textbf{Manutenibilità}
	\item \textbf{Portabilità}
\end{itemize}

\subsection{Funzionalità}
La funzionalità è la capacità di un prodotto di rispondere ad esigenze specifiche.\\
In questo caso le esigenze vengono descritte nel documento Analisi dei Requisiti.
\subsubsection{Obiettivi}
\begin{itemize}
	\item \textbf{Appropriatezza:} Capacità del software di riuscire a svolgere tutte le funzionalità prefissate con l'utente;
	\item \textbf{Accuratezza:} Capacità del software di svolgere correttamente ciò che era stato precedentemente concordato;
	\item \textbf{Interoperabilità:} Capacità del software di operare con più sistemi;
	\item \textbf{Conformità:} Capacità del software di aderire agli standard relativi alla funzionalità;
	\item \textbf{Sicurezza:} Capacità del software di non permettere alle persone non autorizzate di accedere o modificare dati sensibili dell'utente; consente alle persone autorizzate di accedere ai dati.
\end{itemize}

\subsubsection{Metriche}
\textbf{Completezza del Software(Cs)}\\
Viene specificata la completezza del software.
\begin{itemize}
	\item \textbf{Misurazione:} $C = (1-($Funzionalità non implementate$ / $Funzionalità implementate$))$;
	\item \textbf{Valore preferibile:} Cs = 1;
	\item \textbf{Valore accettabile:} Cs = 1.
\end{itemize}
\pagebreak
\subsection{Affidabilità}
L'affidabilità è la capacità di un certo software di mantenere un certo livello di prestazioni in determinate condizioni in un certo periodo.
\subsubsection{Obiettivi}
\begin{itemize}
	\item \textbf{Maturità:} Capacità del prodotto di dare risultati corretti, esenti da malfunzionamenti o errori;
	\item \textbf{Tolleranza agli errori:} Capacità del prodotto di poter essere usabile anche in presenza di malfunzionamenti o casi derivanti un uso scorretto del software;
	\item \textbf{Recuperabilità:} Capacità del prodotto di recuperare almeno le informazioni rilevanti in seguito ad un malfunzionamento;
	\item \textbf{Aderenza:} Capacità del prodotto di aderire a standard inerenti all'affidabilità.
\end{itemize}
\subsubsection{Metriche}
\textbf{Affidabilità del Software A}\\
Viene specificata l'abilità del software di resistere a malfunzionamenti.
\begin{itemize}
	\item \textbf{Misurazione:} $A = $Numero di errori$ / $Numero di test eseguiti;
	\item \textbf{Valore preferibile:} A = 0;
	\item \textbf{Valore accettabile:} A < 0.15.
\end{itemize}

\subsection{Efficienza}
L'efficienza è la capacità del software di poter offrire un determinato livello di prestazioni in date condizioni in un certo periodo.
\subsubsection{Obiettivi}
\begin{itemize}
	\item Comportamento rispetto al tempo: Capacità del prodotto di fornire adeguati livelli di elaborazione, velocità e tempi di risposta;
	\item Utilizzo delle risorse: Capacità del prodotto di utilizzare le risorse in maniera adeguata;
	\item Conformità: Capacità del prodotto di aderire a standard sull'efficienza.
\end{itemize}
\subsubsection{Metriche}
Visto che il proponente non ha incluso dettagli relativi alla qualità dell'efficienza, non verranno proposte metriche per questa sezione.

\subsection{Usabilità}
L'usabilità è la capacità del prodotto di essere compreso ed utilizzato dall'utente senza difficoltà tenendo conto certe condizioni.

\subsubsection{Obiettivi}
\begin{itemize}
	\item \textbf{Comprensibilità:} Capacità del prodotto di visualizzare le varie funzionalità del software e permette all'utente di capire se il software è indicato per le sue esigenze;
	\item \textbf{Apprendibilità:} Capacità del prodotto di aumentare nel tempo l'abilità dell'utente di sfruttare il software;
	\item \textbf{Operabilità:} Capacità del prodotto che permette agli utenti di farne uso per i loro scopi;
	\item \textbf{Attrattiva:} Capacità del prodotto di rendere più piacevolo l'utilizzo del software;
	\item \textbf{Conformità:} Capacità del prodotto di aderire a standard relativi all'usabilità.
\end{itemize}

\subsubsection{Metriche}
\textbf{Numero di tocchi/click necessari C)}\\
Viene specificata la facilità con cui l'utente riesce a raggiungere ciò che vuole attraverso il conteggio del numero di tocchi o click necessari al suo raggiungimento.\\
Si considera la capacità dell'operatore di visualizzare la propria lista delle \gls{task}\textsubscript{G}.
\begin{itemize}
	\item \textbf{Misurazione:} $L = $Numero di tocchi o click necessari per il raggiungimento dell'obiettivo;
	\item \textbf{Valore preferibile:} C < 4;
	\item \textbf{Valore accettabile:} C < 6.
\end{itemize}
\textbf{Numero di secondi necessari S}
Viene specificata la facilità con cui l'utente riesce a raggiungere ciò che vuole attraverso il conteggio dei secondi necessari al suo raggiungimento.\\
Si considera la capacità dell'operatore di visualizzare la propria lista delle \gls{task}\textsubscript{G}.
\begin{itemize}
	\item \textbf{Misurazione:} $L = $Numero di secondi necessari per il raggiungimento dell'obiettivo;
	\item \textbf{Valore preferibile:} S < 15;
	\item \textbf{Valore accettabile:} S < 40.
\end{itemize}
Viene specificata la profondità gerarchica massima dei collegamenti e delle funzionalità presenti all'interno del software.
\begin{itemize}
	\item \textbf{Misurazione:} $P = $Profondità gerarchica massima dei collegamenti e delle funzionalità presenti all'interno del software;
	\item \textbf{Valore preferibile:} P < 4;
	\item \textbf{Valore accettabile:} P < 6.
\end{itemize}
\subsection{Manutenibilità}
Capacità del prodotto di essere modificato anche in futuro.
\subsubsection{Obiettivi}
\begin{itemize}
	\item \textbf{Analizzabilità:} Facilità con cui è possibile interpretare il codice del software;
	\item \textbf{Modificabilità:} Capacità per cui risulta non troppo oneroso modificare il codice del software;
	\item \textbf{Stabilità:} Capacità del software di evitare errori inaspettati derivanti da modifiche errate;
	\item \textbf{Testabilità:} Capacità del prodotto di essere testato al fine di validare le modifiche al codice sorgente.
\end{itemize}
\subsubsection{Metriche}
\textbf{Leggibilità del Software L}\\
Viene specificata l'abilità del software di resistere a malfunzionamenti.
\begin{itemize}
	\item \textbf{Misurazione:} $L = $Numero di linee di codice commentate$ / $Numero di linee di codice;
	\item \textbf{Valore preferibile:} L > 0.15;
	\item \textbf{Valore accettabile:} L > 0.10.
\end{itemize}


\subsection{Portabilità}
La portabilità è la capacità del software di poter funzionare senza tener conto di uno specifico ambiente di lavoro.
\subsubsection{Obiettivi}
\begin{itemize}
	\item \textbf{Adattabilità:} Capacità del prodotto di essere adattato per diversi ambienti operativi;
	\item \textbf{Installabilità:} Capacità del prodotto di essere installato in uno specificato ambiente operativo;
	\item \textbf{Conformità:} Capacità del software di aderire a standard relativi alla portabilità;
	\item \textbf{Sostituibilità:} Capacità del software di sostituire un altro prodotto con le stesse funzionalità.
\end{itemize}
\subsubsection{Metriche}
IL software dovrà eseguire solamente su ambiente Docker, quindi non sono necessarie varie metriche.
\subsection{Tabella Riassuntiva}
%tabella
\begin{table}[H]
	\begin{center}
		\caption{Tabella riassuntiva metriche di processo}
		\begin{tabular}{p{0.25\linewidth} p{0.25\linewidth}c!{\color[HTML]{9b240a}\vrule width 0.05cm}cc}
			\rowcolorhead
			\headertitle{Nome Metrica} & \headertitle{Descrizione} & \headertitle{Tipo Capacità} & \headertitle{Val. Pref.} & \headertitle{Val. Accett.}\\
			
			Completezza del Software Cs & Funzionalità non implementate rispetto alle funzionalità implementate & Funzionalità & Cs = 1 & Cs = 1\\
			Affidabilità del Software A & Errori rispetto al numero di test eseguiti & Affidabilità & A = 0 & A < 0.15\\
			Numero di tocchi/click necessari C & Numero di tocchi o click necessari per visualizzare la propria lista di \gls{task}\textsubscript{G} & Usabilità & C < 4 & C < 6\\
			Numero di secondi necessari S & Numero di secondi necessari per visualizzare la propria lista di \gls{task}\textsubscript{G} & Usabilità & S < 15 & S < 40\\
			Profondità gerarchica P & Profondità gerarchica massima dei collegamenti e funzionalità presenti all'interno del software & Usabilità & P < 4 & P < 6\\
			Leggibilità software L & Numero di linee di codice commentate rispetto al totale di linee di codice & Manutenibilità & L > 0.15 & L > 0.10\\	
		\end{tabular}
		
	\end{center}
\end{table}
