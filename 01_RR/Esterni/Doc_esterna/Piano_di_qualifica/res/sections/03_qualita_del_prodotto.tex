\section{Qualità del prodotto}
Per valutare  la qualità del prodotto, il gruppo Three Way Milkshake ha deciso di avvalersi dello standard ISO/IEC 9126.\\
Questo modello è mirato a  migliorare l'organizzazione e i processi di una società software.\\
Di seguito verrà descritto il modello della qualità del software, in:
\begin{itemize}
	\item Funzionalità
	\item Affidabilità
	\item Efficienza
	\item Usabilità
	\item Manutenibilità
	\item Portabilità
\end{itemize}

\subsection{Funzionalità}
La funzionalità è la capacità di un prodotto di rispondere ad esigenze specifiche.\\
In questo caso le esigenze vengono descritte nel documento Analisi dei Requisiti.
\subsubsection{Obiettivi}
\begin{itemize}
	\item Appropriatezza: Capacità del software di riuscire a svolgere tutte le funzionalità prefissate con l'utente;
	\item Accuratezza: Capacità del software di svolgere correttamente ciò che era stato precedentemente concordato;
	\item Interoperabilità: Capacità del software di operare con più sistemi;
	\item Conformità: Capacità del software di aderire agli standard relativi alla funzionalità;
	\item Sicurezza: Capacità del software di non permettere alle persone non autorizzate di accedere o modificare dati sensibili dell'utente; consente alle persone autorizzate di accedere ai dati.
\end{itemize}

\subsubsection{Metriche}
Viene specificata la completezza del software attraverso la seguente formula:
\begin{center}
	$C = (1-($Funzionalità non implementate$ / $Funzionalità implementate$))$
\end{center}
Con i valori:\\
preferibile = 1;\\
accettabile = 1.

\subsection{Affidabilità}
L'affidabilità è la capacità di un certo software di mantenere un certo livello di prestazioni in determinate condizioni in un certo periodo.
\subsubsection{Obiettivi}
\begin{itemize}
	\item Maturità: Capacità del prodotto di dare risultati corretti, esenti da malfunzionamenti o errori;
	\item Tolleranza agli errori: Capacità del prodotto di poter essere usabile anche in presenza di malfunzionamenti o casi derivanti un uso scorretto del software;
	\item Recuperabilità: Capacità del prodotto di recuperare almeno le informazioni rilevanti in seguito ad un malfunzionamento;
	\item Aderenza: Capacità del prodotto di aderire a standard inerenti all'affidabilità.
\end{itemize}
\subsubsection{Metriche}
Viene specificata l'abilità del software di resistere a malfunzionamenti attraverso la seguente formula:
\begin{center}
	$R = $Numero di errori$ / $Numero di test eseguiti
\end{center}
Con i valori:\\
preferibile = 0;\\
accettabile < 0.15.

\subsection{Efficienza}
L'efficienza è la capacità del software di poter offrire un determinato livello di prestazioni in date condizioni in un certo periodo.
\subsubsection{Obiettivi}
\begin{itemize}
	\item Comportamento rispetto al tempo: Capacità del prodotto di fornire adeguati livelli di elaborazione, velocità e tempi di risposta;
	\item Utilizzo delle risorse: Capacità del prodotto di utilizzare le risorse in maniera adeguata;
	\item Conformità: Capacità del prodotto di aderire a standard sull'efficienza.
\end{itemize}
\subsubsection{Metriche}
Visto che il proponente non ha incluso dettagli relativi alla qualità dell'efficienza, non verranno proposte metriche per questa sezione.

\subsection{Usabilità}
L'usabilità è la capacità del prodotto di essere compreso ed utilizzato dall'utente senza difficoltà tenendo conto certe condizioni.

\subsubsection{Obiettivi}
\begin{itemize}
	\item Comprensibilità: Capacità del prodotto di visualizzare le varie funzionalità del software e permette all'utente di capire se il software è indicato per le sue esigenze;
	\item Apprendibilità: Capacità del prodotto di aumentare nel tempo l'abilità dell'utente di sfruttare il software;
	\item Operabilità: Capacità del prodotto che permette agli utenti di farne uso per i loro scopi;
	\item Attrattiva: Capacità del prodotto di rendere più piacevolo l'utilizzo del software;
	\item Conformità: Capacità del prodotto di aderire a standard relativi all'usabilità.
\end{itemize}

\subsubsection{Metriche}
Viene specificata la facilità con cui l'utente riesce a raggiungere ciò che vuole attraverso il conteggio del numero di tocchi o click necessari al suo raggiungimento.\\
Si considera la capacità dell'operatore di visualizzare la propria lista delle task:\\
Numero di tocchi o click preferibili < 4;\\
accettabile < 6. \\ \space \\
Viene specificata la facilità con cui l'utente riesce a raggiungere ciò che vuole attraverso il conteggio dei secondi necessari al suo raggiungimento.\\
Si considera la capacità dell'operatore di visualizzare la propria lista delle task:\\
valore dei secondi preferibile < 15;\\
accettabile < 40. \\ \space \\
Viene specificata la profondità gerarchica massima dei collegamenti e delle funzionalità presenti all'interno del software:\\
valore preferibile < 4;\\
accettabile < 6.

\subsection{Manutenibilità}
Capacità del prodotto di essere modificato anche in futuro.
\subsubsection{Obiettivi}
\begin{itemize}
	\item Analizzabilità: Facilità con cui è possibile interpretare il codice del software;
	\item Modificabilità: Capacità per cui risulta non troppo oneroso modificare il codice del software;
	\item Stabilità: Capacità del software di evitare errori inaspettati derivanti da modifiche errate;
	\item Testabilità: Capacità del prodotto di essere testato al fine di validare le modifiche al codice sorgente.
\end{itemize}
\subsubsection{Metriche}
Viene specificata la leggibilità del software attraverso la seguente formula:
\begin{center}
	$L = $Numero di linee di codice commentate$ / $Numero di linee di codice
\end{center}
Con i valori:\\
preferibile > 0.15;\\
accettabile > 0.10.

\subsection{Portabilità}
La portabilità è la capacità del software di poter funzionare senza tener conto di uno specifico ambiente di lavoro.
\subsubsection{Obiettivi}
\begin{itemize}
	\item Adattabilità: Capacità del prodotto di essere adattato per diversi ambienti operativi;
	\item Installabilità: Capacità del prodotto di essere installato in uno specificato ambiente operativo;
	\item Conformità: Capacità del software di aderire a standard relativi alla portabilità;
	\item Sostituibilità: Capacità del software di sostituire un altro prodotto con le stesse funzionalità.
\end{itemize}
\subsubsection{Metriche}
IL software dovrà eseguire solamente su ambiente Docker, quindi non sono necessarie varie metriche.
