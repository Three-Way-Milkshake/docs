\section{Introduzione}
\subsection{Scopo del documento}
Il presente documento ha lo scopo di documentare le strategie di verifica e validazione che il gruppo \textbf{\textit{Three Way Milkshake}} ha deciso di adottare relativi al progetto \textbf{PORTACS}, per raggiungere gli obiettivi di qualità processo e prodotto.

\subsection{Scopo del prodotto}
Questo capitolato si concentra sulla realizzazione di un software che coordini lo spostamento di varie unità in una determinata griglia di movimento.
Ogni unità (che può rappresentare un robot, un muletto o un'automobile) ha un punto di partenza nella griglia, una velocità massima e una lista di punti denominati "Points Of Interest" (POI) che deve raggiungere.\\

\subsection{Glossario}
Nel seguente documento sono presenti termini tecnici, specifici o ambigui. Per semplificare la lettura viene fornito un glossario, reperibile nel file Glossario. I termini presenti in questo documento che verranno spiegati meglio nel Glossario vengono contrassengati dalla lettera \textbf{G} come pedice, per esempio Prova$_G$.
\subsection{Riferimenti}

\subsection{Riferimenti normativi}
\begin{itemize}
	\item \textit{Norme di progetto v\_ 1.0.0};
	\item Specifica tecnico-economica e organigramma: \\ \uline{\url{https://www.math.unipd.it/~tullio/IS-1/2020/Progetto/RO.html}}
	\item Regolamento progetto didattico - slide del corso di Ingegneria del Software: \\ \uline{\url{https://www.math.unipd.it/~tullio/IS-1/2020/Dispense/P1.pdf}}
\end{itemize}
\subsection{Riferimenti informativi}
\begin{itemize}
	\item Slide L12 del corso Ingegneria del Software - Qualità del Software:
	\uline{\url{https://www.math.unipd.it/~tullio/IS-1/2020/Dispense/L12.pdf}}
	\item Slide L13 del corso Ingegneria del Software - Qualità di Processo:
	\uline{\url{https://www.math.unipd.it/~tullio/IS-1/2020/Dispense/L13.pdf}}
	\item Slide L14 del corso Ingegneria del Software - Verifica e Validazione: introduzione :
	\uline{\url{https://www.math.unipd.it/~tullio/IS-1/2020/Dispense/L14.pdf}}
\end{itemize}