

\section{Qualità del processo}

\subsection{Scopo}
Per valutare  la qualità del prodotto, il gruppo Three Way Milkshake ha deciso di avvalersi degli standard ISO/IEC$_G$ 12207:1995 e ISO/IEC$_G$ 15504, semplificandoli e riadattandoli in base alle esigenze del gruppo di lavoro.\\
I processi individuati sono presentati di seguito.

\subsection{Processi di Sviluppo}

	\subsubsection{Analisi dei Requisiti}
		\paragraph{Metriche} 
		\begin{enumerate}
		\item []
			 \textbf{Requisiti obbligatori soddisfatti (PROS)}\\
			Indica la quantità di requisiti obbligatori soddisfatti rispetto al totale.
			\begin{itemize}
				\item \textbf{Misurazione:} percentuale; $\frac{requisiti\_obbligatori\_soddisfatti}{requisiti\_obbligatori\_totali}$;
				\item \textbf{Valore preferibile:} 100\%;
				\item \textbf{Valore accettabile:} 100\%.
			\end{itemize}
		\end{enumerate}	
	\subsubsection{Progettazione}	
		\paragraph{Metriche} 
		\begin{enumerate}
		\item []
			\textbf{Accoppiamento tra classi ed oggetti (CBO)}\\ 
			Indica l'accoppiamento tra classi e oggetti; due classi si dicono accoppiate se una utilizza metodi o variabili dell'altra.
			\begin{itemize}
				\item \textbf{Misurazione:} valore intero;
				\item \textbf{Valore preferibile:} 0$\leq$CBO$\leq$1;
				\item \textbf{Valore accettabile:} 0$\leq$CBO$\leq$6.
			\end{itemize}
		\end{enumerate}
	\subsubsection{Codifica}
		\paragraph{Metriche} 
		\begin{enumerate}
		\item[]
		
			\textbf{Profondità delle gerarchie(DEP)}\\
			Indica la profondità delle gerarchie nel codice sviluppato.
			\begin{itemize}
				\item \textbf{Misurazione:} valore intero;
				\item \textbf{Valore preferibile:} DEP$\leq$2;
				\item \textbf{Valore accettabile:} DEP$\leq$3.
			\end{itemize}
\pagebreak
		\item[]
			\textbf{Livello di Annidamento (LEV)}\\
			Indica il livello di annidamento nei vari metodi presenti nel codice prodotto.
			\begin{itemize}
				\item \textbf{Misurazione:} valore intero;
				\item \textbf{Valore preferibile:} 1$\leq$LEV$\leq$3;
				\item \textbf{Valore accettabile:} 1$\leq$LEV$\leq$6.
			\end{itemize}
		\item[]
			\textbf{Parametri per metodo (PAR)}\\
			Indica il numero di parametri presenti nei metodi sviluppati nel codice.
			\begin{itemize}
				\item \textbf{Misurazione:} valore intero;
				\item \textbf{Valore preferibile:} PAR$\leq$4;
				\item \textbf{Valore accettabile:} PAR$\leq$6.
			\end{itemize}
		\item[]
			\textbf{Rapporto Codice-Commenti (RCC)}\\
			Indica il rapporto tra le linee di codice e le linee di commento all'interno del file.
			\begin{itemize}
				\item \textbf{Misurazione:} valore decimale; $\frac{linee\_codice}{linee\_commento}$;
				\item \textbf{Valore preferibile:} RCC$\geq$0.4;
				\item \textbf{Valore accettabile:} RCC$\geq$0.2.
			\end{itemize}
		\end{enumerate}

\subsection{Processi di Supporto}
	\subsubsection{Pianificazione}
		\paragraph{Metriche} 
		\begin{enumerate}
		\item[]
			\textbf{Budget at Completion (BAC)}\\
			Indica il budget totale allocato per il progetto
			\begin{itemize}
				\item \textbf{Misurazione:} valore intero;
				\item \textbf{Valore preferibile:} \textit{preventivo};
				\item \textbf{Valore accettabile:} \textit{preventivo}-5\%$\leq$BAC$\leq$\textit{preventivo}+5\%.
			\end{itemize}
		\item[]
			\textbf{Earned Value (EV)}\\
			Indica la quantità di guadagno ottenuta dal lavoro effettuato fino al momento del calcolo.
			\begin{itemize}
				\item \textbf{Misurazione:} \textit{preventivo} $\cdot$ \%$\_$lavoro$\_$pianificato;
				\item \textbf{Valore preferibile:} EV$\geq$0;
				\item \textbf{Valore accettabile:} EV$\geq$0.
			\end{itemize}
		\item[]
			\textbf{Planned Value (PV)}\\
			Indica la quantità di guadagno stimata sul lavoro pianificato al momento del calcolo.
			\begin{itemize}
				\item \textbf{Misurazione:} \textit{preventivo} $\cdot$ \%$\_$lavoro$\_$pianificato;
				\item \textbf{Valore preferibile:} PV$\geq$0;
				\item \textbf{Valore accettabile:} PV$\geq$0;
			\end{itemize}
		\item[]
			\textbf{Schedule Variance (SV)}\\
			Indica l'anticipo o il ritardo del lavoro effettuato rispetto alla pianificazione.
			\begin{itemize}
				\item \textbf{Misurazione:} EV-PV;
				\item \textbf{Valore preferibile:} SV$\geq$0;
				\item \textbf{Valore accettabile:} SV=0.
			\end{itemize}
		\item[]
			\textbf{Actual Cost (AC)}\\
			Il denaro speso fino al momento del calcolo.
			\begin{itemize}
				\item \textbf{Misurazione:} valore intero;
				\item \textbf{Valore preferibile:} 0$\leq$AC$\leq$PV;
				\item \textbf{Valore accettabile:} 0$\leq$AC$\leq$\textit{budget}.
			\end{itemize}
		\item[]
			\textbf{Cost Variance (CV)}\\
			Il discostamento tra il costo del lavoro ad ora effettuato e il costo preventivato.
			\begin{itemize}
				\item \textbf{Misurazione:} EV-AC;
				\item \textbf{Valore preferibile:} CV$\geq$0;
				\item \textbf{Valore accettabile:} CV$\geq$0.
			\end{itemize}
		\end{enumerate}
	\subsubsection{Verifica}
		\paragraph{Metriche} 
		\begin{enumerate}
		\item[]	
			\textbf{Code Coverage (CC)}\\
			Indica la quantità di codice attraversato durante l'esecuzione dei test; aiuta a valuare la completezza dei test.
			\begin{itemize}
				\item \textbf{Misurazione:} percentuale; $\frac{linee\_codice\_verificate}{linee\_codice\_totali}$;
				\item \textbf{Valore preferibile:} 100\%;
				\item \textbf{Valore accettabile:} 75\%.
			\end{itemize}
		\end{enumerate}
	\subsubsection{Documentazione}
		\paragraph{Metriche} 
		\begin{enumerate}
		\item[]
			\textbf{Indice di Gulpease(IG)}\\
			Esprime una valutazione della qualità prodotta, stimandone la leggibilità.
			\begin{itemize}
				\item \textbf{Misurazione:} [ $89+ \frac{(300-num\_frasi-10\cdot num\_lettere)}{num\_parole}$ ];
				\item \textbf{Valore preferibile:} 80$\leq$IG$\leq$100;
				\item \textbf{Valore accettabile:} 50$\leq$IG$\leq$100.
			\end{itemize}
		\end{enumerate}
\pagebreak
\subsubsection{Tabella riassuntiva}
%tabella
\begin{table}[H]
	\begin{center}
				\caption{Tabella riassuntiva metriche di processo}
		\begin{tabular}{cc!{\color[HTML]{9b240a}\vrule width 0.05cm}cc}
			\rowcolorhead
			\headertitle{Codice} & \headertitle{Tipo Processo} & \headertitle{Valori Preferibili} & \headertitle{Valori Accettabili} \\
			
			PROS & Analisi dei Requisiti & 100\% & 100\%\\
			CBO & Progettazione & 0$\leq$CBO$\leq$1 & 0$\leq$CBO$\leq$6\\
			DEP & Codifica & DEP$\leq$2 & DEP$\leq$3\\
			LEV & Codifica & 1$\leq$LEV$\leq$3 & 1$\leq$LEV$\leq$6\\
			PAR & Codifica & PAR$\leq$4 & PAR$\leq$6\\
			RCC & Codifica & RCC$\geq$0.4 & RCC$\geq$0.2\\
			BAC & Pianificazione & \textit{preventivo} & \textit{preventivo}$\pm$5\%\\
			EV & Pianificazione & EV$\geq$0 & EV$\geq$0\\
			PV & Pianificazione & PV$\geq$0 & PV$\geq$0\\
			SV & Pianificazione & SV$\geq$0 & SV=0\\
			AC & Pianificazione & 0$\leq$AC$\leq$PV & 0$\leq$AC$\leq$\textit{budget}\\
			CV & Pianificazione & CV$\geq$0 & CV$\geq$0\\
			CC & Verifica & 100\% & 75\%\\
			IG & Documentazione & 80$\leq$IG$\leq$100 & 50$\leq$IG$\leq$100\\
		\end{tabular}

	\end{center}
\end{table}