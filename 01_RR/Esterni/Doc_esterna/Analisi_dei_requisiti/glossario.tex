\usepackage[toc,acronym]{glossaries}
\makeglossaries

\newglossaryentry{usecase}{name={caso d'uso}, plural={casi d'uso},%
	description={Un caso d'uso è un insieme di scenari (sequenze di azioni) che hanno in comune uno scopo finale (obiettivo) per un utente (attore)}}

\newglossaryentry{task}{name={task}, plural={tasks},%
	description={Nel contesto del capitolato, con questo termine si identifica un compito da svolgere da parte di un unità (muletto) che consiste nel raggiungere un \acrshort{POI} e caricare o scaricare la merce}}

\newglossaryentry{scenario}{name={scenario}, plural={scenari},%
	description={Sequenza di passi che descrivono interazioni}}

\newglossaryentry{requisito}{name={requisito}, plural={requisiti},%
	description={Esistono due interpretazioni principali di un requisito}}

\newglossaryentry{progetto}{name={progetto}, plural={progetti},%
	description={Insieme di attività e compiti che devono raggiunger determinati obiettivi con specifiche fissate, hanno un inizio e una fine precisi, contano su una disponibilità di risorse limitate e consumano risorse nello svolgimento}}

\newglossaryentry{planimetria}{name={planimetria}, plural={planimetrie},%
	description={Rappresentazione in piano del magazzino con tutte le sue caratteristiche (aree non transitabili, zone di percorrenza, punti di interesse)}}

\newglossaryentry{percorrenza}{name={percorrenza}, plural={percorrenze},%
	description={Si intende i vincoli relativi alle zone di percorrenza quali sensi di marcia e numero massimo di unità che vi possono transitare}}

\newglossaryentry{github}{name={Github},%
	description={Servizio di hosting per progetti software}}

\newglossaryentry{form}{name={form},%
	description={L'interfaccia di un programma che consente a un utente di inserire e inviare uno o più dati}}

\newglossaryentry{combobox}{name={combobox},%
	description={Controllo grafico (widget) che permette all'utente di effettuare una scelta scrivendola in una casella di testo o selezionandola da un elenco}}

\newglossaryentry{capitolato}{name={capitolato}, plural={capitolati},%
	description={Documento tecnico redatto dal cliente in cui vengono specificati i vincoli contrattuali(prezzo e scadenze) per lo sviluppo di un determinato prodotto software. Viene presentato in un bando d’appalto per trovare qualcuno che possa svolgere il lavoro richiesto}}

\newglossaryentry{bug}{name={bug},%
	description={Problema che porta al malfunzionamento del software, per esempio producendo un risultato inatteso o errato, tipicamente dovuto a un errore nella scrittura del codice sorgente di un programma}}

\newglossaryentry{attore}{name={attore}, plural={attori},%
	description={Rappresenta il ruolo che un'entità esterna assume quando interagisce con il sistema}}

\newacronym{uml}{UML}{Unified Modeling Language}

\newacronym{POI}{POI}{Point Of Interest}

\newacronym{api}{API}{Application Programming Interface}

