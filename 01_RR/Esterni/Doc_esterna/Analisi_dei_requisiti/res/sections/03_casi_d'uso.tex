\section{Casi d'uso}
\subsection{Introduzione}
Nella seguente sezione vengono esposti i casi d'uso individuati. Ogni caso d'uso viene descritto attraverso diagrammi dei casi d'uso e rappresenta uno scenario di utilizzo da parte degli attori che si interfacciano con esso.
\subsection{Attori primari}
\begin{itemize}
	\item{\textbf{Utente non autenticato:}\\
	Si riferisce ad un utente generico che non ha ancora effettuato l'accesso all'applicativo.}
	\item{\textbf{Utente autenticato:}\\
	Si riferisce ad un utente generico che ha effettuato l'accesso all'applicativo tramite il codice identificativo generato al momento dell'iscrizione;}
	\item{\textbf{Operatore:}\\
	Si riferisce ad un utente autenticato che intraprende le azioni dirette con la macchina. Può quindi scegliere se guidare l'unità oppure servissi del pilota automatico per raggiungere i vari POI.}
	\item{\textbf{Responsabile:}\\
	Si riferisce ad un utente autenticato che si occupa di inserire la lista di POI da soddisfare.}
	\item{\textbf{Amministratore:}\\
	Si riferisce ad un utente autenticato che ha il compito di creare nuovi account di operatori e di modificare la planimetria o la sua percorrenza in caso di cambiamenti del magazzino.}
\end{itemize}

\begin{itemize}
	\item 	\textbf{Attori primari:}
	\item 	\textbf{Precondizioni:}
	\item 	\textbf{Postcondizioni:}
	\item 	\textbf{Scenario principale:}
	\item 	\textbf{Descrizione:}
	\item 	\textbf{Estensioni:}
\end{itemize}

\subsection{UC1 - Login}

\begin{itemize}
	\item 	\textbf{Attori primari:} utente non autenticato;
	\item 	\textbf{Precondizioni:} l'utente non è autenticato nell'applicativo;
	\item 	\textbf{Postcondizioni:}	l'utente si è autenticato con successo come operatore, responsabile o amministratore. Il sistema rende disponibili diverse pagine e funzionalità a seconda della tipologia di utente;
	\item 	\textbf{Scenario principale:} l'utente richiede il login inserendo nell'apposito form il proprio codice personale identificativo;
	\item 	\textbf{Descrizione:} l'utente tenta di autenticarsi attraverso il suo codice personale identificativo;
	\item 	\textbf{Estensioni:} 
		\begin{itemize}
			\item UC1.1: il codice non è stato inserito correttamente dal sistema e quindi viene visualizzato un messaggio d'errore.
		\end{itemize}
\end{itemize}
\subsubsection{UC1.1 - Visualizzazione messaggio d'errore codice errato}

\begin{itemize}
	\item 	\textbf{Attori primari:} utente non autenticato;
	\item 	\textbf{Precondizioni:} l'utente ha inserito il suo codice personale identificativo;
	\item 	\textbf{Postcondizioni:} viene visualizzato un messaggio d'errore che informa l'utente che il codice identificativo è errato e di riprovare;
	\item 	\textbf{Scenario principale:} l'utente tenta di autenticarsi inserendo un codice non presente nel sistema o errato;
	\item 	\textbf{Descrizione:} l'utente visualizza un messaggio d'errore in seguito al fatto di aver inserito un codice errato o non presente nel sistema e viene chiesto di riprovare o di recarsi dall'amministratore.
	
\end{itemize}

\subsection{UC2 - Registrazione nuovo utente}

\begin{itemize}
	\item 	\textbf{Attori primari:} amministratore;
	\item 	\textbf{Precondizioni:}	l'amministratore intende registrare nel sistema un nuovo lavoratore assunto nell'azienda non ancora registrato nell'applicativo.
	\item 	\textbf{Postcondizioni:} l’utente è registrato nel sistema correttamente come responsabile o operatore;
	\item 	\textbf{Scenario principale:} l'amministratore inserisce i dati personali del lavoratore che vuole registrare nell'applicativo specificandone il ruolo che deve ricoprire all'interno del magazzino (responsabile o operatore);
	\item 	\textbf{Descrizione:} per effettuare l'aggiunta di un nuovo utente, l'amministratore deve compilare i dati dell'account che non deve essere presente all'interno del sistema. Il nuovo utente può essere un nuovo responsabile o un operatore.

\end{itemize}

\subsubsection{UC2.1 - Inserimento dati utente}
\begin{itemize}
	\item 	\textbf{Attori primari:} amministratore;
	\item 	\textbf{Precondizioni:}	l'amministratore sta eseguendo la registrazione del lavoratore nel sistema;
	\item 	\textbf{Postcondizioni:} l'amministratore ha inserito tutti i campi del form di registrazione richiesti;
	\item 	\textbf{Scenario principale:} l'amministratore compila tutti i campi del form richiesti per la registrazione, ovvero:
	\begin{itemize}
		\item inserisce il nome del lavoratore (UC2.1.1);
		\item inserisce il cogonome del lavoratore (UC2.1.2);
		\item inserisce il ruolo del lavoratore, ossia se è un operatore o un responsabile (UC2.1.3);
	\end{itemize}
	\item 	\textbf{Descrizione:} per effettuare la registrazione, l'amministratore deve fornire i seguenti dati dell'utente:
	\begin{itemize}
		\item nome;
		\item cognome
		\item ruolo (responsabile, operatore).
	\end{itemize}

\end{itemize}

\paragraph{UC2.1.1 - Inserimento nome}
\begin{itemize}
	\item 	\textbf{Attori primari:} amministratore;
	\item 	\textbf{Precondizioni:} il sistema ha reso disponibile il campo del form per inserire il nome del lavoratore;
	\item 	\textbf{Postcondizioni:} l'amministratore ha compilato il campo con il nome;
	\item 	\textbf{Scenario principale:} l'amministratore compila il campo del form relativo al nome del lavoratore;
	\item 	\textbf{Descrizione:} per effettuare l'aggiunta di un nuovo utente, l'amministratore deve inserire il nome del lavoratore che si intende registrare.

\end{itemize}

\paragraph{UC2.1.2 - Inserimento cognome}

\begin{itemize}
	\item 	\textbf{Attori primari:} amministratore;
	\item 	\textbf{Precondizioni:} il sistema ha reso disponibile il campo del form per inserire il cognomo del lavoratore;
	\item 	\textbf{Postcondizioni:} l'amministratore ha compilato il campo con il cognome;
	\item 	\textbf{Scenario principale:} l'amministratore compila il campo del form relativo al cognome del lavoratore;
	\item 	\textbf{Descrizione:} per effettuare l'aggiunta di un nuovo utente, l'amministratore deve inserire il cognome del lavoratore che si intende registrare.
	
\end{itemize}

\paragraph{UC2.1.3 - Inserimento ruolo}

\begin{itemize}
	\item 	\textbf{Attori primari:} amministratore;
	\item 	\textbf{Precondizioni:} il sistema ha reso disponibile il campo del form per inserire il ruolo del lavoratore;
	\item 	\textbf{Postcondizioni:} l’amministratore ha compilato il campo con il ruolo;
	\item 	\textbf{Scenario principale:} l’amministratore sceglie tramite una combobox il ruolo che deve intraprendere il nuovo lavoratore che può essere responsabile o operatore;
	\item 	\textbf{Descrizione:} per effettuare l’aggiunta di un nuovo utente, l’amministratore deve inserire il ruolo del lavoratore che si intende registrare. Può scegliere tra responsabile e operatore.
	
\end{itemize}


\subsubsection{UC2.2 - Conferma e invio dei dati}

\begin{itemize}
	\item 	\textbf{Attori primari:} amministratore;
	\item 	\textbf{Precondizioni:} l’amministratore ha compilato il form per l’inserimento dei dati del nuovo utente e rende disponibile un pulsante per la conferma;
	\item 	\textbf{Postcondizioni:} viene visualizzato a video un messaggio con la conferma della ricezione dei dati e il codice identificativo;
	\item 	\textbf{Scenario principale:} l’amministratore preme il pulsante di conferma dopo aver completato tutti i campi del form;
	\item 	\textbf{Descrizione:} l’amministratore preme il pulsante per la conferma e l’invio dei dati. A schermo viene visualizzato un messaggio con l’avvenuta registrazione e il codice identificativo relativo all’account registrato.

\end{itemize}

\subsubsection{UC2.3 - Visualizzazione messaggio d'errore account già presente}

\begin{itemize}
	\item 	\textbf{Attori primari:} amministratore;
	\item 	\textbf{Precondizioni:} i dati del lavoratore sono già presenti nel sistema;
	\item 	\textbf{Postcondizioni:} viene visualizzato a video un messaggio d’errore per informare l’amministratore che il lavoratore è già presente nel sistema;
	\item 	\textbf{Scenario principale:} l’amministratore tenta di registrare nell’applicativo un lavoratore già registrato;
	\item 	\textbf{Descrizione:} l’amministratore visualizza un messaggio d’errore dovuto al fatto di aver inserito i dati di un utente già presente nel sistema.
\end{itemize}

\subsection{UC3 - Modifica utente}

\begin{itemize}
	\item 	\textbf{Attori primari:} amministratore;
	\item 	\textbf{Precondizioni:} l’amministratore intende modificare il profilo dell’utente già registrato all’interno dell’applicativo;
	\item 	\textbf{Postcondizioni:} l’amministratore ha cambiato alcuni campi dell’account di un utente;
	\item 	\textbf{Scenario principale:} l’amministratore modifica un campo del profilo dell’utente;
	\item 	\textbf{Descrizione:} per modificare un campo dell’account di un lavoratore, l’amministratore deve modificare quello corrente con quello corretto;

\end{itemize}

\subsubsection{UC3.1 - Modifica nome}

\begin{itemize}
	\item 	\textbf{Attori primari:} amministratore;
	\item 	\textbf{Precondizioni:} il sistema ha reso disponibile il campo del form per modificare il nome del lavoratore;
	\item 	\textbf{Postcondizioni:}  l’amministratore ha compilato il campo con il nome aggiornato;
	\item 	\textbf{Scenario principale:} l’amministratore compila il campo del form relativo al nome del lavoratore con il nome aggiornato;
	\item 	\textbf{Descrizione:} per effettuare la modifica del campo relativo al nome del lavoratore, l’amministratore aggiorna il form.
\end{itemize}

\subsubsection{UC3.2 - Modifica cognome}

\begin{itemize}
	\item 	\textbf{Attori primari:} amministratore;
	\item 	\textbf{Precondizioni:} il sistema ha reso disponibile il campo del form per modificare il cognome del lavoratore;
	\item 	\textbf{Postcondizioni:} l’amministratore ha compilato il campo con il cognome aggiornato;
	\item 	\textbf{Scenario principale:} l’amministratore compila il campo del form relativo al cognome del lavoratore con il cognome aggiornato;
	\item 	\textbf{Descrizione:} per effettuare la modifica del campo relativo al cognome del lavoratore, l’amministratore aggiorna il form.

\end{itemize}
\subsubsection{UC3.3 - Modifica ruolo}

\begin{itemize}
	\item 	\textbf{Attori primari:} amministratore;
	\item 	\textbf{Precondizioni:} il sistema ha reso disponibile il campo del form per modificare il ruolo del lavoratore;
	\item 	\textbf{Postcondizioni:} l’amministratore ha compilato il campo con il ruolo aggiornato (responsabile o operatore);
	\item 	\textbf{Scenario principale:} l’amministratore compila il campo del form relativo al ruolo del lavoratore con il ruolo aggiornato scegliendo tra responsabile o operatore;
	\item 	\textbf{Descrizione:} per effettuare la modifica del campo relativo al ruolo del lavoratore, l’amministratore aggiorna il form.

\end{itemize}
\subsection{UC4 - Gestione task}

\begin{itemize}
	\item 	\textbf{Attori primari:} responsabile;
	\item 	\textbf{Precondizioni:} il responsabile è autenticato nel sistema e il sistema rende disponibile l’interfaccia per la gestione delle task che verranno assegnati alle unità;
	\item 	\textbf{Postcondizioni:} la lista delle task è stata aggiornata;
	\item 	\textbf{Scenario principale:} il responsabile effettua le operazioni necessarie per la gestione della lista delle task che verranno assegnate dal sistema alle unità, esse possono essere:
	\begin{itemize}
		\item l’inserimento di una nuova task (UC4.1) con la relativa priorità (UC4.2) e il POI a cui fa riferimento, in cui bisogna scaricare la merce (UC4.3);
		\item la conferma dell’inserimento della nuova task (UC4.3);
		\item l’eliminazione di una task dalla lista (UC4.4);
		\item la modifica della priorità di una task esistente (UC4.5);
	\end{itemize}
	\item 	\textbf{Descrizione:} lo scarico delle merci in un determinato punto di interesse viene chiamato task. Il responsabile deve inserire nel sistema quali task devono essere completate e con quale priorità. L’applicativo riceve le informazioni, le ordina e le affida alle unità in base alle esigenze. 

\end{itemize}

\subsubsection{UC4.1 - Inserimento nuova task}
\begin{itemize}
	\item 	\textbf{Attori primari:} responsabile;
	\item 	\textbf{Precondizioni:}è resa disponibile l’interfaccia per l’inserimento di una nuova task;
	\item 	\textbf{Postcondizioni:} il responsabile ha aggiunto con successo la task alla lista. Il sistema assegnerà la nuova task a un muletto la quale la visualizzerà nella propria lista di compiti (UC8);
	\item 	\textbf{Scenario principale:} il responsabile preme l’apposito pulsante per l’aggiunta di una nuova task; 
	\item 	\textbf{Descrizione:} il responsabile inserisce nella lista dei POI da soddisfare dagli operatore un nuovo POI;
	\item 	\textbf{Inclusioni:}
	\begin{itemize}
		\item UC4.2 Inserimento priorità;
		\item UC4.3 Inserimento relativo POI.
	\end{itemize}
\end{itemize}

\subsubsection{UC4.2 - Inserimento priorità}

\begin{itemize}
	\item 	\textbf{Attori primari:} responsabile;
	\item 	\textbf{Precondizioni:} il responsabile sta inserendo una nuova task;
	\item 	\textbf{Postcondizioni:} è stata compilata correttamente la priorità relativa alla task che si vuole inserire;
	\item 	\textbf{Scenario principale:} il responsabile gli assegna una priorità (bassa, media, alta) tramite una combobox;
	\item 	\textbf{Descrizione:} le task possono avere tre diversi gradi di priorità:
	\begin{itemize}
		\item bassa;
		\item media;
		\item alta.
	\end{itemize}
\end{itemize}

\subsubsection{UC4.3 - Inserimento relativo POI}

\begin{itemize}
	\item 	\textbf{Attori primari:} responsabile;
	\item 	\textbf{Precondizioni:} il responsabile sta inserendo una nuova task;
	\item 	\textbf{Postcondizioni:} è stato assegnato correttamente il POI relativo alla nuova task;
	\item 	\textbf{Scenario principale:}
	\begin{itemize}
		\item visualizza la mappa con tutti i POI (UC6) e la lista di tutti i POI di scarico(UC12);
		\item seleziona il POI in cui si vuole scaricare la merce;
		\item viene confermata la selezione;
	\end{itemize}
	\item 	\textbf{Descrizione:} la task rappresenta lo scarico merci in un determinato punto di interesse del magazzino, quindi ad ogni task deve essere affidato il relativo POI.
\end{itemize}
\subsubsection{UC4.4 - Conferma inserimento}

\begin{itemize}
	\item 	\textbf{Attori primari:} responsabile;
	\item 	\textbf{Precondizioni:} il responsabile ha creato la nuova task e il sistema rende disponibile il pulsante di conferma;
	\item 	\textbf{Postcondizioni:} il sistema ha ricevuto la nuova task e lo assegna a un’unità che lo dovrà soddisfare;
	\item 	\textbf{Scenario principale:} il responsabile conferma l’inserimento della nuova task tramite un apposito pulsante;
	\item 	\textbf{Descrizione:} il responsabile visualizza il pulsante di conferma per inserire la task.
\end{itemize}

\subsubsection{UC4.5 - Eliminazione task dalla lista}

\begin{itemize}
	\item 	\textbf{Attori primari:} responsabile;
	\item 	\textbf{Precondizioni:} il responsabile sta visualizzando la lista di task(UC13);
	\item 	\textbf{Postcondizioni:} il responsabile ha eliminato una task dalla lista;
	\item 	\textbf{Scenario principale:} il responsabile seleziona dalla lista delIe task quella che intende eliminare e procede alla cancellazione premendo l’apposito pulsante;
	\item 	\textbf{Descrizione:} il responsabile si accorge che non è più necessario lo svolgimento di un compito, quindi lo elimina dalla lista delle task.

\end{itemize}

\subsubsection{UC4.6 - Modifica priorità}
\begin{itemize}
	\item 	\textbf{Attori primari:} responsabile;
	\item 	\textbf{Precondizioni:} il responsabile sta visualizzando la lista di task(UC13);
	\item 	\textbf{Postcondizioni:} la priorità della task selezionata è stata aggiornata;
	\item 	\textbf{Scenario principale:} il responsabile seleziona la task che intende modificare, viene aperto un menu a tendina nel quale è possibile cambiare la priorità.
	\item 	\textbf{Descrizione:} il responsabile si accorge che la priorità di una task è errata o necessita di un aggiornamento, quindi procede con la modifica.
\end{itemize}

\subsection{UC5 - Logout}

\begin{itemize}
	\item 	\textbf{Attori primari:} utente autenticato;
	\item 	\textbf{Precondizioni:} l’utente si trova in base e quindi può premere sul bottone che gli permette di fare il logout all’applicativo;
	\item 	\textbf{Postcondizioni:} utente viene disconnesso dal sistema;
	\item 	\textbf{Scenario principale:} utente autenticato richiede il logout;
	\item 	\textbf{Descrizione:} l’utente vuole effettuare il logout dall’applicativo:
	\begin{itemize}
		\item se amministratore o responsabile, può effettuare il logout dall’applicativo in qualsiasi momento;
		\item se operatore, può effettuare il logout solo dopo aver raggiunto la base, quindi aver finito il proprio turno.
	\end{itemize}
\end{itemize}


\subsection{UC6 - Visualizzazione mappa}
\begin{itemize}
	\item 	\textbf{Attori primari:} amministratore, responsabile;
	\item 	\textbf{Precondizioni:} gli attori sono autenticati nel sistema;
	\item 	\textbf{Postcondizioni:} gli attori visualizzano la mappa del magazzino.
	\item 	\textbf{Scenario principale:} il responsabile e l’amministratore, una volta autenticati, visualizzano la mappa completa del magazzino;
	\item 	\textbf{Descrizione:}il responsabile e l’amministratore, per portare al termine i loro compiti, devono visualizzare la mappa del magazzino. Gli elementi della mappa sono:
	\begin{itemize}
		\item POI di carico; punto in cui i muletti prelevano il carico da scaricare per soddisfare la propria lista di task. Ogni volta che completano la loro lista di compiti (ma non hanno finito il turno), devono tornare il questo punto;
		\item POI di scarico; punti in cui i muletti devono scaricare la merce prelevata. Sono i luoghi che fanno riferimento le task;
		\item POI di sosta; punto in cui gli operatori partono con il proprio muletto a inizio turno e arrivano alla fine del turno.
		\item zona di percorrenza; sono le strade in cui i muletti possono transitare, hanno delle loro caratteristiche:
		\begin{itemize}
			\item senso di marcia;
			\item numero massimo di unità che può transitare;
		\end{itemize}
		\item aree non transitabili (muri, scaffali..).
	\end{itemize}

\end{itemize}


\subsubsection{UC6.1 - Visualizzazione posizione muletti in real-time}
\begin{itemize}
	\item 	\textbf{Attori primari:} amministratore, responsabile;
	\item 	\textbf{Precondizioni:} gli attori sono autenticati nel sistema e visualizzano correttamente la mappa;
	\item 	\textbf{Postcondizioni:} gli attori visualizzano gli spostamenti dei muletti in real-time nella mappa;
	\item 	\textbf{Scenario principale:} il responsabile e l’amministratore una volta autenticati, visualizzano la mappa completa del magazzino e i muletti muoversi;
	\item 	\textbf{Descrizione:} il responsabile e l’amministratore, per portare al termine i loro compiti, devono visualizzare la mappa del magazzino e gli spostamenti dei muletti;
\end{itemize}

\subsection{UC7 - Gestione mappa}

\begin{itemize}
	\item 	\textbf{Attori primari:} amministratore;
	\item 	\textbf{Precondizioni:} l’amministratore è autenticato nel sistema e viene reso disponibile dal sistema un pulsante per la modifica della mappa;
	\item 	\textbf{Postcondizioni:} la mappa è stata modificata dall’amministratore;
	\item 	\textbf{Scenario principale:} l’amministratore una volta autenticato preme il pulsante per la gestione della mappa dal quale può effettuare le seguenti operazioni:
	\begin{itemize}
		\item modificare la mappa (UC7.1);
		\item gestire i punti d’interesse (UC7.4);
	\end{itemize}
	\item 	\textbf{Descrizione:} l’amministratore ha il compito di gestire la mappa e tenere aggiornati i cambiamenti reali del magazzino nell’applicativo.
\end{itemize}


\subsubsection{UC7.1 - Modifica mappa}
\begin{itemize}
	\item 	\textbf{Attori primari:} amministratore;
	\item 	\textbf{Precondizioni:}viene resa disponibile dal sistema l’interfaccia per la modifica della mappa;
	\item 	\textbf{Postcondizioni:} la mappa è stata modificata dall’amministratore;
	\item 	\textbf{Scenario principale:}l’amministratore dopo aver premuto il pulsante per la modifica, visualizza l’interfaccia per gestire i cambiamenti della mappa tra cui può scegliere tramite un menù a tendina se:
	\begin{itemize}
		\item modificare la planimetria del magazzino (UC7.2);
		\item modificare la percorrenza del magazzino, per esempio i sensi di marcia e le corsie (UC7.3);
		\item gestire i POI (UC7.4);
	\end{itemize}
	e viene visualizzata l’intera mappa (UC6).
Terminata la modifica, l’amministratore salva tramite l’apposito pulsante di conferma;
	\item 	\textbf{Descrizione:} l’amministratore ha il compito di tenere aggiornata la mappa dai cambiamenti reali del magazzino, modificandone la planimetria, la percorrenza e i POI presenti.
	\item 	\textbf{Specializzazione:} 
	\begin{itemize}
		\item UC7.2 - Modifica planimetria
		\item UC7.3 - Modifica percorrenza
		\item UC7.4 - Gestione POI
	\end{itemize}
	\item \textbf{Estensione}:
	\begin{itemize}
		\item UC7.5 - Visualizzazione messaggio d’errore operazione non permessa.
	\end{itemize}Visualizzazione messaggio d’errore operazione non permessa.
\end{itemize}


\subsubsection{UC7.2 - Modifica planimetria}
\begin{itemize}
	\item 	\textbf{Attori primari:} amministratore;
	\item 	\textbf{Precondizioni:} viene resa disponibile dal sistema l’interfaccia per la modifica della planimetria della mappa;
	\item 	\textbf{Postcondizioni:} la mappa è stata modificata dall’amministratore;
	\item 	\textbf{Scenario principale:} vengono visualizzati degli strumenti per la modifica della planimetria:
	\begin{itemize}
		\item ampliamento;
		\item riduzione;
		\item aggiunta, rimozione e modifica zone non transitabili.
	\end{itemize}
	Una volta raggiunto il risultato desiderato, l’amministratore conferma tramite il pulsante di salvataggio;
	\item 	\textbf{Descrizione:} il magazzino con il passare del tempo, può apportare dei cambiamenti nella planimetria. Possono venire modificati:
	\begin{itemize}
		\item la dimensione del magazzino (ampliarlo o diminuirlo);
		\item le zone in cui non è permessa la transizione dei mezzi (scaffali, muri ect).
	\end{itemize}
\end{itemize}

\subsubsection{UC7.3 - Modifica percorrenza}
\begin{itemize}
	\item 	\textbf{Attori primari:} amministratore;
	\item 	\textbf{Precondizioni:}  l’amministratore è autenticato nel sistema e viene reso disponibile dal sistema l’interfaccia per la modifica della percorrenza della mappa;
	\item 	\textbf{Postcondizioni:}la mappa è stata modificata dall’amministratore;
	\item 	\textbf{Scenario principale:} viene visualizzata la mappa con le caratteristiche che ogni corsia ha (senso di marcia e numero massimo di unità). L’amministratore deve premere sopra la corsia che intende cambiare per aprire un pop-up e inserire gli aggiornamenti. Una volta raggiunto il risultato desiderato, l’amministratore conferma tramite il pulsante di salvataggio;
	\item 	\textbf{Descrizione:} il magazzino con il passare del tempo, può apportare dei cambiamenti nella percorrenza, sarà possibile modificare:
	\begin{itemize}
		\item sensi unici;
		\item numero massimo di unità nelle corsie.
	\end{itemize}
\end{itemize}

\subsubsection{UC7.4 - Gestione POI}
\begin{itemize}
	\item 	\textbf{Attori primari:}
	\item 	\textbf{Precondizioni:}
	\item 	\textbf{Postcondizioni:}
	\item 	\textbf{Scenario principale:}
	\item 	\textbf{Descrizione:}
	\item 	\textbf{Estensioni:}
\end{itemize}

\paragraph{UC7.4.1 }
\paragraph{UC7.4.2 }
\paragraph{UC7.4.3 }
\paragraph{UC7.4.4 }
\paragraph{UC7.4.5 }
\paragraph{UC7.4.6 }
\subsubsection{UC7.5 - Visualizzazione messaggio d'errore operazione non permessa}
\begin{itemize}
	\item 	\textbf{Attori primari:}
	\item 	\textbf{Precondizioni:}
	\item 	\textbf{Postcondizioni:}
	\item 	\textbf{Scenario principale:}
	\item 	\textbf{Descrizione:}
	\item 	\textbf{Estensioni:}
\end{itemize}
\subsection{UC8}
\begin{itemize}
	\item 	\textbf{Attori primari:}
	\item 	\textbf{Precondizioni:}
	\item 	\textbf{Postcondizioni:}
	\item 	\textbf{Scenario principale:}
	\item 	\textbf{Descrizione:}
	\item 	\textbf{Estensioni:}
\end{itemize}
\subsection{UC9}
\subsection{UC10}
\subsection{UC11}
\subsection{UC12}
\subsection{UC13}
