\section{Casi d'uso}
\subsection{Introduzione}
Nella seguente sezione vengono esposti i casi d'uso individuati. Ogni caso d'uso viene descritto attraverso diagrammi dei casi d'uso e rappresenta uno scenario di utilizzo da parte degli attori che si interfacciano con esso.
\subsection{Attori primari}
\begin{itemize}
	\item{\textbf{Utente non autenticato:}\\
	Si riferisce ad un utente generico che non ha ancora effettuato l'accesso all'applicativo.}
	\item{\textbf{Utente autenticato:}\\
	Si riferisce ad un utente generico che ha effettuato l'accesso all'applicativo tramite il codice identificativo generato al momento dell'iscrizione;}
	\item{\textbf{Operatore:}\\
	Si riferisce ad un utente autenticato che intraprende le azioni dirette con la macchina. Può quindi scegliere se guidare l'unità oppure servissi del pilota automatico per raggiungere i vari POI.}
	\item{\textbf{Responsabile:}\\
	Si riferisce ad un utente autenticato che si occupa di inserire la lista di POI da soddisfare.}
	\item{\textbf{Amministratore:}\\
	Si riferisce ad un utente autenticato che ha il compito di creare nuovi account di operatori e di modificare la planimetria o la sua percorrenza in caso di cambiamenti del magazzino.}
\end{itemize}

\begin{itemize}
	\item 	\textbf{Attori primari:}
	\item 	\textbf{Precondizioni:}
	\item 	\textbf{Postcondizioni:}
	\item 	\textbf{Scenario principale:}
	\item 	\textbf{Descrizione:}
	\item 	\textbf{Estensioni:}
\end{itemize}

\subsection{UC1 - Login}

\begin{itemize}
	\item 	\textbf{Attori primari:} utente non autenticato;
	\item 	\textbf{Precondizioni:} l'utente non è autenticato nell'applicativo;
	\item 	\textbf{Postcondizioni:}	l'utente si è autenticato con successo come operatore, responsabile o amministratore. Il sistema rende disponibili diverse pagine e funzionalità a seconda della tipologia di utente;
	\item 	\textbf{Scenario principale:} l'utente richiede il login inserendo nell'apposito form il proprio codice personale identificativo;
	\item 	\textbf{Descrizione:} l'utente tenta di autenticarsi attraverso il suo codice personale identificativo;
	\item 	\textbf{Estensioni:} 
		\begin{itemize}
			\item UC1.1: il codice non è stato inserito correttamente dal sistema e quindi viene visualizzato un messaggio d'errore.
		\end{itemize}
\end{itemize}
\subsubsection{UC1.1 - Visualizzazione messaggio d'errore codice errato}

\begin{itemize}
	\item 	\textbf{Attori primari:} utente non autenticato;
	\item 	\textbf{Precondizioni:} l'utente ha inserito il suo codice personale identificativo;
	\item 	\textbf{Postcondizioni:} viene visualizzato un messaggio d'errore che informa l'utente che il codice identificativo è errato e di riprovare;
	\item 	\textbf{Scenario principale:} l'utente tenta di autenticarsi inserendo un codice non presente nel sistema o errato;
	\item 	\textbf{Descrizione:} l'utente visualizza un messaggio d'errore in seguito al fatto di aver inserito un codice errato o non presente nel sistema e viene chiesto di riprovare o di recarsi dall'amministratore.
	
\end{itemize}

\subsection{UC2 - Registrazione nuovo utente}

\begin{itemize}
	\item 	\textbf{Attori primari:} amministratore;
	\item 	\textbf{Precondizioni:}	l'amministratore intende registrare nel sistema un nuovo lavoratore assunto nell'azienda non ancora registrato nell'applicativo.
	\item 	\textbf{Postcondizioni:} l’utente è registrato nel sistema correttamente come responsabile o operatore;
	\item 	\textbf{Scenario principale:} l'amministratore inserisce i dati personali del lavoratore che vuole registrare nell'applicativo specificandone il ruolo che deve ricoprire all'interno del magazzino (responsabile o operatore);
	\item 	\textbf{Descrizione:} per effettuare l'aggiunta di un nuovo utente, l'amministratore deve compilare i dati dell'account che non deve essere presente all'interno del sistema. Il nuovo utente può essere un nuovo responsabile o un operatore.

\end{itemize}

\subsubsection{UC2.1 - Inserimento dati utente}
\begin{itemize}
	\item 	\textbf{Attori primari:} amministratore;
	\item 	\textbf{Precondizioni:}	l'amministratore sta eseguendo la registrazione del lavoratore nel sistema;
	\item 	\textbf{Postcondizioni:} l'amministratore ha inserito tutti i campi del form di registrazione richiesti;
	\item 	\textbf{Scenario principale:} l'amministratore compila tutti i campi del form richiesti per la registrazione, ovvero:
	\begin{itemize}
		\item inserisce il nome del lavoratore (UC2.1.1);
		\item inserisce il cogonome del lavoratore (UC2.1.2);
		\item inserisce il ruolo del lavoratore, ossia se è un operatore o un responsabile (UC2.1.3);
	\end{itemize}
	\item 	\textbf{Descrizione:} per effettuare la registrazione, l'amministratore deve fornire i seguenti dati dell'utente:
	\begin{itemize}
		\item nome;
		\item cognome
		\item ruolo (responsabile, operatore).
	\end{itemize}

\end{itemize}

\paragraph{UC2.1.1 - Inserimento nome}
\begin{itemize}
	\item 	\textbf{Attori primari:} amministratore;
	\item 	\textbf{Precondizioni:} il sistema ha reso disponibile il campo del form per inserire il nome del lavoratore;
	\item 	\textbf{Postcondizioni:} l'amministratore ha compilato il campo con il nome;
	\item 	\textbf{Scenario principale:} l'amministratore compila il campo del form relativo al nome del lavoratore;
	\item 	\textbf{Descrizione:} per effettuare l'aggiunta di un nuovo utente, l'amministratore deve inserire il nome del lavoratore che si intende registrare.

\end{itemize}

\paragraph{UC2.1.2 - Inserimento cognome}
\paragraph{UC2.1.3 - Inserimento ruolo}

\subsubsection{UC2.2 - Conferma e invio dei dati}
\subsubsection{UC2.1 - Visualizzazione messaggio d'errore account già presente}

\subsection{UC3}
\subsection{UC4}
\subsection{UC5}
\subsection{UC6}
\subsection{UC7}
\subsection{UC8}
\subsection{UC9}
\subsection{UC10}
\subsection{UC11}
\subsection{UC12}
\subsection{UC13}
