\section{Requisiti}
\subsection{Introduzione}
In questa sezione vengono riportati i requisiti, strutturati secondo la loro classificazione per tipologia, ovvero requisiti funzionali, requisiti prestazionali, requisiti di qualità e requisiti di vincolo.
\newline 
\subsection{Requisiti funzionali}
\renewcommand{\arraystretch}{1.5}
\rowcolors{2}{pari}{dispari}
\begin{longtable}{ 
		>{\centering}p{0.11\textwidth} 
		>{}p{0.6075\textwidth}
		>{\centering \it}p{0.16\textwidth} }
	\rowcolorhead
	\headertitle{Codice} &
	\centering \headertitle{Descrizione} &	
	\headertitle{\normalfont \textbf{Fonte}}	
	\endfirsthead	
	\endhead
RF-1-O & Un utente deve effettuare il login alla piattaforma tramite il suo codice identificativo & UC1\tabularnewline
RF-2-O & Il processo di login dell’utente non va a buon fine se il codice inserito non è corretto o non è presente nel sistema & UC1.1\tabularnewline
RF-3-O & L’amministratore può registrare un nuovo lavoratore all’interno del sistema & UC2\tabularnewline
RF-4-O & L’amministratore può creare l’account di un responsabile o di un operatore & UC2\tabularnewline
RF-4.1-O & La registrazione di un nuovo utente necessita del nome del lavoratore & UC2.1.1\tabularnewline
RF-4.2-O & La registrazione di un nuovo utente necessita del cognome del lavoratore & UC2.1.2\tabularnewline
RF-4.3-O & La registrazione di un nuovo utente necessita del ruolo del lavoratore (responsabile, operatore) & UC2.1.3\tabularnewline
RF-5-O & La fase di registrazione non va a buon fine se i dati inseriti risultano già presenti nel sistema & UC2.3\tabularnewline
RF-6-O & Il sistema permette la modifica di un utente già registrato & UC3\tabularnewline
RF-6.1-O & L’amministratore può modificare il campo nome di un account esistente & UC3.1\tabularnewline
RF-6.2-O & L’amministratore può modificare il campo cognome di un account esistente & UC3.2\tabularnewline
RF-6.3-O & L’amministratore può modificare il campo ruolo di un account esistente (responsabile, lavoratore) & UC3.3\tabularnewline
RF-7-O & Il responsabile si occupa della gestione della lista delle task & UC4\tabularnewline
RF-8-O & Il responsabile può inserire una nuova task  & UC4.1\tabularnewline
RF-8.1-O & Quando il responsabile inserisce una nuova task dovrà specificare la sua priorità  & UC4.2\tabularnewline
RF-8.2-O & Quando il responsabile inserisce una nuova task dovrà specificare il POI a cui fa riferimento & UC4.3\tabularnewline
RF-8.3-O & Quando il responsabile conferma l’inserimento di una nuova task e il sistema la assegna ad un’unità che la dovrà soddisfare & UC4.4\tabularnewline
RF-9-O & Il responsabile può eliminare una task  & UC4.5\tabularnewline
RF-10-O & Il responsabile può modificare la priorità di una task  & UC4.6\tabularnewline
RF-11-O & Il sistema permette all'utente di fare il logout dall'applicativo & UC5\tabularnewline
RF-12-O & Il sistema abilita il logout all'amministratore in qualsiasi momento & UC5\tabularnewline
RF-13-O & Il sistema abilita il logout al responsabile in qualsiasi momento & UC5\tabularnewline
RF-14-O & Il sistema abilita il logout all'operatore solo quando si trova in base & UC5\tabularnewline
RF-15-O & Il sistema permette la visualizzazione della mappa all’amministratore e ai responsabili & UC6\tabularnewline
RF-15.1-O & Il sistema permette la visualizzazione di tutti i tipi di POI nella mappa all’amministratore e ai responsabili & UC6\tabularnewline
RF-15.2-O & Il sistema permette la visualizzazione delle caratteristiche delle zone di percorrenza (senso di marcia, numero massimo di unità che possono transitare) all’amministratore e ai responsabili & UC6\tabularnewline
RF-15.2.1-O & Il sistema permette la visualizzazione delle zone non transitabili all’amministratore e ai responsabili & UC6\tabularnewline
RF-16-O & Il sistema permette la visualizzazione della posizione dei muletti in real-time sulla mappa & UC6.1\tabularnewline
RF-17-F & Il sistema permette la visualizzazione della posizione delle persone in real-time sulla mappa & Capitolato\tabularnewline
RF-18-O & L’amministratore autenticato può accedere all’interfaccia per gestire la mappa  & UC7\tabularnewline
RF-18.1-O & L’amministratore può modificare planimetria del magazzino & UC7.2\tabularnewline
RF-18.2-O & L’amministratore può modificare la percorrenza del magazzino & UC7.3\tabularnewline
RF-19-O & L’amministratore può gestire i POI & UC7.4\tabularnewline
RF-19.1-O & L'amministratore può modificare la posizione di un POI già esistente & UC7.4.1\tabularnewline
RF-19.2-O & L'amministratore può inserire un nuovo POI & UC7.4.2\tabularnewline
RF-19.2.1-O & Inserendo un nuovo POI, l'amministratore dovrà specificare la sua posizione nella mappa & UC7.4.3\tabularnewline
RF-19.2.2-O & Inserendo un nuovo POI, l'amministratore dovrà specificare il suo codice identificativo & UC7.4.4\tabularnewline
RF-19.2.3-O & Inserendo un nuovo POI, l'amministratore dovrà specificare il tipo di POI inserito (carico, scarico, base) & UC7.4.5\tabularnewline
RF-19.3-O & L'amministratore può eliminare un POI & UC7.4.6\tabularnewline
RF-20-O & La User Interface di una specifica unità attiva implementa una mappa contente i relativi POI presenti nella lista delle task da soddisfare, numerati secondo la lista & UC8.1\tabularnewline
RF-21-O & La User Interface implementa sotto alla mappa una lista ordinata contenente la task rimanenti da eseguire dell’operatore che sta usando l’unità & UC8.2\tabularnewline
RF-22-O & La mappa mostra il prossimo task da soddisfare (POI da raggiungere) & UC8.3\tabularnewline
RF-22.1-O & Nella mappa specifica dell’unità verrà evidenziato con un colore diverso il prossimo POI da raggiungere & UC8.3\tabularnewline
RF-23-O & L’operatore segnala al sistema la conclusione dell’incarico attraverso la user interface & UC9\tabularnewline
RF-24-O & La User Interface che rappresenterà ogni singola unità dovrà prevedere le 4 frecce direzionali che indicano il suggerimento del sistema & UC10\tabularnewline
RF-24.1-O & Il sistema permette all’operatore la visualizzazione di direzione e spostamento del muletto a cui è a bordo, in caso in cui nel muletto sia attiva la guida automatica & UC10\tabularnewline
RF-25-O & Nella user interface è presente un pulsante che permette di passare dalla guida manuale alla guida autonoma dell’unità & UC11.1\tabularnewline
RF-25.1-O & La User Interface del controllo manuale permette di passare alla guida autonoma & UC11.1,\textsc{verbale esterno 1}\tabularnewline
RF-26-O & Nella user interface è presente un pulsante che permette di passare dalla guida autonoma alla guida manuale dell’unità & UC11.2\tabularnewline
RF-26.1-O & La User Interface del controllo automatico permette di passare alla guida manuale & UC11.2, \textsc{verbale esterno 1}\tabularnewline
RF-27-O & Nella user interface è presente un pulsante che permette di segnalare al server un evento eccezionale & UC11.3\tabularnewline
RF-28-O & Nella user interface comparirà  un pulsante per il ritorno alla base dell’unità se l'operatore avrà concluso tutte le task assegnategli e la guida sarà impostata ad autonoma  & UC11.5\tabularnewline
RF-29-O & La User Interface che rappresenterà ogni singola unità dovrà prevedere le 4 frecce direzionali che permettono gli spostamenti manuali ed i pulsanti di start/stop & UC11.4, \textsc{verbale esterno 1}\tabularnewline
RF-30-D & Il pannello permette di visualizzare l’indicatore di velocità attuale (che avrà come massimo la velocità massima anagrafica) & Capitolato\tabularnewline
RF-31-O & Il sistema centrale pilota e coordina tutte le unità per evitare incidenti e ingorghi & Capitolato\tabularnewline
RF-32-F & il sistema fornisce il percorso migliore ad ogni unità tramite algoritmi di ricerca operativa & Capitolato\tabularnewline
RF-33-O & Il sistema permette al responsabile di visualizzare la lista di tutti i POI con il proprio tipo (carico, scarico, base) presenti nella mappa & UC12\tabularnewline
RF-34-O & Il sistema permette all'amministratore di visualizzare la lista di tutti i POI con il proprio tipo (carico, scarico, base) presenti nella mappa & UC12\tabularnewline
RF-35-O & Il responsabile ha a disposizione un pulsante per poter vedere una lista completa delle task & UC13\tabularnewline
RF-36-O & L'amministratore ha a disposizione un'interfaccia su cui può gestire le unità & UC14\tabularnewline
RF-36.1-O & L'amministratore può aggiungere una nuova unità & UC14.1\tabularnewline
RF-36.2-O & L'amministratore può eliminare un'unità & UC14.2\tabularnewline
\end{longtable}.
\newline 
\subsection{Requisiti prestazionali}
In questo progetto non sono stati rilevati alcuni requisiti prestazionali per quanto riguarda i requisiti obbligatori.
\newline 
\subsection{Requisiti di qualità}
\renewcommand{\arraystretch}{1.5}
\rowcolors{2}{pari}{dispari}
\begin{longtable}{ 
		>{\centering}p{0.11\textwidth} 
		>{}p{0.6075\textwidth}
		>{\centering \it}p{0.16\textwidth} }
	\rowcolorhead
	\headertitle{Codice} &
	\centering \headertitle{Descrizione} &	
	\headertitle{\normalfont \textbf{Fonte}}	
	\endfirsthead	
	\endhead
RQ-1-O & Diagrammi UML relativi agli use cases di progetto & Capitolato\tabularnewline
RQ-2-O & Schema design relativo alla base dati (se ritenuta necessaria) & Capitolato\tabularnewline
RQ-3-O & Documentazione delle API che saranno realizzate & Capitolato\tabularnewline
RQ-4-O & Lista dei bug risolti durante la fase di sviluppo & Capitolato\tabularnewline
RQ-5-O & Codice prodotto in formato sorgente utilizzando sistemi di versionamento del codice, quali Github o Bitbucket & Capitolato\tabularnewline
RQ-6-O & Codice sorgente di quanto realizzato & Capitolato\tabularnewline
RQ-7-O & Docker file ("\#"1) con la componente applicativa, rappresentante il motore di calcolo & Capitolato\tabularnewline 
RQ-8-O & Docker file ("\#"2) con la componente applicativa rappresentante il visualizzatore/monitor real-time (in base all’implementazione, potrebbe essere incorporato nel "\#"1) & Capitolato\tabularnewline
RQ-9-O & Docker file ("\#"3), da istanziare N volte, rappresentante la singola unità & Capitolato\tabularnewline
RQ-10-F & Docker file ("\#"4), da istanziare N volte, rappresentante il singolo pedone & Capitolato\tabularnewline 
\end{longtable}.
\newline 
\subsection{Requisiti di vincolo}
\renewcommand{\arraystretch}{1.5}
\rowcolors{2}{pari}{dispari}
\begin{longtable}{ 
		>{\centering}p{0.11\textwidth} 
		>{}p{0.6075\textwidth}
		>{\centering \it}p{0.16\textwidth} }
	\rowcolorhead
	\headertitle{Codice} &
	\centering \headertitle{Descrizione} &	
	\headertitle{\normalfont \textbf{Fonte}}	
	\endfirsthead	
	\endhead
RV-1-O & La geolocalizzazione va simulata & Capitolato\tabularnewline
RV-2-O & L'applicativo propone una mappatura in tempo reale della posizione georeferenziata delle unità & Capitolato\tabularnewline
RV-3-F & L'applicativo propone una mappatura in tempo reale della posizione georeferenziata delle persone & Capitolato\tabularnewline
RV-4-O & Le persone si muovano solo a bordo di mezzi & Decisione interna\tabularnewline
RV-5-O & Il sistema deve prevedere ed evitare le collisioni & Capitolato\tabularnewline
RV-6-O & Ogni zona di percorrenza ha un numero massimo di unità che possono percorrerla in parallelo (dimensione della zona) & Capitolato\tabularnewline
RV-7-O & Ogni zona di percorrenza ha un modo in cui può essere percorsa (senso unico, doppio senso) & Capitolato\tabularnewline
RV-8-O & Ogni unità deve rispettare i vincoli dimensionali delle zone & Capitolato\tabularnewline
RV-9-O & Tutte le unità, quando sono in movimento, viaggiano alla stessa velocità che rimane costante & Capitolato\tabularnewline
RV-10-F & L'applicativo permette di gestire il cambiamento della velocità di un'unità & Capitolato\tabularnewline
RV-11-D & Ogni unità ha una velocità di crociera & Capitolato\tabularnewline
RV-12-D & Ogni unità ha una velocità massima & Capitolato\tabularnewline
RV-13-O & Ogni unità ha un suo identificativo & Capitolato\tabularnewline
RV-14-O & Il sistema centrale conosce la posizione di ogni singola unità & Capitolato\tabularnewline
RV-15-O & Il sistema centrale conosce la direzione di ogni singola unità & Capitolato\tabularnewline
RV-16-D & Il sistema centrale conosce la velocità di ogni singola unità & Capitolato\tabularnewline
RV-17-O & Ogni unità ha una lista di task da risolvere ogni volta che fa carico & Capitolato\tabularnewline
RV-18-O & Ogni task è collegata ad un POI da raggiungere & Capitolato\tabularnewline
RV-19-O & Ogni POI può essere di carico o scarico o base & Decisione interna\tabularnewline
RV-20-O & Ci devono essere più di un POI di scarico & Decisione interna\tabularnewline
RV-21-O & Ci deve essere almeno un POI di base & Decisione interna\tabularnewline
RV-22-O & Ci deve essere almeno un POI di carico & Decisione interna\tabularnewline
RV-23-F & Ci possono essere più POI di base & Decisione interna\tabularnewline
RV-24-F & Ci possono essere più POI di carico & Decisione interna\tabularnewline
RV-25-O & Ogni unità parte da una base. La sua partenza dalla base determina l'inizio del turno di un operatore & Decisione interna\tabularnewline
RV-26-O & Ogni unità torna ad una base quando termina il turno dell’operatore & Decisione interna\tabularnewline
RV-27-O & Ogni unità passa per un’area di carico prima di iniziare la sequenza di scarichi (tasks) & Decisione interna\tabularnewline
RV-28-O & Ogni unità torna ad un'area di carico se ha scaricato tutta la merce (completato i task) e il turno dell’operatore non è terminato & Decisione interna\tabularnewline
RV-29-O & Il sistema centrale conosce ogni spostamento (in avanti, indietro, a destra e a sinistra) di ogni singola unità & Capitolato\tabularnewline
RV-30-O & Il sistema centrale conosce la fermata di ogni singola unità & Capitolato\tabularnewline
RV-31-O & Il sistema centrale conosce la partenza di ogni singola unità & Capitolato\tabularnewline
\end{longtable}.
\newline
\newline
\subsection{Tracciamento}
\subsubsection{Fonti - Requisiti}
\renewcommand{\arraystretch}{1.5}
\rowcolors{2}{pari}{dispari}
\begin{longtable}{ 
		>{\centering}p{0.25\textwidth} 
		>{}p{0.3\textwidth} }
	\rowcolorhead
	\headertitle{Fonte} &
	\headertitle{\normalfont \textbf{Requisiti}}	
	\endfirsthead	
	\endhead
Capitolato & 
RV-1-O\newline RV-2-O\newline
RV-3-F\newline RV-5-O\newline
RV-6-O\newline RV-7-O\newline
RV-8-O\newline RV-9-O\newline
RV-10-F\newline RV-11-D\newline
RV-12-D\newline RV-13-O\newline
RV-14-O\newline RV-15-O\newline
RV-16-D\newline RV-17-O\newline
RV-18-O\newline RV-29-O\newline
RV-30-O\newline RV-31-O\newline
RF-17-F\newline RF-30-D\newline
RF-31-O\newline RF-32-F\newline
RQ-1-O\newline RQ-2-O\newline
RQ-3-O\newline RQ-4-O\newline
RQ-5-O\newline RQ-6-O\newline
RQ-7-O\newline RQ-8-O\newline
RQ-9-O\newline RQ-10-F\tabularnewline
Decisione interna & 
RV-4-O\newline RV-19-O\newline
RV-20-O\newline RV-21-O\newline
RV-22-O\newline RV-23-F\newline
RV-24-F\newline RV-25-O\newline
RV-26-O\newline RV-27-O\newline
RV-28-O\tabularnewline
\textsc{verbale esterno 1} & 
RF-25.1-O\newline RF-26.1-O\newline
RF-29-O\tabularnewline
UC1 & 
RF-1-O\newline RF-2-O\tabularnewline
UC2 & 
RF-3-O\newline RF-4-O\newline
RF-4.1-O\newline RF-4.2-O\newline
RF-4.3-O\newline RF-5-O\tabularnewline
UC3 & 
RF-6-O\newline RF-6.1-O\newline
RF-6.2-O\newline RF-6.3-O\tabularnewline
UC4 &
RF-7-O\newline RF-8-O\newline
RF-8.1-O\newline RF-8.2-O\newline
RF-8.3-O\newline RF-9-O\newline
RF-10-O\tabularnewline
UC5 &
RF-11-O\newline RF-12-O\newline
RF-13-O\newline RF-14-O\tabularnewline
UC6 &
RF-15-O\newline RF-15.1-O\newline
RF-15.2-O\newline RF-15.2.1-O\newline
RF-16-O\tabularnewline
UC7 &
RF-18-O\newline RF-18.1-O\newline
RF-18.2-O\newline RF-19-O\newline
RF-19.1-O\newline RF-19.2-O\newline
RF-19.2.1-O\newline RF-19.2.2-O\newline
RF-19.2.3-O\newline RF-19.3-O\tabularnewline
UC8 &
RF-20-O\newline RF-21-O\newline
RF-22-O\newline RF-22.1-O\tabularnewline
UC9 &
RF-23-O\tabularnewline
UC10 &
RF-24-O\newline RF-24.1-O\tabularnewline
UC11 &
RF-25-O\newline RF-25.1-O\newline
RF-26-O\newline RF-26.1-O\newline
RF-27-O\newline RF-28-O\newline
RF-29-O\tabularnewline
UC12 &
RF-33-O\newline RF-34-O\tabularnewline
UC13 &
RF-35-O\tabularnewline
UC14 &
RF-36-O\newline RF-36.1-O\newline
RF-36.2-O\tabularnewline
\end{longtable}.\newline
\subsubsection{Requisiti - Fonti}
\renewcommand{\arraystretch}{1.5}
\rowcolors{2}{pari}{dispari}
\begin{longtable}{ 
		>{}p{0.15\textwidth} 
		>{}p{0.35\textwidth} }
	\rowcolorhead
	\headertitle{Requisito} &
	\headertitle{\normalfont \textbf{Fonti}}	
	\endfirsthead	
	\endhead
RV-1-O & Capitolato\tabularnewline
RV-2-O & Capitolato\tabularnewline
RV-3-F & Capitolato\tabularnewline
RV-4-O & Decisione interna\tabularnewline
RV-5-O & Capitolato\tabularnewline
RV-6-O & Capitolato\tabularnewline
RV-7-O & Capitolato\tabularnewline
RV-8-O & Capitolato\tabularnewline
RV-9-O & Capitolato\tabularnewline
RV-10-F & Capitolato\tabularnewline
RV-11-D & Capitolato\tabularnewline
RV-12-D & Capitolato\tabularnewline
RV-13-O & Capitolato\tabularnewline
RV-14-O & Capitolato\tabularnewline
RV-15-O & Capitolato\tabularnewline
RV-16-D & Capitolato\tabularnewline
RV-17-O & Capitolato\tabularnewline
RV-18-O & Capitolato\tabularnewline
RV-19-O & Decisione interna\tabularnewline
RV-20-O & Decisione interna\tabularnewline
RV-21-O & Decisione interna\tabularnewline
RV-22-O & Decisione interna\tabularnewline
RV-23-F & Decisione interna\tabularnewline
RV-24-F & Decisione interna\tabularnewline
RV-25-O & Decisione interna\tabularnewline
RV-26-O & Decisione interna\tabularnewline
RV-27-O & Decisione interna\tabularnewline
RV-28-O & Decisione interna\tabularnewline
RV-29-O & Capitolato\tabularnewline
RV-30-O & Capitolato\tabularnewline
RV-31-O & Capitolato\tabularnewline
RF-1-O & UC1\tabularnewline
RF-2-O & UC1.1\tabularnewline
RF-3-O & UC2\tabularnewline
RF-4-O & UC2\tabularnewline
RF-4.1-O & UC2.1.1\tabularnewline
RF-4.2-O & UC2.1.2\tabularnewline
RF-4.3-O & UC2.1.3\tabularnewline
RF-5-O & UC2.3\tabularnewline
RF-6-O & UC3\tabularnewline
RF-6.1-O & UC3.1\tabularnewline
RF-6.2-O & UC3.2\tabularnewline
RF-6.3-O & UC3.3\tabularnewline
RF-7-O & UC4\tabularnewline
RF-8-O & UC4.1\tabularnewline
RF-8.1-O & UC4.2\tabularnewline
RF-8.2-O & UC4.3\tabularnewline
RF-8.3-O & UC4.4\tabularnewline
RF-9-O & UC4.5\tabularnewline
RF-10-O & UC4.6\tabularnewline
RF-11-O & UC5\tabularnewline
RF-12-O & UC5\tabularnewline
RF-13-O & UC5\tabularnewline
RF-14-O & UC5\tabularnewline
RF-15-O & UC6\tabularnewline
RF-15.1-O & UC6\tabularnewline
RF-15.2-O & UC6\tabularnewline
RF-15.2.1-O & UC6\tabularnewline
RF-16-O & UC6.1\tabularnewline
RF-17-F & Capitolato\tabularnewline
RF-18-O & UC7\tabularnewline
RF-18.1-O & UC7.2\tabularnewline
RF-18.2-O & UC7.3\tabularnewline
RF-19-O & UC7.4\tabularnewline
RF-19.1-O & UC7.4.1\tabularnewline
RF-19.2-O & UC7.4.2\tabularnewline
RF-19.2.1-O & UC7.4.3\tabularnewline
RF-19.2.2-O & UC7.4.4\tabularnewline
RF-19.2.3-O & UC7.4.5\tabularnewline
RF-19.3-O & UC7.4.6\tabularnewline
RF-20-O & UC8.1\tabularnewline
RF-21-O & UC8.2\tabularnewline
RF-22-O & UC8.3\tabularnewline
RF-22.1-O & UC8.3\tabularnewline
RF-23-O & UC9\tabularnewline
RF-24-O & UC10\tabularnewline
RF-24.1-O & UC10\tabularnewline
RF-25-O & UC11.1\tabularnewline
RF-25.1-O & UC11.1, \textsc{verbale esterno 1}\tabularnewline
RF-26-O & UC11.2\tabularnewline
RF-26.1-O & UC11.2, \textsc{verbale esterno 1}\tabularnewline
RF-27-O & UC11.3\tabularnewline
RF-28-O & UC11.5\tabularnewline
RF-29-O & UC11.4, \textsc{verbale esterno 1}\tabularnewline
RF-30-D & Capitolato\tabularnewline
RF-31-O & Capitolato\tabularnewline
RF-32-F & Capitolato\tabularnewline
RF-33-O & UC12\tabularnewline
RF-34-O & UC12\tabularnewline
RF-35-O & UC13\tabularnewline
RF-36-O & UC14\tabularnewline
RF-36.1-O & UC14.1\tabularnewline
RF-36.2-O & UC14.2\tabularnewline
RQ-1-O & Capitolato\tabularnewline
RQ-2-O & Capitolato\tabularnewline
RQ-3-O & Capitolato\tabularnewline
RQ-4-O & Capitolato\tabularnewline
RQ-5-O & Capitolato\tabularnewline
RQ-6-O & Capitolato\tabularnewline
RQ-7-O & Capitolato\tabularnewline
RQ-8-O & Capitolato\tabularnewline
RQ-9-O & Capitolato\tabularnewline
RQ-10-F & Capitolato\tabularnewline
\end{longtable}.
\newline
\subsubsection{Riepilogo requisiti}
\renewcommand{\arraystretch}{1.5}
\rowcolors{2}{pari}{dispari}
\begin{longtable}{ 
		>{\centering}p{0.15\textwidth} 
		>{\centering}p{0.15\textwidth}
		>{\centering}p{0.15\textwidth}
		>{\centering}p{0.15\textwidth}
		>{\centering}p{0.1\textwidth} }
	\rowcolorhead
	\headertitle{Tipologia} &
	\centering \headertitle{Obbligatorio} &	
	\centering \headertitle{Facoltativo} &	
	\centering \headertitle{Desiderabile} &	
	\headertitle{\normalfont \textbf{Totale}}	
	\endfirsthead	
	\endhead
Funzionale & 59 & 2 & 1 & 62\tabularnewline
Di Qualità & 9 & 1 & 0 & 10\tabularnewline
Di Vincolo & 24 & 4 & 3 & 31\tabularnewline
\end{longtable}.