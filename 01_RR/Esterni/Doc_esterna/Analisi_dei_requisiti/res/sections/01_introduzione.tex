\section{Introduzione}
\subsection{Scopo del documento}
Il seguente documento ha lo scopo di elencare in modo formale e dettagliato tutti i casi d’uso e i requisiti dedotti dall’analisi del capitolato \textit{C5 Portacs} presentato dalla azienda \textit{Sanmarco Informatica}.

\subsection{Scopo del prodotto}
Il capitolato C5 propone un progetto in cui viene richiesto lo sviluppo di un software per il monitoraggio in tempo reale di unità che si muovono in uno spazio definito. All’interno di questo spazio, creato dall’utente per riprodurre le caratteristiche di un ambiente reale, le unità dovranno essere in grado di circolare in autonomia, o sotto il controllo dell’utente, per raggiungere dei punti di interesse posti nella mappa.  La circolazione è sottoposta a vincoli di viabilità e ad ostacoli propri della topologia dell’ambiente, deve evitare le collisioni con le altre unità e prevedere la gestione di situazioni critiche nel traffico.

\subsection{Glossario}
%da sistemare
All’interno del documento si fa uso di termini specifici o ambigui, quindi per semplificare la lettura e renderla la più corretta possibile si fornisce un glossario reperibile nel file Glossario v. 1.0.0. Inoltre le parole interessate vengono contrassegnate con la lettera “G” posizionata come pedice.

\subsection{Riferimenti}
\subsubsection{Normativi}
\begin{itemize}
\item \textsc{Norme di progetto v1.0.0 }: per qualsiasi convenzione sulla nomenclatura degli elementi presenti all’interno del documento;
\item Specifica tecnico-economica e organigramma: \\ \url{https://www.math.unipd.it/~tullio/IS-1/2020/Progetto/RO.html}
\item Regolamento progetto didattico - slide del corso di Ingegneria del Software: \\ \url{https://www.math.unipd.it/~tullio/IS-1/2020/Dispense/P1.pdf}
\item Specifica sui casi d'uso - slide del corso di Ingegneria del Software: \\ \url{https://www.math.unipd.it/%7Ercardin/swea/2021/Diagrammi%20Use%20Case_4x4.pdf}
\end{itemize}
\subsubsection{Informativi}
\begin{itemize}
\item \textsc{Glossario v1.0.0}: per la definizione dei termini (pedice G) e degli acronimi (pedice A) evidenziati nel documento;
\item Capitolato d'appalto C5-PORTACS: \\
{\url{https://www.math.unipd.it/~tullio/IS-1/2020/Progetto/C5.pdf}}
\item Software Engineering - Iam Sommerville - $10^{th}$ Edition
\item \textsc{Verbale Esterno 1}
 

\end{itemize}