\section{Descrizione generale}

\subsection{Obiettivi del prodotto}
Il progetto Portacs si pone come obiettivo finale di dimostrare la fattibilità di sviluppare un software che permetta il monitoraggio in tempo reale di unità che si muovono in uno spazio per raggiungere una lista ordinata di punti d’interesse. Per facilitare lo sviluppo del progetto e dopo accordo con l'azienda, si è deciso di contestualizzare lo sviluppo ad un magazzino in cui il sistema centrale pilota i vari muletti verso le destinazioni.

\subsection{Caratteristiche del prodotto}
Con questo progetto si vuole sviluppare un software che controlli lo spostamento di unità trasportatrici all’interno di un magazzino rappresentato tramite una mappa, nella quale vengono specificati i percorsi percorribili con annesse corsie parallele e sensi unici, e dove sono indicati i vari \textit{“Points Of Interest”} (POI\ped{G}), ovvero tutti i punti in cui è possibile scaricare le merci. 
Il progetto si può suddividere in tre macro architetture, le cui caratteristiche sono di seguito descritte.
\subsubsection{Unità}
Una prima macro architettura è composta dall’insieme delle varie unità che si muovono nello spazio che rappresentano appunto i muletti all'interno del magazzino. Ognuna di queste unità dispone di un punto di partenza, di una velocità massima non superabile, e di una lista ordinata di POI\ped{G} da raggiungere, ossia un sottoinsieme di tutti i punti di scarico segnati nella mappa.
Inoltre, ogni muletto deve inviare costantemente al sistema centrale la propria posizione, direzione e velocità.
Essi sono guidati dal pilota automatico che decide quale direzione nello spazio fargli percorrere, però in qualsiasi momento l'operatore del mezzo può interrompere il sistema e decidere di guidare autonomamente. 
Per essere messi in moto, c'è bisogno che il guidatore si identifichi all'interno del sistema tramite il proprio codice identificativo, così da poter visualizzare le mosse scelte dall'applicativo e la mappa del magazzino.
Una volta che l'unità ha soddisfatto tutta la lista di POI deve ritornare al punto di partenza in cui gli verranno dati altri compiti oppure il veicolo resterà in attesa di un altro operatore.

\subsubsection{Sistema}
Il sistema centrale si occupa del coordinamento di ogni unità. Considerando la posizione, la direzione e la velocità di ognuna di esse, il sistema calcola la prossima mossa da far eseguire. Le calcola in funzione del successivo POI da raggiungere, della posizione delle altre unità nello spazio al fine di evitare le eventuali collisioni (predittività), dei vincoli dimensionali, i quali i limiti sulle corsie. Esso inoltre deve analizzare e accettare i seguenti input:
\begin{itemize}
	\item{\textbf{Mappa del magazzino:} }
			\begin{itemize}
			\item{definizione della percorrenza e relativi vincoli strutturali, quali corsie parallele o sensi unici ;}
			\item{definizione e posizione dei possibili POI da raggiungere.}
			\end{itemize}
	\item{\textbf{Definizione delle N unità:}}
			\begin{itemize}
			\item{identificativo di sistema;} 
			\item{velocità massima; } 
			\item{posizione iniziale ;} 
			\item{lista dei POI da soddisfare, già ordinata.}
			\end{itemize}
\end{itemize}

\subsubsection{User Interface} 
La user interface è diversa in base al ruolo del lavoratore all'interno del magazzino. Per gli operatori che guidano il muletto è disponibile la visualizzazione della mappa, i comandi per effettuare il passaggio al pilota manuale con le quattro frecce direzionali e un pulsante di start/stop per guidare la vettura e una schermata delle mosse che il sistema intende eseguire sempre identificate con i simboli delle quattro frecce direzionali e lo start/stop. 
L'interfaccia dell'amministratore comprende la mappa e una pagina apposita per l'inserimento di nuovi utenti nella piattaforma, mentre i responsabili visualizzano sempre la mappa e la lista dei POI per selezionare quelli da aggiungere a quelli da soddisfare.

\subsection{Caratteristiche degli utenti}
Il sistema Portacs, per scelta interna, è destinato all'utilizzo in un magazzino. In questa azienda ogni lavoratore avrà un ruolo importante all'interno del sistema: gli operatori potranno controllare le unità nel caso di guida manuale, il responsabile potrà inserire quali POI dovranno essere soddisfatti mentre l'amministratore potrà apportare modifiche alla planimetria e alla percorribilità, in base alle esigenze del magazzino.
\subsection{Vincoli progettuali}
Il prodotto deve soddisfare il vincolo che tutti i POI all'interno della mappa devono essere pubblici e globali, ogni unità deve quindi poter vedere tutti i punti nella mappa.