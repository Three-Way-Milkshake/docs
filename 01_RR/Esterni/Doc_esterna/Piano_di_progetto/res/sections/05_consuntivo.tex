\section{Consuntivo}

La sezione che segue espone le spese effettivamente sostenute, registrate al termine delle fasi di Avvio e di Analisi dei Requisti. In relazione alle spese preventivate, il periodo chiuderà in:
\begin{itemize}
	\item \textbf{positivo} se il preventivo supera il consuntivo;
	\item \textbf{pari} se il preventivo e il consuntivo collimano;
	\item \textbf{negativo} se il consuntivo supera il preventivo.
\end{itemize}


\subsection{Avvio}

\begin{table}[H]
	\begin{center}
		\begin{tabular}{c
				!{\color[HTML]{9b240a}\vrule width 1pt}
				cccccc
				!{\color[HTML]{9b240a}\vrule width 1pt}	
				c}
			\rowcolorhead
			\headertitle{Nome} & \headertitle{R} & \headertitle{V} & \headertitle{An} & \headertitle{Am} & \headertitle{Pr} & \headertitle{Pt} & \headertitle{Tot} \\
			
			Chiarello Sofia & 0 & 0 & 0 & 0 & 0 & 0 & 0\\
			Crivellari Alberto & 0 & 0 & 0 & 0 & 0 & 0 & 0\\
			De Renzis Simone & 0 & 0 & 0 & 0 & 0 & 0 & 0\\
			Greggio Nicolò & 0 & 0 & 0 & 0 & 0 & 0 & 0\\
			Tessari Andrea & 0 & 0 & 0 & 0 & 0 & 0 & 0\\
			Zuccolo Giada & 0 & 0 & 0 & 0 & 0 & 0 & 0\\
		\end{tabular}
		\caption[Consuntivo fase di Avvio]{Per ogni componente, le ore effettivamente spese nella fase di Avvio}
	\end{center}
\end{table}




\subsection{Analisi dei requisiti}

\begin{table}[H]
	\begin{center}
		\begin{tabular}{c
				!{\color[HTML]{9b240a}\vrule width 1pt}
				cccccc
				!{\color[HTML]{9b240a}\vrule width 1pt}	
				c}
			\rowcolorhead
			\headertitle{Nome} & \headertitle{R} & \headertitle{V} & \headertitle{An} & \headertitle{Am} & \headertitle{Pr} & \headertitle{Pt} & \headertitle{Tot} \\
			
			Chiarello Sofia & 0 & 0 & 0 & 0 & 0 & 0 & 0\\
			Crivellari Alberto & 0 & 0 & 0 & 0 & 0 & 0 & 0\\
			De Renzis Simone & 0 & 0 & 0 & 0 & 0 & 0 & 0\\
			Greggio Nicolò & 0 & 0 & 0 & 0 & 0 & 0 & 0\\
			Tessari Andrea & 0 & 0 & 0 & 0 & 0 & 0 & 0\\
			Zuccolo Giada & 0 & 0 & 0 & 0 & 0 & 0 & 0\\
		\end{tabular}
		\caption[Consuntivo fase di Analisi dei Requisiti]{Per ogni componente, le ore effettivamente spese nella fase di Analisi dei Requisiti}
	\end{center}
\end{table}



\subsection{Totale}

\begin{table}[H]
	\centering
	\begin{tabular}{ccc}
		\rowcolorhead
		\headertitle{Ruolo} & \headertitle{Ore} & \headertitle{Costo(€)}\\
		Responsabile & 0 & 0\\
		Verificatore & 0 & 0\\
		Analista & 0 & 0\\				
		Amministratore & 0 & 0\\
		Programmatore & 0 & 0\\
		Progettista & 0 & 0\\
		\hline
		\textbf{Totale preventivo} & \textbf{0} & \textbf{0}\\
		\textbf{Totale consuntivo} & \textbf{0} & \textbf{0}\\
		\textbf{Differenza} & \textbf{0} & \textbf{0}\\
	\end{tabular}
	\caption[Confronto tra preventivo e consuntivo]{Per ogni ruolo, il totale delle ore effettivamente impiegate, con lo scostamento dal preventivo}
\end{table}


\subsection{Preventivo a finire}
Il preventivo a finire comprende i costi consuntivi di tutte le attività terminate più i costi previsti per le attività da eseguire. 



\subsection{Conclusioni}

Il periodo si chiude in \textbf{positivo}, permettendo un risparmio di \textbf{ €}.
