\section{Introduzione}




\subsection{Scopo del documento}

Nel contesto della realizzazione del progetto \textsc{portacs} da parte del gruppo \textit{Three Way Milkshake}, il documento risponde alle seguenti esigenze: 
\begin{itemize}
	\item analizzare i rischi che possono emergere durante lo sviluppo, elaborando strategie per mitigarne gli effetti;
	\item pianificare il lavoro istanziando delle attività a partire dal modello di sviluppo scelto e fissandone le scadenze;
	\item fornire una valutazione preventiva delle risorse necessarie a ciascuna fase in termini di ore di lavoro;
	\item esporre le spese sostenute nelle fasi già attraversate;
	\item verbalizzare i rischi effettivamente riscontrati.
\end{itemize}



\subsection{Scopo del prodotto}
Il prodotto consiste in una piattaforma di coordinamento tra unità a mobilità autonoma all'interno di un ambiente. Le unità, che possono essere di vario tipo (robot, muletto, automobile), vengono istruite di una lista di destinazioni da raggiungere sotto la guida di un server centrale che ne controlla i movimenti e ne evita le collisioni. L'ambiente, fornito in ingresso sotto forma di mappa, presenta punti di interesse, ostacoli e vincoli di viabilità.



\subsection{Glossario}



E' consigliato leggere questo documento con l'ausilio del glossario che ha lo scopo di definire le parole che potrebbero risultare ambigue. Tali termini verranno evidenziati in questo file attraverso l'apposizione di una "G" a pedice della stessa alla sua prima occorrenza, ad esempio: parola$_G$.
La lista dei termini con relativa definizione è fruibile nel documento "Glossario".




\subsection{Riferimenti}



\subsubsection{Normativi}

\begin{itemize}
	\item \textit{Norme di progetto v\_ 1.0.0};
	\item Specifica tecnico-economica e organigramma: \\ \uline{\url{https://www.math.unipd.it/~tullio/IS-1/2020/Progetto/RO.html}}
	\item Regolamento progetto didattico - slide del corso di Ingegneria del Software: \\ \uline{\url{https://www.math.unipd.it/~tullio/IS-1/2020/Dispense/P1.pdf}}
\end{itemize}



\subsubsection{Informativi}
\begin{itemize}
	\item Capitolato d'appalto C5-\textsc{portacs}: \\ \uline{\url{https://www.math.unipd.it/~tullio/IS-1/2020/Progetto/C5.pdf}}
	\item Software Engineering - Iam Sommerville - $10^{th}$ Edition
	\item Slide L5
	\item Slide L6
\end{itemize}