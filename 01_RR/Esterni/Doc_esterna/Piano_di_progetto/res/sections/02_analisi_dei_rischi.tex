\section{Analisi dei rischi}
In un progetto di queste dimensioni bla bla bla... da fare

Per identificare e tracciare i rischi, viene introdotta la seguente nomenclatura:
\begin{itemize}
	\item \textbf{RIS\_ T} Rischio Tecnologico;
	\item \textbf{RIS\_ O} Rischio Organizzativo;
	\item \textbf{RIS\_ I} Rischio Interpersonale;
\end{itemize}



\subsection{Rischi tecnologici}



{\setlength{\parindent}{0cm}
\begin{minipage}{\textwidth} 
\begin{multicols}{4}
\textbf{Nome}: \\ Novità del problema e delle tecnologie \columnbreak

\textbf{Codice}: \\ RIS\_ T \_ 1 
\columnbreak

\textbf{Occorrenza}: \\ Alta 
\columnbreak

\textbf{Pericolosità}: \\ Media

\end{multicols}

\begin{multicols}{3}

\textbf{Descrizione}: \\ Il capitolato non pone vincoli sull'utilizzo delle tecnologie da adottare. Se da un lato questo permette libertà nell'implementazione, dall'altro può causare disorientamento nei membri meno esperti. Vista la novità del problema da trattare, le tecnologie da impiegare potranno risultare nuove per molti.
\columnbreak

\textbf{Rilevamento}: \\ Il responsabile si occuperà di censire le conoscenze e competenze dei membri del gruppo, al fine di individuare particolari lacune. I membri, qualora dovessero riscontrare difficoltà, lo comunicheranno al resto del gruppo.  
\columnbreak

\textbf{Piano di contingenza}: \\ Dopo un'esplorazione generale delle tecnologie che si prestano a risolvere questo tipo di problemi, ci si confronterà con il proponente per confermare la bontà delle scelte adottate. I membri che hanno più esperienza guideranno lo studio di queste tecnologie.\\

\columnbreak
\end{multicols}
\end{minipage}} \\


\noindent\rule{\textwidth}{1pt}\\

{\setlength{\parindent}{0cm}
\begin{minipage}{\textwidth} 
\begin{multicols}{4}
\textbf{Nome}: \\ Novità del problema e delle tecnologie \columnbreak

\textbf{Codice}: \\ RIS\_ T \_ 1 
\columnbreak

\textbf{Occorrenza}: \\ Alta 
\columnbreak

\textbf{Pericolosità}: \\ Media

\end{multicols}

\begin{multicols}{3}

\textbf{Descrizione}: \\ Il capitolato non pone vincoli sull'utilizzo delle tecnologie da adottare. Se da un lato questo permette libertà nell'implementazione, dall'altro può causare disorientamento nei membri meno esperti. Vista la novità del problema da trattare, le tecnologie da impiegare potranno risultare nuove per molti.
\columnbreak

\textbf{Rilevamento}: \\ Il responsabile si occuperà di censire le conoscenze e competenze dei membri del gruppo, al fine di individuare particolari lacune. I membri, qualora dovessero riscontrare difficoltà, lo comunicheranno al resto del gruppo.  
\columnbreak

\textbf{Piano di contingenza}: \\ Dopo un'esplorazione generale delle tecnologie che si prestano a risolvere questo tipo di problemi, ci si confronterà con il proponente per confermare la bontà delle scelte adottate. I membri che hanno più esperienza guideranno lo studio di queste tecnologie.\\

\columnbreak
\end{multicols}
\end{minipage}}



\subsection{Rischi organizzativi}



\subsection{Rischi interpersonali}

