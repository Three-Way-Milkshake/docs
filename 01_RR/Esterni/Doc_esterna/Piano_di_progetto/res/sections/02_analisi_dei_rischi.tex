\section{Analisi dei rischi}

In un progetto di queste dimensioni è possibile incontrare problemi di varia natura. Per cercare di marginare ciò si possono gestire attentamente 4 attività:

\begin{itemize}
	\item \textbf{Individuazione dei rischi}: individuare i fattori che possono introdurre criticità nello svolgimento del progetto;
	\item \textbf{Analisi dei rischi}: esaminare i fattori di rischio deducendone la probabilità che la criticità si manifesti, l'impatto che ha e le sue conseguenze nel progetto;
	\item \textbf{Pianificazione per il controllo}: pianificare delle misure atte a impedire il verificarsi del problema e ad arginarne le conseguenze;
	\item \textbf{Monitoraggio dei rischi}: controllare attivamente e in modo costante i fattori di rischio al fine di prevenirne o intercettarne in modo tempestivo gli effetti.
\end{itemize}


Per identificare e tracciare i rischi, viene introdotta la seguente nomenclatura:

\begin{itemize}
	\item \textbf{RIS\_ T} Rischio Tecnologico;
	\item \textbf{RIS\_ O} Rischio Organizzativo;
	\item \textbf{RIS\_ I} Rischio Interpersonale;
\end{itemize}

\subsection{Rischi tecnologici}



%------------------------------------RIS\_ T \_ 1---------------------------------------

\renewcommand{\arraystretch}{1.5}
\rowcolors{2}{pari}{dispari}
\begin{longtable} { 
		>{\raggedright}p{0.33\textwidth} 
		>{\raggedright}p{0.33\textwidth} 
		>{\raggedright}p{0.33\textwidth}    }
		
		\caption{RIS\_ T \_ 1} \endhead	


	\textbf{Nome}: \\ Novità del problema e delle tecnologie
	& \textbf{Codice}: \\ RIS\_ T \_ 1  
	& \textbf{Occorrenza}: Alta \\ \textbf{Pericolosità}: Media
	
	\tabularnewline
	
	\textbf{Descrizione}: \\ Il capitolato non pone vincoli sull'utilizzo delle tecnologie da adottare. Se da un lato questo permette libertà nell'implementazione, dall'altro può causare disorientamento nei membri meno esperti. Vista la novità del problema da trattare, le tecnologie da impiegare potranno risultare nuove per molti.
	& 
	\textbf{Rilevamento}: \\ Il responsabile si occuperà di censire le conoscenze e competenze dei membri del gruppo, al fine di individuare particolari lacune. I membri, qualora dovessero riscontrare difficoltà, lo comunicheranno al resto del gruppo. 	
	&  
	\textbf{Piano di contingenza}: \\ Dopo un'esplorazione generale delle tecnologie che si prestano a risolvere questo tipo di problemi, ci si confronterà con il proponente per confermare la bontà delle scelte adottate. I membri che hanno più esperienza guideranno lo studio di queste tecnologie.

\end{longtable}

%------------------------------------RIS\_ T \_ 2---------------------------------------

\renewcommand{\arraystretch}{1.5}
\rowcolors{2}{pari}{dispari}
\begin{longtable} { 
		>{\raggedright}p{0.33\textwidth} 
		>{\raggedright}p{0.33\textwidth} 
		>{\raggedright}p{0.33\textwidth}    }
	
	\caption{RIS\_ T \_ 2} \endhead	
	
	
	\textbf{Nome}: \\ Malfunzionamento dei dispositivi
	& \textbf{Codice}: \\ RIS\_ T \_ 2  
	& \textbf{Occorrenza}: Bassa \\ \textbf{Pericolosità}: Bassa
	
	\tabularnewline
	
	\textbf{Descrizione}: \\ I computer dei componenti del gruppo di lavoro possono andare incontro a guasti software o hardware. Questo può compromettere parte del lavoro svolto o rallentarne l'avanzamento.
	& 
	\textbf{Rilevamento}: \\ Il membro interessato dal guasto avviserà tempestivamente il gruppo se l'imprevisto dovesse causare difficoltà nel proseguimento del lavoro o se parte di esso fosse stato perso.
		
	&  
	\textbf{Piano di contingenza}: \\ \'E caldamente consigliato mantenere una copia di backup del lavoro in corso di svolgimento. L'interessato dal guasto si adopererà con urgenza a ripristinare il funzionamento del proprio dispositivo. Se non fosse possibile recuperare il lavoro svolto, esso verrà suddiviso tra i membri ed elaborato nuovamente.
	
\end{longtable}



\subsection{Rischi organizzativi}

%------------------------------------RIS\_ O \_ 1---------------------------------------

\renewcommand{\arraystretch}{1.5}
\rowcolors{2}{pari}{dispari}
\begin{longtable} { 
		>{\raggedright}p{0.33\textwidth} 
		>{\raggedright}p{0.33\textwidth} 
		>{\raggedright}p{0.33\textwidth}    }
	
	\caption{RIS\_ O \_ 1} \endhead	
	
	
	\textbf{Nome}: \\ Organizzazione e preventivazione
	& \textbf{Codice}: \\ RIS\_ O \_ 1
	& \textbf{Occorrenza}: Alta \\ \textbf{Pericolosità}: Media
	
	\tabularnewline
	
	\textbf{Descrizione}: \\ Preventivare le ore necessarie a svolgere le attività future è difficile se non si ha maturato esperienza nello sviluppo di progetti complessi.
	& 
	\textbf{Rilevamento}: \\ I costi preventivati possono non corrispondere alle ore effettivamente spese, e accumulare ritardi può compromettere la buona riuscita del lavoro.
	
	&  
	\textbf{Piano di contingenza}: \\ I preventivi delle attività future saranno sufficientemente generali da permettere di raffinarli quando le attività si fanno più prossime, e sufficientemente precisi da dettare le scadenze oltre le quali i ritardi diventano critici. Il Responsabile di Progetto guiderà il lavoro riferendosi costantemente con il cruscotto di progetto. 
	
\end{longtable}




\subsection{Rischi interpersonali}



\renewcommand{\arraystretch}{1.5}
\rowcolors{2}{pari}{dispari}
\begin{longtable} { 
		>{\raggedright}p{0.33\textwidth} 
		>{\raggedright}p{0.33\textwidth} 
		>{\raggedright}p{0.33\textwidth}    }
	
	\caption{RIS\_ I \_ 1} \endhead	
	
	
	\textbf{Nome}: \\ Divergenze tra membri del gruppo di lavoro
	& \textbf{Codice}: \\ RIS\_ I \_ 1
	& \textbf{Occorrenza}: Media \\ \textbf{Pericolosità}: Media
	
	\tabularnewline
	
	\textbf{Descrizione}: \\ I membri del gruppo, se soggetti a situazioni stressanti come può essere lo svolgimento di un lavoro impegnativo, potrebbero trovare difficoltà nella cooperazione e generare contrasti all'interno del team.
	& 
	\textbf{Rilevamento}: \\ Il lavoro subisce rallentamenti a causa di conflitti tra i componenti.
	
	&  
	\textbf{Piano di contingenza}: \\ Ogni membro del gruppo è tenuto a tenere un atteggiamento aperto al dialogo e al compromesso, conscio del fatto che la buona riuscita del progetto è imprescindibile da una stretta collaborazione interna. Dopo ogni revisione si terrà un'attività di verifica in cui ogni membro avrà l'opportunità di esporre in modo costruttivo eventuali critiche nel lavoro o nel comportamento degli altri componenti. Se i conflitti risultassero impossibili da gestire internamente, verrà interpellato il professor Tullio Vardanega.
	
\end{longtable}



{\setlength{\parindent}{0cm}
\begin{minipage}{\textwidth} 
\begin{multicols}{4}
\textbf{Nome}: \\ Divergenze tra membri del gruppo di lavoro \columnbreak

\textbf{Codice}: \\ RIS\_ I \_ 1 
\columnbreak

\textbf{Occorrenza}: \\ Bassa 
\columnbreak

\textbf{Pericolosità}: \\ Media

\end{multicols}

\begin{multicols}{3}

\textbf{Descrizione}: \\ Potrebbero incorrere dei rischi legati alla personalità differente di ciascun membro del gruppo di lavoro.
\columnbreak

\textbf{Rilevamento}: \\ Si verificheranno discordie tra due o più membri del gruppo che potranno sfociare in insulti e/o liti.
\columnbreak

\textbf{Piano di contingenza}: \\ Il responsabile di progetto cercherà di dividere il più possibile le persone coinvolte nella lite. In caso non ci riuscisse dovrà informare tempestivamente il professor Tullio Vardanega.\\

\columnbreak
\end{multicols}
\end{minipage}} \\

\noindent\rule{\textwidth}{1pt}\\

{\setlength{\parindent}{0cm}
\begin{minipage}{\textwidth} 
\begin{multicols}{4}
\textbf{Nome}: \\ Scarsa disponibilità dei membri a partecipare attivamente al progetto \columnbreak

\textbf{Codice}: \\ RIS\_ I \_ 2 
\columnbreak

\textbf{Occorrenza}: \\ Alta 
\columnbreak

\textbf{Pericolosità}: \\ Alta

\end{multicols}

\begin{multicols}{3}

\textbf{Descrizione}: \\ Ogni membro ha i suoi impegni ed è difficile avere una disponibilità completa per il progetto.
\columnbreak

\textbf{Rilevamento}: \\ Scarsa partecipazione ai meeting e/o lavoro arretrato da parte di uno o più membri del gruppo.
\columnbreak

\textbf{Piano di contingenza}: \\ Se la situazione persiste a svantaggio di tutti gli altri membri, sarà dovere del Responsabile di progetto ridistribuire i compiti del soggetto a tutti gli altri membri, informando il professor Tullio Vardanega.\\

\columnbreak
\end{multicols}
\end{minipage}} \\

\noindent\rule{\textwidth}{1pt}\\


