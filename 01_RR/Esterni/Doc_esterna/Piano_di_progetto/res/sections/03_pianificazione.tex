\section{Pianificazione}

\subsection{Modello di sviluppo}

Il modello di sviluppo scelto è il modello \textbf{incrementale}. Esso si adatta al sistema di revisioni a cui vanno incontro gli artefatti prodotti nel corso del progetto. Il modello incrementale infatti prevede che Analisi dei requisiti e Progettazione architetturale si svolgano una volta sola: queste fasi servono studiare il problema e a strutturarne la soluzione. Se si dovesse tornare su queste fasi sarà solo per raffinarne i contenuti in base a nuove evidenze individuate nelle fasi successive.

La progettazione di dettaglio e la codifica invece si svilupperanno attraverso cicli di incremento atti a integrare il sistema di nuove funzionalità: si partirà dal soddisfacimento dei requisiti più importanti, per poi eventualmente incrementare con requisiti secondari e opzionali. 

Queste modalità permettono, gettate le basi del prodotto, di accrescerne le funzionalità producendo valore fin da subito, in modo da avere riscontro quasi immediato sull'operato e poterne indirizzare gli sviluppi successivi in base ai feedback ricevuti (anche dal proponente) e alle risorse disponibili.

Il team adotterà anche alcune tecniche tipiche dello sviluppo Agile: viene fatto uso di una Kanban board, strumento che permette di pianificare in dettaglio e visualizzare gli obiettivi a cui ciascun membro del team si dedica. Questa tecnica riflette la modalità con cui il team si organizza nel contesto di un incremento: un meeting a cadenza settimanale permette di pianificare l'avanzamento e stabilire le future assegnazioni, in modo da affrontare eventuali ritardi o difficoltà prima che possano causare problemi allo sviluppo complessivo.



\subsection{Scadenze}

Il gruppo stabilisce di affrontare le revisioni di avanzamento nelle seguenti date:
\begin{itemize}
	\item \textbf{Revisione dei Requisiti}: 18 gennaio 2021
	\item \textbf{Revisione di Progettazione}: 8 marzo 2021 
	\item \textbf{Revisione di Qualifica}: 9 aprile 2021
	\item \textbf{Revisione di Accettazione}: 10 maggio 2021	
\end{itemize}





\subsection{Fasi}

A fronte del modello di sviluppo scelto e delle scadenze fissate, lo sviluppo procederà attraverso le seguenti fasi:
\begin{itemize}
	\item \textbf{Avvio}
	\item \textbf{Analisi dei requisiti}
	\item \textbf{Progettazione architetturale}
	\item \textbf{Progettazione di dettaglio e codifica}
	\item \textbf{Validazione e collaudo}
\end{itemize}

Di seguito vengono descritte le attività da realizzarsi nel contesto di ogni fase.

\subsubsection{Avvio}

\textit{Dal 12/11/2020 al 13/12/2020}
\\\\
Questa fase inizia in corrispondenza del primo seminario tecnologico tenuto da una delle aziende proponenti e termina con la scelta del capitolato a cui il gruppo intende avanzare la propria offerta nella relativa gara d'appalto.
Durante questo periodo, vengono svolte le seguenti attività:
\begin{itemize}
	\item \textbf{Visione dei seminari}: i seminari tecnologici costituiscono un fattore importante nel contesto della scelta del capitolato, fanno luce sui requisiti e sulla fattibilità dei progetti;
	\item \textbf{Norme di progetto}: inizia in questa fase la definizione delle norme che il gruppo intende adottare. Si studiano e testano gli strumenti che permetteranno l'organizzazione interna, il tracciamento, la stesura dei documenti e il loro versionamento, la gestione dei meeting e la loro calendarizzazione;
	\item \textbf{Studio di fattibilità}: l'analisi del materiale di ogni capitolato permette ai gruppi di farsi una prima idea sulle preferenze nella scelta del capitolato. Con il termine della visione dei seminari, si effettua uno studio approfondito di ogni progetto e si redige il documento studio di fattibilità, nel quale si esprime la propria preferenza definitiva;
	\item \textbf{Piano di progetto}: inizia la definizione delle fasi in cui si articola il progetto e la redazione del documento piano di progetto;
\end{itemize}  





\subsubsection{Analisi dei requisiti}



\subsubsection{Progettazione architetturale}



\subsubsection{Progettazione di dettaglio e codifica}



\subsubsection{Validazione e collaudo}



