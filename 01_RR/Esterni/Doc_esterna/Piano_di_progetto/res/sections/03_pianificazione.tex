\section{Pianificazione}

\subsection{Modello di sviluppo}

Il modello di sviluppo scelto è il modello \textbf{incrementale}. Esso si adatta al sistema di revisioni a cui vanno incontro gli artefatti prodotti nel corso del progetto. Il modello incrementale infatti prevede che Analisi dei requisiti e Progettazione architetturale si svolgano una volta sola: queste fasi servono studiare il problema e a strutturarne la soluzione. Se si dovesse tornare su queste fasi sarà solo per raffinarne i contenuti in base a nuove evidenze individuate nelle fasi successive.

La progettazione di dettaglio e la codifica invece si svilupperanno attraverso cicli di incremento atti a integrare il sistema di nuove funzionalità: si partirà dal soddisfacimento dei requisiti più importanti, per poi eventualmente incrementare con requisiti secondari e opzionali. 

Queste modalità permettono, gettate le basi del prodotto, di accrescerne le funzionalità producendo valore fin da subito, in modo da avere riscontro quasi immediato sull'operato e poterne indirizzare gli sviluppi successivi in base ai feedback ricevuti (anche dal proponente) e alle risorse disponibili.

Il team adotterà anche alcune tecniche tipiche dello sviluppo Agile: viene fatto uso di una Kanban board, strumento che permette di pianificare in dettaglio e visualizzare gli obiettivi a cui ciascun membro del team si dedica. Questa tecnica riflette la modalità con cui il team si organizza nel contesto di un incremento: un meeting a cadenza settimanale permette di pianificare l'avanzamento e stabilire le future assegnazioni, in modo da affrontare eventuali ritardi o difficoltà prima che possano causare problemi allo sviluppo complessivo.



\subsection{Scadenze}

Il gruppo stabilisce di affrontare le revisioni di avanzamento nelle seguenti date:
\begin{itemize}
	\item \textbf{Revisione dei Requisiti}: 18 gennaio 2021
	\item \textbf{Revisione di Progettazione}: 8 marzo 2021 
	\item \textbf{Revisione di Qualifica}: 9 aprile 2021
	\item \textbf{Revisione di Accettazione}: 10 maggio 2021	
\end{itemize}





\subsection{Fasi}

A fronte del modello di sviluppo scelto e delle scadenze fissate, lo sviluppo procederà attraverso le seguenti fasi:
\begin{itemize}
	\item \textbf{Avvio}
	\item \textbf{Analisi dei requisiti}
	\item \textbf{Progettazione architetturale}
	\item \textbf{Progettazione di dettaglio e codifica}
	\item \textbf{Validazione e collaudo}
	

\end{itemize}
Nella settimana che intercorre tra la consegna degli artefatti per la revisione e la presentazione degli stessi, il gruppo è impegnato nelle seguenti attività: 
\begin{itemize}
	\item \textbf{Preparazione alla presentazione}: viene preparato il materiale necessario alla presentazione;
	\item \textbf{Verifica della fase precedente}: il gruppo si vede coinvolto in un confronto dal quale vorranno emergere le criticità riscontrate nel periodo appena trascorso, al fine di migliorare lo svolgimento delle fasi successive;
	\item \textbf{Approfondimento personale}: ogni membro del gruppo spende alcune ore per formare e consolidare una conoscenza di base degli strumenti e tecniche da impiegare nella fase successiva.
\end{itemize}
Queste attività non verranno esplicitate nella descrizione di dettaglio che segue, in quanto ripetitive.



\subsubsection{Avvio}

\textit{Dal 12/11/2020 al 13/12/2020}
\\\\
Questa fase inizia in corrispondenza del primo seminario tecnologico tenuto da una delle aziende proponenti e termina con la scelta del capitolato a cui il gruppo intende avanzare la propria offerta nella relativa gara d'appalto.
Durante questo periodo, vengono svolte le seguenti attività:
\begin{itemize}
	\item \textbf{Visione dei seminari}: i seminari tecnologici costituiscono un fattore importante nel contesto della scelta del capitolato, fanno luce sui requisiti e sulla fattibilità dei progetti;
	\item \textbf{Norme di progetto}: inizia in questa fase la definizione delle norme che il gruppo intende adottare. Si studiano e testano gli strumenti che permetteranno l'organizzazione interna, il tracciamento, la stesura dei documenti e il loro versionamento, la gestione dei meeting e la loro calendarizzazione;
	\item \textbf{Studio di fattibilità}: l'analisi del materiale di ogni capitolato permette ai gruppi di farsi una prima idea sulle preferenze nella scelta del capitolato. Con il termine della visione dei seminari, si effettua uno studio approfondito di ogni progetto e si redige il documento Studio di fattibilità, nel quale si esprime la propria preferenza definitiva;
	\item \textbf{Piano di progetto}: inizia la redazione del documento Piano di progetto nelle sue parti fondamentali, a partire da una prima definizione delle fasi e dei rischi;
\end{itemize}  





\subsubsection{Analisi dei requisiti}

\textit{Dal 13/12/2020 al 11/01/2021}
\\\\
Questa fase inizia al termine della fase di Avvio e si conclude con la consegna dei documenti per la Revisione dei Requisiti.
Durante questo periodo, vengono svolte le seguenti attività:
\begin{itemize}
	\item \textbf{Norme di progetto}: vengono consolidate le norme soprattutto per quando riguarda la nomenclatura e le convenzioni da adottare per gli artefatti in produzione. L'omonimo documento è redatto dall'Amministratore;
	\item \textbf{Piano di progetto}: il Responsabile di progetto redige il documento Piano di progetto scandendo le fasi in cui si articolerà il lavoro, presentando il preventivo dei periodi e il consuntivo delle prime 3 fasi;
	\item \textbf{Analisi dei requisiti}: gli Analisti effettuano uno studio approfondito del capitolato e ne individuano i requisiti: l'analisi si caratterizza da contatti frequenti con il proponente che fornirà supporto nella comprensione del problema. Si completa la redazione del documento Analisi dei requisiti. Quest'attività è bloccante per la prosecuzione del progetto;
	\item \textbf{Piano di Qualifica}: in questa attività si individuano i criteri che garantiscono la qualità del prodotto. Il documento Piano di Qualifica è redatto, nella sua prima versione, dall'Amministratore;
	\item \textbf{Glossario}: il documento Glossario conterrà i termini a cui si riterrà necessario darne definizione. \`E redatto da tutti i membri del team;
	\item \textbf{Verifica dei documenti}: quest'attività si concentra nella settimana che precede la presentazione e ha l'obiettivo di verificare e certificare la qualità di tutti i documenti prodotti;
	\item \textbf{Lettera di presentazione}: avviene la stesura della lettera con cui il gruppo si candida alla Revisione dei Requisiti.
\end{itemize}



\subsubsection{Progettazione architetturale}

\textit{Dal 18/01/2021 al 28/02/2021}
\\\\
Inizia il giorno successivo alla presentazione della Revisione dei Requisiti e termina in corrispondenza della consegna degli artefatti per la Revisione di Progettazione.
\begin{itemize}
	\item \textbf{Allegato Tecnico}: viene redatto il documento Allegato tecnico, nel quale viene presentata la Technology Baseline, ovvero l'architettura ad alto livello del software. Redatto dai progettisti;
	\item \textbf{Proof of Concept}: una prima implementazione della soluzione permette di valutarne la bontà: viene realizzato un prototipo del software che verrà sviluppato;
	\item \textbf{Integrazione della documentazione}: l'avanzamento nello sviluppo del prodotto chiarirà alcuni aspetti che nella fase di Analisi risultavano oscuri, e potrebbe evidenziare delle criticità non inizialmente considerate. Se necessario, viene raffinato il documento Analisi dei Requisiti. Anche il documento Piano di Progetto viene migliorato fornendo maggior dettaglio, oltre che integrato con il consuntivo della fase trascorsa. Il documento Norme di Progetto riguarda ora anche gli strumenti necessari alla progettazione architetturale, e il documento Glossario si vede integrato con nuovi termini. Il documento Piano di Qualifica prevede ora anche i criteri di qualità per la progettazione. Generali miglioramenti sono apportati in base alle indicazioni ricevute con la Revisione dei Requisiti. L'integrazione avviene ad opera delle figure interessate alla stesura dei documenti nelle fasi precedenti.
\end{itemize}



\subsubsection{Progettazione di dettaglio e codifica}

\textit{Dal 8/03/2021 al 2/04/2021}
\\\\
Inizia il giorno successivo alla presentazione della Revisione di Progettazione e termina in corrispondenza della consegna degli artefatti per la Revisione di Qualifica.
\begin{itemize}
	\item \textbf{Allegato Tecnico}: viene integrato il documento Allegato Tecnico, che presenterà ora anche la Product Baseline, nella quale il software è scomposto e analizzato nelle sue unità. Redatto dai Programmatori;
	\item \textbf{Codifica}: la scrittura del codice ad opera dei Programmatori segue i criteri di qualità stabiliti nel documento Piano di Qualifica;
	\item \textbf{Integrazione della documentazione}: se necessario, viene raffinato il documento Analisi dei Requisiti. Il Piano di Progetto viene integrato con il consuntivo della fase trascorsa.  Il documento Norme di Progetto riguarda ora anche gli strumenti necessari alla codifica, e il documento Glossario comprende nuovi termini. Il documento Piano di Qualifica prevede ora anche i criteri di qualità per la codifica. Generali miglioramenti sono apportati in base alle indicazioni ricevute con la Revisione di Progettazione. L'integrazione avviene ad opera delle figure interessate alla stesura dei documenti nelle fasi precedenti.
\end{itemize}



\subsubsection{Validazione e collaudo}

\textit{Dal 9/04/2021 al 3/05/2021}
\\\\
Inizia il giorno successivo alla presentazione della Revisione di Qualifica e termina in corrispondenza della consegna degli artefatti per la Revisione di Accettazione.
\begin{itemize}
	\item \textbf{Validazione e collaudo}: vengono eseguiti ulteriori test per consolidare e garantire la qualità del prodotto. Il Piano di Qualifica è il documento di riferimento per quest'attività. \`E svolta dai Progettisti e dai Programmatori;
	\item \textbf{Manuale Utente}: il documento Manuale Utente, la cui stesura è affidata ai Progettisti e agli Analisti, specifica le modalità d'uso del software agli utenti utilizzatori;
	\item \textbf{Integrazione della documentazione}: generali miglioramenti sono apportati in base alle indicazioni ricevute con la Revisione di Qualifica. L'integrazione avviene ad opera delle figure interessate alla stesura dei documenti nelle fasi precedenti.
\end{itemize}

