\usepackage[toc,acronym]{glossaries}
\makeglossaries

\newglossaryentry{requisito}{name={requisito}, plural={requisiti},%
	description={Esistono due interpretazioni principali di un requisito}}

\newglossaryentry{periodo}{name={periodo},%
	description={Nel documento, il lasso di tempo che intercorre tra due revisioni successive}}

\newglossaryentry{modellosviluppo}{name={modello di sviluppo},%
	description={principio teorico che indica il metodo da seguire nel progettare e nello scrivere un programma. Essi simulano la realtà per vedere cosa accadrebbe e al fine di ridurre gli errori e ottimizzare prestazioni e risultati}}

\newglossaryentry{cruscotto}{name={cruscotto}, plural={cruscotti},%
	description={Interfaccia utente grafica che fornisce viste a colpo d'occhio di indicatori chiave di prestazione rilevanti per un particolare obiettivo o processo aziendale}}

\newglossaryentry{capitolato}{name={capitolato}, plural={capitolati},%
	description={Documento tecnico redatto dal cliente in cui vengono specificati i vincoli contrattuali(prezzo e scadenze) per lo sviluppo di un determinato prodotto software. Viene presentato in un bando d’appalto per trovare qualcuno che possa svolgere il lavoro richiesto}}

\newglossaryentry{attivita}{name={attività},%
	description={Insieme di una o più azioni il cui completamento porta ad un avanzamento nel complesso}}

