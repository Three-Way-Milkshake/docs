\section{Verbale della riunione}

\subsection{Diagrammi FE}
Sono stati discussi gli aggiustamenti dei diagrammi di classe del front end.
\\Si è deciso di:
\begin{itemize}
	\item non avere a che fare con MVC;
	\item aggiustare i diagrammi con le ultime modifiche;
	\item iniziare a fare unit test.
\end{itemize}


\subsection{Diagrammi BE}
Si è discusso a riguardo di:
\begin{itemize}
	\item \href{https://docs.oracle.com/javase/8/docs/api/java/util/Observer.html}{observer};
	\item \href{https://docs.oracle.com/javase/8/docs/api/java/util/Observable.html}{observable};
\end{itemize}
Sono state eseguite le seguenti modifiche:
\begin{itemize}
	\item rimozione delle classi Subject e Observer, per usare java.util.*;
	\item i singleton diventano bean di Spring;
	\item rimozione di facade.
\end{itemize}
A questo punto è necessario:
\begin{itemize}
	\item correggere i diagrammi;
	\item finire implementazione engine;
	\item finire implementazione connection \\→ cambiare connection handler/accepter dependency;
	\item finire implementazione persistenza\\+ capire la differenza tra il pattern chain of responsibility e il pattern pipeline.
\end{itemize}

\subsection{Diagrammi di sequenza}
Sono stati ricontrollati i diagrammi di sequenza. Dalla discussione è emerso che:
\begin{itemize}
	\item bisogna riordinare la struttura (frecce che non si sovrappongano e siano squadrate);
	\item continuare più possibile implementazione collisioni.
\end{itemize}

\subsection{Riorganizzazione presentazione PB}
In vista della prossima presentazione della PB, si è deciso come ristrutturare le slide. In particolar modo risulta \textbf{necessario ridurre il numero delle slide} della presentazione, perciò si è deciso di:
\begin{itemize}
	\item lasciare una sola slide per il FE;
	\item ridurre al minor numero possibile le slide riguardo i diagrammi di sequenza; 
	\item rimuovere la slide iniziale e la slide finale dal conteggio; 
\end{itemize}
