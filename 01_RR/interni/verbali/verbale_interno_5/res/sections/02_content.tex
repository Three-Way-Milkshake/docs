\section{Verbale della riunione}

\subsection{Confronto con l'azienda}
In questo paragrafo viene illustrata la conversazione avvenuta tramite Google Chat tra il gruppo Three Way Milkshake e l'azienda Sanmarco Informatica, esponendo le nostre domande e le risposte ricevute.
\subsubsection{Domande effettuate}
\begin{enumerate}
	\item Riguardo i vincoli tra tipologie di mezzo e corsie: dobbiamo prevedere la presenza di mezzi di tipo diverso che possono quindi avere capienza di merce diversa? 
	
	\item Sempre riguardo la merce: 1. dobbiamo considerare la quantità di merci da scaricare per ogni POI ? 2. dobbiamo considerare dimensioni diverse per le merci che quindi occupano spazi diversi sui muletti? 
	
	\item Pensavamo di porre come vincolo il fatto che una persona fisica si muove a piedi solo per caricare/scaricare le merci, per il resto si muove a bordo di un muletto. Il che implicherebbe la geolocalizzazione solo per i muletti. Va bene questa assunzione? 
	
	\item I POI devono essere differenziati a seconda del loro scopo. Pensavamo di dividerli in: carico merci (sul muletto), scarico merci (dal muletto, nel magazzino), base deposito muletti. È necessario considerare più punti di carico e più basi di deposito muletti all'interno di uno stesso magazzino? 
	
	\item Siccome le unità in movimento hanno sempre un operatore a bordo, pensavamo di identificare quell'unità con l'ID univoco di quella persona a bordo in quel momento finché svolge la sua funzione. Può andare?
\end{enumerate}
\subsubsection{Risposte ricevute}
\begin{enumerate}
	\item Non serve, si può ipotizzare che tutti i mezzi siano "uguali" e che occupino un numero fisso di posizioni (anche 1x1 può andar bene).
	
	\item No, non sono temi legati a questo capitolato, in quanto si andrebbero ad aprire numerosi altri fronti (tipo problema dello zaino, impilamento, etc...).
	
	\item Il tema dei pedoni è un tema facoltativo, se decidete di implementarlo, dovete prevedere che i pedoni si possano muovere in tutta l'area di lavoro. 
	
	\item La caratterizzazione dei POI non è richiesta, tuttavia facilita la chiarezza del tema. Per cui, secondo me, è possibile prevedere più POI per ogni tipologia all'interno di un magazzino.
	
	\item Consiglio di identificare l'unità tramite l'id univoco del muletto stesso. Al cambio turno, ad esempio, il muletto resta sempre nella stessa posizione mentre chi lascia o inizia il turno si muove. Il muletto, in questo frangente, continua ad "occupare spazio" e quindi deve essere tracciato.
\end{enumerate}

\subsection{Controllo e discussione avanzamento vari documenti}
Si è iniziato il meeting controllando l'avanzamento dei vari documenti, dove tutti sono a buon punto e con data di scadenza 2021-01-07.\\
In particolare:
\begin{itemize}
	\item \textsc{Piano Di Progetto :} Correggere la fase di codifica dei requisiti desiderabili in desiderabili+opzionali;
	\item \textsc{Piano di Qualifica :} Manca solo la sezione dei test.
	\item \textsc{Analisi dei Requisiti :} In avanzamento;
	\item \textsc{Norme di Progetto :} In avanzamento.
\end{itemize}

\subsection{Compilazione e discussione della tabella sul consuntivo dei costi}
Si è discussa la sezione Consuntivo del \textsc{Piano di Progetto}, in particolare ogni membro ha compilato la tabella, e ci si è consultati sulla divisione delle ore e dei costi tra i vari membri.

\subsection{Discussione sui ruoli per frontespizio e registro delle modifiche}
Si sono decisi i ruoli di frontespizio e per il registro delle modifiche dei vari documenti.\newline Questi sono i ruoli su ci siamo accordati:
\begin{itemize}
	\item \textbf{Redattori:} Aggiungono contenuto al documento, non possono essere anche verificatori del documento;
	\item \textbf{Verificatori:} Verificano le sezioni e apportano piccole modifiche ove necessario;
	\item \textbf{Approvatore/Responsabile di Progetto :} Approva il documento prima della sua pubblicazione. E` descritto come \textit{responsabile di progetto} nel registro delle modifiche e \textit{approvatore} nel frontespizio.
\end{itemize}

\subsection{Decisione numerazione della versione dei documenti}
Per la numerazione della versione, si è deciso di adottare la convenzione \textit{x.y.z}.\\
\begin{itemize}
	\item \textbf{{per x:}} Incrementato dal responsabile a seguito di un’approvazione importante, non si azzera mai;
	\item \textbf{{per y:}} Incrementato a seguito di un insieme discreto di modifiche, si azzera ad ogni incremento di x;
	\item \textbf{{per z:}} Incrementato per ogni modifica o verifica minore, si azzera ad ogni modifica di x o y.
\end{itemize}
\subsection{Discussione su rischi e sottoprocessi da aggiungere nelle norme di progetto}
Si è discusso sulle integrazioni da fare nel documento \textsc{Norme di Progetto}, in particolare:
\begin{itemize}
	\item Discussione sull'aggiunta nella sezione Processi di Supporto, dei sottoprocessi di qualità, verifica e validazione;
	\item Discussione sull'aggiunta nella sezione Processi Organizzativi (sottosezione Miglioramento), dei:
	\begin{itemize}
		\item Rischi Tecnologici RIS$\_$T-1;
		\item Rischi Organizzativi RIS$\_$O-1;
		\item Rischi Interpersonali RIS$\_$O-1.
	\end{itemize}
\end{itemize}

\subsection{Divisione verificatori e approvatori}
Si è decisa la suddivisione di verificatori, per documento.\\
\textbf{Verbali interni(5) :}
\begin{enumerate}
	\item Greggio Nicolò;
	\item Zuccolo Giada;
	\item Chiarello Sofia;
	\item Crivellari Alberto;
	\item Tessari Andrea.
\end{enumerate}
\textbf{Verbali esterni(2) :}
\begin{enumerate}
	\item De Renzis Simone;
	\item De Renzis Simone.
\end{enumerate}
\textbf{Studio di fattibilità(7 capitolati) :}
\begin{enumerate}
	\item Chiarello Sofia;
	\item De Renzis Simone;
	\item Tessari Andrea;
	\item Crivellari Alberto;
	\item Zuccolo Giada;
	\item Greggio Nicolò;
	\item Crivellari Alberto.
\end{enumerate}
\textbf{\textsc{Piano di Progetto :}}
\begin{itemize}
	\item Chiarello Sofia.
\end{itemize}
\textbf{\textsc{Analisi dei Requisiti :}}
\begin{itemize}
	\item De Renzis Simone;
	\item Tessari Andrea.
\end{itemize}
\textbf{\textsc{Norme di Progetto :}}
\begin{itemize}
	\item Crivellari Alberto;
	\item Zuccolo Giada.
\end{itemize}
\textbf{\textsc{Piano di Qualifica :}}
\begin{itemize}
	\item Greggio Nicolò.
\end{itemize}
\textbf{\textsc{Glossario :}}
\begin{itemize}
	\item Greggio Nicolò;
	\item De Renzis Simone.
\end{itemize}
\textbf{Responsabili approvatori :}
\begin{itemize}
	\item Greggio Nicolò;
	\item De Renzis Simone.
\end{itemize}