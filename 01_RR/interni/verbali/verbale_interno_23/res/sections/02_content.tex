\section{Verbale della riunione}

\subsection{Punto della situazione su funzionalità non ancora trattate}
    Si riassumono le funzionalità non trattate:
    \begin{itemize}
        \item login, registrazione e logout;
        \item gestione account:
            \begin{itemize}
                \item modifica/elimina utente;
                \item visualizzazione lista utenti;
                \item modifica nome/cognome/password;
                \item reset password;
            \end{itemize}
        \item inserimento liste di task (con possibilità di aggiungere e rimuovere task durante la compilazione);
        \item eliminazione lista task non ancora presa in carico;
        \item differenziazione muletti nella vista globale e loro direzione;
        \item notifica evento eccezionale;
        \item gestione planimetria e percorrenze;
        \item gestione POI:
        \begin{itemize}
            \item modifica posizione;
            \item inserimento nuovo
            \begin{itemize}
                \item posizione;
                \item codice (?);
                \item tipo carico/scarico: eliminare distinzione fra questi due;
                \item tipo base: varco che porta all’esterno del magazzino dove c'è il parcheggio muletti;
            \end{itemize}
            \item rimozione;
        \end{itemize}
        \item visualizzazione prossima task nella mappa (verificare consistenza in analisi);
        \item visualizzazione lista task operatore;
        \item segnalazione completamento task;
        \item verifica visualizzazione spostamenti guida automatica;
        \item gestione guida:
        \begin{itemize}
            \item switch guida manuale;
            \item comandi guida;
            \item segnalazione evento eccezionale;
        \end{itemize}
        \item visualizzazione liste di liste di task ordinate (responsabile);
        \item possibilità di aggiungere/rimuovere unità per l'amministratore (?).
    \end{itemize}



\subsection{Interfacce mancanti}
\label{ui}
    Si discutono le possibili implementazioni delle interfacce utente mancanti.
    \subsubsection{Responsabile}
        \begin{itemize}
            \item Dopo login indirizza a dashboard, la quale:
            \begin{itemize}
                \item mostra le 2 liste di task:
                \begin{itemize}
                    \item ancora da assegnare, elimina intera lista con bottone;
                    \item già assegnate, non più rimovibili;
                \end{itemize}
                \item inserimento lista task:
                \begin{itemize}
                    \item nuova pagina per inserimento/eliminazione di singole task;
                    \item mappa statica con POI i quali sono bottoni selezionabili per generare la lista ordinata;
                \end{itemize}
                \item mappa globale (già realizzata, da sistemare);
            \end{itemize}
            \item pagina gestione account con relative funzionalità già dette sopra e pulsante logout.
        \end{itemize}

    \subsubsection{Amministratore}
    \begin{itemize}
        \item Dashboard:
        \begin{itemize}
            \item mappa globale;
            \item notifiche eventi eccezionali (?);
            \item inserimento nuovo utente con pagina dedicata;
            \item gestione account esistenti;
            \item gestione mappa come discusso in precedenza;
            \item aggiunta unità (apparizione in una base?);
            \item pagina gestione account con relative funzionalità già dette sopra e pulsante logout.
        \end{itemize}
    \end{itemize}




\subsection{Adattamenti necessari alla comunicazione e funzionalità critiche}
    Si discutono le modifiche e le evoluzioni necessarie alle parti già sviluppate.
    \begin{itemize}
        \item adattare collision detection per considerare ostacoli e non solo altre unità in movimento;
        \item se unità in manuale e mossa effettuata si discosta da suggerita, richiedere ricalcolo:
        \begin{itemize}
            \item modificare comportamento comando PATH nel protcollo di comunicazione per gestire correttamente questa eventualità (Greggio Nicolò \& Tessari Andrea);
        \end{itemize}
        \item adattare algoritmo di path finding per gestire sensi unici di percorrenza (Greggio Nicolò);
        \item gestione liste task:
        \begin{itemize}
            \item realizzare interfaccia FE;
            \item lato server:
            \begin{itemize}
                \item ritorno a carico o base?
                \item assegnare nuova lista su terminazione?
                \item disconnessione su raggiunta base?
            \end{itemize}
        \end{itemize}
        \item sistemare POI aggiungendo attributi necessari (base, carico/scarico);
        \item rivedere gestione mappa server:
        \begin{itemize}
            \item oggetti dedicati;
            \item rappresentazione interna;
            \item comunicazione e notifiche modifiche;
        \end{itemize}
        \item capire secure communication:
        \begin{itemize}
            \item vedere se sufficiente sostituire Socket/ServerSocket con implementazione SSL o se sono necessarie ulteriori operazioni;
        \end{itemize}
        \item capire docker.
    \end{itemize}













