VI\_6.1 & Suddividere i documenti in sezioni e incaricare il singolo componente della verifica di parti affini o collegate.
\tabularnewline 
VI\_6.2 & Riferirsi con più attenzione alle \textsc{Norme di Progetto}.
\tabularnewline 
VI\_6.3 & Dedicare più tempo alla verifica e prestare maggior attenzione.
\tabularnewline
VI\_6.4 & Utilizzare esclusivamente Slack per la comunicazione riguardante il progetto
\tabularnewline
VI\_6.5 & Fare ampio uso della strutturazione a \textit{thread} delle discussioni nei canali di Slack
\tabularnewline
VI\_6.6 & Utilizzare Telegram per comunicazioni informali, in caso in cui il destinatario non sia reperibile su Slack.
\tabularnewline
VI\_6.7 & Favorire più riunioni di durata limitata a poche riunioni lunghe
\tabularnewline
VI\_6.8 & Favorire la parallelizzazione delle riunioni che si terranno solo tra i componenti effettivamente interessati dal lavoro da trattare.
\tabularnewline
VI\_6.9 & Approfondire il dettaglio delle decisioni prese.
\tabularnewline
VI\_6.10 & I membri non direttamente interessati nella discussione cureranno la formalizzazione delle decisioni nei fogli virtuali di Confluence.
\tabularnewline
VI\_6.11 & Verrà posta una sezione dedicata al tracciamento delle decisioni sui fogli virtuali di Confluence dedicati ai verbali.
\tabularnewline
VI\_6.12 & Intensificare i contatti con l'azienda proponente.
\tabularnewline
VI\_6.13 & Entro Giovedì 14 Gennaio sarà necessario aver concluso la realizzazione delle slide per la presentazione.
\tabularnewline
VI\_6.14 &	La realizzazione delle slide avverrà ad opera di Alberto, Sofia, Giada, Andrea e Simone. La sezione di Elevator Pitch è in carico a Nicolò.
\tabularnewline
VI\_6.15 & Nelle slide, scrivere poco testo preferendo immagini rappresentative.
