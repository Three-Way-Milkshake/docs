\section{Verbale della riunione}

\subsection{Verifica del periodo trascorso}

Il gruppo si è interrogato sull'andamento del periodo appena trascorso e ha esposto le criticità che si sono manifestate durante il lavoro. Per ognuna ha individuato delle possibili soluzioni per mitigare il riproporsi dei problemi nei periodi successivi.  

\subsubsection{Verifica dei documenti}

Sono state riscontrate diverse lacune nella verifica dei documenti. \'E stato dedicato poco tempo alla verifica, a causa della vicinanza della scadenza di consegna: sono emerse, a posteriori, difformità per quanto riguarda le convenzioni adottate nella scrittura dei documenti ed errori anche di particolare evidenza. Il gruppo propone di:
\begin{itemize}
	\item assegnare più persone alla verifica di ogni artefatto: in particolare, suddividere i documenti in sezioni e incaricare il singolo componente della verifica di parti in qualche modo affini o collegate;
	\item riferirsi con più attenzione alle \textsc{Norme di Progetto} per garantire al massimo l'uniformità nello stile di scrittura;
	\item dedicare più tempo alla verifica e prestare maggior attenzione.
\end{itemize}

\subsubsection{Comunicazione asincrona}

Il gruppo rileva un uso talvolta non deterministico delle piattaforme per la comunicazione asincrona tra i componenti: vengono utilizzate spesso indifferentemente gli applicativi Telegram e Slack, ma senza precisi criteri che discriminino quando usare l'uno o l'altro. Questo ha causato talvolta confusione e difficoltà nel reperimento delle informazioni scambiate. \'E stato deciso di:
\begin{itemize}
	\item utilizzare esclusivamente Slack per la comunicazione riguardante il progetto, secondo i canali istituiti e, all'interno di questi, facendo ampio uso della strutturazione a \textit{thread} delle discussioni;
	\item utilizzare Telegram per comunicazioni informali, o per comunicazioni riguardanti il progetto solo nel caso in cui il destinatario non sia momentaneamente reperibile o non risponda su Slack.
\end{itemize}


\subsubsection{Comunicazione sincrona}

Per quanto riguarda le riunioni sincrone, si è notata la difficoltà di rispettare i limiti di durata stabiliti: talvolta gli incontri sono durati più di un'ora e mezza, e con il prolungarsi della durata l'efficacia degli stessi degradava in modo rilevante. Inoltre, spesso la riunione non vedeva impegnati tutti i componenti del gruppo, ma interessava solo alcuni. Vengono proposti questi approcci:
\begin{itemize}
	\item limitare la durata del meeting a quanto prestabilito: se il tempo non consente di soddisfare tutti i punti dell'ordine del giorno, predisporre un altra riunione nella stessa giornata. Quindi, favorire più riunioni di durata limitata a poche riunioni lunghe;
	\item dove non ci siano decisioni da prendere in gruppo, favorire la parallelizzazione delle riunioni che si terranno solo tra i componenti effettivamente interessati dal lavoro da trattare.
\end{itemize}



\subsubsection{Verbali}

\'E capitato che decisioni prese in sede di riunione si rivelassero essere state male interpretate dai membri del gruppo. Per arginare questo problema, si propone di:
\begin{itemize}
	\item approfondire il dettaglio delle decisioni prese, e appurare che tutti i componenti ne abbiano compreso il contenuto;
	\item migliorare la precisione con cui le riunioni vengono formalizzate in verbali: il gruppo utilizza la piattaforma Confluence per appuntare il contenuto delle riunioni. Si incoraggiano i membri non direttamente interessati nella discussione in corso, a partecipare attivamente nella formalizzazione delle decisioni nei fogli virtuali di Confluence. Questo permette maggior dettaglio e assicura che anche i membri non coinvolti nella discussione abbiano compreso la decisione. Al termine della pagina, verrà posta una sezione dedicata al tracciamento delle decisioni, per facilitare anche la trascrizione nel relativo verbale in \LaTeX{}.
\end{itemize}

\'E stata rilevata la necessità di intensificare i contatti con l'azienda proponente.







\subsection{Preparazione alla presentazione}

\'E stata pianificata l'organizzazione per la presentazione dei documenti alla RR: si stabilisce che entro Giovedì 14 Gennaio sarà necessario aver concluso la realizzazione delle slide. La creazione delle stesse avverrà ad opera di Alberto, Sofia, Giada, Andrea e Simone. La sezione di Elevator Pitch è in carico a Nicolò.

Il gruppo condivide alcune buone pratiche per la realizzazione di slide efficaci: in particolare la necessità di scrivere poco testo, preferendo immagini rappresentative.

