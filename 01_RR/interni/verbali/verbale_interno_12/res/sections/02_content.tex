\section{Verbale della riunione}

\subsection{Controllo stato cruscotto}
Breve discussione sullo stato dei lavori per il cruscotto. In particolare:
    \begin{itemize}
        \item controllo collettivo della correttezza cruscotto e discussione del suo funzionamento;
        \item effettuate alcune correzioni;
        \item discussione sul suo mantenimento.
    \end{itemize}

\subsection{Ultimi controlli sul \textsc{Piano di Qualifica}}
Fatto il punto della situazione sulla correzione del documento:
    \begin{itemize}
   		\item controllo collettivo del documento sistemato;
   		\item indicate alcune correzioni da fare sull'appendice (viste le correzioni precedentemente eseguite nel cruscotto);
		\item assegnata la verifica del documento;
		\item ipotizzata una data di consegna se la verifica andasse subito a buon fine.
	\end{itemize}
	
\subsection{Presentazione delle tecnologie}
Ogni coppia ha presentato al resto del gruppo, con l'ausilio di documenti su Confluence, precedentemente preparati contenenti le varie basi da conoscere, alcune informazioni riguardo le tecnologie approfondite per l'imminente sviluppo del PoC:
    \begin{itemize}
   		\item Java;
   		\item NodeJS;
		\item Angular.
	\end{itemize}
	
\subsection{Come procedere per la codifica}
Organizzato come procedere per lo sviluppo del PoC.
\begin{itemize}
\item Angular: bozza interfaccia
	\begin{itemize}
		\item schermata principale con frecce + mappa;
		\item capire il funzionamento dell'interazione con NodeJS;
	\end{itemize}
	
\item NodeJS: 
 \begin{itemize}
	\item capire interazione con interfaccia;
	\item capire setup chiamate per funzioni possibili;
\end{itemize}
	
\item Java: 
\begin{itemize}
	\item capire uso timer task;
	\item esporre collegamento socket per node;
\end{itemize}
	
\end{itemize}
	