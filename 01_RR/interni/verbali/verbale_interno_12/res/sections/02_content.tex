\section{Verbale della riunione}

\subsection{Controllo stato cruscotto}
Breve discussione sullo stato dei lavori per il cruscotto. in particolare:
    \begin{itemize}
        \item controllo collettivo della correttezza cruscotto e discussione del suo funzionamento;
        \item effettuate alcune correzioni;
        \item discussione del suo mantenimento.
    \end{itemize}

\subsection{Ultimi controlli sul \textsc{Piano di Qualifica}}
Fatto punto della situazione sulla correzione del documento:
    \begin{itemize}
   		\item controllo collettivo del documento sistemato;
   		\item indicate alcune correzioni da fare sull'appendice (viste le correzioni precedentemente fatte nel cruscotto);
		\item assegnata la verifica del documento;
		\item ipotizzata una data di consegna se la verifica andasse subito a buon fine.
	\end{itemize}
	
\subsection{Presentazione delle tecnologie}
Ogni coppia ha presentato al resto del gruppo, con l'ausilio di documenti su Confluence precedentemente preparati contenenti varie le basi da conoscere, alcune informazioni riguardo le tecnologie approfondite:
    \begin{itemize}
   		\item Java;
   		\item NodeJS;
		\item Angular.
	\end{itemize}
	
\subsection{Come procedere per la codifica}
Organizzato come procedere per lo sviluppo del PoC.
\begin{itemize}
\item Angular: bozza interfaccia
	\subitem schermata principale con frecce + mappa;
	\subitem capire il funzionamento dell'interazione con NodeJS;
\item NodeJS: 
	\subitem capire interazione con interfaccia;
	\subitem capire setup chiamate per funzioni possibili;
\item Java: 
	\subitem capire uso timer task;
	\subitem esporre collegamento socket per node;
\end{itemize}
	