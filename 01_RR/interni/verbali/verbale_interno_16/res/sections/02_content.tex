\section{Verbale della riunione}

\subsection{Nuova gestione inserimento task}
\label{newtask}
Durante l'ultimo confronto con il proponente (\textsc{VE\_4.2}) si rende necessaria una rivisitazione del processo di inserimento task da parte del responsabile:
    \begin{itemize}
        \item inserimento di una lista ordinata di task, invece che singole task
        \begin{itemize}
            \item sottocaso inserimento singole task;
            \item durante compilazione lista si possono aggiungere/rimuovere task;
            \item una volta confermata va nell'insieme delle liste di task da evadere;
            \item non può più essere modificata;
            \item può venire cancellata fintantoché non è stata presa in carico;
        \end{itemize}
        \item il sistema assegna le liste alle unità nell'ordine in cui sono state inserite (FIFO);
        \item visualizzazione responsabile:
            \begin{itemize}
                \item elenco di liste non ancora assegnate (che può rimuovere arbitrariamente);

                \item quelle assegnate;

                \item visioni separate:
                \begin{itemize}
                    \item \textbf{"nuove task": }elenco di liste di task non ancora prese in carico, dove si possono eliminare esistenti  o aggiungere nuovi;
                    \item \textbf{"situazione attuale": } mappa real time con muletti che si spostano, ogni unità ha identificativo corrispondente con quanto si trova nella lista disponibile sotto che mostra ogni unità cos'ha in carico.
                \end{itemize}
            \end{itemize}
    \end{itemize}
    \subsubsection{Variazioni necessarie}
    \begin{itemize}
        \item UC4 e UC13;
        \item piccole modifiche ai diagrammi di attività già presenti.
    \end{itemize}

\subsection{Nuova definizione dell'attore operatore}
    In seguito al ricevimento con il docente Vardanega e agli ultimi confronti con il proponente, è emersa la necessità di variare l'interpretazione corrente dell'attore operatore. Fatte le seguenti considerazioni:
    \begin{itemize}
        \item non è necessario il login per chi manovra i muletti;
        \item le unità si occupano di interagire direttamente con il server in tutte le casistiche;
        \item l'operatore umano interagisce unicamente tramite l'interfaccia grafica e compie un sottoinsieme di tutte le operazioni che l'unità può invece avere con il server;
    \end{itemize}
Il gruppo ritiene utile esporre il problema ed il contesto ai docenti Vardanega e Cardin, ed attendere un loro riscontro prima di intraprendere sostanziali modifiche all'\textsc{Analisi dei Requisiti}.

\subsection{Movimenti relativi ad UC11}
\label{moves}
    Durante i lavori di implementazione del PoC sono emerse delle discrepanze fra necessità pratiche e quanto inizialmente rilevato. \\Il comportamento inerente agli spostamenti delle unità ora deciso può così essere riassunto:
    \begin{itemize}
        \item ad ogni istante un unità può:
        \begin{itemize}
            \item muoversi di 1 casella nella direzione in cui si trova;
            \item ruotare su sé stessa di 90° a destra o a sinistra o di 180°;
            \item restare ferma;
        \end{itemize}
        \item sigle:
        \begin{itemize}
            \item \textbf{M}ove;
            \item \textbf{R}ight, \textbf{L}eft e \textbf{T}urnaround;
            \item \textbf{S}top.
        \end{itemize}
    \end{itemize}
    A fronte di ciò, sarà necessario modificare l'interfaccia front end ora presente.

\subsection{Redistribuzione verbali}
    A causa di alcune difficoltà causate da un impegno richiesto maggiore di quanto preventivato per altri compiti, si è deciso di redistribuire la stesura dei verbali ad ora assegnati a De Renzis Simone, e ripianificarne la verifica, come segue:
    \begin{itemize}
        \item \textsc{Verbale Interno 11}:
            \subitem -- \textbf{stesura: }Crivellari Alberto;
            \subitem -- \textbf{verifica: }Greggio Nicolò;
        \item \textsc{Verbale Esterno 4}:
            \subitem -- \textbf{stesura: }Tessari Andrea;
            \subitem -- \textbf{verifica: }Crivellari Alberto;
    \end{itemize}

\subsection{Pulsante completamento task}
    Il completamento di una task verrà così implementato:
    \begin{itemize}
        \item lista mosse pathToNextTask vuota $\implies$ POI raggiunto, appare bottone conferma "fatto";
        \item per permettere simulazioni con unità lasciate totalmente in automatico, prevedere timeout dopo il quale verrà comunque confermato il completamento.
    \end{itemize}

\subsection{Parti del cruscotto live non ancora automatiche}
    \begin{itemize}
        \item alcune sezioni del cruscotto live (le tabelle che indicano il superamento delle metriche) non sono ancora completamente automatizzate;
        \item Crivellari Alberto si assume la responsabilità di controllarne validità e funzionamento fintantoché non si provvederà a realizzare l'automazione completa;
    \end{itemize}

\subsection{Avanzamento PoC}
\label{poc}
    \begin{itemize}
        \item si procede con il lavoro visti i punti discussi in precedenza;
        \item si organizzano incontri a gruppi ridotti per discutere e decidere:
        \begin{itemize}
            \item rilevazione collisioni;
            \item strategia da adottare per scegliere le unità a cui inviare lo stop;
            \item protocollo interno di comunicazione fra server ed unità per gestire lo scambio di tutte le informazioni necessarie.
        \end{itemize}
    \end{itemize}









