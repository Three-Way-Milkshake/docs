
\section{Verbale della riunione}

\subsection{Controllo livello avanzamento diagrammi}
Controllo dell'avanzamento dei vari diagrammi per la Product Baseline.
\begin{itemize}
	\item {FE:} diagrammi della parte front-end, completati;
	\item {BE:} diagrammi della parte back-end, in svolgimento, manca la ristrutturazione della parte collision detection;
	\item {Sequenza:} diagrammi di sequenza, vanno riviste in base alle versioni finali dei diagrammi di classe;
	\item {Attività:} diagrammi di attività, alcune idee su diagrammi di attività da aggiungere, come aggiunta lista task da parte del responsabile.
\end{itemize}

\subsection{Discussione strutturazione allegato tecnico}
Discussione sul come strutturare l'allegato tecnico.
Suddivisione in:
\begin{itemize}
	\item {Introduzione:} con scopo del documento, scopo del prodotto e riferimenti;
	\item {Architettura del prodotto:} con descrizione generale, client e interazione tra client e server, suddividendo i vari diagrammi in queste 3 parti.
\end{itemize}
Discussione sulla presentazione:
\begin{itemize}
	\item breve presentazione, una decina di slide;
	\item 1/2 min per slide;
	\item focus sulla spiegazione dei design pattern utilizzati.
\end{itemize}

\subsection{Discussione requisiti mancanti}
Discussione su altri requisiti da implementare.
\begin{itemize}
	\item login e logout;
	\item registrazione responsabile;
	\item gestione account (modifica, eliminazione utente e visualizzazione lista utenti);
	\item inserimento, modifica e rimozione lista task;
	\item visualizzazione mappa totale muletti, non si differenziano i muletti e loro direzione;
	\item implementare notifica evento eccezionale;
	\item implementare modifica planimetria, percorrenza e gestione POI;
	\item visualizzazione lista di tutti i task, da parte del responsabile;
	\item aggiunta e rimozione unità, da parte del responsabile.	
\end{itemize}
