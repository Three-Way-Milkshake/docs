\section{Verbale della riunione}

\subsection{Decisione utilizzo di Rest}
Si è deciso di non utilizzare Rest per la comunicazione e continuare con i Socket.

\subsection{Discussione diagrammi per RP}
Si è data come scadenza il completamento dei diagrammi entro il 3 Aprile.
Sono stati visti insieme quelli prodotti finora e corretti gli eventuali errori.

\subsection{Protocollo di comunicazione}
Si è discusso il nuovo protocollo di comunicazione per le nuove operazioni tra Admin, Manager e Forklift.
Si trova al seguente \href{https://threewaymilkshake.atlassian.net/wiki/spaces/PORTACS/pages/237764625/Collegamento+NodeJs+Java}{link}.

\subsection{Decisioni data RQ}
Si è deciso di fissare la PB il 2021-04-12, mentre la RQ il 2021-04-19.

\subsection{Domande per il proponente}
\begin{itemize}
	\item Chiedere se è opportuno cambiare il fatto che ogni POI è rappresentato come un tipo char, oppure se cambiarlo in string;
	\item conferma prodotto con Docker.
\end{itemize}



	\item Controllo livello avanzamento diagrammi;
\item Discussione strutturazione allegato tecnico;
\item Discussione requisiti mancanti.

\section{Verbale della riunione}

\subsection{Controllo livello avanzamento diagrammi}
Controllo dell'avanzamento dei vari diagrammi per la Product Baseline.
\begin{itemize}
	\item {FE:} diagrammi della parte front-end, completati;
	\item {BE:} diagrammi della parte back-end, in svolgimento, manca la ristrutturazione della parte collision detection;
	\item {Sequenza:} diagrammi di sequenza, vanno riviste in base alle versioni finali dei diagrammi di classe;
	\item {Attività:} diagrammi di attività, alcune idee su diagrammi di attività da aggiungere, come aggiunta lista task da parte del responsabile.
\end{itemize}

\subsection{Discussione strutturazione allegato tecnico}
Discussione sul come strutturare l'allegato tecnico.
Suddivisione in:
\begin{itemize}
	\item {Introduzione:} con scopo del documento, scopo del prodotto e riferimenti;
	\item {Architettura del prodotto:} con descrizione generale, client e interazione tra client e server, suddivendo i vari diagrammi in queste 3 parti.
\end{itemize}
Discussione sulla presentazione:
\begin{itemize}
	\item breve presentazione, una decina di slide;
	\item 1/2 min per slide;
	\item focus sulla spiegazione dei design pattern utilizzati.
\end{itemize}

\subsection{Discussione requisiti mancanti}
Discussione su altri requisiti da implementare.
\begin{itemize}
	\item login e logout;
	\item registrazione responsabile;
	\item gestione account (modifica, eliminazione utente e visualizzazione lista utenti);
	\item inserimento, modifica e rimozione lista task;
	\item visualizzazione mappa totale muletti, non si differeziano i muletti e loro direzione;
	\item implementare notifica evento eccezionale;
	\item implementare modifica planimetria, percorrenza e gestione POI;
	\item visualizzazione lista di tutti i task, da parte del responsabile;
	\item aggiunta e rimozione unità, da parte del responsabile.	
\end{itemize}
