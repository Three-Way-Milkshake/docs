\section{Verbale della riunione}

\subsection{Funzioni critiche}
\subsubsection{Lato front end}
Si è discusso riguardo gli aggiornamenti degli aggiustamenti delle funzioni critiche dell'applicativo lato front end.
\\Risulta da ultimare:
\begin{itemize}
	\item modifica ed eliminazione dei POI;
	\item tabella task-muletto in parte alla mappa;
	\item modifica POI quando si rimuovono righe/colonne dalla mappa.
\end{itemize}
Rimane in sospeso la decisione di dove parcheggiare le unità a fine turno, e da dove farle entrare nella mappa ad inizio turno.
\subsubsection{Test}
\begin{itemize}
	\item prettier/eslint da ultimare;
	\item code coverage: \\lato server per ottenere risultati aggiornati bisogna fare un merge;\\lato client da sistemare l'automazione.
\end{itemize}

\subsection{Funzioni extra}
Sono state discusse eventuali funzioni aggiuntive da implementare. L'unica contrassegnata da capire se fare o meno è:
\begin{itemize}
	\item Mettere un hover sui pulsanti di movimento quando in modalità guida automatica.
\end{itemize}

\subsection{Controllo stato documentazione}
Sono stati ricontrollati i documenti da presentare e le modifiche da fare su ognuno di essi. 
\subsubsection{Manuale utente}
\begin{itemize}
	\item controllare consistenza di quanto precedentemente scritto;
	\item inserire screen interfaccia manager e admin;
	\item inserire screen delle nuove funzionalità.	
\end{itemize}
\subsubsection{Manuale manutentore}
Risulta necessario sistemare e ampliare:
\begin{itemize}
	\item sezione riguardante installazione di Docker;
	\item sezione riguardante requisiti hardware;
	\item sezione riguardante requisiti software;
	\item sezione riguardante la comunicazione.	
\end{itemize}
\subsubsection{Analisi dei requisiti}
Sistemare versione di Java, cambiando in Java 15.
\subsubsection{Piano di qualifica}
Fare un incremento nelle sezioni riguardanti i test e l'appendice riguardante le osservazioni.
\subsubsection{Piano di progetto}
Inserire sezioni riguardanti questo periodo.

