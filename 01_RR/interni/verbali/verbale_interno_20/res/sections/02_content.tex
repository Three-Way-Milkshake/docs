\section{Verbale della riunione}

\subsection{ Elaborati da produrre per Product Baseline e RQ}
\subsubsection{Product Baseline}
Bisogna consegnare al prof. Cardin:
\begin{itemize}
	\item \textsc{Allegato Tecnico}: diagrammi vari (classi, package, sequenza, attività), scelte architetturali, design pattern notevoli, requisiti soddisfatti e non;
	\item presentazione: slides con estratti da \textsc{Allegato Tecnico};
\end{itemize}

\subsubsection{Revisione di Qualifica}
Per la RQ bisogna consegnare:
\begin{itemize}
	\item \textsc{Manuale Manutentore}: esso comprende 
	\begin{itemize}
		\item tutti i diagrammi per ogni componente;
		\item la descrizione dell'architettura utilizzata;
		\item informazioni e requisiti delle varie tecnologie;
		\item requisiti tecnici vari (hardware o altro);
		\item istruzioni (avvio, test, manutenzione);
	\end{itemize}
	\item \textsc{Manuale Utente}: illustra come utilizzare il prodotto e le sue funzionalità.
\end{itemize}
Inoltre sarà necessario avanzare con il codice implementando almeno i requisiti obbligatori e i test.


\subsection{Discussione \textsc{Piano di Qualifica}}
Sono state discusse alcune modifiche per quanto riguarda il PdQ:
\begin{itemize}
	\item abbassare la soglia RTPI a 0.14;
	\item le considerazione andranno fatte per fase e generali sull'andamento complessivo. Necessario elaborare in modo più approfondito i punti negativi.
\end{itemize}

\subsection{Gestione repository GitHub}
Ricordato al gruppo di controllare le configurazioni in locale della repository e di non lavorare mai direttamente nei branch develop e main.

\subsection{Discussione gestione test}
La creazione dei test sarà eseguita dai membri del gruppo che non hanno scritto il codice da testare, così:
	\begin{itemize}
		\item non saranno influenzati;
		\item vi sarà una copertura più ampia delle funzionalità;
		\item le aspettative saranno reali;
		\item verranno testati anche problemi di comprensione o interpretazione.
	\end{itemize}
Tessari si occuperà di come si eseguono i test in Node.js, mentre Chiarello e Zuccolo per quelli in Angular.js.
\subsection{Discussione \textsc{Norme di Progetto}}
\begin{itemize}
	\item Deciso utilizzo dello standard Google per le convenzioni di tutte le tecnologie utilizzate;
	\item necessità di trovare strumenti che verifichino e controllino tale standard;
	\item inserimento di sezione per l'utilizzo degli IDE.
\end{itemize}
\subsection{Tracciamento metriche di prodotto}
Deciso che Crivellari si occuperà di trovare degli strumenti automatici per tracciare ogni metrica di prodotto.

\subsection{Integrazioni PoC}
Deciso di migliore il PoC prima di procedere con gli altri casi d'uso da implementare:
	\begin{itemize}
		\item aggiustare ritardo nella visualizzazione in real time del responsabile:
			\begin{itemize}
				\item possibile soluzione: prima spostamento e poi invio della posizione;
			\end{itemize}
		\item integrare algoritmo per la rilevazione delle collisioni.
	\end{itemize}
\subsection{Discussione prossimi obiettivi del progetto}
I prossimi passi da eseguire per il completamento del software sono:
	\begin{itemize}
		\item implementare il login;
		\item implementare modifiche alla planimetria;
		\item guida manuale;
		\item collisioni contro ostacoli:
		\begin{itemize}
			\item verificare che basti il ricalcolo percorso;
			\item possibile soluzione: se l'unità cambia il suo percorso ottimale durante la guida manuale, segnala al server e richiede un nuovo percorso tramite PATH;
		\end{itemize}
		\item sensi di percorrenza: rivedere l'algoritmo path finding.
	\end{itemize}
\subsection{Problema cruscotto}
Sistemazione arrotondamento per periodi con durate inferiori ad un'ora.