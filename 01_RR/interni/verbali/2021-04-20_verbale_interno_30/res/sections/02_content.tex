\section{Verbale della riunione}

\subsection{Codifica}
\begin{itemize}
	\item FE:
	\begin{itemize}
		\item logout;
		
		\item registrazione nuovo utente;
		
		\item registrazione nuova unità;
		
		\item gestione unità già presenti;
		
		\item gestione account già presenti;
		
		\item evento eccezionale;
		
		\item visualizzazione task;
		
		\item gestione (aggiunta liste, rimozione liste non prese in carico);
		
		\item discussione se usare tecnologia PrimeNG.
	\end{itemize}
\end{itemize}

\subsection{Manuale utente}
	\begin{itemize}
		\item Introduzione classica;
		
		\item requisiti di sistema hardware/software;
		
		\item come installare/far partire le varie robe (anche docker);
		
		\item funzionalità: dividiamo i 3 attori;
		\begin{itemize}
			\item operatore;
		
			\item admin;
		
			\item manager;
		\end{itemize}
		
		\item glossario built-in minimale.
	\end{itemize}

\subsection{Manuale manutentore}
	\begin{itemize}
		\item Introduzione;
		
		\item tecnologie e librerie e framework coinvolti;
		\begin{itemize}
		
		
		\item server:
		\begin{itemize}
		
		\item java;
		
		\item spring;
		
		\item junit;
		
		\item mockito;
		
		\item gson;
		
		\item json;
		
		\item docker;
		
		\item gradle;
		
		\item shadow jar;
	\end{itemize}
		
		\item client:
		\begin{itemize}
		
		\item node.js;
		
		\item angular (html, css, js/ts);
		
		\item framework primeNG;
	\end{itemize}
	\item altro:
	\begin{itemize}
		\item scm con git su github;
		
		\item build automation con github action;
	\end{itemize}
	\end{itemize}
	\item requisiti di sistema (simil manuale utente);
	
	\item installazione componenit (manuale utente);
	
	\item test;
	
	\item architettura:
	\begin{itemize}
	
	
	\item diag classi server + loro descrizione ed analisi;
	\begin{itemize}
	\item minimal visione complessiva;
	
	\item zoom su:
	\begin{itemize}
	\item persistence;
	
	\item map;
	
	\item client;
	
	\item connection;
\end{itemize}
\end{itemize}
	\item diag classi client + loro descrizione ed analisi:
	\begin{itemize}
	\item resp/admin;
	
	\item muletto;
\end{itemize}
	\item diagrammi di sequenza:
	\begin{itemize}
		\item 2 delle slide + eventualmente altri;
	\end{itemize}
\end{itemize}
	\item come estendere PORTACS:
	\begin{itemize}
	
	\item strategy per l’algoritmo path finding;
	
	\item nuovi utenti / nuovi client:
	\begin{itemize}
	
	\item server → estendere User / Client;
	
	\item client → nuova cartellaa con codice specifico + usare generic;
\end{itemize}
	
	\item persistenza;
	
	\item nuovo handler per collision / ma anche l’intero algoritmo;
\end{itemize}
	
	\item glossario.
	\end{itemize}

\subsection{Cruscotto}
Nella verifica del prodotto inserire i test d'unità e l'analisi statica.