\section{Verbale della riunione}

\subsection{Modifica workflow dei verbali, pull request dopo la verifica}
Abbiamo deciso di modificare il workflow aggiungendo una \textbf{fase di pull request} per integrare i verbali al ramo principale.
Prima questa fase si faceva lo stesso ma non faceva parte del workflow del ticket del verbale, ma si creava un nuovo task di approvazione e inclusione nel branch principale del verbale.
Dobbiamo inoltre modificare la sezione relativa delle norme.
Workflow:
\begin{itemize}
	\item fine verbale;
	\item assegnare l'approvazione al prossimo componente relativamente all'adeguato foglio su cofluence, relativo alla stesura dei verbali;
	\item compilare l'approvazione con  frontespizio e changelog con \textbf{approvingTable};
	\item vado nel branch del verbale e creo una pull request, per unire il branch del verbale al ramo principale (\textbf{preferibile:} assegnare a Nicolò il task di gestire le pull request dei verbali).
	
\end{itemize}
\subsection{Discussione su modifiche per il PdQ}
Abbiamo discusso cosa fare riguardo l'incremento del \textsc{Piano di Qualifica}, e abbiamo identificato alcuni punti da migliorare:
\begin{itemize}
	\item abbassare il limite minimo della metrica REI a 0.2;
	\item abbassare il limite minimo della metrica RTPI a 0.14;
	\item aggiornare appendice B, sezione 1 con considerazioni sul cruscotto più dettagliate e divise per fasi;
	\item aggiornare appendice C, sezione 1, aggiungendo altri problemi incontrati e suddividendo per fasi;
	\item aggiornare appendice C, sezione 2, integrando il ruolo programmatore;
	\item aggiornare appendice C, sezione 3, aggiungendo tutti gli altri strumenti utilizzati.
\end{itemize}

\subsection{Discussione su come procedere con il design}
Dobbiamo contattare il proponente, probabilmente attraverso un meeting telematico, per chiarire alcuni dubbi sui design pattern del nostro applicativo.
Abbiamo deciso di proseguire con il design creando i vari diagrammi, e ci siamo suddivisi il lavoro:
\begin{itemize}
	\item diagrammi di classe relativi a Java: \textbf{Greggio Nicolò} e \textbf{De Renzis Simone};
	\item diagrammi di classe relativi a NodeJS: \textbf{Tessari Andrea};
	\item diagrammi di classe relativi a Angular: \textbf{Chiarello Sofia} e \textbf{Zuccolo Giada};
	\item diagrammi dei package: \textbf{De Renzis Simone};
	\item diagrammi di attività: \textbf{Crivellari Alberto};
	\item diagrammi di sequenza: \textbf{Crivellari Alberto}.
\end{itemize}

\subsection{Discussione su come procedere con il PoC}
Abbiamo riscontrato alcuni aspetti da migliorare o aggiungere al nostro applicativo:
\begin{itemize}
	\item implementare il login;
	\item modificare la planimetria, ad esempio a magazzino fermo;
	\item implementare la guida manuale;
	\item gestire la collisione con muri e ostacoli;
	\item rivedere l'algoritmo pathfinding per i sensi di percorrenza;
	\item implementare interfaccia aggiunta dei task;
	\item capire nuove tecnologie da implementare, ad esempio secure communication e docker;
	\item sistemare \textbf{service} di angular.
\end{itemize}