\section{Verbale della riunione}

\subsection{Nuova convenzioni verbali}
Usare [data]\_verbale\_[interno/esterno]\_[numero verbale] per i file .tex, il nome della cartella e il nome del branch su GitHub.

\subsection{Esito RP}

\subsubsection{Indicazioni generali}
\begin{itemize}
	\item Non usare parola "sezione" ma solo \S\;
	\item nei verbali esterni:
	\begin{itemize}
		\item aggiungere frase dopo il tracciamento delle tematiche: "per le decisioni relative… si rimanda al verbale interno successivo alla data d'incontro di questo.";
		\item nel verbale interno successivo prendere decisioni dove necessario riguardo i temi trattati nel verbale esterno, facendo riferimento ai codici della tabella Q\&A VEx\_n …;
	\end{itemize}
	\item nelle \textsc{Norme di Progetto} rimuovere la parola "Riferimento" quando ripetuta nelle sezioni varie;
	\item si è ricordato al gruppo di usare correttamente gli accenti e le maiuscole.
\end{itemize}

\subsubsection{Piano di Progetto}
Dividere l'appendice A in:
\begin{itemize}
	\item riscontro rischi;
	\item valutazione misure di mitigazione adottate;
	\item adattare \S\ 3.4 per richiamare nuova appendice A che parla di standard di qualità (guardare \S\ successiva).
\end{itemize}

\subsubsection{Piano di Qualifica}
\begin{itemize}
	\item Tabella obiettivi in \S\ 2.3 → rimuovere colonna metriche;
	
	\item \S\ 2.4 rimuovere e spostare tabella a fine \S\ 3 come riassunto;
	
	\item \S\ 3 diventa definizione univoca delle metriche:
	\begin{itemize}
		\item alla struttura attuale, per ogni metrica, aggiungere nella sua descrizione valori preferibili/accettabili e obiettivo di riferimento, come da tabella già presente;
		\item mettere tabella riassuntiva ora presente in \S\ 2.4 a conclusione della sezione;
	\end{itemize}
	\item trasferire appendice D \textsc{Piano di Qualifica} → appendice A \textsc{Norme di Progetto};
	
	\item adattare \textsc{Piano di Qualifica} a tale cambiamento dove si fa eventualmente riferimento a ciò;
	
	\item appendice C:
	\begin{itemize}
		\item continuare ad integrarla al termine delle fasi;
		
		\item rinominare sottosezioni → rendere consistente con nostre fasi e non revisioni.
	\end{itemize}
	
\end{itemize}

\subsubsection{Incongruenza modello di sviluppo}
\textbf{Problema:}\\
La scelta di frammentare la pianificazione in distinte sezioni di primo livello, poi internamente suddivise in Periodi, e di dotare ciascuno di essi di un suo proprio corredo di preventivo e consuntivo, parcellizza (e quindi offusca invece di esporre) la visione d'insieme della pianificazione del progetto e del suo andamento effettivo.\\
\textbf{Soluzione:}\\
non modifichiamo perché, visto lo stato di avanzamento del progetto e data la priorità dello sviluppo delle funzionalità, il gruppo non ritiene praticabile modificare ulteriormente la struttura del \textsc{Piano di Progetto}, in quanto questo porterebbe un (ulteriore) rilevante aumento dei costi e introdurrebbe potenziali ritardi. Vengono riconosciuti gli errori commessi e tramite l'esperienza acquisita nei mesi di lavoro il gruppo identifica nella pianificazione di periodo ora adottata un discreto strumento di organizzazione e suddivisione dei lavori. 

\subsubsection{Riferimenti ai documenti}
Ricordarsi di aggiornare i riferimenti ai documenti.\\
Dare una versione anche al \textsc{Glossario} quando esportato.

\subsubsection{Analisi dei Requisiti}
\begin{itemize}
	\item togliere distinzione POI di carico/scarico;
	
	\item UC3.4 reset password → cambiare di conseguenza scrivendo meglio gli scenari;
	
	\item UC 3 → togliere responsabile → gestione account presenti;
	\begin{itemize}
		\item modifica profili account;
		
		\item elimina;
		
		\item visualizzazione;
		
		\item reset password;
	\end{itemize}
	UC X → utente autenticato → modifica proprio account;
	\begin{itemize}
		\item modifica nome;
		
		\item modifica cognome;
		
		\item modifica password;
	\end{itemize}
	\item correzione figura 4.2.1 non termina flusso in un ramo;
	
	\item correzione requisiti:
	\begin{itemize}
		\item quelli di qualità sui docker file → vincoli;
		\item primi 3 attuali vincolo → funzionali;
	\end{itemize}
\end{itemize}

\subsection{Guida manuale}
Permettere all'unità rotazioni e turnaround anche da fermi.\\
Indicare stop quando viene raggiunto un POI.

\subsection{Cruscotto Web}
Aggiornare ed impostare le sezioni del Cruscotto Web con la nuova codifica -> assegnato a Crivellari Alberto.

\subsection{Ritorno alla base dell'unità}
Due possibilità:
\begin{itemize}
	\item torna alla base uscendo dalla mappa e successivamente si disconnette;
	\item richiede una nuova lista di task.
\end{itemize}

\subsection{RestAPI}
Viene incentivata una ricerca sulla documentazione Angular riguardante le RestAPI -> assegnata a Tessari Andrea