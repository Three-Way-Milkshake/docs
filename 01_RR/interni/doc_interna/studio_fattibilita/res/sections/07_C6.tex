\subsection{Capitolato C6 - RGP: Realtime Gaming Platform}
\subsubsection{Informazioni generali}
	\begin{itemize}
	\item \textbf{Nome:} RGP: Realtime Gaming Platform;
	\item \textbf{Proponente:} Zero12 s.r.l.;
	\item \textbf{Committente:}  Prof. Tullio Vardanega e Prof. Riccardo Cardin;
	\end{itemize}
\subsubsection{Descrizione del capitolato}
    Il \gls{capitolato}\textsubscript{G} proposto prevede la realizzazione di un videogioco a scorrimento verticale, fruibile da dispositivi mobile, con la possibilità di giocare in multiplayer real time.
    La modalità di gioco, per quanto riguarda il gameplay, sarà simile ad Aero Fighters\footnote{Esempio Aero Fighters su youtube: \url{https://youtu.be/5Phj-735p30?t=28}}, mentre la grafica potrà essere scelta liberamente dal gruppo o fornita dal team di Zero12.

\subsubsection{Finalità del progetto}
    Sono previste le modalità di gioco single e multiplayer, ma
    il cuore del \gls{progetto}\textsubscript{G} sono le sfide tra più giocatori: il gioco è ad eliminazione, l'ultimo rimasto vince.
    Durante la partita deve essere possibile vedere, in tempo reale, i movimenti del rivale e per garantire l'equità bisognerà prevedere la sincronizzazione di eventuali nemici e powerup.
    La modalità single player invece consiste in una serie infinita di livelli a difficoltà crescente. Il gioco termina quando l'utente ha concluso le vite oppure se non ha raccolto power-up sufficienti a mantenere il proprio oggetto attivo.
    Il \gls{progetto}\textsubscript{G} è finalizzato allo sviluppo di un'applicazione mobile superando dei vincoli quali:
    \begin{itemize}
    	\item ricerca delle tecnologie AWS per capire quale si può adattare meglio ad un gioco con \glspl{requisito}\textsubscript{G} di realtime, raccogliendo le motivazioni a supporto della scelta;
    	\item realizzazione della componente server-side;
    	\item implementazione del gioco per piattaforma mobile.
    \end{itemize}
\subsubsection{Tecnologie interessate}
    Le tecnologie interessate sono:
    \begin{itemize}
    	\item \textbf{AWS: }per la realizzazione dei servizi per la gestione dei giochi;
    	\item \textbf{NodeJs: }per lo sviluppo di codice server side;
    	\item \textbf{\href{https://swift.org/about/}{Swift} o \href{https://kotlinlang.org/}{Kotlin}: }per l'implementazione delle applicazioni mobile, rispettivamente per iOS e Android;
    	\item \textbf{\href{https://git-scm.com/about}{Git}: }per il versionamento del codice sviluppato.
    \end{itemize}
\subsubsection{Aspetti positivi}
    Il \gls{capitolato}\textsubscript{G} permette di entrare nel mondo dello sviluppo mobile nativo e della gestione di comunicazioni real time fra più dispositivi. Inoltre consente anche di familiarizzare con le tecnologie serverless anche tramite i servizi di Amazon AWS, per i quali l'azienda proponente prevede corsi di formazione.
\subsubsection{Criticità e fattori di rischio}
    L'ambiente AWS è molto vasto ed è critica la scelta iniziale del servizio sul quale basarsi non avendo alcuna esperienza pregressa: una scelta errata può portare ad una realizzazione non ottimale, rischiando così di non soddisfare pienamente i \glspl{requisito}\textsubscript{G} real time della parte multiplayer del videogioco. Inoltre lo sviluppo di applicazioni mobile native è già un compito arduo di per sé, per cui aggiungere comunicazioni real time e sincronizzazioni tra dispositivi diversi aumenta ulteriormente il grado di difficoltà e le possibilità d'errore.
\subsubsection{Conclusioni}
    I corsi di formazione offerti dall'azienda potrebbero essere interessanti, ma valutata l'elevata difficoltà e lo scarso interesse da parte del gruppo, la scelta non si pone qui.