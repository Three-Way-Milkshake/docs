\subsection{Capitolato C3 - GDP - Gathering Detection Platform}


\subsubsection{Informazioni generali}

\begin{itemize}
	\item{\textbf{Nome:}} GDP - Gathering Detection Platform
	\item{\textbf{Proponente:}} SyncLab
	\item{\textbf{Committente:}} Prof. Vardanega Tullio, Prof. Cardin Riccardo
\end{itemize}



\subsubsection{Descrizione del capitolato}
In seguito al lockdown generato a causa del virus COVID-19, si sono venute a creare situazioni, lungo l'arco della vita quotidian,a in cui le persone entrano in contatto tra di loro, generando assembramento.\\
L’informatica può aiutare a governare meglio questa situazione.\\
Il software GDP (Gathering Detection Platform) è costituito da una piattaforma che rappresenta, mediante visualizzazione grafica, le zone potenzialmente a rischio di assembramento col fine di prevenirne di nuove, attraverso sensori e varie sorgenti di dati.\\


\subsubsection{Finalità del progetto}

Il software finale prevede l'acquisizione di informazione da sensoristica e da altre sorgenti:
\begin{itemize}
	\item{\textbf{Sensori:}}
	    \begin{itemize}
	        \item telecamere
	        \item dispositivi contapersone
	        \item etc.
	    \end{itemize}
	\item{\textbf{Sorgenti varie e eterogenee:}}
	    \begin{itemize}
	        \item flussi di prenotazioni Uber
	        \item orari dei mezzi di trasporto con capienze medie per corsia (autobus, metro, treno)
	        \item etc.
	\end{itemize}
\end{itemize}
Gli utilizzatori del software potranno vedere una rappresentazione delle zone a rischio (attuali o possibilmente in futuro), attraverso una heat-map.\\
In questo modo potranno accedere alla situazione globale dei vari flussi:
\begin{itemize}
    \item in tempo reale, con bassa latenza
    \item flussi previsti in futuro.
    \item flussi vecchi raccolti e storicizzati.
\end{itemize}


\subsubsection{Tecnologie interessate}

\begin{itemize}
	\item{\textbf{Java, \href{https://angular.io/}{Angular:}}} linguaggi suggeriti dal proponente per lo sviluppo del server back-end e della componente Web Application del sistema;
	\item{\textbf{\href{https://leafletjs.com/}{framework Leaflet}:}} framework\textsubscript{G} da utilizzare per la gestione delle mappe, ad esempio heatmap;
	\item{\textbf{protocolli asincroni:}} per la comunicazione tra le varie componenti;
	\item{\textbf{pattern Publisher/Subscriber e protocollo \href{https://mqtt.org/}{MQTT}:}}\newline (Message Queue Telemetry Transport), consigliato per essere open, di facile implementazione, di ampia diffusione in applicazioni IoT e M2M.
\end{itemize}

Inoltre il proponente specifica i seguenti requisito\textsubscript{G} minimi:
\begin{itemize}
	\item \textbf{responsive} deve sempre rispondere a una richiesta di servizio, anche in caso di guasto;
	\item \textbf{resilient} I servizi devono essere sempre ripristinabili a seguito di guasti;
	\item \textbf{elastic} I servizi devono scalare in base alla domanda;
	\item \textbf{message-driven} I servizi devono rispondere al mondo, senza controllarlo.
\end{itemize}
Un sistema che soddisfa questi requisito\textsubscript{G} è definito un Sistema Reattivo.


\subsubsection{Aspetti positivi}

Il progetto\textsubscript{G} si pone in un contesto molto tecnologico, di grande interesse per l'attuale mercato, molto sensibile e maturo su questa problematica.\\
Anche le tecnologie impiegate sono attuali e valide, il cui studio porterebbe all'acquisizione di competenze molto utili in ambito lavorativo.



\subsubsection{Criticità}

Nonostante il capitolato\textsubscript{G} abbia riscosso un buon interesse da parte del gruppo, le tecnologie da impiegare, in quanto alcune nuove e diverse da qualsiasi altro linguaggio imparato durante il percorso di studi, gravano molto sulla facilità di realizzazione del progetto\textsubscript{G}.\\
In particolare le criticità principali sono:
\begin{itemize}
    \item \textbf{Leaflet:} Tecnologia relativa alle heat-map e framework\textsubscript{G} Leaflet;
    \item \textbf{Linguaggio Angular:} Linguaggio di programmazione che viene eseguito su web browser;
    \item \textbf{Protocolli asincroni, Pattern Publisher/Subscriber e protocollo MQTT: } Varie tecnologie, alcune sconosciute, che servono alla realizzazione del progetto\textsubscript{G}.
\end{itemize}


\subsubsection{Conclusioni}

Nonostante l'elevata attenzione che questi sistemi hanno attirato, il gruppo non ha preso in considerazione questo capitolato\textsubscript{G} in quanto molti altri gruppi di lavoro avevano già segnalato un alto interesse per questo progetto\textsubscript{G}. \\
