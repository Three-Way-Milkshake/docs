\subsection{Capitolato C7 - Soluzioni di sincronizzazione Desktop}
\subsubsection{Informazioni generali}
	\begin{itemize}
	\item \textbf{Nome:} Soluzioni di sincronizzazione Desktop;
	\item \textbf{Proponente:} Zextras;
	\item \textbf{Committente:}  Prof. Vardanega Tullio e Prof. Cardin Riccardo;
	\end{itemize}
\subsubsection{Descrizione del capitolato}
Il capitolato\textsubscript{G} richiede di sviluppare un algoritmo di sincronizzazione Desktop solido ed efficiente\textsubscript{G} in grado di garantire il salvataggio in cloud del lavoro, in modo da poter raggiungere in qualsiasi momento e da qualsiasi dispositivo il proprio lavoro. Inoltre deve essere sviluppata un'interfaccia multipiattaforma per l'uso dell'algoritmo nei più importanti sistemi operativi esistenti (MacOs, Windows, Linux) senza richiedere all'utente l'installazione di ulteriori prodotti per il funzionamento.
\subsubsection{Finalità del progetto}
L'obiettivo è quello di creare questo algoritmo di sincronizzazione che funzioni in più piattaforme in grado di garantire il salvataggio in cloud del lavoro e contemporaneamente la sincronizzazione dei cambiamenti presenti in cloud. Inoltre viene richiesto l'utilizzo dell’algoritmo sviluppato per richiedere e fornire i cambiamenti ai contenuti in sincronizzazione verso il prodotto Zextras Drive.
L'algoritmo deve avere le seguenti principali funzionalità:
\begin{itemize}
\item {Configurazione ed autenticazione dell’utente};
\item {Gestione di cosa sincronizzare e di cosa ignorare nelle cartelle cloud};
\item {Gestione di cosa sincronizzare e di cosa ignorare nelle cartelle locali};
\item {Sincronizzazione costante dei cambiamenti, siano essi locali o remoti};
\item {Possibilità di modifica delle preferenze a posteriori};
\item {Sistema di notifica utente dei cambiamenti}.
\end{itemize}
\subsubsection{Tecnologie interessate}
L'azienda esprime la necessità di sviluppare l'algoritmo per i più importanti sistemi operativi esistenti (MacOs, Windwos, Linux) e consiglia l'utilizzo:
\begin{itemize}
\item \textbf{\href{https://wiki.qt.io/About_Qt}{Qt Framework}:} strumento basato sul linguaggio C++ orientato ad oggetti, da utilizzare per creare l'interfaccia poiché fortemente supportato e documentato;
\item \textbf{Python:} strumento consigliato per lo sviluppo della Business Logic, linguaggio ad alto livello adatto allo sviluppo di applicazioni distribuite.
\end{itemize}
\subsubsection{Aspetti positivi}
Il progetto\textsubscript{G} si pone in un contesto molto utilizzato sia da utenti privati che da aziende. Inoltre le tecnologie consigliate fanno già parte delle conoscenze di gran parte dei membri del gruppo e ci sono molti esempi, anch'essi ben conosciuti, a cui ispirarsi per la creazione dell'algoritmo (Dropbox, Google Drive, ect).
\subsubsection{Criticità e fattori di rischio}
Il confronto con tecnologie già esistenti di questo tipo può essere svantaggioso poiché si potrebbe creare un prodotto molto più inefficiente. Inoltre le richieste sono numerose e abbastanza complesse.
\subsubsection{Conclusioni}
Si è deciso di puntare su altri capitolato\textsubscript{G} poiché questo non ha suscitato grande interesse per quanto riguarda la tematica che affronta e il settore interessato.