\subsection{Capitolato C4 - HD Viz, visualizzazione di dati con molte dimensioni}
\subsubsection{Informazioni generali}
	\begin{itemize}
	\item \textbf{Nome:} HD Viz, visualizzazione di dati con molte dimensioni;
	\item \textbf{Proponente:} Zucchetti;
	\item \textbf{Committente:}  Prof. Vardanega Tullio e Prof. Cardin Riccardo;
	\end{itemize}
\subsubsection{Descrizione del capitolato}
Il capitolato\textsubscript{G} propone la realizzazione di un'applicazione chiamata "HD Viz", utile per l'interpretazione dell'analisi di dati a molte dimensioni, raccolti sfruttando tecniche di statistica, machine learning e intelligenza artificiale.
Richiede quindi lo sviluppo di un'applicazione per la visualizzazione di dati con molte dimensioni (ovvero i dati da visualizzare dovranno poter avere almeno fino a 15 dimensioni e deve però essere possibile visualizzare anche dati con meno dimensioni) a supporto della fase\textsubscript{G} esplorativa di questi dati da interpretare, forniti al sistema di visualizzazione sia con query che da file in formato CSV.
La visualizzazione di dati con il software "HD Viz" deve essere presentata almeno con i grafici "Scatter plot Matrix", "Force Field", "Heat Map" e "Proiezione Lineare Multi Asse". In particolare dovrà ordinare i punti nel grafico "Heat Map" per evidenziare i "cluster" presenti nei dati.
Inoltre sono proposti altri requisito\textsubscript{G} valutati come opzionali.
\subsubsection{Finalità del progetto}
L'obiettivo è sviluppare un'applicazione per la visualizzazione di dati, di molte dimensioni, raccolti con la finalità di conseguire poi con un'analisi dei dati.
Il software "HD Viz" dovra presentare almeno le visualizzazioni con:
\begin{itemize}
\item \textbf{Scatter plot Matrix} (fino ad un massimo di 5 dimensioni);
\item \textbf{Force Field};
\item \textbf{Heat Map} (obbligatoriamente dovrà organizzare i punti per evidenziare i "cluster" presenti nei dati);
\item \textbf{Proiezione Lineare Multi Asse};
\end{itemize}
\subsubsection{Tecnologie interessate}
L'azienda indica lo sviluppo dell'applicazione "HD Viz" con:
\begin{itemize}
\item \textbf{HTML/CSS/\href{https://developer.mozilla.org/en-US/docs/Web/JavaScript/About_JavaScript}{JavaScript}} utilizzando la \textit{libreria D3.js};
\item \textbf{Java con server \href{http://tomcat.apache.org/}{Tomcat}} o \textbf{JavaScript con server Node.js} per la parte server di supporto alla presentazione nel browser e alle query ad un database SQL o NoSQL.
\end{itemize}
\subsubsection{Aspetti positivi}
Questo capitolato\textsubscript{G} propone lo sviluppo un progetto\textsubscript{G} che si inserisce in un ambito relativamente nuovo, cioè la raccolta, la gestione, la memorizzazione e la visualizzazione di grandi quantità di dati, i Big Data, al giorno d'oggi utilizzabili in tutti i vari campi.
Lo sviluppo di questo progetto\textsubscript{G} avrebbe permesso l'approfondimento di questo nuovo ma più complesso ambito, utilizzando delle tecnologie non totalmente nuove alla gran parte dei membri del gruppo.
\subsubsection{Criticità e fattori di rischio}
Sebbene le tecnologie interessate per lo sviluppo dell'applicazione non siamo totalmente nuove alla gran parte dei membri del gruppo, applicarle per il soddisfacimento delle richieste obbligatorie e dei requisito\textsubscript{G} opzionali, potrebbe portare ad una situazione svantaggiosa in quanto, sia le richieste obbligatorie sia i requisito\textsubscript{G} opzionali, siano più complessi da sviluppare e nel complesso anche numerose.
\subsubsection{Conclusioni}
Malgrado l'ambito abbastanza stimolante, il capitolato\textsubscript{G} ha riscosso un misurato interesse da parte del gruppo.
Per questo motivo si è scelto di selezionare altri capitolato\textsubscript{G} che riscontravano maggiore curiosità nel gruppo e che si avvalgono di tecnologie nuove.