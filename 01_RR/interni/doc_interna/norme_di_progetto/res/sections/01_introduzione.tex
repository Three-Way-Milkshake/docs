\section{Introduzione}
\subsection{Scopo del documento}
    Questo documento ha lo scopo di fissare e definire tutte le regole, convenzioni e buone pratiche utili a formare un way of working condiviso alla base da tutti i componenti del gruppo per assicurare una collaborazione efficiente ed efficace. Si discuteranno inoltre i vari strumenti che verranno adottati per facilitare lo sviluppo del progetto\textsubscript{G} e per promuovere un'organizzazione adeguata.
    Si ritiene inoltre che la definizione ed il mantenimento per incremento di un documento condiviso all'interno del gruppo di lavoro, che definisca e raccolga quanto descritto in maniera formale e centralizzata, possa favorire, in un contesto dove i membri possano variare, l'inserimento di nuovi componenti facilitandone l'ambientamento. Pur non essendo questo il contesto di lavoro attuale, è comunque una buona pratica da sperimentare e consolidare.

\subsection{Scopo del prodotto}
Il capitolato\textsubscript{G} C5 propone un progetto\textsubscript{G} in cui viene richiesto lo sviluppo di un software per il monitoraggio in tempo reale di unità che si muovono in uno spazio definito. All'interno di questo spazio, creato dall'utente per riprodurre le caratteristiche di un ambiente reale, le unità dovranno essere in grado di circolare in autonomia, o sotto il controllo dell'utente, per raggiungere dei punti di interesse posti nella mappa.  La circolazione è sottoposta a vincoli di viabilità e ad ostacoli propri della topologia dell'ambiente, deve evitare le collisioni con le altre unità e prevedere la gestione di situazioni critiche nel traffico.

\subsection{Termini, abbreviazioni ed altri documenti}
    Tutti i termini che necessitano di una spiegazione, per fornire un'adeguata comprensione, o perché possono causare ambiguità nel contesto, sono definiti nel glossario. A dispetto di tutti gli altri documenti presenti, il glossario è sotto la veste di una pagina web, più nello specifico in una wiki\textsubscript{G} di GitHub, con vocaboli suddivisi tra "Acronimi" e "Termini", disposti in ordine alfabetico al fine di facilitare la navigazione. Le voci di glossario saranno seguite da una G pedice mentre gli acronimi da una A (e.g.: voce di glossario\textsubscript{G} ; acronimo\textsubscript{A}).
    Ad ogni rilascio verrà creato un file glossario.pdf contenente i vari vocaboli con la loro definizione, per una consultazione offline.
    Inoltre quando si farà riferimento ad un altro documento o al documento stesso, il nome di questo sarà in maiuscoletto (e.g.: \textsc{esempio nome documento}).

\subsection{Riferimenti}
\label{ref}
    \subsubsection{Riferimenti Normativi}
        \begin{itemize}
            \item Standard ISO 12207: \url{https://www.math.unipd.it/~tullio/IS-1/2009/Approfondimenti/ISO_12207-1995.pdf};
            \item Standard uml\textsubscript{A} 2.0: \url{https://www.omg.org/spec/UML/2.0/Superstructure/PDF};
            \item Diagrammi dei usecase\textsubscript{G}: \url{https://www.math.unipd.it/~rcardin/swea/2021/Diagrammi\%20Use\%20Case_4x4.pdf};
            \item Diagrammi delle classi: \url{https://www.math.unipd.it/~rcardin/swea/2021/Diagrammi\%20delle\%20Classi_4x4.pdf};
            \item Diagrammi dei package: \url{https://www.math.unipd.it/~rcardin/swea/2021/Diagrammi\%20dei\%20Package_4x4.pdf};
            \item Diagrammi di attivita\textsubscript{G}: \url{https://www.math.unipd.it/~rcardin/swea/2021/Diagrammi\%20di\%20Attivit\%c3\%a0_4x4.pdf};
            \item Diagrammi di sequenza: \url{https://www.math.unipd.it/~rcardin/swea/2021/Diagrammi\%20di\%20Sequenza_4x4.pdf};
            \item Standard ISO 8601: \url{https://www.iso.org/iso-8601-date-and-time-format.html}.
        \end{itemize}

    \subsubsection{Riferimenti Informativi}
        \begin{itemize}
        	\item \textsc{\href{https://github.com/Three-Way-Milkshake/docs/wiki/Glossario}{Glossario}}: per la definizione dei termini (pedice G) e degli acronimi (pedice A) evidenziati nel documento;
            \item \url{https://www.math.unipd.it/~tullio/IS-1/2020/Dispense/L03.pdf}
            \item \url{https://www.math.unipd.it/~tullio/IS-1/2020/Dispense/L06.pdf}
            \item \url{https://www.math.unipd.it/~tullio/IS-1/2020/Dispense/FC2.pdf}
            \item \url{https://www.math.unipd.it/~rcardin/swea/2021/SOLID\%20Principles\%20of\%20Object-Oriented\%20Design_4x4.pdf}
            \item \url{https://www.math.unipd.it/~tullio/IS-1/2020/Dispense/L09.pdf}
            \item \url{https://docs.github.com/en/free-pro-team@latest/actions}
            \item \url{https://www.latex-project.org/}
            \item \url{http://www.texstudio.org/}
            \item \url{https://www.xm1math.net/texmaker/}
            \item \url{https://www.atlassian.com/git/tutorials/comparing-workflows/gitflow-workflow}
            \item \url{https://github.com/about}
            \item \url{https://github.com/about}
            \item \url{https://www.atlassian.com/software/jira}
            \item \url{https://www.atlassian.com/}
            \item \url{https://meet.google.com/}
            \item \url{https://slack.com/intl/en-it/about}
            \item \url{https://telegram.org/}
            \item \url{https://workspace.google.com/products/chat/}
        \end{itemize}

\pagebreak
